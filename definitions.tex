\section{Sync examples}

The first of these is seen in hardware.  All are allowed by \PTX.
Showing $\rrfx$ that is not included in the order using a dashed arrow.
\begin{gather*}
  {
    \PW{x}{1}
    \SEMI
    \PW[\mREL]{y}{1}
  }\PAR{
    \PR[\mACQ]{y}{r}
    \SEMI
    \PW{z}[\sSYS]{r}
  }\LPAR[\bScp]{
    \PR[\mACQ]{z}[\sSYS]{r}
    \SEMI
    \PR{x}{s}
  }
  \\
  \tag{$\lesync$}
  \hbox{\begin{tikzinline}[node distance=1.5em]
      \event{a1}{\DW{x}{1}[\aScp]}{}
      \event{a2}{\DW[\mREL]{y}{1}[\aScp]}{right=of a1}
      \event{b1}{\DR[\mACQ]{y}{1}[\aScp]}{right=3em of a2}
      \event{b2}{\DW{z}[\sSYS]{1}[\aScp]}{right=of b1}
      \event{c1}{\DR[\mACQ]{z}[\sSYS]{1}[\bScp]}{right=3em of b2}
      \event{c2}{\DR{x}{0}[\bScp]}{right=of c1}
      \sync{a1}{a2}
      \sync{b1}{b2}
      \sync{c1}{c2}
      \rf{a2}{b1}
      \rfint{b2}{c1}
      %\wk[out=-165,in=-15]{c2}{a1}
    \end{tikzinline}}
\end{gather*}

\begin{gather*}
  {
    \PW{x}{1}
    \SEMI
    \PW[\mREL]{y}{1}
  }\PAR{
    \PR[\mACQ]{y}{r}
    \SEMI
    \PW{z}{r}
  }\PAR{
    \PR[\mACQ]{z}{r}
    \SEMI
    \PR{x}{s}
  }
  % \\
  % \tag{$\ledep$}
  % \hbox{\begin{tikzinline}[node distance=1.5em]
  %     \event{a1}{\DW{x}{1}}{}
  %     \event{a2}{\DW[\mREL]{y}{1}}{right=of a1}
  %     \event{b1}{\DR[\mACQ]{y}{1}}{right=3em of a2}
  %     \event{b2}{\DW{z}{1}}{right=of b1}
  %     \event{c1}{\DR[\mACQ]{z}{1}}{right=3em of b2}
  %     \event{c2}{\DR{x}{0}}{right=of c1}
  %     %\sync{a1}{a2}
  %     \po{b1}{b2}
  %     %\sync{c1}{c2}
  %     \rf{a2}{b1}
  %     \rf{b2}{c1}
  %     %\wk[out=-165,in=-15]{c2}{a1}
  %   \end{tikzinline}}
  \\
  \tag{$\lesync$}
  \hbox{\begin{tikzinline}[node distance=1.5em]
      \event{a1}{\DW{x}{1}}{}
      \event{a2}{\DW[\mREL]{y}{1}}{right=of a1}
      \event{b1}{\DR[\mACQ]{y}{1}}{right=3em of a2}
      \event{b2}{\DW{z}{1}}{right=of b1}
      \event{c1}{\DR[\mACQ]{z}{1}}{right=3em of b2}
      \event{c2}{\DR{x}{0}}{right=of c1}
      \sync{a1}{a2}
      \sync{b1}{b2}
      \sync{c1}{c2}
      \rf{a2}{b1}
      \rfint{b2}{c1}
      %\wk[out=-165,in=-15]{c2}{a1}
    \end{tikzinline}}
  % \\
  % \tag{$\leloc$}
  % \hbox{\begin{tikzinline}[node distance=1.5em]
  %     \event{a1}{\DW{x}{1}}{}
  %     \event{a2}{\DW[\mREL]{y}{1}}{right=of a1}
  %     \event{b1}{\DR[\mACQ]{y}{1}}{right=3em of a2}
  %     \event{b2}{\DW{z}{1}}{right=of b1}
  %     \event{c1}{\DR[\mACQ]{z}{1}}{right=3em of b2}
  %     \event{c2}{\DR{x}{0}}{right=of c1}
  %     %\sync{a1}{a2}
  %     %\sync{b1}{b2}
  %     %\sync{c1}{c2}
  %     \rf{a2}{b1}
  %     \rf{b2}{c1}
  %     \wk[out=-165,in=-15]{c2}{a1}
  %   \end{tikzinline}}
\end{gather*}

\begin{gather*}
  {
    \PW{x}{1}
    \SEMI
    \PW[\mREL]{y}{1}
  }\PAR{
    \PR{y}{r}
    \SEMI
    \PW[\mREL]{z}{r}
  }\PAR{
    \PR[\mACQ]{z}{r}
    \SEMI
    \PR{x}{s}
  }
  \\
  \tag{$\lesync$}
  \hbox{\begin{tikzinline}[node distance=1.5em]
      \event{a1}{\DW{x}{1}}{}
      \event{a2}{\DW[\mREL]{y}{1}}{right=of a1}
      \event{b1}{\DR{y}{1}}{right=3em of a2}
      \event{b2}{\DW[\mREL]{z}{1}}{right=of b1}
      \event{c1}{\DR[\mACQ]{z}{1}}{right=3em of b2}
      \event{c2}{\DR{x}{0}}{right=of c1}
      \sync{a1}{a2}
      \sync{b1}{b2}
      \sync{c1}{c2}
      \rfint{a2}{b1}
      \rf{b2}{c1}
      %\wk[out=-165,in=-15]{c2}{a1}
    \end{tikzinline}}
\end{gather*}

To get publication using fences we need an additional closure property for
$\rrfx$ on sync order:
\begin{gather*}
  {
    \PW{x}{1}
    \SEMI
    \PF{\fREL}
    \SEMI
    \PW{y}{1}
  }\PAR{
    \PR{y}{r}
    \SEMI
    \PF{\fACQ}
    \SEMI
    \PR{x}{s}
  }
  \\
  \tag{$\lesync$}
  \hbox{\begin{tikzinline}[node distance=1.5em]
      \event{a1}{\DW{x}{1}}{}
      \event{a2}{\DF{\fREL}}{right=of a1}
      \event{a3}{\DW{y}{1}}{right=of a2}
      \event{b1}{\DR{y}{1}}{right=3em of a3}
      \event{b2}{\DF{\fACQ}}{right=of b1}
      \event{b3}{\DR{x}{0}}{right=of b2}
      \sync{a1}{a2}
      \sync{a2}{a3}
      \sync{b1}{b2}
      \sync{b2}{b3}
      \rfint{a3}{b1}
      %\wk[out=-165,in=-15]{b3}{a1}
    \end{tikzinline}}
\end{gather*}
Current def of candidate requires:
\begin{itemize}
\item[(\ref{rf-lesync})]
  if $\bEv\xrfx\aEv$ and $\labeling(\bEv) \rsmatches \labeling(\aEv)$ then $\bEv \lesync \aEv$.
\end{itemize}
This is not good enough for fences.
A possible fix is the following closure condition:
\begin{itemize}
\item[(\ref{rf-lesync}$'$)]
  if $\bEv'\lesync\bEv\xrfx\aEv\lesync\aEv'$ and $\labeling(\bEv') \rsmatches \labeling(\aEv')$ then $\bEv' \lesync \aEv'$.
\end{itemize}
With that we have the following, using $\xliftrf$ for edges induced by closure
when $\bEv'\neq\bEv$ or $\aEv'\neq\aEv$:
\begin{gather*}
  {
    \PW{x}{1}
    \SEMI
    \PF{\fREL}
    \SEMI
    \PW{y}{1}
  }\PAR{
    \PR{y}{r}
    \SEMI
    \PF{\fACQ}
    \SEMI
    \PR{x}{s}
  }
  \\
  \tag{$\lesync$}
  \hbox{\begin{tikzinline}[node distance=1.5em]
      \event{a1}{\DW{x}{1}}{}
      \event{a2}{\DF{\fREL}}{right=of a1}
      \event{a3}{\DW{y}{1}}{right=of a2}
      \event{b1}{\DR{y}{1}}{right=3em of a3}
      \event{b2}{\DF{\fACQ}}{right=of b1}
      \event{b3}{\DR{x}{0}}{right=of b2}
      \sync{a1}{a2}
      \sync{a2}{a3}
      \sync{b1}{b2}
      \sync{b2}{b3}
      \rfint{a3}{b1}
      \liftrf[out=-15,in=-165]{a2}{b2}
      %\wk[out=-165,in=-15]{b3}{a1}
    \end{tikzinline}}
\end{gather*}
This seems to work for the above examples, but it could be too strong in general.
\begin{itemize}
\item One possibility is to restrict to preceding and following things in the
  same thread:
  \begin{itemize}
  \item[(\ref{rf-lesync}$''$)]
    if $\bEv'\lesyncpo\bEv\xrfx\aEv\lesyncpo\aEv'$ and $\labeling(\bEv') \rsmatches \labeling(\aEv')$ then $\bEv' \lesync \aEv'$.
  \end{itemize}
  where $\lesyncpo$ is the obvious restriction of $\lesync$ to actions on the
  same thread.
\item With either (\ref{rf-lesync}$'$) or (\ref{rf-lesync}$''$) is it too strong to require $\lesync$ that be
  transitive?   In particular:
  \begin{itemize}
  \item if we restrict to $\lesyncpo$, the closure condition
    (\ref{rf-lesync}$''$) could add order between actions on the same thread
    via cross-thread reads.
  \item How does transitivity interact with scopes?
  \end{itemize}
\end{itemize}
Anton proposes:
\begin{itemize}
\item[(\ref{pom-rmw-lesync}$'$)]
  if $\bEv\xrmw\aEv$ then %$\bEv \lesync \aEv$ and
  $\bEv \leloc \aEv$,    
\item[(\ref{rf-lesync}$'''$)]
  if $\bEv'\lesync\bEv\mathrel{(\xrfx;(\xrmw;\xrfx)^{*})}\aEv\lesync\aEv'$ and $\labeling(\bEv') \rsmatches \labeling(\aEv')$ then $\bEv' \lesync \aEv'$.
\end{itemize}

The following behavior is allowed by Arm, IMM, and C11, but forbidden by \PTX.
\PTX{} forbids it since acquire reads work as fences for po-previous reads from
the same location (symmetrically to release writes for po-latter writes to
the same location in \IMM, \cXI, and \PTX).
\begin{gather*}
  {
    \PW{x}{1}
    \SEMI
    \PW[\mREL]{y}{1}
  }\PAR{
    \PR{y}{r}
    \SEMI
    \PW{y}{2}
    \SEMI
    \PR[\mACQ]{y}{s}
     \SEMI
    \PR{x}{t}
  }
  \\
  \tag{$\lesync$}
  \hbox{\begin{tikzinline}[node distance=1.5em]
      \event{a}{\DW{x}{1}}{}
      \raevent{b}{\DW[\mREL]{y}{1}}{right=of a}
      \event{c}{\DR{y}{1}}{right=3em of b}
      \event{d}{\DW{y}{2}}{right=of c}
      \raevent{e}{\DR[\mACQ]{y}{2}}{right=of d}
      \event{f}{\DR{x}{0}}{right=of e}
      \sync{a}{b}
      \rfint{b}{c}
      \sync[out=15,in=165]{c}{e}
      %\wk{c}{d}
      \rfint{d}{e}
      \sync{e}{f}
      %\wk[out=-165,in=-15]{f}{a}
      \liftrf[out=-15,in=-165]{b}{e}
    \end{tikzinline}}
\end{gather*}
To allow this on for IMM, we need to drop
\begin{math}
  (\DR{\aLoc}{}, \DR[\gemode\mACQ]{\aLoc}{})
\end{math}
from $\rsyncdelays$.

The following is allowed by \cXI{}, but not \IMM{} or \PTX.
The goal here is to construct a cycle
$a\xrfx b \xhb c \xrfx d \xhb a$
where $\rrfx$  will be included in synch-relation.
In relational notation, the cycle has the following form:
\begin{displaymath}
  \PBRbig{{\rrmw} ; ({\rrfe} ; {\rrmw})^2 ; {\rppo} ; \Wclass[\mREL]; {\rrfe} ; \Rclass[\mACQ] ; {\rppo}}^2
\end{displaymath}
\begin{gather*}
  {
    \PR[\mACQ]{x}{r}
    \SEMI
    \PINC{y}{}
  }\PAR{
    \PINC{y}{}
  }\PAR{
    \PINC{y}{}
    \SEMI
    \PW[\mREL]{z}{1}
  }\PAR{
    \PR[\mACQ]{z}{s}
    \SEMI
    \PINC{w}{}
  }\PAR{
    \PINC{w}{}
  }\PAR{
    \PINC{w}{}
    \SEMI
    \PW[\mREL]{x}{1}
  }
  \\
  \hbox{\begin{tikzinline}[node distance=1.5em]
      \event{a1}{\DR[\mACQ]{x}{1}}{}
      \event{a2}{\DR{y}{0}}{below=of a1}
      \event{a3}{\DW{y}{1}}{below=of a2}
      \sync{a1}{a2}
      \rmw{a2}{a3}
      \event{b1}{\DR{y}{1}}{right=of a1}
      \event{b2}{\DW{y}{2}}{below=of b1}
      \rmw{b1}{b2}
      \event{c1}{\DR{y}{2}}{right=of b1}
      \event{c2}{\DW{y}{3}}{below=of c1}
      \event{c3}{\DW[\mREL]{z}{1}}{below=of c2}
      \rmw{c1}{c2}
      \sync{c2}{c3}
      \event{d1}{\DR[\mACQ]{z}{1}}{right=of c1}
      \event{d2}{\DR{w}{0}}{below=of d1}
      \event{d3}{\DW{w}{1}}{below=of d2}
      \sync{d1}{d2}
      \rmw{d2}{d3}
      \event{e1}{\DR{w}{1}}{right=of d1}
      \event{e2}{\DW{w}{2}}{below=of e1}
      \rmw{e1}{e2}
      \event{f1}{\DR{w}{2}}{right=of e1}
      \event{f2}{\DR{w}{3}}{below=of f1}
      \event{f3}{\DW[\mREL]{x}{1}}{below=of f2}
      \rmw{f1}{f2}
      \sync{f2}{f3}
      \rf{a3}{b1}
      \rf{b2}{c1}
      \rf{c3}{d1}
      \rf{d3}{e1}
      \rf{e2}{f1}
      \rf{f3}{a1}
    \end{tikzinline}}
\end{gather*}

\section{Model}
\label{sec:model}

\subsection{Preliminaries}
\label{sec:prelim}
The syntax is built from
\begin{itemize}
\item a set of \emph{values} $\Val$, ranged over by
  $\aVal$, $\bVal$, $\cVal$, $\dVal$,
\item a set of \emph{registers} $\Reg$, ranged over by
  $\aReg$, $\bReg$,
\item a set of \emph{expressions} $\Exp$, ranged over by
  $\aExp$, $\bExp$,  $\cExp$,
\item a set of \emph{thread ids} $\Thrd$, ranged over by
  $\aThrd$, $\bThrd$.
\end{itemize}

\emph{Memory references} are tagged values, written $\REF{\cVal}$.  Let $\Loc$
be the set of memory references, ranged over by $\aLoc$, $\bLoc$, $\cLoc$.

We require that
\begin{itemize}
\item values and registers are disjoint, 
\item values include at least the constants $0$ and $1$,  
\item expressions include at least registers and values, 
\item expressions do \emph{not} include references: $\aExp[\bExp/\aLoc]=\aExp$,
\item there are registers
  $\uRegs{\AllEvents}=\{\uReg{\aEv}\mid\aEv\in\AllEvents\}$,
\item registers $\uRegs{\AllEvents}$ do not appear in programs: $\aCmd[\bExp/\uReg{\aEv}]=\aCmd$.
\end{itemize}
Alternative to the last assumption, we sometimes assume each register is
assigned at most once.\footnote{We make this assumption when discussing any
  semantics of load ($\PRREF[\amode]{\cExp}[\ascope]{\aReg}$) that does not
  include the substitution $[\uReg{\aEv}/\aReg]$.}

We model the following language.
\begin{align*}
  \amode \BNFDEF& \mWK
  \BNFSEP \mRLX
  \BNFSEP \mRA 
  \BNFSEP \mSC
  &
  \fmode \BNFDEF& \fACQ 
  \BNFSEP \fREL
  \BNFSEP \fSC
  &
  \ascope,\bscope \BNFDEF& \sCTA
  \BNFSEP \sGPU
  \BNFSEP \sSYS
\end{align*}
\begin{align*}
  \aCmd
  \BNFDEF& \SKIP
  \BNFSEP \LET{\aReg}{\aExp}
  \BNFSEP \PR[\amode]{\REF{\cExp}}[\ascope]{\aReg}
  \BNFSEP \PW[\amode]{\REF{\cExp}}[\ascope]{\aExp}
  \BNFSEP \PF[\ascope]{\fmode}
  \BNFSEP \IF{\aExp} \THEN \aCmd_1 \ELSE \aCmd_2 \FI
  \BNFSEP \aCmd_1 \SEMI \aCmd_2
  \\[-.5ex]
  \BNFSEP& \aCmd_1 \LPAR[\bThrd] \aCmd_2
  \BNFSEP \PCAS[\amode_1][\amode_2]{\REF{\cExp}}[\ascope]{\aReg}{\aExp}{\bExp}
  \BNFSEP \PFADD[\amode_1][\amode_2]{\REF{\cExp}}[\ascope]{\aReg}{\aExp}
  \BNFSEP \PEXCHG[\amode_1][\amode_2]{\REF{\cExp}}[\ascope]{\aReg}{\aExp}
\end{align*}

\emph{Access modes}, $\amode$, are {weak} ($\mWK$), are {relaxed} ($\mRLX$),
{release-acquire} ($\mRA$), and {sequentially consistent} ($\mSC$).
$\mRA$/$\mSC$ accesses are collectively known as \emph{synchronized
  accesses}.

\emph{Fence modes}, $\bmode$, are {acquire} ($\fACQ$), {release} ($\fREL$), 
and {acquire-release} ($\fSC$).  

\emph{Scopes}, $\ascope$, are thread group ($\sCTA$), processor ($\sGPU$) and
system ($\sSYS$).

\emph{Commands}, aka \emph{statements}, $\aCmd$, include memory accesses at a
given mode, as well as the usual structural constructs.  Following
\cite{DBLP:conf/icfp/FerreiraHJ96}, $\LPAR$ denotes parallel composition.  If
$(\aCmd_1 \LPAR[\bThrd] \aCmd_2)$ is executed with thread \ID{} $\aThrd$, then
$\aCmd_2$ runs with \ID{} $\bThrd$ and $\aCmd_1$ continues under \ID{} $\aThrd$.
Top level programs run with thread \ID{} $\topThrd$.  In examples, we usually
drop thread \ID{}s.  We use the symmetric $\PAR$ operator when there is no
continuation after the parallel composition.


The semantics is built from the following.
\begin{itemize}
\item a set of \emph{events} $\AllEvents$, ranged over by $\aEv$, $\bEv$,
  $\cEv$, $\dEv$, 
\item a set of \emph{actions} $\Act$, ranged over by $\aAct$, 
\item a set of \emph{logical formulae} $\Formulae$, ranged over by $\aForm$,
  $\bForm$, $\cForm$.
\end{itemize}
Subsets of $\AllEvents$ are ranged over by $\aEvs$, $\bEvs$, $\cEvs$,
$\dEvs$.

We require that:
\begin{itemize}
\item formulae include equalities $(\aExp{=}\bExp)$ and $(\aLoc{=}\aExp)$,
\item formulae are closed under negation, conjunction, disjunction, and
  substitutions $[\aExp/\aReg]$, $[\aExp/\aLoc]$,
\item there is a relation $\rimpliesdef$ between
  formulae, capturing entailment, 
\item $\rimpliesdef$ has the expected semantics for $=$, $\lnot$, $\land$, $\lor$,
  $\limplies$ and substitution.
\end{itemize}

Logical formulae include equations over registers, such as
$(\aReg{=}\bReg{+}1)$.  For \xLIR{}, we also include equations over memory
references, such as $(\aLoc{=}1)$.  Formulae are subject to substitutions;
actions are not.  We use expressions as formulae, coercing $\aExp$ to
$\aExp{\neq}0$.  Equations have precedence over logical operators; thus
$\aReg{=}\aVal\limplies\bReg{>}\bVal$ is read
$(\aReg{=}\aVal)\limplies(\bReg{>}\bVal)$.  As usual, implication associates
to the right; thus $\aForm\limplies\bForm\limplies\cForm$ is read
$\aForm\limplies(\bForm\limplies\cForm)$.

We say
$\aForm$ is a \emph{tautology} if $\TRUE\vDash\aForm$.
We say
$\aForm$ is \emph{unsatisfiable} if $\aForm\vDash\FALSE$.



We require several binary relations between actions, detailed in the next
subsection: 
$\roverlapsdef$, 
$\rmatchesdef$, 
$\rsmatchesdef$, 
$\rblocksdef$, 
$\rsblocksdef$,
$\rsyncdelaysdef$
and $\rcodelaysdef$.
We also require that there is a subsets of actions, distinguishing
$\sreaddef$ and $\sreleasedef$ actions, and an operator
$\fmerge{}{}:\Act\times\Act\fun2^{\Act}$.


\subsection{Actions}
\label{sec:actions}

We combine access and fence modes into a single order:
\begin{align*}
  \begin{tikzcenter}
    \node (wk)  at (-1, 0) {$\mathstrut\mWK$};
    \node (rlx) at (0, 0) {$\mathstrut\mRLX$};
    \node (ra)  at (1, 0) {$\mathstrut\mRA$};
    \node (sc)  at (2, 0) {$\mathstrut\mSC$};
    \draw[->](wk)to(rlx);
    \draw[->](rlx)to(ra);
    \draw[->](ra)to(sc);
  \end{tikzcenter}
  &&
  \begin{tikzcenter}
    \node (fsc) at (3, 0) {$\mathstrut\fSC$};
    \node (rel) at (2, -0.2) {$\mathstrut\fREL$};
    \node (acq) at (2,  0.2) {$\mathstrut\fACQ$};
    \draw[->](rel)to(fsc);
    \draw[->](acq)to(fsc);
  \end{tikzcenter}
\end{align*}
We write $\amode\lemode\bmode$ for this order.
Let $\amode\lubmode\bmode$ denote the least upper bound of $\amode$ and $\bmode$.

Let actions be reads, writes and fences:
\begin{displaymath}
  \aAct,\bAct \BNFDEF \DW[\amode]{\aLoc}[\ascope]{\aVal}[\aThrd]
  \BNFSEP \DR[\amode]{\aLoc}[\ascope]{\aVal}[\aThrd]
  \BNFSEP \DF[\ascope]{\fmode}[\aThrd]
\end{displaymath}
In definitions, we drop elements of actions that are existentially
quantified.  In examples, we drop elements of actions, using defaults.  We
write $\DXP[\amode]{}[\ascope]{}[\aThrd]$ to stand for $\DWP[\amode]{}[\ascope]{}[\aThrd]$, $\DRP[\amode]{}[\ascope]{}[\aThrd]$,
or $\DFP[\ascope]{\amode}[\aThrd]$.  We write $\DWP[\gemode\mREL]{}{}$ to stand for either
$\DWP[\mREL]{}{}$ or $\DWP[\mSC]{}{}$, and similarly for other actions and
modes.

We say $\aAct \rmatchesdef \bAct$ if $\aAct=\DWP{\aLoc}{\aVal}$ and $\bAct=\DRP{\aLoc}{\aVal}$.

We say $\aAct \rblocksdef \bAct$ if $\aAct=\DWP{\aLoc}{}$ and $\bAct=\DRP{\aLoc}{}$, regardless of value.

We say $\aAct \roverlapsdef \bAct$ if they access the same location.

We say $\aAct \rcodelaysdef \bAct$ if
\begin{math}
  (\aAct,\bAct)\in
  \{(\DW{\aLoc}{}, \DW{\aLoc}{}),\;(\DR{\aLoc}{}, \DW{\aLoc}{}),\;(\DW{\aLoc}{}, \DR{\aLoc}{}),\;(\DX[\mSC]{}{},\DX[\mSC]{}{})\}
\end{math}.

We say $\aAct \rsyncdelaysdef \bAct$ if
\begin{math}
  (\aAct,\bAct)\in
  \{(\aAct,             \DW[\gemode\mREL]{}{}     )\Cc
  (\aAct,               \DF{\gemode\fREL}        )\Cc
  (\DR{}{},             \DF{\gemode\fACQ}        )\Cc
  (\DR[\gemode\mACQ]{}{},\bAct                    )\Cc
  (\DF{\gemode\fACQ},   \bAct                    )\Cc
  (\DF{\gemode\fREL},   \DW{}{}                  )\Cc
  (\DW[\gemode\mREL]{\aLoc}{},\DW{\aLoc}{})\}
\end{math}.\footnote{For \PTX, one can additionally include
  \begin{math}
    (\DR{\aLoc}{}, \DR[\gemode\mACQ]{\aLoc}{}).
  \end{math}}

Let $\DWP[\gemode\mREL]{}{}$ and $\DFP{\gemode\fREL}$ be
$\sreleasedef$ actions.  Actions $\DRP{}{}$ are $\sreaddef$ actions.

Let
$\fmerge{}{}:\Act\times\Act\fun2^{\Act}$
be defined as follows.
Let
\begin{math}
  \fmerge{\DR[\amode]{\aLoc}{\aVal}}{\DR[\bmode]{\aLoc}{\aVal}} = \{
  \DR[\amode\lubmode\bmode]{\aLoc}{\aVal} \}
\end{math},
\begin{math}
  \fmerge{\DW[\amode]{\aLoc}{\aVal}}{\DW[\bmode]{\aLoc}{\bVal}} = \{
  \DW[\amode\lubmode\bmode]{\aLoc}{\bVal} \}
\end{math},
\begin{math}
  \fmerge{\DW[\amode]{\aLoc}{\aVal}}{\DR[\bmode\lemode\mRLX]{\aLoc}{\aVal}} = \{
  \DW[\amode\lubmode\bmode]{\aLoc}{\aVal} \}
\end{math},
\begin{math}
  \fmerge{\DF{\amode}}{\DF{\bmode}} = \{ \DF{\amode\lubmode\bmode} \}
\end{math},
and
\begin{math}
  \fmerge{\aAct}{\bAct} = \emptyset, 
\end{math}
otherwise.

If $\aAct_0\in\fmerge{\aAct_1}{\aAct_2}$, then $\aAct_1$ and $\aAct_2$ can
coalesce, resulting in $\aAct_0$.  This allows optimizations such as
$(\PW{x}{1}\SEMI \PW{x}{2})$ to $(\PW{x}{2})$ and
$(\PW{x}{1}\SEMI \PR{x}{r})$ to $(\PW{x}{1}\SEMI \PR{1}{r})$.  For
associativity of sequential composition, it is important that $\fmerge{}{}$
always take an upper bound on the modes of the two actions.  For example, it
would invalidate associativity to allow
$\DWP{\aLoc}{\aVal}\in\fmerge{\DW{\aLoc}{\aVal}}{\DR[\mACQ]{\aLoc}{\aVal}}$,
although this is considered safe.\footnote{A list of safe merge operations
  can be found in \cite[\textsection E]{DBLP:conf/cgo/ChakrabortyV17} and
  \cite[\textsection7.1]{Kang19}.  For examples of unsafe merges and
  reorderings, see \cite[\textsection D]{DBLP:conf/cgo/ChakrabortyV17}.}


Default mode should be RLX, default scope SYS.

\begin{definition}
  When modeling \IMM, we ban access mode $\mWK$; the default access mode is
  $\mRLX$.  We also ban scopes $\sCTA$ and $\sGPU$; the only allowed scope is
  $\sSYS$.  We assume there is only one thread \ID{} ($\fcard{\Thrd}=1$), which
  we elide.  Let $\rsblocksdef$ be $\Act\times\Act$.
  We say $\aAct \rsmatchesdef \bAct$ if (1) $\aAct \roverlaps \bAct$ and (2) neither has mode $\mRLX$.
\end{definition}
\begin{definition}
  When modeling \PTX, the default access mode is $\mWK$.  The default scope
  is $\sCTA$.  We assume two equivalences:
  $\PBR{\xgpu}\subseteq(\Thrd\times\Thrd)$ partitions threads by
  \emph{processor}, and $\PBR{\xcta}\subseteq\PBR{\xgpu}$ refines the processor
  partitioning into \emph{thread groups}.  
  We say
  $\DXP[\amode]{}[\ascope]{}[\aThrd] \rsblocksdef \DXP[\bmode]{}[\bscope]{}[\bThrd]$
  when either (1) $\aThrd=\bThrd$ or (2) all of the following hold:
  \begin{enumerate}[,label=(2\alph*),ref=2\alph*]
  \item $\amode,\bmode\neq\mWK$,
  \item if $\ascope=\sCTA$ or $\bscope=\sCTA$ then $\aThrd\xcta\bThrd$,
  \item if $\ascope=\sGPU$ or $\bscope=\sGPU$ then $\aThrd\xgpu\bThrd$, 
  \item if either action is an access then they overlap.
  \end{enumerate}
  % We say $\DXP[\amode]{}[\ascope]{}[\aThrd] \rsmatchesdef \DXP[\bmode]{}[\bscope]{}[\bThrd]$ if
  % (1) $\DXP[\amode]{}[\ascope]{}[\aThrd] \rsblocks \DXP[\bmode]{}[\bscope]{}[\bThrd]$ and
  % either (2a) $\aThrd=\bThrd$ or (2b) $\amode$ is an acquire and $\bmode$ is a release.
  We say $\aAct \rsmatchesdef \bAct$ if (1) $\aAct \rsblocks \bAct$ and
  either (2a) they have the same thread (2b) $\aAct$ is an acquire and $\bAct$ is a release.
\end{definition}




\subsection{Pomsets with Predicate Transformers}
\label{sec:pomsets}

\begin{definition}
  \label{def:trans}
  A \emph{predicate transformer} is a   function
  $\aTr{}{}:\Formulae\fun\Formulae$ such that
  \begin{enumerate}
  \item $\aTr{}{\FALSE}$ is $\FALSE$,    
  \item $\aTr{}{\bForm_1\land\bForm_2}$ is $\aTr{}{\bForm_1}\land\aTr{}{\bForm_2}$,    
  \item $\aTr{}{\bForm_1\lor\bForm_2}$ is $\aTr{}{\bForm_1}\lor\aTr{}{\bForm_2}$, 
  \item if $\aForm \rimplies \bForm$, then $\aTr{}{\aForm} \rimplies \aTr{}{\bForm}$.
  \end{enumerate}
\end{definition}

\begin{definition}
  \label{def:family}
  A \emph{family of predicate transformers} for $\aEvs$ consists of a
  predicate transformer $\aTr{\bEvs}{}$ for each $\bEvs\subseteq\AllEvents$,
  such that if $\cEvs \cap \aEvs \subseteq \bEvs$ then
  $\aTr{\cEvs}{\bForm} \rimplies \aTr{\bEvs}{\bForm}$.
\end{definition}

\begin{definition}
  \label{def:pomset}
  A \emph{pomset with predicate transformers} 
  is a tuple $(\Event, \labeling, \labelingForm, \aTr{}{}, {\aTerm}, {\ledep}, {\lesync}, {\leloc}, {\rrmw})$ where
  \begin{enumerate}[,label=(\textsc{m}\arabic*),ref=\textsc{m}\arabic*]
  \item \label{pom-E}
    $\Event\subset\AllEvents$ is a set of \emph{events},
  \item \label{pom-lambda}
    $\labeling: \Event \fun \Act$ defines a \emph{label} for each event,
  \item \label{pom-kappa}
    $\labelingForm:\aEvs\fun\Formulae$ defines a \emph{precondition} for each event,
  \item \label{pom-tau}
    $\aTr{}{}:2^{\AllEvents}\fun\Formulae \fun\Formulae$ is a \emph{family of
      predicate transformers} over $\aEvs$, 
  \item \label{pom-term}
    $\aTerm:\Formulae$ defines a \emph{termination condition},
  \item
    \label{pom-ledep}
    ${\ledep} : (\Event\times\Event)$ is a partial order capturing \emph{dependency},
  \item
    \label{pom-lesync}
    ${\lesync} : (\Event\times\Event)$ is a partial order capturing
    \emph{synchronization}, 
  \item \label{pom-leloc}
    ${\leloc} : (\Event\times\Event)$ is a partial order capturing
    \emph{per-location order}, such that 
    \begin{enumerate}
    \item \label{pom-leloc-lesync}
      if $\labeling(\bEv)\roverlaps\labeling(\aEv)$ 
      then $\bEv\lesync\aEv$ implies $\bEv\leloc\aEv$,      
    \end{enumerate}
  \item \label{pom-rmw}
    ${\rrmw} : \Event\fun\Event$ is a partial function capturing
    read-modify-write \emph{atomicity}, such that
    \begin{enumerate}
    \item \label{pom-rmw-block}
      if $\bEv\xrmw\aEv$ then $\labeling(\aEv) \rblocks \labeling(\bEv)$,
    \item \label{pom-rmw-lesync} \label{pom-rmw-leloc}
      if $\bEv\xrmw\aEv$ then $\bEv \lesync \aEv$ and $\bEv \leloc \aEv$,    
    \item \label{pom-rmw-atomic}
      if $\labeling(\cEv)\roverlaps\labeling(\bEv)$ then
      \begin{enumerate}        
      \item \label{pom-rmw-atomic1}
        if $\bEv \xrmw \aEv$ then
        $\cEv\ledep \aEv$ implies $\cEv\ledep \bEv$,
        $\cEv\lesync \aEv$ implies $\cEv\lesync \bEv$,
        $\cEv\leloc \aEv$ implies $\cEv\leloc \bEv$,
      \item \label{pom-rmw-atomic2}
        if $\bEv \xrmw \aEv$ then
        $\bEv\ledep \cEv$ implies $\aEv\ledep \cEv$,
        $\bEv\lesync \cEv$ implies $\aEv\lesync \cEv$,
        $\bEv\leloc \cEv$ implies $\aEv\leloc \cEv$.
      \end{enumerate}
    \end{enumerate}
  \end{enumerate}

  A pomset is a \emph{candidate} if there is an injective relation
  ${\rrfx} : \Event\times\Event$, capturing \emph{reads-from}, such that
  \begin{enumerate}[,label=(\textsc{c}\arabic*),ref=\textsc{c}\arabic*]
  \item \label{rf-match}
    if $\bEv\xrfx\aEv$ then $\labeling(\bEv) \rmatches \labeling(\aEv)$,
  \item \label{rf-block}
    if $\bEv\xrfx\aEv$ and $\labeling(\cEv) \rblocks \labeling(\aEv)$ then either $\cEv\leexists\bEv$ or $\aEv\leexists\cEv$,\\
    where $\bEv'\leexists\aEv'$ when
    \labeltext[\textsc{c}2a]{(\textsc{c}2a)}{rf-block-not-leloc}
    $\aEv'\not\ltloc\bEv'$ and
    \labeltext[\textsc{c}2b]{(\textsc{c}2b)}{rf-block-strongly}
    if $\labeling(\bEv') \rsblocks \labeling(\aEv')$ then $\bEv'\leloc\aEv'$,
  \item \label{rf-ledep} \label{rf-leloc}
    if $\bEv\xrfx\aEv$ then $\bEv \ledep \aEv$ and $\bEv \leloc \aEv$,
  \item \label{rf-lesync}
    if $\bEv\xrfx\aEv$ and $\labeling(\bEv) \rsmatches \labeling(\aEv)$ then $\bEv \lesync \aEv$,
  \item \label{top-sc}
    if $\labeling(\bEv) \rblocks \labeling(\aEv)$ and both are $\mSC$ fences
    then either $\bEv\lesync\aEv$ or $\aEv\lesync\bEv$.
  \end{enumerate}

  A pomset is \emph{top-level} if
  % \labeltext[\textsc{t}1]{(\textsc{t}1)}{top-term} $\aTerm$ is a tautology and \labeltext[\textsc{t}2]{(\textsc{t}2)}{top-ev} for every $\aEv\in\aEvs$,
  % \begin{enumerate}[label=(\textsc{t}\arabic*),ref=\textsc{t}\arabic*]
  %   \setcounter{enumi}{2}
  % \item[]
  %   \begin{enumerate}[leftmargin=0pt]
  %   \item \label{top-kappa}
  %     $\labelingForm(\aEv)$ is a tautology,    
  %   \item \label{top-rf}
  %     if $\labeling(\aEv)$ is a $\sread$ then there is some $\bEv\xrfx\aEv$.
  %   \end{enumerate}
  % \end{enumerate}
  \begin{enumerate}[,label=(\textsc{t}\arabic*),ref=\textsc{t}\arabic*]
  \item \label{top-term} $\aTerm$ is a tautology,
  \item \label{top-kappa} $\labelingForm(\aEv)$ is a tautology (for every $\aEv\in\aEvs$),
  \item \label{top-rf} if $\labeling(\aEv)$ is a $\sread$ then there is some $\bEv\xrfx\aEv$.
  \end{enumerate}
\end{definition}

Note that for the \IMM{} model, \ref{rf-block} is equivalent to:
\begin{center}
  if $\bEv\xrfx\aEv$ and $\labeling(\cEv) \rblocks \labeling(\aEv)$ then
  either $\cEv\leloc\bEv$ or $\aEv\leloc\cEv$.
\end{center}

Let $\aPS$ range over pomsets, and $\aPSS$ over sets of pomsets.

We lift terminology from actions to events.  For example, we say that $\aEv$
writes $\aLoc$ if $\labeling(\aEv)$ writes $\aLoc$.  We also drop quantifiers
when clear from context, such as
$(\forall\aEv\in\Event)(\forall\aLoc\in\Loc)$.
We write $\bEv\ltsync\aEv$ when $\bEv\lesync\aEv$ and $\bEv\neq\aEv$, and similarly
for $\ltdep$ and $\ltloc$.

\begin{definition}
  $\aPSS_1$ \emph{refines} $\aPSS_2$ if $\aPSS_1\subseteq\aPSS_2$.
\end{definition}

\subsection{Semantics}


\begin{definition}
  \noindent
  If $\aPS\in\sSKIP$ then $\aEvs = \emptyset$ and
  $\aTr{\bEvs}{\bForm} \rimplies \bForm$.

  \noindent
  If $\aPS \in \sLPAR{\aPSS_1}{\aPSS_2}$ then  
  $(\exists\aPS_1\in\aPSS_1)$ $(\exists\aPS_2\in\aPSS_2)$
  \begin{enumerate}[topsep=0pt,label=(\textsc{p}\arabic*),ref=\textsc{p}\arabic*]
  \item \label{par-E}
    $\aEvs = (\aEvs_1\cup\aEvs_2)$,
    ${\ledep}\supseteq\PBR{{\ledep_1}\cup{\ledep_2}}$, 
    ${\lesync}\supseteq\PBR{{\lesync_1}\cup{\lesync_2}}$, 
    ${\leloc}\supseteq\PBR{{\leloc_1}\cup{\leloc_2}}$, 
    ${\rrmw}=\PBR{{\rrmw_1}\cup{\rrmw_2}}$, 
  \item \label{par-lambda}
    ${\labeling}=\PBR{{\labeling_1}\cup {\labeling_2}}$, 
  \item[] \stepcounter{enumi} \labeltext[\textsc{p}3]{}{par-kappa}
    \begin{enumerate}[leftmargin=0pt]
    \item \label{par-kappa1} if $\aEv\in\aEvs_1$ then $\labelingForm(\aEv) \rimplies \labelingForm_1(\aEv)$,
    \item \label{par-kappa2} if $\aEv\in\aEvs_2$ then $\labelingForm(\aEv) \rimplies \labelingForm_2(\aEv)$,
    \end{enumerate}
  \item \label{par-tau}
    $\aTr{\bEvs}{\bForm} \rimplies \aTr[1]{\bEvs}{\bForm}$,
  \item \label{par-term}
    $\aTerm \rimplies \aTerm[1]\land\aTerm[2]$,
  \item \label{par-disjoint}
    $\aEvs_1$ and $\aEvs_2$ are disjoint.
  \end{enumerate}
  \medskip

  If $\aPS \in \sSEMI{\aPSS_1}{\aPSS_2}$ then
  $(\exists\aPS_1\in\aPSS_1)$ $(\exists\aPS_2\in\aPSS_2)$
  \begin{enumerate}[topsep=0pt,label=(\textsc{s}\arabic*),ref=\textsc{s}\arabic*]
  \item as in \ref{par-E},
  \item[] \stepcounter{enumi} \labeltext[\textsc{s}2]{}{seq-lambda}
    \begin{enumerate}[leftmargin=0pt]
    \item \label{seq-lambda1}
      if $\aEv\in\aEvs_1\setminus\aEvs_2$ then $\labeling(\aEv)=\labeling_1(\aEv)$,
    \item \label{seq-lambda2}
      if $\aEv\in\aEvs_2\setminus\aEvs_1$ then $\labeling(\aEv)=\labeling_2(\aEv)$,
    \item \label{seq-lambda12}
      if $\aEv\in\aEvs_1\cap\aEvs_2$ then $\labeling(\aEv)\in \fmerge{\labeling_1(\aEv)}{\labeling_2(\aEv)}$,
    \end{enumerate}
  \item[] \stepcounter{enumi} \labeltext[\textsc{s}3]{}{seq-kappa}
    \begin{enumerate}[leftmargin=0pt]
    \item \label{seq-kappa1}
      if $\aEv\in\aEvs_1\setminus\aEvs_2$ then $\labelingForm(\aEv) \rimplies \labelingForm_1(\aEv)$,
    \item \label{seq-kappa2}
      if $\aEv\in\aEvs_2\setminus\aEvs_1$ then $\labelingForm(\aEv) \rimplies \labelingForm'_2(\aEv)$,
    \item \label{seq-kappa12}
      if $\aEv\in\aEvs_1\cap\aEvs_2$ then $\labelingForm(\aEv) \rimplies \labelingForm_1(\aEv)\lor\labelingForm'_2(\aEv)$,
      where
      $\labelingForm'_2(\aEv)=\aTr[1]{\Cdown{\aEv}}{\labelingForm_2(\aEv})$,\\
      where $\Cdown{\aEv}=\{ \cEv \mid \cEv \ltdep \aEv \}$ if $\labeling(\aEv)$ is a write,
      and $\Cdown{\aEv}=\aEvs_1$, otherwise,
    \item \label{seq-release}
      if $\labeling_2(\aEv)$ is a $\srelease$ then $\labelingForm(\aEv) \rimplies \aTerm[1]$,
    \end{enumerate}
  \item \label{seq-tau}
    $\aTr{\bEvs}{\bForm} \rimplies \aTr[1]{\bEvs}{\aTr[2]{\bEvs}{\bForm}}$,
  \item \label{seq-term}
    $\aTerm \rimplies \aTerm[1]\land\aTr[1]{\aEvs_1}{\aTerm[2]}$,
  \item[] \stepcounter{enumi} \labeltext[\textsc{s}6]{}{seq-delay}
    \begin{enumerate}[leftmargin=0pt]
    \item \label{seq-delay-sync}
      if 
      $\labeling_1(\bEv)\rsyncdelays\labeling_2(\aEv)$ then $\bEv\lesync\aEv$,
    \item \label{seq-delay-co}
      if 
      $\labeling_1(\bEv)\rcodelays\labeling_2(\aEv)$ then $\bEv\leloc\aEv$.
    \end{enumerate}
  \end{enumerate}
  \medskip

  \noindent
  If $\aPS \in \sIFTHEN{\aForm}{\aPSS_1}{\aPSS_2}$ then
  $(\exists\aPS_1\in\aPSS_1)$ $(\exists\aPS_2\in\aPSS_2)$
  \begin{enumerate}[topsep=0pt,label=(\textsc{i}\arabic*),ref=\textsc{i}\arabic*]
  \item as in \ref{par-E},
  \item 
    ${\labeling}=\PBR{{\labeling_1}\cup {\labeling_2}}$, 
  \item[] \stepcounter{enumi} \labeltext[\textsc{i}3]{}{if-kappa}
    \begin{enumerate}[leftmargin=0pt]
    \item \label{if-kappa1}
      if $\aEv\in\aEvs_1\setminus\aEvs_2$ then $\labelingForm(\aEv) \rimplies \aForm\land\labelingForm_1(\aEv)$,
    \item \label{if-kappa2}
      if $\aEv\in\aEvs_2\setminus\aEvs_1$ then $\labelingForm(\aEv) \rimplies \neg\aForm\land\labelingForm_2(\aEv)$, 
      \item \label{if-kappa12}
        if $\aEv\in\aEvs_1\cap\aEvs_2$\\ then
        $\labelingForm(\aEv) \rimplies (\aForm\land\labelingForm_1(\aEv))\lor(\neg\aForm\land\labelingForm_2(\aEv))$,
      \end{enumerate}
    \item \label{if-tau}
      $\aTr{\bEvs}{\bForm} \rimplies (\aForm\land\aTr[1]{\bEvs}{\bForm})\lor(\neg\aForm\land\aTr[2]{\bEvs}{\bForm})$,
    \item \label{if-term}
      $\aTerm \rimplies (\aForm\land\aTerm[1])\lor(\neg\aForm\land\aTerm[2])$.
  \end{enumerate}
  \medskip

  \noindent
  If $\aPS\in\sLET{\aReg}{\aExp}$ then $\aEvs = \emptyset$ and
  $\aTr{\bEvs}{\bForm} \rimplies \bForm[\aExp/\aReg]$.

  \noindent
  If $\aPS \in \sFENCE[\ascope]{\amode}[\aThrd]$ then
  \begin{enumerate}[topsep=0pt,label=(\textsc{f}\arabic*),ref=\textsc{f}\arabic*]
  \item \label{fence-E}
    if $\bEv,\aEv\in\aEvs$ then $\bEv=\aEv$,
  \item \label{fence-lambda}
    $\labelingAct(\aEv) = \DF[\ascope]{\amode}[\aThrd]$,
    \stepcounter{enumi}
  \item \label{fence-tau}
    $\aTr{\bEvs}{\bForm} \rimplies \bForm$,
  \item \label{fence-term}
    if $\aEvs=\emptyset$ then $\aTerm \rimplies \FALSE$.
  \end{enumerate}
  \medskip

  \noindent
  If $\aPS \in \sLOAD[\amode]{\aReg}[\ascope]{\aLoc}[\aThrd]$ then
  $(\exists\aVal\in\Val)$
  \begin{enumerate}[topsep=0pt,label=(\textsc{r}\arabic*),ref=\textsc{r}\arabic*]
  \item \label{read-E}
    if $\bEv,\aEv\in\aEvs$ then $\bEv=\aEv$,
  \item \label{read-lambda}
    $\labelingAct(\aEv) = \DR[\amode]{\aLoc}[\ascope]{\aVal}[\aThrd]$,      
    \stepcounter{enumi}
    \stepcounter{enumi}
  \item[] \labeltext[\textsc{r}4]{}{read-tau}
    \begin{enumerate}[leftmargin=0pt]
    \item \label{read-tau-dep}
      if $(\aEvs\cap\bEvs)\neq\emptyset$ then
      \begin{math}
        \aTr{\bEvs}{\bForm} \rimplies
        \aVal{=}\uReg{\aEv}
        \limplies \bForm[\uReg{\aEv}/\aReg]
      \end{math},    
    \item \label{read-tau-ind}
      if $\aEvs\neq\emptyset$ and $(\aEvs\cap\bEvs)=\emptyset$ then
      \begin{math}
        \aTr{\bEvs}{\bForm} \rimplies
        \PBR{\aVal{=}\uReg{\aEv} \lor \aLoc{=}\uReg{\aEv}} \limplies
        \bForm[\uReg{\aEv}/\aReg],
      \end{math}
    \item \label{read-tau-empty}
      if $\aEvs=\emptyset$ then
      \begin{math}
        (\forall\bReg)
        \aTr{\bEvs}{\bForm} \rimplies
        \bForm[\bReg/\aReg],
      \end{math}
    \end{enumerate}
  \item \label{read-term}
    if $\aEvs=\emptyset$ and $\amode\gemode\mACQ$ then $\aTerm \rimplies \FALSE$. 
  \end{enumerate}
  \medskip

  \noindent
  If $\aPS \in \sSTORE[\amode]{\aLoc}[\ascope]{\aExp}[\aThrd]$ then
  $(\exists\aVal\in\Val)$
  \begin{enumerate}[topsep=0pt,label=(\textsc{w}\arabic*),ref=\textsc{w}\arabic*]
  \item \label{write-E}
    if $\bEv,\aEv\in\aEvs$ then $\bEv=\aEv$,
  \item \label{write-lambda}
    $\labelingAct(\aEv) = \DW[\amode]{\aLoc}[\ascope]{\aVal}[\aThrd]$,
  \item \label{write-kappa}
    \begin{math}
      \labelingForm(\aEv) \rimplies
      \aExp{=}\aVal
    \end{math},    
  \item \label{write-tau}
    \begin{math}
      \aTr{\bEvs}{\bForm} \rimplies 
      \bForm
    \end{math},
  \item[] \stepcounter{enumi} \labeltext[\textsc{w}5]{}{write-term}
    \begin{enumerate}[leftmargin=0pt]
    \item \label{write-term-empty}
      if $\aEvs=\emptyset$ then $\aTerm \rimplies \FALSE$,
    \item \label{write-term-nonempty}
      if $\aEvs\neq\emptyset$ then $\aTerm \rimplies \aExp{=}\aVal$.
    \end{enumerate}
  \end{enumerate}

  \begin{align*}
    \begin{aligned}
      \sem[\aThrd]{\LET{\aReg}{\aExp}} &= \sLET{\aReg}{\aExp}
      \\
      \sem[\aThrd]{\PR[\amode]{\aLoc}{\aReg}} &= \sLOAD[\amode]{\aReg}[\ascope]{\aLoc}[\aThrd]
      \\
      \sem[\aThrd]{\PW[\amode]{\aLoc}{\aExp}} &= \sSTORE[\amode]{\aLoc}[\ascope]{\aExp}[\aThrd]
      \\
      \sem[\aThrd]{\PF[\ascope]{\fmode}} &= \sFENCE[\ascope]{\fmode}[\aThrd]
    \end{aligned}
    &&
    \begin{aligned}
      \sem[\aThrd]{\SKIP} &= \sSKIP 
      \\
      \sem[\aThrd]{\aCmd_1 \LPAR[\bThrd] \aCmd_2} &= \sLPAR{\sem[\bThrd]{\aCmd_1}}{\sem[\aThrd]{\aCmd_2}}
      \\
      \sem[\aThrd]{\aCmd_1 \SEMI \aCmd_2} &= \sSEMI{\sem[\aThrd]{\aCmd_1}}{\sem[\aThrd]{\aCmd_2}}
      \\
      \sem[\aThrd]{\IF{\aExp}\THEN\aCmd_1\ELSE\aCmd_2\FI} &= \sIFTHEN{\aExp{\neq}0}{\sem[\aThrd]{\aCmd_1}}{\sem[\aThrd]{\aCmd_2}}
    \end{aligned}
  \end{align*}
\end{definition}

In diagrams, we use different shapes and colors for arrows and events.  These
are included only to help the reader understand why order is included.  We
adopt the following conventions:
\begin{itemize}  
\item \makebox{$\aEv\xpo\bEv$} arises from control/data/address \emph{dependency} \eqref{seq-kappa},
\item \makebox{$\aEv\xsync\bEv$} arises from $\rsyncdelaysdef$ \eqref{seq-delay-sync},
\item \makebox{$\aEv\xwki\bEv$} arises from $\rcodelaysdef$ \eqref{seq-delay-co},
\item \makebox{$\aEv\xwk\bEv$} arises from \emph{blocking} \eqref{rf-block},
\item \makebox{$\aEv\xrf\bEv$} arises from \emph{matching} \eqref{rf-ledep} \eqref{rf-lesync}.
\end{itemize}

\begin{definition}
  \label{def:semaddr}
  Address Calculation.

  \noindent
  If $\aPS \in \sSTORE[\amode]{\cExp}[\ascope]{\aExp}[\aThrd]$ then
  $(\exists\cVal\in\Val)$
  $(\exists\aVal\in\Val)$
  \begin{multicols}{2}
    \begin{enumerate}[topsep=0pt,label=(\textsc{w}\arabic*),ref=\textsc{w}\arabic*]
    \item \label{write-E-addr}
      if $\bEv,\aEv\in\aEvs$ then $\bEv=\aEv$,
    \item \label{write-lambda-addr}
      $\labelingAct(\aEv) = \DW[\amode]{\REF{\cVal}}[\ascope]{\aVal}[\aThrd]$,
    \item \label{write-kappa-addr}
      \begin{math}
        \labelingForm(\aEv) \rimplies
        \cExp{=}\cVal
        \land \aExp{=}\aVal
      \end{math},      
      \stepcounter{enumi}
    \item[] \labeltext[\textsc{w}4]{}{write-tau-addr}
      \begin{enumerate}[leftmargin=0pt]
      \item \label{write-tau-dep-addr}
        if $\aEvs\neq\emptyset$ then 
        \begin{math}
          \aTr{\bEvs}{\bForm} \rimplies 
          (\cExp{=}\cVal)
          \limplies 
          \bForm[\aExp/\REF{\cVal}]
        \end{math},
      \item \label{write-tau-empty-addr}
        if $\aEvs=\emptyset$ then \\
        \begin{math}
          (\forall\dVal)
        \end{math}        
        \begin{math}
          \aTr{\bEvs}{\bForm} \rimplies 
          (\cExp{=}\dVal)
          \limplies 
          \bForm
          [\aExp/\REF{\dVal}]
        \end{math}  
      \end{enumerate}  
      \stepcounter{enumi}
    \item[] \labeltext[\textsc{w}5]{}{write-term-addr}
      \begin{enumerate}[leftmargin=0pt]
      \item \label{write-term-nonempty-addr}
        if $\aEvs\neq\emptyset$ then $\aTerm \rimplies \cExp{=}\cVal \land \aExp{=}\aVal$,
      \item \label{write-term-empty-addr}
        if $\aEvs=\emptyset$ then $\aTerm \rimplies \FALSE$.
      \end{enumerate}
    \end{enumerate}
  \end{multicols}

  \medskip
  \noindent
  If $\aPS \in \sLOAD[\amode]{\aReg}[\ascope]{\cExp}[\aThrd]$ then
  $(\exists\cVal\in\Val)$
  $(\exists\aVal\in\Val)$
  \begin{enumerate}[topsep=0pt,label=(\textsc{r}\arabic*),ref=\textsc{r}\arabic*]
  \item \label{read-E-addr}
    if $\bEv,\aEv\in\aEvs$ then $\bEv=\aEv$,
  \item \label{read-lambda-addr}
    $\labelingAct(\aEv) = \DR[\amode]{\REF{\cVal}}[\ascope]{\aVal}[\aThrd]$
  \item \label{read-kappa-addr}
    \begin{math}
      \labelingForm(\aEv) 
      \land \cExp{=}\cVal
    \end{math},
    \stepcounter{enumi}
  \item[] \labeltext[\textsc{r}4]{}{read-tau-addr}
    \begin{enumerate}[leftmargin=0pt]
    \item \label{read-tau-dep-addr}
      \begin{math}
        (\forall\aEv\in\aEvs\cap\bEvs)
      \end{math}
      \begin{math}
        \aTr{\bEvs}{\bForm} \rimplies
        (\cExp{=}\cVal\limplies\aVal{=}\uReg{\aEv})
        \limplies \bForm[\uReg{\aEv}/\aReg]
      \end{math},      
    \item \label{read-tau-ind-addr}
      \begin{math}
        (\forall\aEv\in\aEvs\setminus\bEvs)
      \end{math}
      \begin{math}
        \aTr{\bEvs}{\bForm} \rimplies
        \PBR{(\cExp{=}\cVal\limplies\aVal{=}\uReg{\aEv}) \lor (\cExp{=}\cVal\limplies\REF{\cVal}{=}\uReg{\aEv})}
        \limplies
        \bForm[\uReg{\aEv}/\aReg]
      \end{math},      
    \item \label{read-tau-empty-addr}
      \begin{math}
        (\forall\bReg)
      \end{math}
      if $\aEvs=\emptyset$ then 
      \begin{math}
        \aTr{\bEvs}{\bForm} \rimplies 
        \bForm[\bReg/\aReg],
      \end{math}  
    \end{enumerate}  
  \item \label{read-term-addr}
    if $\aEvs=\emptyset$ and $\amode\neq\mRLX$ then $\aTerm \rimplies \FALSE$. 
  \end{enumerate}
\end{definition}


\begin{definition}
  \label{def:semca}
  If-closure
  
  \noindent
  If $\aPS \in \sSTORE[\amode]{\aLoc}[\ascope]{\aExp}[\aThrd]$ then
  $(\exists\aVal:\aEvs\fun\Val)$
  $(\exists\cForm:\aEvs\fun\Formulae)$
  \begin{enumerate}[topsep=0pt,label=(\textsc{w}\arabic*),ref=\textsc{w}\arabic*]
  \item \label{write-E-ca}
    if $\cForm_\bEv\land\cForm_\aEv$ is satisfiable then $\bEv=\aEv$,
  \item \label{write-lambda-ca}
    $\labelingAct(\aEv) = \DW[\amode]{\aLoc}[\ascope]{\aVal_\aEv}[\aThrd]$,
  \item \label{write-kappa-ca}
    \begin{math}
      \labelingForm(\aEv) \rimplies
      \cForm_\aEv
      \land \aExp{=}\aVal_\aEv
    \end{math},
    
    
  \item \label{write-tau-ca}
    \begin{math}
      \aTr{\bEvs}{\bForm} \rimplies 
      \cForm_\aEv
      \limplies 
      \bForm[\aExp/\aLoc]
    \end{math},
  \item \label{write-term-ca}
    $\aTerm \rimplies \cForm_\aEv \limplies \aExp{=}\aVal_\aEv$,
  \end{enumerate}

  \medskip
  \noindent
  If $\aPS \in \sLOAD[\amode]{\aReg}[\ascope]{\aLoc}[\aThrd]$ then
  $(\exists\aVal:\aEvs\fun\Val)$
  $(\exists\cForm:\aEvs\fun\Formulae)$ 
  \begin{enumerate}[topsep=0pt,label=(\textsc{r}\arabic*),ref=\textsc{r}\arabic*]
  \item \label{read-E-ca}
    if $\cForm_\bEv\land\cForm_\aEv$ is satisfiable then $\bEv=\aEv$,
  \item \label{read-lambda-ca}
    $\labelingAct(\aEv) = \DR[\amode]{\aLoc}[\ascope]{\aVal_\aEv}[\aThrd]$
  \item \label{read-kappa-ca}
    \begin{math}
      \labelingForm(\aEv) \rimplies
      \cForm_\aEv
    \end{math},
    \stepcounter{enumi}
  \item[] \labeltext[\textsc{r}4]{}{read-tau-ca}
    \begin{enumerate}[leftmargin=0pt]
    \item \label{read-tau-dependent-ca}
      \begin{math}
        (\forall\aEv\in\aEvs\cap\bEvs)
      \end{math}
      \begin{math}
        \aTr{\bEvs}{\bForm} \rimplies
        \cForm_\aEv
        \limplies \aVal_\aEv{=}\uReg{\aEv}
        \limplies \bForm[\uReg{\aEv}/\aReg]
      \end{math},
      
    \item \label{read-tau-independent-ca}
      \begin{math}
        (\forall\aEv\in\aEvs\setminus\bEvs)
      \end{math}
      \begin{math}
        \aTr{\bEvs}{\bForm} \rimplies
        \cForm_\aEv 
        \limplies
        \PBR{\aVal_\aEv{=}\uReg{\aEv} \lor \aLoc{=}\uReg{\aEv}}
        \limplies
        \bForm[\uReg{\aEv}/\aReg]
      \end{math},
      
    \item \label{read-tau-empty-ca}
      \begin{math}
        (\forall\bReg)
      \end{math}
      \begin{math}
        \aTr{\bEvs}{\bForm} \rimplies 
        (\bigwedge_{\aEv\in\aEvs}\lnot\cForm_\aEv)
        \limplies 
        \bForm[\bReg/\aReg],
      \end{math}  
    \end{enumerate}  
  \item \label{read-term-ca}
    if $\aEvs=\emptyset$ and $\amode\neq\mRLX$ then $\aTerm \rimplies \FALSE$. 
  \end{enumerate}
\end{definition}
\begin{definition}
  \label{def:semcaaddr}
  Both.
  
  \noindent
  If $\aPS \in \sSTORE[\amode]{\cExp}[\ascope]{\aExp}[\aThrd]$ then
  $(\exists\cVal:\aEvs\fun\Val)$
  $(\exists\aVal:\aEvs\fun\Val)$
  $(\exists\cForm:\aEvs\fun\Formulae)$
  \begin{multicols}{2}
    \begin{enumerate}[topsep=0pt,label=(\textsc{w}\arabic*),ref=\textsc{w}\arabic*]
    \item \label{write-E-ca-addr}
      if $\cForm_\bEv\land\cForm_\aEv$ is satisfiable then $\bEv=\aEv$,
    \item \label{write-lambda-ca-addr}
      $\labelingAct(\aEv) = \DW[\amode]{\REF{\cVal}}[\ascope]{\aVal_\aEv}[\aThrd]$,
    \item \label{write-kappa-ca-addr}
      \begin{math}
        \labelingForm(\aEv) \rimplies
        \cForm_\aEv
        \land \cExp{=}\cVal_\aEv
        \land \aExp{=}\aVal_\aEv
      \end{math},      
      \stepcounter{enumi}
    \item[] \labeltext[\textsc{w}4]{}{write-tau-ca-addr}
      \begin{enumerate}[leftmargin=0pt]
      \item \label{write-tau-dep-ca-addr}
        \begin{math}
          \aTr{\bEvs}{\bForm} \rimplies 
          \cForm_\aEv
          \limplies (\cExp{=}\cVal)
          \limplies 
          \bForm[\aExp/\REF{\cVal}]
        \end{math},
      \item \label{write-tau-empty-ca-addr}
        \begin{math}
          (\forall\dVal)
        \end{math}
        \\
        \begin{math}
          \aTr{\bEvs}{\bForm} \rimplies 
          (\bigwedge_{\aEv\in\aEvs}\lnot\cForm_\aEv)
          \limplies (\cExp{=}\dVal)
          \limplies 
          \bForm
          [\aExp/\REF{\dVal}]
        \end{math}  
      \end{enumerate}  
      \stepcounter{enumi}
    \item[] \labeltext[\textsc{w}5]{}{write-term-ca-addr}
      \begin{enumerate}[leftmargin=0pt]
      \item \label{write-term-nonempty-ca-addr}
        $\aTerm \rimplies \cForm_\aEv \limplies \cExp{=}\cVal_\aEv \land \aExp{=}\aVal_\aEv$,
      \item \label{write-term-empty-ca-addr}
        $\aTerm \rimplies \bigvee_{\aEv\in\aEvs}\cForm_\aEv$.
      \end{enumerate}
    \end{enumerate}
  \end{multicols}

  \medskip
  \noindent
  If $\aPS \in \sLOAD[\amode]{\aReg}[\ascope]{\cExp}[\aThrd]$ then
  $(\exists\cVal:\aEvs\fun\Val)$
  $(\exists\aVal:\aEvs\fun\Val)$
  $(\exists\cForm:\aEvs\fun\Formulae)$ 
  \begin{enumerate}[topsep=0pt,label=(\textsc{r}\arabic*),ref=\textsc{r}\arabic*]
  \item \label{read-E-ca-addr}
    if $\cForm_\bEv\land\cForm_\aEv$ is satisfiable then $\bEv=\aEv$,
  \item \label{read-lambda-ca-addr}
    $\labelingAct(\aEv) = \DR[\amode]{\REF{\cVal}}[\ascope]{\aVal_\aEv}[\aThrd]$
  \item \label{read-kappa-ca-addr}
    \begin{math}
      \labelingForm(\aEv) \rimplies
      \cForm_\aEv
      \land \cExp{=}\cVal_\aEv
    \end{math},
    \stepcounter{enumi}
  \item[] \labeltext[\textsc{r}4]{}{read-tau-ca-addr}
    \begin{enumerate}[leftmargin=0pt]
    \item \label{read-tau-dependent-ca-addr}
      \begin{math}
        (\forall\aEv\in\aEvs\cap\bEvs)
      \end{math}
      \begin{math}
        \aTr{\bEvs}{\bForm} \rimplies
        \cForm_\aEv
        \limplies (\cExp{=}\cVal_\aEv\limplies\aVal_\aEv{=}\uReg{\aEv})
        \limplies \bForm[\uReg{\aEv}/\aReg]
      \end{math},      
    \item \label{read-tau-independent-ca-addr}
      \begin{math}
        (\forall\aEv\in\aEvs\setminus\bEvs)
      \end{math}
      \begin{math}
        \aTr{\bEvs}{\bForm} \rimplies
        \cForm_\aEv 
        \limplies
        \PBR{(\cExp{=}\cVal_\aEv\limplies\aVal_\aEv{=}\uReg{\aEv}) \lor (\cExp{=}\cVal_\aEv\limplies\REF{\cVal}{=}\uReg{\aEv})}
        \limplies
        \bForm[\uReg{\aEv}/\aReg]
      \end{math},      
    \item \label{read-tau-empty-ca-addr}
      \begin{math}
        (\forall\bReg)
      \end{math}
      \begin{math}
        \aTr{\bEvs}{\bForm} \rimplies 
        (\bigwedge_{\aEv\in\aEvs}\lnot\cForm_\aEv)
        \limplies 
        \bForm[\bReg/\aReg],
      \end{math}  
    \end{enumerate}  
  \item \label{read-term-ca-addr}
    if $\aEvs=\emptyset$ and $\amode\neq\mRLX$ then $\aTerm \rimplies \FALSE$. 
  \end{enumerate}
\end{definition}

\begin{definition}
  Let $\sLOADP{}{}$ be defined as for $\sLOAD{}{}$, adding the constraint:
  \begin{itemize}
  \item[{\labeltext[\textsc{r}4d]{(\textsc{r}4d)}{read-tau-rmw}}]
    if $(\aEvs\cap\bEvs)=\emptyset$ then
    \begin{math}
      \aTr{\bEvs}{\bForm} \rimplies
      \bForm.
    \end{math}
  \end{itemize}
  If $\aPS\in\mathit{FADD}(\aReg,\aLoc,\aExp,\amode_1,\amode_2)$ then
  $(\exists \aPS_1\in\sSEMI{\sLOADP[\amode_1]{\aReg}{\aLoc}}{\sSTORE[\amode_2]{\aLoc}{\aReg{+}\aExp}})$
  \begin{enumerate}[topsep=0pt,label=(\textsc{u}\arabic*),ref=\textsc{u}\arabic*]
  \item if $\labeling_1(\aEv)$ is a write then there is a read $\labeling_1(\bEv)$ such that 
    $\labelingForm(\aEv)\rimplies\labelingForm(\bEv)$ and
    $\bEv\xrmw\aEv$.
  \end{enumerate}
  If $\aPS\in\mathit{EXCHG}(\aReg,\aLoc,\aExp,\amode_1,\amode_2)$ then
  $(\exists \aPS_1\in\sSEMI{\sLOADP[\amode_1]{\aReg}{\aLoc}}{\sSTORE[\amode_2]{\aLoc}{\aExp}})$
  \begin{enumerate}[topsep=0pt,label=(\textsc{u}\arabic*),ref=\textsc{u}\arabic*]
  \item if $\labeling_1(\aEv)$ is a write then there is a read $\labeling_1(\bEv)$ such that 
    $\labelingForm(\aEv)\rimplies\labelingForm(\bEv)$ and
    $\bEv\xrmw\aEv$.
  \end{enumerate}
  If $\aPS\in\mathit{CAS}(\aReg,\aLoc,\aExp,\bExp,\amode_1,\amode_2)$ then
  $(\exists \aPS_1\in\sSEMI{\sLOADP[\amode_1]{\aReg}{\aLoc}}{\sIF{\aReg{=}\aExp}\sTHEN\sSTORE[\amode_2]{\aLoc}{\bExp}\sELSE\sSKIP\sFI})$
  \begin{enumerate}[topsep=0pt,label=(\textsc{u}\arabic*),ref=\textsc{u}\arabic*]
  \item if $\labeling_1(\aEv)$ is a write then there is a read $\labeling_1(\bEv)$ such that 
    $\labelingForm(\aEv)\rimplies\labelingForm(\bEv)$ and
    $\bEv\xrmw\aEv$.
  \end{enumerate}
\end{definition}
\begin{example}
  Consider \iriw{} with all $\mRA$ access:
  \begin{gather*}
    \PW[\mREL]{x}{1}
    \PAR
    \PR[\mACQ]{x}{r}\SEMI \PR[\mACQ]{y}{s}
    \PAR
    \PW[\mREL]{y}{1}
    \PAR
    \PR[\mACQ]{y}{r}\SEMI \PR[\mACQ]{x}{s}
    \taglabel{IRIW-acq-acq}
    \\
    \tag{\cmark\ppc,\cXI}
    \hbox{\begin{tikzinline}[node distance=1.5em]
        \raevent{wx1}{\DW[\mREL]{x}{1}}{}
        \raevent{rx1}{\DR[\mACQ]{x}{1}}{right=2.5em of wx1}
        \raevent{ry0}{\DR[\mACQ]{y}{0}}{right=of rx1}
        \raevent{wy1}{\DW[\mREL]{y}{1}}{right=2.5em of ry0}
        \raevent{ry1}{\DR[\mACQ]{y}{1}}{right=2.5em of wy1}
        \raevent{rx0}{\DR[\mACQ]{x}{0}}{right=of ry1}
        \sync{rx1}{ry0}
        \sync{ry1}{rx0}
        \rf{wx1}{rx1}
        \rf{wy1}{ry1}
        \wk[out=-165,in=-15]{rx0}{wx1}
        \wk{ry0}{wy1}
      \end{tikzinline}}
  \end{gather*}
  We allow this execution:
  \begin{gather*}
    \tag{$\ledep$}
    \hbox{\begin{tikzinline}[node distance=1.5em]
        \raevent{wx1}{\DW[\mREL]{x}{1}}{}
        \raevent{rx1}{\DR[\mACQ]{x}{1}}{right=2.5em of wx1}
        \raevent{ry0}{\DR[\mACQ]{y}{0}}{right=of rx1}
        \raevent{wy1}{\DW[\mREL]{y}{1}}{right=2.5em of ry0}
        \raevent{ry1}{\DR[\mACQ]{y}{1}}{right=2.5em of wy1}
        \raevent{rx0}{\DR[\mACQ]{x}{0}}{right=of ry1}
        \rf{wx1}{rx1}
        \rf{wy1}{ry1}
      \end{tikzinline}}
    \\
    \tag{$\lesync$}
    \hbox{\begin{tikzinline}[node distance=1.5em]
        \raevent{wx1}{\DW[\mREL]{x}{1}}{}
        \raevent{rx1}{\DR[\mACQ]{x}{1}}{right=2.5em of wx1}
        \raevent{ry0}{\DR[\mACQ]{y}{0}}{right=of rx1}
        \raevent{wy1}{\DW[\mREL]{y}{1}}{right=2.5em of ry0}
        \raevent{ry1}{\DR[\mACQ]{y}{1}}{right=2.5em of wy1}
        \raevent{rx0}{\DR[\mACQ]{x}{0}}{right=of ry1}
        \sync{rx1}{ry0}
        \sync{ry1}{rx0}
        \rf{wx1}{rx1}
        \rf{wy1}{ry1}
      \end{tikzinline}}
    \\
    \tag{$\leloc$}
    \hbox{\begin{tikzinline}[node distance=1.5em]
        \raevent{wx1}{\DW[\mREL]{x}{1}}{}
        \raevent{rx1}{\DR[\mACQ]{x}{1}}{right=2.5em of wx1}
        \raevent{ry0}{\DR[\mACQ]{y}{0}}{right=of rx1}
        \raevent{wy1}{\DW[\mREL]{y}{1}}{right=2.5em of ry0}
        \raevent{ry1}{\DR[\mACQ]{y}{1}}{right=2.5em of wy1}
        \raevent{rx0}{\DR[\mACQ]{x}{0}}{right=of ry1}
        \rf{wx1}{rx1}
        \rf{wy1}{ry1}
        \wk[out=-165,in=-15]{rx0}{wx1}
        \wk{ry0}{wy1}
      \end{tikzinline}}
  \end{gather*}
  \ref{IRIW-acq-sc}, is allowed by trailing-sync compilation to power
  \cite[\textsection 1]{DBLP:conf/pldi/LahavVKHD17}.
  \begin{gather*}
    \PW[\mSC]{x}{1}
    \PAR
    \PR[\mACQ]{x}{r}\SEMI \PR[\mSC]{y}{s}
    \PAR
    \PW[\mSC]{y}{1}
    \PAR
    \PR[\mACQ]{y}{r}\SEMI \PR[\mSC]{x}{s}
    \taglabel{IRIW-acq-sc}
    \\
    \tag{\cmark\ppc,\xmark\cXI}
    \hbox{\begin{tikzinline}[node distance=1.5em]
        \scevent{wx1}{\DW[\mSC]{x}{1}}{}
        \raevent{rx1}{\DR[\mACQ]{x}{1}}{right=2.5em of wx1}
        \scevent{ry0}{\DR[\mSC]{y}{0}}{right=of rx1}
        \scevent{wy1}{\DW[\mSC]{y}{1}}{right=2.5em of ry0}
        \raevent{ry1}{\DR[\mACQ]{y}{1}}{right=2.5em of wy1}
        \scevent{rx0}{\DR[\mSC]{x}{0}}{right=of ry1}
        \sync{rx1}{ry0}
        \sync{ry1}{rx0}
        \rf{wx1}{rx1}
        \rf{wy1}{ry1}
        \wk[out=-165,in=-15]{rx0}{wx1}
        \wk{ry0}{wy1}
      \end{tikzinline}}
  \end{gather*}
  To model this it is convenient that synchronization is not included in
  dependency order:
  \begin{itemize}
  \item add $\mSC$ bullet to def of $\leexists$ in \ref{rf-block},
  \item add SC access to $\rsyncdelaysdef$.
  \end{itemize}
  \begin{gather*}
    \tag{$\ledep$}
    \hbox{\begin{tikzinline}[node distance=1.5em]
        \scevent{wx1}{\DW[\mSC]{x}{1}}{}
        \raevent{rx1}{\DR[\mACQ]{x}{1}}{right=2.5em of wx1}
        \scevent{ry0}{\DR[\mSC]{y}{0}}{right=of rx1}
        \scevent{wy1}{\DW[\mSC]{y}{1}}{right=2.5em of ry0}
        \raevent{ry1}{\DR[\mACQ]{y}{1}}{right=2.5em of wy1}
        \scevent{rx0}{\DR[\mSC]{x}{0}}{right=of ry1}
        \rf{wx1}{rx1}
        \rf{wy1}{ry1}
        \wk[out=-165,in=-15]{rx0}{wx1}
        \wk{ry0}{wy1}
      \end{tikzinline}}    
    \\
    \tag{$\lesync$}
    \hbox{\begin{tikzinline}[node distance=1.5em]
        \scevent{wx1}{\DW[\mSC]{x}{1}}{}
        \raevent{rx1}{\DR[\mACQ]{x}{1}}{right=2.5em of wx1}
        \scevent{ry0}{\DR[\mSC]{y}{0}}{right=of rx1}
        \scevent{wy1}{\DW[\mSC]{y}{1}}{right=2.5em of ry0}
        \raevent{ry1}{\DR[\mACQ]{y}{1}}{right=2.5em of wy1}
        \scevent{rx0}{\DR[\mSC]{x}{0}}{right=of ry1}
        \sync{rx1}{ry0}
        \sync{ry1}{rx0}
        \rf{wx1}{rx1}
        \rf{wy1}{ry1}
      \end{tikzinline}}    
    \\
    \tag{$\leloc$}
    \hbox{\begin{tikzinline}[node distance=1.5em]
        \scevent{wx1}{\DW[\mSC]{x}{1}}{}
        \raevent{rx1}{\DR[\mACQ]{x}{1}}{right=2.5em of wx1}
        \scevent{ry0}{\DR[\mSC]{y}{0}}{right=of rx1}
        \scevent{wy1}{\DW[\mSC]{y}{1}}{right=2.5em of ry0}
        \raevent{ry1}{\DR[\mACQ]{y}{1}}{right=2.5em of wy1}
        \scevent{rx0}{\DR[\mSC]{x}{0}}{right=of ry1}
        \rf{wx1}{rx1}
        \rf{wy1}{ry1}
        \wk[out=-165,in=-15]{rx0}{wx1}
        \wk{ry0}{wy1}
      \end{tikzinline}}    
  \end{gather*}
  This correctly forbids the all $\mSC$ version:
  \begin{gather*}
    \PW[\mSC]{x}{1}
    \PAR
    \PR[\mSC]{x}{r}\SEMI \PR[\mSC]{y}{s}
    \PAR
    \PW[\mSC]{y}{1}
    \PAR
    \PR[\mSC]{y}{r}\SEMI \PR[\mSC]{x}{s}
    \taglabel{IRIW-sc-sc}
    \\
    \tag{$\ledep$}
    \hbox{\begin{tikzinline}[node distance=1.5em]
        \scevent{wx1}{\DW[\mSC]{x}{1}}{}
        \scevent{rx1}{\DR[\mSC]{x}{1}}{right=2.5em of wx1}
        \scevent{ry0}{\DR[\mSC]{y}{0}}{right=of rx1}
        \scevent{wy1}{\DW[\mSC]{y}{1}}{right=2.5em of ry0}
        \scevent{ry1}{\DR[\mSC]{y}{1}}{right=2.5em of wy1}
        \scevent{rx0}{\DR[\mSC]{x}{0}}{right=of ry1}
        \sync{rx1}{ry0}
        \sync{ry1}{rx0}
        \rf{wx1}{rx1}
        \rf{wy1}{ry1}
        \wk[out=-165,in=-15]{rx0}{wx1}
        \wk{ry0}{wy1}
      \end{tikzinline}}
  \end{gather*}
  
\end{example}  

\begin{example}
  Thin air with an SC antidependency:
  \begin{gather*}
    \PW[\mSC]{y}{\PR{x}{}}
    \PAR \PW[\mSC]{y}{2}
    \PAR \PW{x}{\PR{y}{}{-}1}
    \\
    \tag{$\ledep$}
    \hbox{\begin{tikzinline}[node distance=1.5em]
        \event{a}{\DR{x}{1}}{}
        \scevent{b}{\DW[\mSC]{y}{1}}{right=of a}
        \scevent{c}{\DW[\mSC]{y}{2}}{right=2.5em of b}
        \event{d}{\DR{y}{2}}{right=2.5em of c}
        \event{e}{\DW{x}{1}}{right=of d}
        \po{a}{b}
        \wk{b}{c}
        \rf{c}{d}
        \po{d}{e}
        \rf[out=-165,in=-15]{e}{a}
      \end{tikzinline}}
  \end{gather*}
\end{example}

\subsection{Fulfillment}
[This is old.]



\begin{definition}
  \label{def:fulfilled}
  Define $\leexists$ as follows.
  \begin{align*}
    \bEv\leexists\aEv &\textwhen                      
    \begin{cases}
      \bEv\leloc\aEv &\text{if}\; \bEv \;\text{is morally strong with}\; \aEv
      \\
      \aEv\not\ltloc\bEv &\text{otherwise}
    \end{cases}
  \end{align*}    

  

  A read event $\aEv$ is \emph{strongly fulfilled} if there is a
  $\bEv\xrfx\aEv$ and 
  \begin{center}
    for any $\cEv$ that can block $\aEv$, either $\cEv\leloc\bEv$ or
    $\aEv\leloc\cEv$.
  \end{center}


  A read event $\aEv$ is \emph{weakly fulfilled} if there is a
  $\bEv\xrfx\aEv$ and 
  \begin{center}
    for any $\cEv$ that can block $\aEv$, either $\cEv\leexists\bEv$ or
    $\aEv\leexists\cEv$.
  \end{center}

\end{definition}





If all accesses are morally strong with each other, weak fulfillment
degenerates to
\begin{center}
  $\forall\labelingAct(\cEv)=\DWP[]{x}{}$ either
  $\cEv \leloc \bEv$ or $\aEv \leloc \cEv$
\end{center}

If no accesses are morally strong with each other, weak fulfillment
degenerates to
\begin{center}
  $\not\mkern-5mu\exists\labelingAct(\cEv)=\DWP[]{x}{}$ 
  both $\bEv \ltloc \cEv$ and $\cEv \ltloc \aEv$
\end{center}

Note that the difference between strong and weak fulfillment is limited to $\leloc$.
We sometimes write $\lelocstrong$ for strong fulfillment and
$\lelocweak$ for weak fulfillment.


\endinput

\subsection{Generalized access modes}
This is an option.

\begin{displaymath}
  \begin{tikzpicture}
    \node (wk)  at (-1, 0) {$\mathstrut\mWK$};
    \node (rlx) at (0, 0) {$\mathstrut\mRLX$};
    \node (ra)  at (1, 0) {$\mathstrut\mRA$};
    \node (sc)  at (2, 0) {$\mathstrut\mSC$};
    \node (fsc) at (3, 0) {$\mathstrut\fSC$};
    \node (rel) at (2, -0.3) {$\mathstrut\fREL$};
    \node (acq) at (2,  0.3) {$\mathstrut\fACQ$};
    \draw[->](wk)to(rlx);
    \draw[->](rlx)to(ra);
    \draw[->](ra)to(sc);
    \draw[->](ra)to(rel);
    \draw[->](ra)to(acq);
    \draw[->](sc)to(fsc);
    \draw[->](rel)to(fsc);
    \draw[->](acq)to(fsc);
  \end{tikzpicture}
\end{displaymath}
\begin{align*}
  \fmerge{\DF{\amode}}{\DR[\bmode]{\aLoc}{\aVal}}
  = \fmerge{\DR[\amode]{\aLoc}{\aVal}}{\DF{\bmode}}
  &= \{ \DR[\amode\lubmode\bmode]{\aLoc}{\aVal} \}
  \\
  \fmerge{\DF{\amode}}{\DW[\bmode]{\aLoc}{\bVal}}
  = \fmerge{\DW[\amode]{\aLoc}{\bVal}}{\DF{\bmode}}
  &= \{ \DW[\amode\lubmode\bmode]{\aLoc}{\bVal} \}
\end{align*}
\begin{scope}
  \begin{align*}
    {\reorderra}
    &=
    \{(\DW[\amode]{}{}, \DR[\bmode]{}{}) \mid \amode\not\gemode\mSC \lor \bmode\not\gemode\mSC\}
    \cup\{(\DW[\amode]{}{}, \DW[\mRLX]{}{} \mid \amode\not\gemode\fREL \}
    \\&
    \cup\{(\DR[\amode]{}{}, \DW[\bmode]{}{}) \mid \amode=\mRLX \land \bmode=\mRLX\}
    \cup\{(\DR[\mRLX]{}{},  \DR[\bmode]{}{}) \mid \bmode\not\gemode\fACQ \}
    \\&
    \cup\{(\DF{\fREL},      \DF{\fACQ}    ) \}
    \cup\{(\DF{\fREL},      \DR[\bmode]{}{}) \}%\mid \amode=\fREL \}
    \cup\{(\DW[\amode]{}{}, \DF{\fACQ}     ) \}%\mid \bmode=\fACQ \}
  \end{align*}  
\end{scope}
In terms of reordering, $\DF{\fREL}$ is the same as $\DW[\fREL]{}{}$, and
$\DF{\fACQ}$ is the same as $\DR[\fACQ]{}{}$.
\begin{center}
  \setlength{\tabcolsep}{4pt}
  \begin{tabular}{c|ccccc|ccccc}
    &  \multicolumn{9}{|c}{$2^{\text{nd}}$} \\
    \hline
    $1^{\text{st}}$
    & $\DR[\mRLX]{}{}$  & $\DR[\mACQ]{}{}$ & $\DR[\fACQ]{}{}$& $\DR[\mSC]{}{}$  & $\DR[\fSC]{}{}$ & $\DW[\mRLX]{}{}$& $\DW[\mREL]{}{}$ & $\DW[\fREL]{}{}$& $\DW[\mSC]{}{}$& $\DW[\fSC]{}{}$\\% & $\DF{\fREL}$&$\DF{\fACQ}$ &$\DF{\fSC}$\\
    \hline                                                                                                                                                                                                                            
    $\DR[\mRLX]{}{}$ & \cmark            & \cmark          & \xmark          & \cmark           & \xmark           & \cmark          & \xmark          & \xmark          & \xmark         & \xmark         \\%  & \xmark      &\xmark       & \xmark    \\
    $\DR[\mACQ]{}{}$  & \xmark            & \xmark          & \xmark          & \xmark           & \xmark           & \xmark          & \xmark          & \xmark          & \xmark         & \xmark         \\%  & \xmark      &\xmark       & \xmark    \\
    $\DR[\fACQ]{}{}$ & \xmark            & \xmark          & \xmark          & \xmark           & \xmark           & \xmark          & \xmark          & \xmark          & \xmark         & \xmark         \\%  & \xmark      &\xmark       & \xmark    \\
    $\DR[\mSC]{}{}$  & \xmark            & \xmark          & \xmark          & \xmark           & \xmark           & \xmark          & \xmark          & \xmark          & \xmark         & \xmark         \\%  & \xmark      &\xmark       & \xmark    \\
    $\DR[\fSC]{}{}$ & \xmark            & \xmark          & \xmark          & \xmark           & \xmark           & \xmark          & \xmark          & \xmark          & \xmark         & \xmark         \\%  & \xmark      &\xmark       & \xmark    \\
    \hline                                                                                                                                                                                                                           
    $\DW[\mRLX]{}{}$ & \cmark            & \cmark          & \cmark          & \cmark           & \xmark           & \cmark          & \xmark          & \xmark          & \xmark         & \xmark         \\%  & \xmark      &\cmark       & \xmark    \\
    $\DW[\mREL]{}{}$  & \cmark            & \cmark          & \cmark          & \cmark           & \xmark           & \cmark          & \xmark          & \xmark          & \xmark         & \xmark         \\%  & \xmark      &\cmark       & \xmark    \\
    $\DW[\fREL]{}{}$ & \cmark            & \cmark          & \cmark          & \cmark           & \xmark           & \xmark          & \xmark          & \xmark          & \xmark         & \xmark         \\%  & \xmark      &\cmark       & \xmark    \\
    $\DW[\mSC]{}{}$  & \cmark            & \cmark          & \cmark          & \xmark           & \xmark           & \cmark          & \xmark          & \xmark          & \xmark         & \xmark         \\%  & \xmark      &\cmark       & \xmark    \\
    $\DW[\fSC]{}{}$ & \xmark            & \xmark          & \xmark          & \xmark           & \xmark           & \xmark          & \xmark          & \xmark          & \xmark         & \xmark         \\%  & \xmark      &\cmark       & \xmark    \\
  \end{tabular}
\end{center}

