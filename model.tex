\section{Sync examples}

The first of these is seen in hardware.  All are allowed by \PTX.
Showing $\rrfx$ that is not included in the order using a dashed arrow.
\begin{gather*}
  {
    \PW{x}{1}
    \SEMI
    \PW[\mREL]{y}{1}
  }\PAR{
    \PR[\mACQ]{y}{r}
    \SEMI
    \PW{z}[\sSYS]{r}
  }\LPAR[\bScp]{
    \PR[\mACQ]{z}[\sSYS]{r}
    \SEMI
    \PR{x}{s}
  }
  \\
  \tag{$\lesync$}
  \hbox{\begin{tikzinline}[node distance=1.5em]
      \event{a1}{\DW{x}{1}[\aScp]}{}
      \event{a2}{\DW[\mREL]{y}{1}[\aScp]}{right=of a1}
      \event{b1}{\DR[\mACQ]{y}{1}[\aScp]}{right=3em of a2}
      \event{b2}{\DW{z}[\sSYS]{1}[\aScp]}{right=of b1}
      \event{c1}{\DR[\mACQ]{z}[\sSYS]{1}[\bScp]}{right=3em of b2}
      \event{c2}{\DR{x}{0}[\bScp]}{right=of c1}
      \sync{a1}{a2}
      \sync{b1}{b2}
      \sync{c1}{c2}
      \rf{a2}{b1}
      \rfint{b2}{c1}
      %\wk[out=-165,in=-15]{c2}{a1}
    \end{tikzinline}}
\end{gather*}

\begin{gather*}
  {
    \PW{x}{1}
    \SEMI
    \PW[\mREL]{y}{1}
  }\PAR{
    \PR[\mACQ]{y}{r}
    \SEMI
    \PW{z}{r}
  }\PAR{
    \PR[\mACQ]{z}{r}
    \SEMI
    \PR{x}{s}
  }
  % \\
  % \tag{$\ledep$}
  % \hbox{\begin{tikzinline}[node distance=1.5em]
  %     \event{a1}{\DW{x}{1}}{}
  %     \event{a2}{\DW[\mREL]{y}{1}}{right=of a1}
  %     \event{b1}{\DR[\mACQ]{y}{1}}{right=3em of a2}
  %     \event{b2}{\DW{z}{1}}{right=of b1}
  %     \event{c1}{\DR[\mACQ]{z}{1}}{right=3em of b2}
  %     \event{c2}{\DR{x}{0}}{right=of c1}
  %     %\sync{a1}{a2}
  %     \po{b1}{b2}
  %     %\sync{c1}{c2}
  %     \rf{a2}{b1}
  %     \rf{b2}{c1}
  %     %\wk[out=-165,in=-15]{c2}{a1}
  %   \end{tikzinline}}
  \\
  \tag{$\lesync$}
  \hbox{\begin{tikzinline}[node distance=1.5em]
      \event{a1}{\DW{x}{1}}{}
      \event{a2}{\DW[\mREL]{y}{1}}{right=of a1}
      \event{b1}{\DR[\mACQ]{y}{1}}{right=3em of a2}
      \event{b2}{\DW{z}{1}}{right=of b1}
      \event{c1}{\DR[\mACQ]{z}{1}}{right=3em of b2}
      \event{c2}{\DR{x}{0}}{right=of c1}
      \sync{a1}{a2}
      \sync{b1}{b2}
      \sync{c1}{c2}
      \rf{a2}{b1}
      \rfint{b2}{c1}
      %\wk[out=-165,in=-15]{c2}{a1}
    \end{tikzinline}}
  % \\
  % \tag{$\leloc$}
  % \hbox{\begin{tikzinline}[node distance=1.5em]
  %     \event{a1}{\DW{x}{1}}{}
  %     \event{a2}{\DW[\mREL]{y}{1}}{right=of a1}
  %     \event{b1}{\DR[\mACQ]{y}{1}}{right=3em of a2}
  %     \event{b2}{\DW{z}{1}}{right=of b1}
  %     \event{c1}{\DR[\mACQ]{z}{1}}{right=3em of b2}
  %     \event{c2}{\DR{x}{0}}{right=of c1}
  %     %\sync{a1}{a2}
  %     %\sync{b1}{b2}
  %     %\sync{c1}{c2}
  %     \rf{a2}{b1}
  %     \rf{b2}{c1}
  %     \wk[out=-165,in=-15]{c2}{a1}
  %   \end{tikzinline}}
\end{gather*}

\begin{gather*}
  {
    \PW{x}{1}
    \SEMI
    \PW[\mREL]{y}{1}
  }\PAR{
    \PR{y}{r}
    \SEMI
    \PW[\mREL]{z}{r}
  }\PAR{
    \PR[\mACQ]{z}{r}
    \SEMI
    \PR{x}{s}
  }
  \\
  \tag{$\lesync$}
  \hbox{\begin{tikzinline}[node distance=1.5em]
      \event{a1}{\DW{x}{1}}{}
      \event{a2}{\DW[\mREL]{y}{1}}{right=of a1}
      \event{b1}{\DR{y}{1}}{right=3em of a2}
      \event{b2}{\DW[\mREL]{z}{1}}{right=of b1}
      \event{c1}{\DR[\mACQ]{z}{1}}{right=3em of b2}
      \event{c2}{\DR{x}{0}}{right=of c1}
      \sync{a1}{a2}
      \sync{b1}{b2}
      \sync{c1}{c2}
      \rfint{a2}{b1}
      \rf{b2}{c1}
      %\wk[out=-165,in=-15]{c2}{a1}
    \end{tikzinline}}
\end{gather*}

To get publication using fences we need an additional closure property for
$\rrfx$ on sync order:
\begin{gather*}
  {
    \PW{x}{1}
    \SEMI
    \PF{\fREL}
    \SEMI
    \PW{y}{1}
  }\PAR{
    \PR{y}{r}
    \SEMI
    \PF{\fACQ}
    \SEMI
    \PR{x}{s}
  }
  \\
  \tag{$\lesync$}
  \hbox{\begin{tikzinline}[node distance=1.5em]
      \event{a1}{\DW{x}{1}}{}
      \event{a2}{\DF{\fREL}}{right=of a1}
      \event{a3}{\DW{y}{1}}{right=of a2}
      \event{b1}{\DR{y}{1}}{right=3em of a3}
      \event{b2}{\DF{\fACQ}}{right=of b1}
      \event{b3}{\DR{x}{0}}{right=of b2}
      \sync{a1}{a2}
      \sync{a2}{a3}
      \sync{b1}{b2}
      \sync{b2}{b3}
      \rfint{a3}{b1}
      %\wk[out=-165,in=-15]{b3}{a1}
    \end{tikzinline}}
\end{gather*}
Previous def of candidate requires:
\begin{itemize}
\item[(\ref{cand-lesync-rf})]
  if $\bEv\xrfx\aEv$ and $\labeling(\bEv) \rsmatches \labeling(\aEv)$ then $\bEv \lesync \aEv$.
\end{itemize}
This is not good enough for fences.
A possible fix is the following closure condition:
\begin{itemize}
\item[(\ref{cand-lesync-rf}$'$)]
  if $\bEv'\lesync\bEv\xrfx\aEv\lesync\aEv'$ and $\labeling(\bEv') \rsmatches \labeling(\aEv')$ then $\bEv' \lesync \aEv'$.
\end{itemize}
With that we have the following, using $\xliftrf$ for edges induced by closure
when $\bEv'\neq\bEv$ or $\aEv'\neq\aEv$:
\begin{gather*}
  {
    \PW{x}{1}
    \SEMI
    \PF{\fREL}
    \SEMI
    \PW{y}{1}
  }\PAR{
    \PR{y}{r}
    \SEMI
    \PF{\fACQ}
    \SEMI
    \PR{x}{s}
  }
  \\
  \tag{$\lesync$}
  \hbox{\begin{tikzinline}[node distance=1.5em]
      \event{a1}{\DW{x}{1}}{}
      \event{a2}{\DF{\fREL}}{right=of a1}
      \event{a3}{\DW{y}{1}}{right=of a2}
      \event{b1}{\DR{y}{1}}{right=3em of a3}
      \event{b2}{\DF{\fACQ}}{right=of b1}
      \event{b3}{\DR{x}{0}}{right=of b2}
      \sync{a1}{a2}
      \sync{a2}{a3}
      \sync{b1}{b2}
      \sync{b2}{b3}
      \rfint{a3}{b1}
      \liftrf[out=-15,in=-165]{a2}{b2}
      %\wk[out=-165,in=-15]{b3}{a1}
    \end{tikzinline}}
\end{gather*}
This seems to work for the above examples, but it could be too strong in general.
\begin{itemize}
\item One possibility is to restrict to preceding and following things in the
  same thread:
  \begin{itemize}
  \item[(\ref{cand-lesync-rf}$''$)]
    if $\bEv'\lesyncpo\bEv\xrfx\aEv\lesyncpo\aEv'$ and $\labeling(\bEv') \rsmatches \labeling(\aEv')$ then $\bEv' \lesync \aEv'$.
  \end{itemize}
  where $\lesyncpo$ is the obvious restriction of $\lesync$ to actions on the
  same thread.
\item With either (\ref{cand-lesync-rf}$'$) or (\ref{cand-lesync-rf}$''$) is it too strong to require $\lesync$ that be
  transitive?   In particular:
  \begin{itemize}
  \item if we restrict to $\lesyncpo$, the closure condition
    (\ref{cand-lesync-rf}$''$) could add order between actions on the same thread
    via cross-thread reads.
  \item How does transitivity interact with scopes?
  \end{itemize}
\end{itemize}
Anton proposes:
\begin{itemize}
\item[(\ref{pom-rmw-lesync}$'$)]
  if $\bEv\xrmw\aEv$ then %$\bEv \lesync \aEv$ and
  $\bEv \leloc \aEv$,    
\item[(\ref{cand-lesync-rf}$'''$)]
  if $\bEv'\lesync\bEv\mathrel{(\xrfx;(\xrmw;\xrfx)^{*})}\aEv\lesync\aEv'$ and $\labeling(\bEv') \rsmatches \labeling(\aEv')$ then $\bEv' \lesync \aEv'$.
\end{itemize}

The following behavior is allowed by Arm, IMM, and C11, but forbidden by \PTX.
\PTX{} forbids it since acquire reads work as fences for po-previous reads from
the same location (symmetrically to release writes for po-latter writes to
the same location in \IMM, \cXI, and \PTX).
\begin{gather*}
  {
    \PW{x}{1}
    \SEMI
    \PW[\mREL]{y}{1}
  }\PAR{
    \PR{y}{r}
    \SEMI
    \PW{y}{2}
    \SEMI
    \PR[\mACQ]{y}{s}
     \SEMI
    \PR{x}{t}
  }
  \\
  \tag{$\lesync$}
  \hbox{\begin{tikzinline}[node distance=1.5em]
      \event{a}{\DW{x}{1}}{}
      \raevent{b}{\DW[\mREL]{y}{1}}{right=of a}
      \event{c}{\DR{y}{1}}{right=3em of b}
      \event{d}{\DW{y}{2}}{right=of c}
      \raevent{e}{\DR[\mACQ]{y}{2}}{right=of d}
      \event{f}{\DR{x}{0}}{right=of e}
      \sync{a}{b}
      \rfint{b}{c}
      \sync[out=15,in=165]{c}{e}
      %\wk{c}{d}
      \rfint{d}{e}
      \sync{e}{f}
      %\wk[out=-165,in=-15]{f}{a}
      \liftrf[out=-15,in=-165]{b}{e}
    \end{tikzinline}}
\end{gather*}
To allow this on for IMM, we need to drop
\begin{math}
  (\DR{\aLoc}{}, \DR[\gemode\mACQ]{\aLoc}{})
\end{math}
from $\rsyncdelays$.

The following is allowed by \cXI{}, but not \IMM{} or \PTX.
The goal here is to construct a cycle
$a\xrfx b \xhb c \xrfx d \xhb a$
where $\rrfx$  will be included in synch-relation.
In relational notation, the cycle has the following form:
\begin{displaymath}
  \PBRbig{{\rrmw} ; ({\rrfe} ; {\rrmw})^2 ; {\rppo} ; \Wclass[\mREL]; {\rrfe} ; \Rclass[\mACQ] ; {\rppo}}^2
\end{displaymath}
\begin{gather*}
  {
    \PR[\mACQ]{x}{r}
    \SEMI
    \PINC{y}{}
  }\PAR{
    \PINC{y}{}
  }\PAR{
    \PINC{y}{}
    \SEMI
    \PW[\mREL]{z}{1}
  }\PAR{
    \PR[\mACQ]{z}{s}
    \SEMI
    \PINC{w}{}
  }\PAR{
    \PINC{w}{}
  }\PAR{
    \PINC{w}{}
    \SEMI
    \PW[\mREL]{x}{1}
  }
  \\
  \hbox{\begin{tikzinline}[node distance=1.5em]
      \event{a1}{\DR[\mACQ]{x}{1}}{}
      \event{a2}{\DR{y}{0}}{below=of a1}
      \event{a3}{\DW{y}{1}}{below=of a2}
      \sync{a1}{a2}
      \rmw{a2}{a3}
      \event{b1}{\DR{y}{1}}{right=of a1}
      \event{b2}{\DW{y}{2}}{below=of b1}
      \rmw{b1}{b2}
      \event{c1}{\DR{y}{2}}{right=of b1}
      \event{c2}{\DW{y}{3}}{below=of c1}
      \event{c3}{\DW[\mREL]{z}{1}}{below=of c2}
      \rmw{c1}{c2}
      \sync{c2}{c3}
      \event{d1}{\DR[\mACQ]{z}{1}}{right=of c1}
      \event{d2}{\DR{w}{0}}{below=of d1}
      \event{d3}{\DW{w}{1}}{below=of d2}
      \sync{d1}{d2}
      \rmw{d2}{d3}
      \event{e1}{\DR{w}{1}}{right=of d1}
      \event{e2}{\DW{w}{2}}{below=of e1}
      \rmw{e1}{e2}
      \event{f1}{\DR{w}{2}}{right=of e1}
      \event{f2}{\DR{w}{3}}{below=of f1}
      \event{f3}{\DW[\mREL]{x}{1}}{below=of f2}
      \rmw{f1}{f2}
      \sync{f2}{f3}
      \rf{a3}{b1}
      \rf{b2}{c1}
      \rf{c3}{d1}
      \rf{d3}{e1}
      \rf{e2}{f1}
      \rf{f3}{a1}
    \end{tikzinline}}
\end{gather*}

\section{Relating IMM and PTX}
It looks like we cannot prove compilation correctness from IMM to PTX.
(In this email I assume that all threads are in the same CTA, so any relation is a morally strong one if it is applicable.)
The problem is in the LB-data-rel example:
\begin{comment}
a := [x]  || b := [y]
[y] := a  || [x]_rel := 1
\end{comment}
\begin{gather*}
  \PR{x}{r}\SEMI
  \PW{y}{r}
  \PAR
  \PR{y}{s}\SEMI
  \PW[\mREL]{x}{1}
  \\
  \hbox{\begin{tikzinline}[node distance=1.5em]
      \event{a}{\DR{x}{1}}{}
      \event{b}{\DW{y}{1}}{right=of a}
      \event{c}{\DR{y}{1}}{right=3em of b}
      \raevent{d}{\DW[\mREL]{x}{1}}{right=of c}
      \data{a}{b}
      \rfe{b}{c}
      \bob{c}{d}
      \rfe[out=-165,in=-15]{d}[above]{a}
    \end{tikzinline}}
  \\
  \tag{$\ledep$}  
  \hbox{\begin{tikzinline}[node distance=1.5em]
      \event{a}{\DR{x}{1}}{}
      \event{b}{\DW{y}{1}}{right=of a}
      \event{c}{\DR{y}{1}}{right=3em of b}
      \raevent{d}{\DW[\mREL]{x}{1}}{right=of c}
      \po{a}{b}
      \rf{b}{c}
      %\sync{c}{d}
      \rf[out=-165,in=-15]{d}{a}
    \end{tikzinline}}
  \\
  \tag{$\lesync$}  
  \hbox{\begin{tikzinline}[node distance=1.5em]
      \event{a}{\DR{x}{1}}{}
      \event{b}{\DW{y}{1}}{right=of a}
      \event{c}{\DR{y}{1}}{right=3em of b}
      \raevent{d}{\DW[\mREL]{x}{1}}{right=of c}
      %\po{a}{b}
      %\rf{b}{c}
      \sync{c}{d}
      %\rf[out=-165,in=-15]{d}{a}
    \end{tikzinline}}
  \\
  \tag{$\leloc$}  
  \hbox{\begin{tikzinline}[node distance=1.5em]
      \event{a}{\DR{x}{1}}{}
      \event{b}{\DW{y}{1}}{right=of a}
      \event{c}{\DR{y}{1}}{right=3em of b}
      \raevent{d}{\DW[\mREL]{x}{1}}{right=of c}
      %\po{a}{b}
      %\rf{b}{c}
      %\sync{c}{d}
      %\rf[out=-165,in=-15]{d}{a}
    \end{tikzinline}}
\end{gather*}

IMM forbids it, but PTX allows it. The point is that IMM mixes dependencies and release/acquire-induced po-order in its NoOOTA axiom,
whereas PTX doesn't --- release/acquire are only used to have coherence.

The problem is related to the one we have already discussed in the context of the C++ model -- if you don't have acquire reads in the
program, then you can erase release annotations from writes. In this regard, PTX is closer to PL memory models than to hardware ones.

AFAIU for the same reason we won't be able to show compilation correctness from the Pomset model to PTX even directly,
if the Pomset model mixes release/acquire induced order with dependencies in the same causality relation.

The previous example in the section shows that IMM's acquires are stronger
than PTX for this pattern.  The next example shows that acquiring reads in
PTX are stronger than in IMM for a different pattern.  Thus the acquires in
PTX and IMM are incomparable.

The following behavior is allowed by IMM and C11, but forbidden by PTX.  PTX
forbids it since acquire reads work as fences for po-previous reads from the
same location (symmetrically to release writes for po-latter writes to the
same location in IMM, C11, and PTX).
\begin{gather*}
  \PW{x}{1}\SEMI
  \PW[\mREL]{y}{1}
  \PAR
  \PR{y}{r}\SEMI
  \PW{y}{2}\SEMI
  \PR[\mACQ]{y}{s}\SEMI
  \PR{x}{t}
  \\
  \hbox{\begin{tikzinline}[node distance=1.5em]
      \event{a}{\DW{x}{1}}{}
      \raevent{b}{\DW[\mREL]{y}{1}}{right=of a}
      \event{c}{\DR{y}{1}}{right=3em of b}
      \event{d}{\DW{y}{2}}{right=of c}
      \raevent{e}{\DR[\mACQ]{y}{2}}{right=of d}
      \event{f}{\DR{x}{0}}{right=of e}
      \sync{a}{b}
      \rf{b}{c}
      \wk{c}{d}
      \rf{d}{e}
      \sync{e}{f}
      \wk[out=-165,in=-15]{f}{a}
    \end{tikzinline}}
\end{gather*}


\section{Thin Air}

Need $\ledep$ to prevent thin air on $\mRLX$:
\begin{gather*}
  \PW{y}{\PR{x}{}}\PAR
  \PW{x}{\PR{y}{}}
  \\
  \tag{$\ledep$}
  \hbox{\begin{tikzinline}[node distance=1.5em]
      \event{a}{\DR{x}{1}}{}
      \event{b}{\DW{y}{1}}{right=of a}
      \event{c}{\DR{y}{1}}{right=2.5em of b}
      \event{d}{\DW{x}{1}}{right=of c}
      \po{a}{b}
      \rf{b}{c}
      \po{c}{d}
      \rf[out=-165,in=-15]{d}{a}
    \end{tikzinline}}
  \\
  \tag{$\lesync$}
  \hbox{\begin{tikzinline}[node distance=1.5em]
      \event{a}{\DR{x}{1}}{}
      \event{b}{\DW{y}{1}}{right=of a}
      \event{c}{\DR{y}{1}}{right=2.5em of b}
      \event{d}{\DW{x}{1}}{right=of c}
      \po{a}{b}
      \po{c}{d}
    \end{tikzinline}}
  \\
  \tag{$\lelocstrong$}
  \hbox{\begin{tikzinline}[node distance=1.5em]
      \event{a}{\DR{x}{1}}{}
      \event{b}{\DW{y}{1}}{right=of a}
      \event{c}{\DR{y}{1}}{right=2.5em of b}
      \event{d}{\DW{x}{1}}{right=of c}
      \rf{b}{c}
      \rf[out=-165,in=-15]{d}{a}
    \end{tikzinline}}
\end{gather*}

\section{IMM Examples}

Disallowed by \IMM{}:
\begin{gather*}
  \taglabel{pub-rel-acq-coe}
  \PW{x}{2}\SEMI 
  \PW[\mREL]{y}{1} \PAR
  \PR[\mACQ]{y}{r}\SEMI
  \PW{x}{1}
  \\
  \tag{\xmark\IMM}
  \hbox{\begin{tikzinline}[node distance=1.5em]
      \event{a}{\DW{x}{2}}{}
      \raevent{b}{\DW[\mREL]{y}{1}}{right=of a}
      \raevent{c}{\DR[\mACQ]{y}{1}}{right=2.5em of b}
      \event{d}{\DW{x}{1}}{right=of c}
      \bob{a}{b}
      \rfe{b}{c}
      \bob{c}{d}
      \coe[out=-165,in=-15]{d}{a}
    \end{tikzinline}}
  \\
  \tag{${\ledep}={\lesync}$}
  \hbox{\begin{tikzinline}[node distance=1.5em]
      \event{a}{\DW{x}{2}}{}
      \raevent{b}{\DW[\mREL]{y}{1}}{right=of a}
      \raevent{c}{\DR[\mACQ]{y}{1}}{right=2.5em of b}
      \event{d}{\DW{x}{1}}{right=of c}
      \sync{a}{b}
      \rf{b}{c}
      \sync{c}{d}
      %\wk[out=-165,in=-15]{d}{a}
    \end{tikzinline}}
  \\
  \tag{$\lelocstrong$}
  \hbox{\begin{tikzinline}[node distance=1.5em]
      \event{a}{\DW{x}{2}}{}
      \raevent{b}{\DW[\mREL]{y}{1}}{right=of a}
      \raevent{c}{\DR[\mACQ]{y}{1}}{right=2.5em of b}
      \event{d}{\DW{x}{1}}{right=of c}
      %\sync{a}{b}
      \rf{b}{c}
      %\sync{c}{d}
      \po[out=15,in=165]{a}{d}
      %\wk[out=-165,in=-15]{d}{a}
    \end{tikzinline}}
\end{gather*}

Allowed by \IMM, but not by Power/ARMv7/ARMv8/TSO:
\begin{gather*}
  \taglabel{pub-rel-rlx-coe}
  \PW{x}{2}\SEMI 
  \PW[\mREL]{y}{1} \PAR
  \PR{y}{r}\SEMI
  \PW{x}{1}
  \\
  \tag{\cmark\IMM}
  \hbox{\begin{tikzinline}[node distance=1.5em]
      \event{a}{\DW{x}{2}}{}
      \raevent{b}{\DW[\mREL]{y}{1}}{right=of a}
      \event{c}{\DR{y}{1}}{right=2.5em of b}
      \event{d}{\DW{x}{1}}{right=of c}
      \bob{a}{b}
      \rfe{b}{c}
      \data{c}{d}
      \coe[out=-165,in=-15]{d}{a}
    \end{tikzinline}}
  \\
  \tag{$\ledep$}
  \hbox{\begin{tikzinline}[node distance=1.5em]
      \event{a}{\DW{x}{2}}{}
      \raevent{b}{\DW[\mREL]{y}{1}}{right=of a}
      \event{c}{\DR{y}{1}}{right=2.5em of b}
      \event{d}{\DW{x}{1}}{right=of c}
      \sync{a}{b}
      \rf{b}{c}
      \po{c}{d}
      %\wk[out=-165,in=-15]{d}{a}
    \end{tikzinline}}
  \\
  \tag{$\lesync$}
  \hbox{\begin{tikzinline}[node distance=1.5em]
      \event{a}{\DW{x}{2}}{}
      \raevent{b}{\DW[\mREL]{y}{1}}{right=of a}
      \event{c}{\DR{y}{1}}{right=2.5em of b}
      \event{d}{\DW{x}{1}}{right=of c}
      \sync{a}{b}
      %\rfe{b}{c}
      \po{c}{d}
      %\wk[out=-165,in=-15]{d}{a}
    \end{tikzinline}}
  \\
  \tag{$\lelocstrong$}
  \hbox{\begin{tikzinline}[node distance=1.5em]
      \event{a}{\DW{x}{2}}{}
      \raevent{b}{\DW[\mREL]{y}{1}}{right=of a}
      \event{c}{\DR{y}{1}}{right=2.5em of b}
      \event{d}{\DW{x}{1}}{right=of c}
      %\sync{a}{b}
      \rf{b}{c}
      %\po{c}{d}
      %\wk[out=-165,in=-15]{d}{a}
    \end{tikzinline}}
\end{gather*}


Example from talk:
\begin{gather*}
  \taglabel{arm7-weak}
  \PR{x}{r}\SEMI \PW{x}{1}
  \PAR
  \PW{y}{x} 
  \PAR
  \PW{x}{y} 
  \\[-1.2ex]
  \tag{$\ledep$}
  \hbox{\begin{tikzinline}[node distance=1.5em]
      \event{a}{\DR{x}{1}}{}
      \event{b}{d:\DW{x}{1}}{right=of a}
      %\wk{a}{b}
      \event{c}{\DR{x}{1}}{right=3em of b}
      \event{d}{\DW{y}{1}}{right=of c}
      \po{c}{d}
      \event{e}{\DR{y}{1}}{right=3em of d}
      \event{f}{e:\DW{x}{1}}{right=of e}
      \po{e}{f}
      \rf{b}{c}
      \rf{d}{e}
      \rf[out=172,in=8]{f}{a}
    \end{tikzinline}}
  \\
  \tag{$\lesync$}
  \hbox{\begin{tikzinline}[node distance=1.5em]
      \event{a}{\DR{x}{1}}{}
      \event{b}{d:\DW{x}{1}}{right=of a}
      %\wk{a}{b}
      \event{c}{\DR{x}{1}}{right=3em of b}
      \event{d}{\DW{y}{1}}{right=of c}
      \po{c}{d}
      \event{e}{\DR{y}{1}}{right=3em of d}
      \event{f}{e:\DW{x}{1}}{right=of e}
      \po{e}{f}
      %\rf{b}{c}
      %\rf{d}{e}
      %\rf[out=172,in=8]{f}{a}
    \end{tikzinline}}
  \\
  \tag{$\lelocstrong$}
  \hbox{\begin{tikzinline}[node distance=1.5em]
      \event{a}{\DR{x}{1}}{}
      \event{b}{d:\DW{x}{1}}{right=of a}
      \wk{a}{b}
      \event{c}{\DR{x}{1}}{right=3em of b}
      \event{d}{\DW{y}{1}}{right=of c}
      %\po{c}{d}
      \event{e}{\DR{y}{1}}{right=3em of d}
      \event{f}{e:\DW{x}{1}}{right=of e}
      % \po{e}{f}
      %\po[out=-15,in=-165]{c}{f}
      \rf{b}{c}
      \rf{d}{e}
      \rf[out=172,in=8]{f}{a}
    \end{tikzinline}}
  \\
  \tag{$\lelocweak$}
  \hbox{\begin{tikzinline}[node distance=1.5em]
      \event{a}{\DR{x}{1}}{}
      \event{b}{d:\DW{x}{1}}{right=of a}
      \wk{a}{b}
      \event{c}{\DR{x}{1}}{right=3em of b}
      \event{d}{\DW{y}{1}}{right=of c}
      %\po{c}{d}
      \event{e}{\DR{y}{1}}{right=3em of d}
      \event{f}{e:\DW{x}{1}}{right=of e}
      % \po{e}{f}
      %\po[out=-15,in=-165]{c}{f}
      %\rf{b}{c}
      %\rf{d}{e}
      %\rf[out=172,in=8]{f}{a}
    \end{tikzinline}}
\end{gather*}

\section{PTX Examples}
Based on \cite{DBLP:conf/asplos/LustigSG19,nvidia}.

In examples, all threads in different $\sCTA$s.

$(\DR{x}{0})$ must be forbidden.
Before fulfilling the read:
\begin{gather*}
  \taglabel[sys]{pub1}
  \PW[\mWK]{x}[\sCTA]{0}\SEMI 
  \PW[\mWK]{x}[\sCTA]{1}\SEMI
  \PW[\mREL]{y}{1} \PAR
  \PR[\mACQ]{y}{r}\SEMI
  \PR[\mWK]{x}[\sCTA]{s}
  \\
  \tag{${\ledep}={\lesync}$}
  \hbox{\begin{tikzinline}[node distance=1.5em]
      \event{wx0}{\DW[\mWK]{x}[\sCTA]{0}}{}
      \event{wx1}{\DW[\mWK]{x}[\sCTA]{1}}{right=of wx0}
      \raevent{wy1}{\DW[\mREL]{y}{1}}{right=of wx1}
      \raevent{ry1}{\DR[\mACQ]{y}{1}}{right=2.5em of wy1}
      \event{rx}{\DR[\mWK]{x}[\sCTA]{}}{right=of ry1}
      \sync[out=-15,in=-165]{wx0}{wy1}
      \sync{wx1}{wy1}
      \sync{ry1}{rx}
      \rf{wy1}{ry1}
    \end{tikzinline}}
  \\
  \tag{$\leloc$}
  \hbox{\begin{tikzinline}[node distance=1.5em]
      \event{wx0}{\DW[\mWK]{x}[\sCTA]{0}}{}
      \event{wx1}{\DW[\mWK]{x}[\sCTA]{1}}{right=of wx0}
      \raevent{wy1}{\DW[\mREL]{y}[\sSYS]{1}}{right=of wx1}
      \raevent{ry1}{\DR[\mACQ]{y}[\sSYS]{1}}{right=2.5em of wy1}
      \event{rx}{\DR[\mWK]{x}[\sCTA]{}}{right=of ry1}
      \rf{wy1}{ry1}
      \wki{wx0}{wx1}
      \rf[out=-15,in=-165]{wx1}{rx}
    \end{tikzinline}}
\end{gather*}
$(\DW{x}{1})\leexists(\DR{x}{})$ is required by \ref{cand-leloc-block}, enforcing publication.

$(\DR{x}{0})$ must be allowed:
\begin{gather*}
  \taglabel[cta]{pub1}
  \PW[\mWK]{x}[\sCTA]{0}\SEMI 
  \PW[\mWK]{x}[\sCTA]{1}\SEMI
  \PW[\mREL]{y}[\sCTA]{1} \PAR
  \PR[\mACQ]{y}[\sCTA]{r}\SEMI
  \PR[\mWK]{x}[\sCTA]{s}
  \\
  \tag{${\ledep}$}
  \hbox{\begin{tikzinline}[node distance=1.5em]
      \event{wx0}{\DW[\mWK]{x}[\sCTA]{0}}{}
      \event{wx1}{\DW[\mWK]{x}[\sCTA]{1}}{right=of wx0}
      \raevent{wy1}{\DW[\mREL]{y}[\sCTA]{1}}{right=of wx1}
      \raevent{ry1}{\DR[\mACQ]{y}[\sCTA]{1}}{right=2.5em of wy1}
      \event{rx}{\DR[\mWK]{x}[\sCTA]{}}{right=of ry1}
      \sync[out=-15,in=-165]{wx0}{wy1}
      \sync{wx1}{wy1}
      \sync{ry1}{rx}
      \rf{wy1}{ry1}
    \end{tikzinline}}
  \\
  \tag{${\lesync}$}
  \hbox{\begin{tikzinline}[node distance=1.5em]
      \event{wx0}{\DW[\mWK]{x}[\sCTA]{0}}{}
      \event{wx1}{\DW[\mWK]{x}[\sCTA]{1}}{right=of wx0}
      \raevent{wy1}{\DW[\mREL]{y}[\sCTA]{1}}{right=of wx1}
      \raevent{ry1}{\DR[\mACQ]{y}[\sCTA]{1}}{right=2.5em of wy1}
      \event{rx}{\DR[\mWK]{x}[\sCTA]{}}{right=of ry1}
      \sync[out=-15,in=-165]{wx0}{wy1}
      \sync{wx1}{wy1}
      \sync{ry1}{rx}
      % \rf{wy1}{ry1}
    \end{tikzinline}}
  \\
  \tag{$\leloc$}
  \hbox{\begin{tikzinline}[node distance=1.5em]
      \event{wx0}{\DW[\mWK]{x}[\sCTA]{0}}{}
      \event{wx1}{\DW[\mWK]{x}[\sCTA]{1}}{right=of wx0}
      \raevent{wy1}{\DW[\mREL]{y}[\sCTA]{1}}{right=of wx1}
      \raevent{ry1}{\DR[\mACQ]{y}[\sCTA]{1}}{right=2.5em of wy1}
      \event{rx}{\DR[\mWK]{x}[\sCTA]{}}{right=of ry1}
      % \rf{wy1}{ry1}
      \wki{wx0}{wx1}
      % \wk[out=-15,in=-165]{wx1}{rx}
    \end{tikzinline}}
\end{gather*}
We do not have $(\DW[\mREL]{y}{1})\lesync (\DR[\mACQ]{y}{1})$ since \ref{cand-lesync-rf} only
requires order for things that are morally strong.  

Another example that may be of interest (nothing morally strong).  Can this $(\DR{x}{0})$?
\begin{gather*}
  %\taglabel[cta]{pub1}
  \PW{x}{0} \SEMI
  \PW{x}{1} \PAR 
  \PW{y}{\PR{x}{}} \PAR
  \IF{\PR{y}{}}\THEN \PR{x}{r} \FI
\end{gather*}

\PTX{} allows TC16 for events that are not mutually strong (\ref{tc16wk}),
but disallows it when events are mutually strong (\ref{tc16sys}).  Note that
$\lesync$ imposes no requirements here.  Fulfillment imposes no order.  This
example shows that \ref{cand-leloc-block} cannot be strengthened to replace
$\leexists$ with $\leloc$.
\begin{gather*}
  \taglabel[wk]{tc16}
  \PR[\mWK]{x}[\sCTA]{r} \SEMI
  \PW[\mWK]{x}[\sCTA]{1}
  \PAR
  \PR[\mWK]{x}[\sCTA]{s} \SEMI
  \PW[\mWK]{x}[\sCTA]{2}
  \\
  \tag{$\ledep$}
  \hbox{\begin{tikzinline}[node distance=1.5em]
      \event{a1}{\DR[\mWK]{x}[\sCTA]{2}}{}
      \event{a2}{\DW[\mWK]{x}[\sCTA]{1}}{right=of a1}
      \event{b1}{\DR[\mWK]{x}[\sCTA]{1}}{right=3em of a2}
      \event{b2}{\DW[\mWK]{x}[\sCTA]{2}}{right=of b1}
      \rf{a2}{b1}
      \rf[out=-165,in=-15]{b2}{a1}
    \end{tikzinline}}
  \\
  \tag{$\lesync$}
  \hbox{\begin{tikzinline}[node distance=1.5em]
      \event{a1}{\DR[\mWK]{x}[\sCTA]{2}}{}
      \event{a2}{\DW[\mWK]{x}[\sCTA]{1}}{right=of a1}
      \event{b1}{\DR[\mWK]{x}[\sCTA]{1}}{right=3em of a2}
      \event{b2}{\DW[\mWK]{x}[\sCTA]{2}}{right=of b1}
    \end{tikzinline}}
  \\
  \tag{$\leloc$}
  \hbox{\begin{tikzinline}[node distance=1.5em]
      \event{a1}{\DR[\mWK]{x}[\sCTA]{2}}{}
      \event{a2}{\DW[\mWK]{x}[\sCTA]{1}}{right=of a1}
      \event{b1}{\DR[\mWK]{x}[\sCTA]{1}}{right=3em of a2}
      \event{b2}{\DW[\mWK]{x}[\sCTA]{2}}{right=of b1}
      \wki{a1}{a2}
      \wki{b1}{b2}
      % \rf{a2}{b1}
      % \rf[out=-165,in=-15]{b2}{a1}
    \end{tikzinline}}
\end{gather*}
\begin{gather*}
  \taglabel[sys]{tc16}
  \PR{x}{r} \SEMI \PW{x}{1}
  \PAR                                              
  \PR{x}{s} \SEMI \PW{x}{2}
  \\
  \tag{${\ledep}={\lesync}$}
  \hbox{\begin{tikzinline}[node distance=1.5em]
      \event{a1}{\DR{x}{2}}{}
      \event{a2}{\DW{x}{1}}{right=of a1}
      \event{b1}{\DR{x}{1}}{right=3em of a2}
      \event{b2}{\DW{x}{2}}{right=of b1}
      \rf{a2}{b1}
      \rf[out=-165,in=-15]{b2}{a1}
    \end{tikzinline}}
  \\
  \tag{$\leloc$}
  \hbox{\begin{tikzinline}[node distance=1.5em]
      \event{a1}{\DR{x}{2}}{}
      \event{a2}{\DW{x}{1}}{right=of a1}
      \event{b1}{\DR{x}{1}}{right=3em of a2}
      \event{b2}{\DW{x}{2}}{right=of b1}
      \wk{a1}{a2}
      \wk{b1}{b2}
      \rf{a2}{b1}
      \rf[out=-165,in=-15]{b2}{a1}
    \end{tikzinline}}
\end{gather*}

About Release-Acquire semantics.  Anton confirms that the following example
is allowed in C11, but disallowed in the \IMM{}.  It is apparently allowed in
C11 with the intention to allow releasing writes to be downgraded to relaxed
in the case that only fulfill relaxed reads.
\begin{gather*}
  \taglabel{LB-REL}
  \PR{x}{r} \SEMI \PW[\mREL]{y}{1}
  \PAR                                             
  \PR{y}{s} \SEMI \PW[\mREL]{x}{1}
  \\
  \tag{${\ledep}={\lesync}$}
  \hbox{\begin{tikzinline}[node distance=1.5em]
      \event{a1}{\DR{x}{1}}{}
      \raevent{a2}{\DW[\mREL]{y}{1}}{right=of a1}
      \event{b1}{\DR{y}{1}}{right=3em of a2}
      \raevent{b2}{\DW[\mREL]{x}{1}}{right=of b1}
      \rf{a2}{b1}
      \rf[out=-165,in=-15]{b2}{a1}
      \sync{a1}{a2}
      \sync{b1}{b2}
    \end{tikzinline}}
\end{gather*}

Another example from Anton.  This is allowed in PTX because it does not
include synchronization in the no-tar axiom, only in coherence and causality.
\begin{gather*}
  \taglabel{LB-data-rel}
  \PR{x}{r} \SEMI \PW{y}{r}
  \PAR                                             
  \PR{y}{s} \SEMI \PW[\mREL]{x}{1}
  \\
  \tag{${\ledep}={\lesync}$}
  \hbox{\begin{tikzinline}[node distance=1.5em]
      \event{a1}{\DR{x}{1}}{}
      \event{a2}{\DW{y}{1}}{right=of a1}
      \event{b1}{\DR{y}{1}}{right=3em of a2}
      \raevent{b2}{\DW[\mREL]{x}{1}}{right=of b1}
      \rf{a2}{b1}
      \rf[out=-165,in=-15]{b2}{a1}
      \po{a1}{a2}
      \sync{b1}{b2}
    \end{tikzinline}}
\end{gather*}


\section{RFI Examples}

Bad example:
\begin{gather*}
  \PEXCHG{x}{r}{2}\SEMI 
  \PR{x}{s}\SEMI
  \PW{y}{s{-}1} \PAR
  \PR{y}{r}\SEMI
  \PW{x}{r}
  \\
  \tag{\cmark\armeight}
  \hbox{\begin{tikzinline}[node distance=1.5em]
      \event{a}{\DR{x}{1}}{}
      \event{b}{\DW{x}{2}}{right=of a}
      \event{c}{\DR{x}{2}}{right=of b}
      \event{d}{\DW{y}{1}}{right=of c}
      \event{e}{\DR{y}{1}}{right=3em of d}
      \event{f}{\DW{x}{1}}{right=of e}
      \rmw{a}{b}
      \rfi{b}{c}
      \dob{c}{d}
      \rfe{d}{e}
      \dob{e}{f}
      \rfe[out=-165,in=-15]{f}{a}
    \end{tikzinline}}
  \\
  \tag{$\ledep$}
  \hbox{\begin{tikzinline}[node distance=1.5em]
      \event{a}{\DR{x}{1}}{}
      \event{b}{\DW{x}{2}}{right=of a}
      \event{c}{\DR{x}{2}}{right=of b}
      \event{d}{\DW{y}{1}}{right=of c}
      \event{e}{\DR{y}{1}}{right=3em of d}
      \event{f}{\DW{x}{1}}{right=of e}
      %\rmw{a}{b}
      \rf{b}{c}
      \po{c}{d}
      \rf{d}{e}
      \po{e}{f}
      \rf[out=-165,in=-15]{f}{a}
    \end{tikzinline}}
  \\
  \tag{${\lesync}$}
  \hbox{\begin{tikzinline}[node distance=1.5em]
      \event{a}{\DR{x}{1}}{}
      \event{b}{\DW{x}{2}}{right=of a}
      \event{c}{\DR{x}{2}}{right=of b}
      \event{d}{\DW{y}{1}}{right=of c}
      \event{e}{\DR{y}{1}}{right=3em of d}
      \event{f}{\DW{x}{1}}{right=of e}
      \rmw{a}{b}
      % \rf{b}{c}
      % \po{c}{d}
      % \rf{d}{e}
      % \po{e}{f}
      % \rf[out=-165,in=-15]{f}{a}
    \end{tikzinline}}
  \\
  \tag{$\leloc$}
  \hbox{\begin{tikzinline}[node distance=1.5em]
      \event{a}{\DR{x}{1}}{}
      \event{b}{\DW{x}{2}}{right=of a}
      \event{c}{\DR{x}{2}}{right=of b}
      \event{d}{\DW{y}{1}}{right=of c}
      \event{e}{\DR{y}{1}}{right=3em of d}
      \event{f}{\DW{x}{1}}{right=of e}
      \wki{a}{b}
      \rf{b}{c}
      % \po{c}{d}
      \rf{d}{e}
      % \po{e}{f}
      \rf[out=-165,in=-15]{f}{a}
      % \wk[out=15,in=165]{c}{f}
    \end{tikzinline}}
\end{gather*}
\begin{gather*}
  \PR{x}{r}\SEMI 
  \PW{x}{2}\SEMI
  \PR{x}{s}\SEMI
  \PW{y}{s{-}1} \PAR
  \PR{y}{r}\SEMI
  \PW{x}{r}
  \\
  \tag{${\ledep}$}
  \hbox{\begin{tikzinline}[node distance=1.5em]
      \event{a}{\DR{x}{1}}{}
      \event{b}{\DW{x}{2}}{right=of a}
      \event{c}{\DR{x}{2}}{right=of b}
      \event{d}{\DW{y}{1}}{right=of c}
      \event{e}{\DR{y}{1}}{right=3em of d}
      \event{f}{\DW{x}{1}}{right=of e}
      % \wk{a}{b}
      \rf{b}{c}
      \po{c}{d}
      \rf{d}{e}
      \po{e}{f}
      \rf[out=-165,in=-15]{f}{a}
    \end{tikzinline}}
  \\
  \tag{${\lesync}$}
  \hbox{\begin{tikzinline}[node distance=1.5em]
      \event{a}{\DR{x}{1}}{}
      \event{b}{\DW{x}{2}}{right=of a}
      \event{c}{\DR{x}{2}}{right=of b}
      \event{d}{\DW{y}{1}}{right=of c}
      \event{e}{\DR{y}{1}}{right=3em of d}
      \event{f}{\DW{x}{1}}{right=of e}
      % \wk{a}{b}
      % \rf{b}{c}
      % \po{c}{d}
      % \rf{d}{e}
      % \po{e}{f}
      % \rf[out=-165,in=-15]{f}{a}
    \end{tikzinline}}
  \\
  \tag{$\leloc$}
  \hbox{\begin{tikzinline}[node distance=1.5em]
      \event{a}{\DR{x}{1}}{}
      \event{b}{\DW{x}{2}}{right=of a}
      \event{c}{\DR{x}{2}}{right=of b}
      \event{d}{\DW{y}{1}}{right=of c}
      \event{e}{\DR{y}{1}}{right=3em of d}
      \event{f}{\DW{x}{1}}{right=of e}
      \wki{a}{b}
      \rf{b}{c}
      % \po{c}{d}
      \rf{d}{e}
      % \po{e}{f}
      \rf[out=-165,in=-15]{f}{a}
      % \wk[out=15,in=165]{c}{f}
    \end{tikzinline}}
\end{gather*}


Anton example 1 (Allowed by ARM) \texttt{[rfi-coe-coe]}
\begin{gather*}
  \taglabel{rfi-coe-coe}
  \PW{x}{2}\SEMI 
  \PR[\mACQ]{x}{r}\SEMI
  \PW{y}{1} \PAR
  \PW{y}{2}\SEMI
  \PW[\mREL]{x}{1}
  \\
  \tag{\cmark\armeight}
  \hbox{\begin{tikzinline}[node distance=1.5em]
      \event{a}{\DW{x}{2}}{}
      \raevent{b}{\DR[\mACQ]{x}{2}}{right=of a}
      \event{c}{\DW{y}{1}}{right=of b}
      \event{d}{\DW{y}{2}}{right=2.5em of c}
      \raevent{e}{\DW[\mREL]{x}{1}}{right=of d}
      \rfi{a}{b}
      \bob{b}{c}
      \coe{c}{d}
      \bob{d}{e}
      \coe[out=-165,in=-15]{e}{a}
    \end{tikzinline}}
  \\
  \tag{$\ledep$}
  \hbox{\begin{tikzinline}[node distance=1.5em]
      \event{a}{\DW{x}{2}}{}
      \raevent{b}{\DR[\mACQ]{x}{2}}{right=of a}
      \event{c}{\DW{y}{1}}{right=of b}
      \event{d}{\DW{y}{2}}{right=2.5em of c}
      \raevent{e}{\DW[\mREL]{x}{1}}{right=of d}
      \rf{a}{b}
      %\sync{b}{c}
      %\wk{c}{d}
      %\sync{d}{e}
      %\wk[out=-165,in=-15]{e}{a}
    \end{tikzinline}}
  \\
  \tag{$\lesync$}
  \hbox{\begin{tikzinline}[node distance=1.5em]
      \event{a}{\DW{x}{2}}{}
      \raevent{b}{\DR[\mACQ]{x}{2}}{right=of a}
      \event{c}{\DW{y}{1}}{right=of b}
      \event{d}{\DW{y}{2}}{right=2.5em of c}
      \raevent{e}{\DW[\mREL]{x}{1}}{right=of d}
      %\rf{a}{b}
      \sync{b}{c}
      %\wk{c}{d}
      \sync{d}{e}
      %\wk[out=-165,in=-15]{e}{a}
    \end{tikzinline}}
  \\
  \tag{$\leloc$}
  \hbox{\begin{tikzinline}[node distance=1.5em]
      \event{a}{\DW{x}{2}}{}
      \raevent{b}{\DR[\mACQ]{x}{2}}{right=of a}
      \event{c}{\DW{y}{1}}{right=of b}
      \event{d}{\DW{y}{2}}{right=2.5em of c}
      \raevent{e}{\DW[\mREL]{x}{1}}{right=of d}
      \rf{a}{b}
      %\sync{b}{c}
      \wk{c}{d}
      %\sync{d}{e}
      \wk[out=-165,in=-15]{e}{a}
    \end{tikzinline}}
\end{gather*}
Internal reads survive acquires \texttt{[rfi-acq-coe-coe]} (where SC read =
\texttt{LDAR})
\begin{gather*}
  \taglabel{rfi-acq-coe-coe}
  \PW{x}{2}\SEMI 
  \PR[\mSC]{z}{s}\SEMI
  \PR[\mSC]{x}{r}\SEMI
  \PW{y}{1} \PAR
  \PW{y}{2}\SEMI
  \PW[\mREL]{x}{1}
  \\
  \tag{\cmark\armeight}
  \hbox{\begin{tikzinline}[node distance=1.5em]
      \event{a}{\DW{x}{2}}{}
      \scevent{b0}{\DR[\mSC]{z}{0}}{right=of a}
      \scevent{b}{\DR[\mSC]{x}{2}}{right=of b0}
      \event{c}{\DW{y}{1}}{right=of b}
      \event{d}{\DW{y}{2}}{right=2.5em of c}
      \raevent{e}{\DW[\mREL]{x}{1}}{right=of d}
      \rfi[out=20,in=160]{a}{b}
      %\bob[out=20,in=160]{b0}{c}
      \bob{b0}{b}
      \bob{b}{c}
      \coe{c}{d}
      \bob{d}{e}
      \coe[out=-165,in=-15]{e}{a}
    \end{tikzinline}}
\end{gather*}
And release-acquire pairs \texttt{[rfi-ra-coe-coe]} (where acquiring read
= \texttt{LDAPR})
\begin{gather*}
  \taglabel{rfi-ra-coe-coe2}
  \PW{x}{2}\SEMI 
  \PW[\mREL]{w}{1}\SEMI
  \PR[\mACQ]{z}{s}\SEMI
  \PR[\mACQ]{x}{r}\SEMI
  \PW{y}{1}
  \\[-1ex]\PAR
  \PW{y}{2}\SEMI
  \PW[\mREL]{x}{1}
  \PAR
  \PR{w}{r}\SEMI
  \PW{z}{1}\SEMI
  \\
  \tag{\cmark\armeight}
  \hbox{\begin{tikzinline}[node distance=1.5em]
      \event{a}{\DW{x}{2}}{}
      \raevent{b1}{\DW[\mREL]{w}{1}}{right=of a}
      \raevent{b0}{\DR[\mACQ]{z}{1}}{right=of b1}
      \raevent{b}{\DR[\mACQ]{x}{2}}{right=of b0}
      \event{c}{\DW{y}{1}}{right=of b}
      \event{d}{\DW{y}{2}}{right=2.5em of c}
      \raevent{e}{\DW[\mREL]{x}{1}}{right=of d}
      \rfi[out=20,in=160]{a}{b}
      %\bob[out=20,in=160]{b0}{c}
      \bob{a}{b1}
      %\bob{b1}{b0}
      \bob{b0}{b}
      \bob{b}{c}
      \coe{c}{d}
      \bob{d}{e}
      \coe[out=-165,in=-15]{e}{a}
      \event{f1}{\DR{w}{1}}{below=of b1}
      \event{f0}{\DW{z}{1}}{below=of b0}
      %\data{f1}{f0}
      \rfe{b1}{f1}
      \rfe{f0}{b0}
    \end{tikzinline}}
\end{gather*}
% \begin{gather*}
%   \taglabel{rfi-ra-coe-coe}
%   \PW{x}{2}\SEMI 
%   \PW[\mREL]{w}{1}\SEMI
%   \PR[\mACQ]{z}{s}\SEMI
%   \PR[\mACQ]{x}{r}\SEMI
%   \PW{y}{1} \PAR
%   \PW{y}{2}\SEMI
%   \PW[\mREL]{x}{1}
%   \\
%   \tag{\cmark\armeight}
%   \hbox{\begin{tikzinline}[node distance=1.5em]
%       \event{a}{\DW{x}{2}}{}
%       \raevent{b1}{\DW[\mREL]{w}{1}}{right=of a}
%       \raevent{b0}{\DR[\mACQ]{z}{0}}{right=of b1}
%       \raevent{b}{\DR[\mACQ]{x}{2}}{right=of b0}
%       \event{c}{\DW{y}{1}}{right=of b}
%       \event{d}{\DW{y}{2}}{right=2.5em of c}
%       \raevent{e}{\DW[\mREL]{x}{1}}{right=of d}
%       \rfi[out=20,in=160]{a}{b}
%       %\bob[out=20,in=160]{b0}{c}
%       \bob{a}{b1}
%       %\bob{b1}{b0}
%       \bob{b0}{b}
%       \bob{b}{c}
%       \coe{c}{d}
%       \bob{d}{e}
%       \coe[out=-165,in=-15]{e}{a}
%     \end{tikzinline}}
% \end{gather*}
But not if either acquire is strengthened to SC (where SC read =
\texttt{LDAR}).  The execution is also disallowed if an external thread
places order between the $\mRA$ accesses:
\begin{gather*}
  \taglabel{rfi-ra-data-coe-coe}
  \PW{x}{2}\SEMI 
  \PW[\mREL]{w}{1}\SEMI
  \PR[\mACQ]{z}{s}\SEMI
  \PR[\mACQ]{x}{r}\SEMI
  \PW{y}{1}
  \\[-1ex]\PAR
  \PW{y}{2}\SEMI
  \PW[\mREL]{x}{1}
  \PAR
  \PR{w}{r}\SEMI
  \PW{z}{r}\SEMI
  \\
  \tag{\xmark\armeight}
  \hbox{\begin{tikzinline}[node distance=1.5em]
      \event{a}{\DW{x}{2}}{}
      \raevent{b1}{\DW[\mREL]{w}{1}}{right=of a}
      \raevent{b0}{\DR[\mACQ]{z}{1}}{right=of b1}
      \raevent{b}{\DR[\mACQ]{x}{2}}{right=of b0}
      \event{c}{\DW{y}{1}}{right=of b}
      \event{d}{\DW{y}{2}}{right=2.5em of c}
      \raevent{e}{\DW[\mREL]{x}{1}}{right=of d}
      \rfi[out=20,in=160]{a}{b}
      %\bob[out=20,in=160]{b0}{c}
      \bob{a}{b1}
      %\bob{b1}{b0}
      \bob{b0}{b}
      \bob{b}{c}
      \coe{c}{d}
      \bob{d}{e}
      \coe[out=-165,in=-15]{e}{a}
      \event{f1}{\DR{w}{1}}{below=of b1}
      \event{f0}{\DW{z}{1}}{below=of b0}
      \data{f1}{f0}
      \rfe{b1}{f1}
      \rfe{f0}{b0}
    \end{tikzinline}}
\end{gather*}

To allow this, weaken $\mRA$ to $\mRLX$ when read fulfilled by relaxed write
of same thread (don't need to allow this when the write is part of an \RMW{}).
\begin{gather*}
  \PW{x}{2}\SEMI 
  \PR[\mACQ]{x}{r}\SEMI
  \PW{y}{1} \PAR
  \PW{y}{2}\SEMI
  \PW[\mREL]{x}{1}
  \\
  \hbox{\begin{tikzinline}[node distance=1.5em]
      \event{a}{\DW{x}{2}}{}
      \event{b}{\DR{x}{2}}{right=of a}
      \event{c}{\DW{y}{1}}{right=of b}
      \event{d}{\DW{y}{2}}{right=2.5em of c}
      \raevent{e}{\DW[\mREL]{x}{1}}{right=of d}
      \rf{a}{b}
      %\sync{b}{c}
      \wk{c}{d}
      \sync{d}{e}
      \wk[out=-165,in=-15]{e}{a}
    \end{tikzinline}}
\end{gather*}

RF variant \texttt{[rfi-rfe-coe]}:
\begin{gather*}
  \taglabel{rfi-rfe-coe}
  \PW{x}{2}\SEMI 
  \PR[\mACQ]{x}{r}\SEMI
  \PW{y}{1} \PAR
  \PR{y}{s}\SEMI
  \PW[\mREL]{x}{1}
  \\
  \tag{\cmark\armeight}
  \hbox{\begin{tikzinline}[node distance=1.5em]
      \event{a}{\DW{x}{2}}{}
      \raevent{b}{\DR[\mACQ]{x}{2}}{right=of a}
      \event{c}{\DW{y}{1}}{right=of b}
      \event{d}{\DR{y}{1}}{right=2.5em of c}
      \raevent{e}{\DW[\mREL]{x}{1}}{right=of d}
      \rfi{a}{b}
      \bob{b}{c}
      \rfe{c}{d}
      \bob{d}{e}
      \coe[out=-165,in=-15]{e}{a}
    \end{tikzinline}}
\end{gather*}

\tso{} variant \texttt{[rfi-fre-coe2]}:
\begin{gather*}
  \taglabel{rfi-coe-coe2}
  \PW{x}{2}\SEMI 
  \PR[\mACQ]{x}{r}\SEMI
  \PR{y}{s} \PAR
  \PW{y}{2}\SEMI
  \PW[\mREL]{x}{1}
  \\
  \tag{\cmark\armeight}
  \hbox{\begin{tikzinline}[node distance=1.5em]
      \event{a}{\DW{x}{2}}{}
      \raevent{b}{\DR[\mACQ]{x}{2}}{right=of a}
      \event{c}{\DR{y}{0}}{right=of b}
      \event{d}{\DW{y}{2}}{right=2.5em of c}
      \raevent{e}{\DW[\mREL]{x}{1}}{right=of d}
      \rfi{a}{b}
      \bob{b}{c}
      \fre{c}{d}
      \bob{d}{e}
      \coe[out=-165,in=-15]{e}{a}
    \end{tikzinline}}
  \\
  \tag{\cmark\tso}
  \hbox{\begin{tikzinline}[node distance=1.5em]
      \event{a}{\DW{x}{2}}{}
      \event{b}{\DR{x}{2}}{right=of a}
      \event{c}{\DR{y}{0}}{right=of b}
      \event{d}{\DW{y}{2}}{right=2.5em of c}
      \event{e}{\DW{x}{1}}{right=of d}
      %\rfi{a}{b}
      \lob{b}{c}
      \fre{c}{d}
      \lob{d}{e}
      \coe[out=-165,in=-15]{e}{a}
    \end{tikzinline}}
\end{gather*}
Note that \tso{} does not order W to R in local order, even in poloc.
Nonetheless, \tso{} disallows the following because of local visibility in first thread.
\begin{gather*}
  \PW{x}{2}\SEMI 
  \PR{x}{r} \PAR
  \PW{x}{1}\SEMI
  \PR{x}{s}
  \\
  \tag{\xmark\tso}
  \hbox{\begin{tikzinline}[node distance=1.5em]
      \event{a}{\DW{x}{2}}{}
      \event{b}{\DR{x}{1}}{right=of a}
      \event{c}{\DW{x}{1}}{right=2.5em of b}
      \event{d}{\DR{x}{2}}{right=of c}
      \coe[out=-165,in=-15]{c}{a}
      \rfe{c}[above]{b}
      \rfe[out=15,in=165]{a}{d}
      \fr{b}[above]{a}
    \end{tikzinline}}
\end{gather*}
\cite{DBLP:conf/hipc/HighamK00} describe \tso{} as a linearization of partial
order including:
\begin{itemize}
\item ${\rpoloc}$
\item lws = ${\rpox};[\mathsf{W}]$
\item $\bEv\xpox\aEv$ when $\cEv\xrfe\bEv\xpox\aEv$
\end{itemize}
\cite{armed-cats} describe \tso{} as linearization of partial order
satisfying internal visibility and including
\begin{itemize}
\item $[\mathsf{W}];\rpox;[\mathsf{W}]$
\item $\bEv\xpox\aEv$ when $\cEv\xrfe\bEv\xpox\aEv$, from \verb|(range(rfe) * _)|
\item $[\mathsf{R}];\rpox;[\mathsf{W}]$, from \verb|(rfi^-1; lob)|
\end{itemize}
Ignoring fences and \RMW{}s:
\begin{verbatim}
let rec lob = po \ ([W]; po; [R])
let IM0 = loc & ((IW * (M\IW)) | ((W\FW) * FW))
let gc-req = (W * _) | ((R * _) & ((range(rfe) * _) | (rfi^-1; lob))
let preorder-gcb = IM0 | lob & gc-req
\end{verbatim}
% \begin{verbatim}
% let rec lob = po \ ([W]; po; [R])
%         | [W]; po; [MFENCE]; po; [R]
%         | [W]; po; [R & X]
%         | [W & X]; po; [R]
%         | lob; lob
% let IM0 = loc & ((IW * (M\IW)) | ((W\FW) * FW))
% let gc-req = (W * _) | ((R * _) & ((range(rfe) * _) | (rfi^-1; lob))
% let preorder-gcb = IM0 | lob & gc-req
% \end{verbatim}


Double FRE variant \texttt{[rfi-fre-fre]}:
\begin{gather*}
  \taglabel{rfi-fre-fre}
  \PW{x}{2}\SEMI 
  \PR[\mACQ]{x}{r}\SEMI
  \PR{y}{s} \PAR
  \PW{y}{2}\SEMI
  \PF{}\SEMI
  \PR{x}{r}
  \\
  \tag{\cmark\armeight}
  \hbox{\begin{tikzinline}[node distance=1.5em]
      \event{a}{\DW{x}{2}}{}
      \raevent{b}{\DR[\mACQ]{x}{2}}{right=of a}
      \event{c}{\DR{y}{0}}{right=of b}
      \event{d}{\DW{y}{2}}{right=2.5em of c}
      \event{e}{\DF{}}{right=of d}
      \event{f}{\DR{x}{0}}{right=of e}
      \rfi{a}{b}
      \bob{b}{c}
      \fre{c}{d}
      \bob{d}{e}
      \bob{e}{f}
      \fre[out=-165,in=-15]{f}{a}
    \end{tikzinline}}
\end{gather*}

It does not seem possible to do this only with $\rrfe$.
ARM disallows this \texttt{[data-rfi-rfe-rfe]}:
\begin{gather*}
  \taglabel{data-rfi-rfe-rfe}
  \PW{x}{\PR{z}{}} \SEMI
  \PR[\mACQ]{x}{r}\SEMI
  \PW{y}{1} \PAR
  \PW{z}{\PR{y}{}}
  \\
  \tag{\xmark\armeight}
  \hbox{\begin{tikzinline}[node distance=1.5em]
      \event{a}{\DR{z}{1}}{}
      \event{b}{\DW{x}{1}}{right=of a}
      \raevent{c}{\DR[\mACQ]{x}{1}}{right=of b}
      \event{d}{\DW{y}{1}}{right=of c}
      \event{e}{\DW{y}{1}}{right=2.5em of d}
      \event{f}{\DW{z}{1}}{right=of e}
      \data{a}{b}
      \rfi{b}{c}
      \bob{c}{d}
      \data{e}{f}
      \rfe[out=-165,in=-15]{f}{a}
      \rfe{d}{e}
    \end{tikzinline}}
\end{gather*}

It also disallows \texttt{[ctrl-rfi-rfe-rfe]}:
\begin{gather*}
  \taglabel{ctrl-rfi-rfe-rfe}
  \IF{\PR{z}{}}\THEN\FI \SEMI
  \PW{x}{1} \SEMI
  \PR[\mACQ]{x}{r}\SEMI
  \PW{y}{1}
  \PAR
  \PW{z}{\PR{y}{}}
  \\
  \tag{\xmark\armeight}
  \hbox{\begin{tikzinline}[node distance=1.5em]
      \event{a}{\DR{z}{1}}{}
      \event{b}{\DW{x}{1}}{right=of a}
      \raevent{c}{\DR[\mACQ]{x}{1}}{right=of b}
      \event{d}{\DW{y}{1}}{right=of c}
      \event{e}{\DW{y}{1}}{right=2.5em of d}
      \event{f}{\DW{z}{1}}{right=of e}
      \ctrl[out=15,in=165]{a}{d}
      \rfi{b}{c}
      \bob{c}[below]{d}
      \data{e}{f}
      \rfe[out=-165,in=-15]{f}{a}
      \rfe{d}{e}
    \end{tikzinline}}
\end{gather*}

ARM allows some counterintuitive results for SC access \texttt{[ctrl-rfi-fre-rfe]}:
\begin{gather*}
  \taglabel{ctrl-rfi-fre-rfe}
  \IF{\PR{x}{}}\THEN\FI\SEMI
  \PW{x}{2} \SEMI
  \PR[\mSC]{x}{r}\SEMI
  \PR[\mSC]{y}{s} \PAR
  \PW[\mSC]{y}{2}\SEMI
  \PW[\mSC]{x}{1}
  \\
  \tag{\cmark\armeight}
  \hbox{\begin{tikzinline}[node distance=1.5em]
      \event{a}{\DR{x}{1}}{}
      \event{b}{\DW{x}{2}}{right=of a}
      \scevent{c}{\DR[\mSC]{x}{2}}{right=of b}
      \scevent{d}{\DR[\mSC]{y}{0}}{right=of c}
      \scevent{e}{\DW[\mSC]{y}{2}}{right=2.5em of d}
      \scevent{f}{\DW[\mSC]{x}{1}}{right=of e}
      \ctrl{a}{b}
      \rfi{b}{c}
      \bob{c}{d}
      \bob{e}{f}
      \fre{d}{e}
      \rfe[out=-165,in=-15]{f}{a}
    \end{tikzinline}}
\end{gather*}
Not possible with $\rcoe$ \texttt{[ctrl-rfi-coe-rfe]}:
\begin{gather*}
  \taglabel{ctrl-rfi-coe-rfe}
  \IF{\PR{x}{}}\THEN\FI\SEMI
  \PW{x}{2} \SEMI
  \PR[\mSC]{x}{r}\SEMI
  \PW[\mSC]{y}{1} \PAR
  \PW[\mSC]{y}{2}\SEMI
  \PW[\mSC]{x}{1}
  \\
  \tag{\xmark\armeight}
  \hbox{\begin{tikzinline}[node distance=1.5em]
      \event{a}{\DR{x}{1}}{}
      \event{b}{\DW{x}{2}}{right=of a}
      \scevent{c}{\DR[\mSC]{x}{2}}{right=of b}
      \scevent{d}{\DW[\mSC]{y}{1}}{right=of c}
      \scevent{e}{\DW[\mSC]{y}{2}}{right=2.5em of d}
      \scevent{f}{\DW[\mSC]{x}{1}}{right=of e}
      \ctrl[out=15,in=165]{a}{d}
      \rfi{b}{c}
      \bob{c}{d}
      \bob{e}{f}
      \coe{d}{e}
      \rfe[out=-165,in=-15]{f}{a}
    \end{tikzinline}}
\end{gather*}

This is not allowed with a data dependency instead of a control dependency \texttt{[data-rfi-fre-rfe]}:
\begin{gather*}
  \taglabel{data-rfi-fre-rfe}
  \PW{x}{\PR{x}{}{+}1} \SEMI
  \PR[\mSC]{x}{r}\SEMI
  \PR[\mSC]{y}{s} \PAR
  \PW[\mSC]{y}{1}\SEMI
  \PW[\mSC]{x}{1}
  \\
  \tag{\xmark\armeight}
  \hbox{\begin{tikzinline}[node distance=1.5em]
      \event{a}{\DR{x}{1}}{}
      \event{b}{\DW{x}{2}}{right=of a}
      \scevent{c}{\DR[\mSC]{x}{2}}{right=of b}
      \scevent{d}{\DR[\mSC]{y}{0}}{right=of c}
      \scevent{e}{\DW[\mSC]{y}{1}}{right=2.5em of d}
      \scevent{f}{\DW[\mSC]{x}{1}}{right=of e}
      \data{a}{b}
      \rfi{b}{c}
      \bob{c}{d}
      \bob{e}{f}
      \fre{d}{e}
      \rfe[out=-165,in=-15]{f}{a}
    \end{tikzinline}}
\end{gather*}

\section{SC Examples}

\begin{example}
  Consider \iriw{} with all $\mRA$ access:
  \begin{gather*}
    \PW[\mREL]{x}{1}
    \PAR
    \PR[\mACQ]{x}{r}\SEMI \PR[\mACQ]{y}{s}
    \PAR
    \PW[\mREL]{y}{1}
    \PAR
    \PR[\mACQ]{y}{r}\SEMI \PR[\mACQ]{x}{s}
    \taglabel{IRIW-acq-acq}
    \\
    \tag{\cmark\ppc,\cXI}
    \hbox{\begin{tikzinline}[node distance=1.5em]
        \raevent{wx1}{\DW[\mREL]{x}{1}}{}
        \raevent{rx1}{\DR[\mACQ]{x}{1}}{right=2.5em of wx1}
        \raevent{ry0}{\DR[\mACQ]{y}{0}}{right=of rx1}
        \raevent{wy1}{\DW[\mREL]{y}{1}}{right=2.5em of ry0}
        \raevent{ry1}{\DR[\mACQ]{y}{1}}{right=2.5em of wy1}
        \raevent{rx0}{\DR[\mACQ]{x}{0}}{right=of ry1}
        \sync{rx1}{ry0}
        \sync{ry1}{rx0}
        \rf{wx1}{rx1}
        \rf{wy1}{ry1}
        \wk[out=-165,in=-15]{rx0}{wx1}
        \wk{ry0}{wy1}
      \end{tikzinline}}
  \end{gather*}
  We allow this execution:
  \begin{gather*}
    \tag{$\ledep$}
    \hbox{\begin{tikzinline}[node distance=1.5em]
        \raevent{wx1}{\DW[\mREL]{x}{1}}{}
        \raevent{rx1}{\DR[\mACQ]{x}{1}}{right=2.5em of wx1}
        \raevent{ry0}{\DR[\mACQ]{y}{0}}{right=of rx1}
        \raevent{wy1}{\DW[\mREL]{y}{1}}{right=2.5em of ry0}
        \raevent{ry1}{\DR[\mACQ]{y}{1}}{right=2.5em of wy1}
        \raevent{rx0}{\DR[\mACQ]{x}{0}}{right=of ry1}
        \rf{wx1}{rx1}
        \rf{wy1}{ry1}
      \end{tikzinline}}
    \\
    \tag{$\lesync$}
    \hbox{\begin{tikzinline}[node distance=1.5em]
        \raevent{wx1}{\DW[\mREL]{x}{1}}{}
        \raevent{rx1}{\DR[\mACQ]{x}{1}}{right=2.5em of wx1}
        \raevent{ry0}{\DR[\mACQ]{y}{0}}{right=of rx1}
        \raevent{wy1}{\DW[\mREL]{y}{1}}{right=2.5em of ry0}
        \raevent{ry1}{\DR[\mACQ]{y}{1}}{right=2.5em of wy1}
        \raevent{rx0}{\DR[\mACQ]{x}{0}}{right=of ry1}
        \sync{rx1}{ry0}
        \sync{ry1}{rx0}
        \rf{wx1}{rx1}
        \rf{wy1}{ry1}
      \end{tikzinline}}
    \\
    \tag{$\leloc$}
    \hbox{\begin{tikzinline}[node distance=1.5em]
        \raevent{wx1}{\DW[\mREL]{x}{1}}{}
        \raevent{rx1}{\DR[\mACQ]{x}{1}}{right=2.5em of wx1}
        \raevent{ry0}{\DR[\mACQ]{y}{0}}{right=of rx1}
        \raevent{wy1}{\DW[\mREL]{y}{1}}{right=2.5em of ry0}
        \raevent{ry1}{\DR[\mACQ]{y}{1}}{right=2.5em of wy1}
        \raevent{rx0}{\DR[\mACQ]{x}{0}}{right=of ry1}
        \rf{wx1}{rx1}
        \rf{wy1}{ry1}
        \wk[out=-165,in=-15]{rx0}{wx1}
        \wk{ry0}{wy1}
      \end{tikzinline}}
  \end{gather*}
  \ref{IRIW-acq-sc1}, is allowed by trailing-sync compilation to power
  \cite[\textsection 1]{DBLP:conf/pldi/LahavVKHD17}.
  \begin{gather*}
    \PW[\mSC]{x}{1}
    \PAR
    \PR[\mACQ]{x}{r}\SEMI \PR[\mSC]{y}{s}
    \PAR
    \PW[\mSC]{y}{1}
    \PAR
    \PR[\mACQ]{y}{r}\SEMI \PR[\mSC]{x}{s}
    \taglabel{IRIW-acq-sc1}
    \\
    \tag{\cmark\ppc,\xmark\cXI}
    \hbox{\begin{tikzinline}[node distance=1.5em]
        \scevent{wx1}{\DW[\mSC]{x}{1}}{}
        \raevent{rx1}{\DR[\mACQ]{x}{1}}{right=2.5em of wx1}
        \scevent{ry0}{\DR[\mSC]{y}{0}}{right=of rx1}
        \scevent{wy1}{\DW[\mSC]{y}{1}}{right=2.5em of ry0}
        \raevent{ry1}{\DR[\mACQ]{y}{1}}{right=2.5em of wy1}
        \scevent{rx0}{\DR[\mSC]{x}{0}}{right=of ry1}
        \sync{rx1}{ry0}
        \sync{ry1}{rx0}
        \rf{wx1}{rx1}
        \rf{wy1}{ry1}
        \wk[out=-165,in=-15]{rx0}{wx1}
        \wk{ry0}{wy1}
      \end{tikzinline}}
  \end{gather*}
  To model this it is convenient that synchronization is not included in
  dependency order:
  \begin{itemize}
  \item add $\mSC$ bullet to def of $\leexists$ in \ref{cand-leloc-block},
  \item add SC access to $\rsyncdelaysdef$.
  \end{itemize}
  \begin{gather*}
    \tag{$\ledep$}
    \hbox{\begin{tikzinline}[node distance=1.5em]
        \scevent{wx1}{\DW[\mSC]{x}{1}}{}
        \raevent{rx1}{\DR[\mACQ]{x}{1}}{right=2.5em of wx1}
        \scevent{ry0}{\DR[\mSC]{y}{0}}{right=of rx1}
        \scevent{wy1}{\DW[\mSC]{y}{1}}{right=2.5em of ry0}
        \raevent{ry1}{\DR[\mACQ]{y}{1}}{right=2.5em of wy1}
        \scevent{rx0}{\DR[\mSC]{x}{0}}{right=of ry1}
        \rf{wx1}{rx1}
        \rf{wy1}{ry1}
        \wk[out=-165,in=-15]{rx0}{wx1}
        \wk{ry0}{wy1}
      \end{tikzinline}}    
    \\
    \tag{$\lesync$}
    \hbox{\begin{tikzinline}[node distance=1.5em]
        \scevent{wx1}{\DW[\mSC]{x}{1}}{}
        \raevent{rx1}{\DR[\mACQ]{x}{1}}{right=2.5em of wx1}
        \scevent{ry0}{\DR[\mSC]{y}{0}}{right=of rx1}
        \scevent{wy1}{\DW[\mSC]{y}{1}}{right=2.5em of ry0}
        \raevent{ry1}{\DR[\mACQ]{y}{1}}{right=2.5em of wy1}
        \scevent{rx0}{\DR[\mSC]{x}{0}}{right=of ry1}
        \sync{rx1}{ry0}
        \sync{ry1}{rx0}
        \rf{wx1}{rx1}
        \rf{wy1}{ry1}
      \end{tikzinline}}    
    \\
    \tag{$\leloc$}
    \hbox{\begin{tikzinline}[node distance=1.5em]
        \scevent{wx1}{\DW[\mSC]{x}{1}}{}
        \raevent{rx1}{\DR[\mACQ]{x}{1}}{right=2.5em of wx1}
        \scevent{ry0}{\DR[\mSC]{y}{0}}{right=of rx1}
        \scevent{wy1}{\DW[\mSC]{y}{1}}{right=2.5em of ry0}
        \raevent{ry1}{\DR[\mACQ]{y}{1}}{right=2.5em of wy1}
        \scevent{rx0}{\DR[\mSC]{x}{0}}{right=of ry1}
        \rf{wx1}{rx1}
        \rf{wy1}{ry1}
        \wk[out=-165,in=-15]{rx0}{wx1}
        \wk{ry0}{wy1}
      \end{tikzinline}}    
  \end{gather*}
  This correctly forbids the all $\mSC$ version:
  \begin{gather*}
    \PW[\mSC]{x}{1}
    \PAR
    \PR[\mSC]{x}{r}\SEMI \PR[\mSC]{y}{s}
    \PAR
    \PW[\mSC]{y}{1}
    \PAR
    \PR[\mSC]{y}{r}\SEMI \PR[\mSC]{x}{s}
    \taglabel{IRIW-sc-sc}
    \\
    \tag{$\ledep$}
    \hbox{\begin{tikzinline}[node distance=1.5em]
        \scevent{wx1}{\DW[\mSC]{x}{1}}{}
        \scevent{rx1}{\DR[\mSC]{x}{1}}{right=2.5em of wx1}
        \scevent{ry0}{\DR[\mSC]{y}{0}}{right=of rx1}
        \scevent{wy1}{\DW[\mSC]{y}{1}}{right=2.5em of ry0}
        \scevent{ry1}{\DR[\mSC]{y}{1}}{right=2.5em of wy1}
        \scevent{rx0}{\DR[\mSC]{x}{0}}{right=of ry1}
        \sync{rx1}{ry0}
        \sync{ry1}{rx0}
        \rf{wx1}{rx1}
        \rf{wy1}{ry1}
        \wk[out=-165,in=-15]{rx0}{wx1}
        \wk{ry0}{wy1}
      \end{tikzinline}}
  \end{gather*}
  
\end{example}  

\begin{example}
  Thin air with an SC antidependency:
  \begin{gather*}
    \PW[\mSC]{y}{\PR{x}{}}
    \PAR \PW[\mSC]{y}{2}
    \PAR \PW{x}{\PR{y}{}{-}1}
    \\
    \tag{$\ledep$}
    \hbox{\begin{tikzinline}[node distance=1.5em]
        \event{a}{\DR{x}{1}}{}
        \scevent{b}{\DW[\mSC]{y}{1}}{right=of a}
        \scevent{c}{\DW[\mSC]{y}{2}}{right=2.5em of b}
        \event{d}{\DR{y}{2}}{right=2.5em of c}
        \event{e}{\DW{x}{1}}{right=of d}
        \po{a}{b}
        \wk{b}{c}
        \rf{c}{d}
        \po{d}{e}
        \rf[out=-165,in=-15]{e}{a}
      \end{tikzinline}}
  \end{gather*}
\end{example}


\ref{IRIW-acq-sc2} is allowed by trailing-sync compilation to power
\cite[\textsection 1]{DBLP:conf/pldi/LahavVKHD17}.
\begin{gather*}
  \PW[\mSC]{x}{1}
  \PAR
  \PR[\mACQ]{x}{r}\SEMI \PR[\mSC]{y}{s}
  \PAR
  \PW[\mSC]{y}{1}
  \PAR
  \PR[\mACQ]{y}{r}\SEMI \PR[\mSC]{x}{s}
  \taglabel{IRIW-acq-sc2}
  \\
  \tag{\cmark\ppc,\rcXI}
  \hbox{\begin{tikzinline}[node distance=1.5em]
      \scevent{wx1}{\DW[\mSC]{x}{1}}{}
      \raevent{rx1}{\DR[\mACQ]{x}{1}}{right=2.5em of wx1}
      \scevent{ry0}{\DR[\mSC]{y}{0}}{right=of rx1}
      \scevent{wy1}{\DW[\mSC]{y}{1}}{right=2.5em of ry0}
      \raevent{ry1}{\DR[\mACQ]{y}{1}}{right=2.5em of wy1}
      \scevent{rx0}{\DR[\mSC]{x}{0}}{right=of ry1}
      \sync{rx1}{ry0}
      \sync{ry1}{rx0}
      \rf{wx1}{rx1}
      \rf{wy1}{ry1}
      \wk[out=-165,in=-15]{rx0}{wx1}
      \wk{ry0}{wy1}
    \end{tikzinline}}
\end{gather*}
This example is hard to get right for power because it must be allowed with
$\mRA$ reads, but disallowed with $\mSC$ reads.  This seems unsolvable: To
allow the version with $\mRA$, we would need to weaken the order between the
reads in each thread for the $\mRA$ case, and that would break publication.



Leading sync is also unsound in \cXI{} with \RMW{}
\cite[\textsection 2.1]{DBLP:conf/pldi/LahavVKHD17}.
\begin{gather*}
  \PW[\mSC]{x}{1} \SEMI \PW[\mREL]{y}{1}
  \PAR
  \PFADD[\mSC][\mSC]{y}{}{1} \SEMI \PR{y}{s}
  \PAR
  \PW[\mSC]{y}{3} \SEMI \PR[\mSC]{x}{s}
  \taglabel{Z6.U}
  \\
  \tag{\cmark\ppc,\rcXI}
  \hbox{\begin{tikzinline}[node distance=1.5em]
      \scevent{a}{\DW[\mSC]{x}{1}}{}
      \raevent{b}{\DW[\mREL]{y}{1}}{right=of a}
      \scevent{c1}{\DR[\mSC]{y}{1}}{right=2.5em of b}
      \scevent{c2}{\DW[\mSC]{y}{2}}{right=of c1}
      \event{d}{\DR{y}{3}}{right=of c2}
      \scevent{e}{\DW[\mSC]{y}{3}}{right=2.5 em of d}
      \scevent{f}{\DR[\mSC]{x}{0}}{right=of e}
      \sync{a}{b}
      \rf{b}{c1}
      \rf{e}{d}
      \rmw{c1}{c2}
      %\wk{c2}{d}
      \wk[out=-15,in=-165]{c2}{e}
      % \sync[out=-15,in=-165]{c1}{d}
      %\wk{c2}{d}
      \sync{e}{f}
      \wk[out=-165,in=-15]{f}{a}
    \end{tikzinline}}
\end{gather*}
Leading sync is also unsound in \cXI{} with SC fences
\cite[\textsection A.1]{DBLP:conf/pldi/LahavVKHD17}.
\begin{gather*}
  \PW{x}{2} \SEMI \PF{\mSC} \SEMI \PR{y}{r}
  \PAR
  \PW[\mSC]{y}{1}
  \PAR
  \PR[\mACQ]{y}{r} \SEMI \PW[\mREL]{x}{1}  \SEMI \PR{x}{s}
  \PAR
  \PR[\mSC]{x}{r}
   \taglabel{rsync+rsc}
  \\
  \tag{\cmark\rcXI}
  \hbox{\begin{tikzinline}[node distance=1.5em]
      \event{a}{\DW{x}{2}}{}
      \event{b}{\DF{\mSC}}{right=of a}
      \event{c}{\DR{y}{0}}{right=of b}
      \scevent{d}{\DW[\mSC]{y}{1}}{right=2.5em of c}
      \raevent{e}{\DR[\mACQ]{y}{1}}{right=2.5em of d}
      \raevent{f}{\DW[\mREL]{x}{1}}{right=of e}
      \event{g}{\DR{x}{2}}{right=of f}
      \scevent{h}{\DR[\mSC]{x}{1}}{right=2.5em of g}
      \sync{a}{b}
      \sync{b}{c}
      \rf{d}{e}
      \rf[out=-15,in=-165]{f}{h}
      \wk[in=-15,out=-165]{f}{a}
      %\rf[out=-15,in=-165]{a}{g}
      \wk{c}{d}
      \wki{f}{g}
      \sync{e}{f}
      %\sync[out=15,in=165]{e}{g}
    \end{tikzinline}}
\end{gather*}
Fulfillment of $(\DR{x}{2})$ requires that either
\begin{math}
  (\DW[\mREL]{x}{1})
  \xwk
  (\DW{x}{2})
\end{math}
or 
\begin{math}
  (\DR{x}{2})
  \xwk
  (\DW[\mREL]{x}{1}).
\end{math}
It's interesting that in the pomset, $(\DR[\mSC]{x}{1})$ is not needed to get
a cycle.

There is a long discussion of this in \cite[\textsection 5.2,
Fig.~17]{DBLP:journals/pacmpl/BenderP19}, where they also discuss this example:
\begin{gather*}
  \PW[\mSC]{x}{1}\SEMI \PW{x}{2}
  \PAR
  \PW[\mSC]{y}{1}\SEMI \PW{y}{2}
  \PAR
  \PR[\mACQ]{x}{r}\SEMI \PR[\mSC]{y}{s}
  \PAR
  \PR[\mACQ]{y}{r}\SEMI \PR[\mSC]{x}{s}
  \taglabel{IRIW-sc-rlx-acq}
  \\
  \tag{\cmark\rcXI}
  \hbox{\begin{tikzinline}[node distance=1.5em]
      \scevent{wx1}{\DW[\mSC]{x}{1}}{}
      \event{wx2}{\DW{x}{2}}{right=of wx1}
      \scevent{wy1}{\DW[\mSC]{y}{1}}{below=4ex of wx1}
      \event{wy2}{\DW{y}{2}}{right=of wy1}
      \raevent{ry1}{\DR[\mACQ]{y}{2}}{right=2.5em of wy2}
      \scevent{rx0}{\DR[\mSC]{x}{0}}{right=of ry1}
      \raevent{rx1}{\DR[\mACQ]{x}{2}}{right=2.5 em of wx2}
      \scevent{ry0}{\DR[\mSC]{y}{0}}{right=of rx1}
      \sync{rx1}{ry0}
      \sync{ry1}{rx0}
      \rf{wx2}{rx1}
      \rf{wy2}{ry1}
      \wk{rx0}{wx1}
      \wk{ry0}{wy1}
      \wk{wx1}{wx2}
      \wk{wy1}{wy2}
    \end{tikzinline}}
\end{gather*}


\cite[\textsection A.2]{DBLP:conf/pldi/LahavVKHD17} claims that \armeight{}
allows this \texttt{[RWC+acq+sc]}, but \href{http://diy.inria.fr/www/?record=aarch64}{herd7} rejects it.
%\verbatiminput{litmus/RWC+acq+sc.litmus}
% More legibly:
% \begin{verbatim}
% STLR#1,[x]     | LDR a, [x] /1    | STLR #1, [y] 
%                | DMB LD           | LDAR c, [x] /0
%                | LDAR b, [y] /0
% \end{verbatim}
Reason: they are citing the flowing/pop model
\cite{DBLP:conf/popl/FlurGPSSMDS16} rather than
\cite{DBLP:journals/pacmpl/PulteFDFSS18}.
\begin{gather*}
  \taglabel{rwc+acq+sc}
  \PW[\mSC]{x}{1} \PAR
  \PR{x}{r}\SEMI
  \PF{\fACQ}\SEMI
  \PR[\mSC]{y}{s} \PAR
  \PW[\mSC]{y}{1}\SEMI
  \PR[\mSC]{x}{r}
  \\
  \tag{\xmark\armeight}
  \hbox{\begin{tikzinline}[node distance=1.5em]
      \scevent{a}{\DW[\mSC]{x}{1}}{}
      \event{b}{\DR{x}{1}}{right=2.5em of a}
      \event{c}{\DF{\fACQ}}{right=of b}
      \scevent{d}{\DR[\mSC]{y}{0}}{right=of c}
      \scevent{e}{\DW[\mSC]{y}{1}}{right=2.5em of d}
      \scevent{f}{\DR[\mSC]{x}{0}}{right=of e}
      \rfe{a}{b}
      \sync{b}{c}
      \sync{c}{d}
      \fre[out=-165,in=-15]{f}{a}
      \fre{d}{e}
      \sync{e}{f}
    \end{tikzinline}}
\end{gather*}

\section{Additional RMW Examples}

It is not possible for two \RMW{}s to see the same write.
\begin{gather*}
  \begin{gathered}
    \PW{x}{0} \SEMI \bigl(\PFADD[\mRLX][\mRLX]{x}{}{1} \PAR \PFADD[\mRLX][\mRLX]{x}{}{1}\bigr)
    \\
    \hbox{\begin{tikzinline}[node distance=2em]
        \event{a0}{\DW{x}{0}}{}
        \event{a1}{\DR{x}{0}}{right=3em of a0}
        \event{a2}{\DW{x}{1}}{right=of a1}
        \event{b1}{\DR{x}{0}}{right=3em of a2}
        \event{b2}{\DW{x}{1}}{right=of b1}
        \rmw{a1}{a2}
        \rf{a0}{a1}
        \rf[out=-15,in=-165]{a0}{b1}
        \wk[out=-15,in=-165]{a1}{b2}
        \wk{b1}{a2}
        \graywk[bend left]{a2}{b1}
        \rmw{b1}{b2}
      \end{tikzinline}}
  \end{gathered}
  \taglabel{rmw0}
\end{gather*}
The gray arrow is required the \RMW{} atomicity axioms.

\citet{DBLP:conf/pldi/LeeCPCHLV20} introduce \PS{2.0} to refine the treatment of
\RMW{}s in the promising semantics (\PS{}).  Their examples have the expected
results here, with far less work.  First they recall that \PS{} requires
quantification over multiple futures in order to disallow executions such as
\ref{CDRF}:
\begin{gather*}
  \taglabel{CDRF}
    \begin{gathered}
      \PFADD[\mACQ][\mREL]{x}{r}{1}\SEMI \IF{r{=}0}\THEN \PW{y}{1} \FI
      \PAR
      \PFADD[\mACQ][\mREL]{x}{r}{1}\SEMI \IF{r{=}0}\THEN \IF{y}\THEN \PW{x}{0} \FI \FI
      \\
      \hbox{\begin{tikzinline}[node distance=2em]
          \event{a1}{\DR[\mACQ]{x}{0}}{}
          \event{a1b}{\DW[\mREL]{x}{1}}{below=1em of a1}
          \event{a2}{\DW{y}{1}}{right=of a1}
          \event{b0}{\DR[\mACQ]{x}{0}}{right=3em of a2}
          \event{b0b}{\DW[\mREL]{x}{1}}{below=1em of b0}
          \event{b1}{\DR{y}{1}}{right=of b0}
          \event{b2}{\DW{x}{0}}{right=of b1}
          \rmw{a1}{a1b}
          \rmw{b0}{b0b}
          \rf[out=-13,in=-163]{a2}{b1}
          \po{a1}{a2}
          \sync{b0}{b1}
          \po{b1}{b2}
          \rf[out=-165,in=-12]{b2}{a1}
        \end{tikzinline}}
    \end{gathered}
  \end{gather*}
This execution is clearly impossible, due to the cycle above.  In this
diagram, we have not drawn order adjacent to the writes of the \RMW{}s, since
this is not necessary to produce the cycle.
If \ref{CDRF} is allowed then \drfra{} fails.


  
\PS{} does not support global value range analysis, as modeled by \ref{GA+E} below.  Our
semantics permits \ref{GA+E}:
\begin{gather*}
  \taglabel{GA+E}
    \begin{gathered}
      \PW{x}{0} \SEMI
      \bigl(
        \PCAS[\mRLX][\mRLX]{x}{r}{0}{1}\SEMI \IF{r{<}10}\THEN \PW{y}{1} \FI
        \PAR
        \PW{x}{42}\SEMI \PW{x}{y}
      \bigr)
      \\
      \hbox{\begin{tikzinline}[node distance=2em]
          \event{a0}{\DW{x}{0}}{}
          \event{a1}{\DR{x}{1}}{right=3em of a0}
          \event{a2}{0{<}10\mid\DW{y}{1}}{right=of a1}
          \event{b0}{\DW{x}{42}}{right=3em of a2}
          \event{b1}{\DR{y}{1}}{right=of b0}
          \event{b2}{\DW{x}{1}}{right=of b1}
          %\rmw{a1}{a2}
          \rf[out=-15,in=-160]{a2}{b1}
          \po{b1}{b2}
          \rf[out=-165,in=-15]{b2}{a1}
          \wk[out=10,in=170]{a0}{b0}
          \wk[out=15,in=165]{b0}{b2}
        \end{tikzinline}}
    \end{gathered}
\end{gather*}
\PS{} also does not support register promotion, as modeled by \ref{RP} below.    Our
semantics permits \ref{RP}:
\begin{gather*}
  \taglabel{RP}
    \begin{gathered}
      \PR{x}{r}\SEMI
      \PFADD[\mRLX][\mRLX]{z}{s}{r}\SEMI \PW{y}{s{+}1}
      \PAR
      \PW{x}{y}
      \\
      \hbox{\begin{tikzinline}[node distance=2em]
          \event{a0}{\DR{x}{1}}{}
          \event{a1}{\DR{z}{0}}{right=of a0}
          \event{a1b}{\DW{z}{1}}{right=of a1}
          \event{a2}{\DW{y}{1}}{right=of a1b}
          \event{b0}{\DR{y}{1}}{right=3em of a2}
          \event{b1}{\DW{x}{1}}{right=of b0}
          \rmw{a1}{a1b}
          \po[out=20,in=160]{a0}{a1b}
          \po[out=20,in=160]{a1}{a2}
          \po{b0}{b1}
          \rf{a2}{b0}
          \rf[out=-165,in=-15]{b1}{a0}
        \end{tikzinline}}
    \end{gathered}
\end{gather*}



These following examples are from ``Modular Data-Race-Freedom Guarantees in
the Promising Semantics'' to appear in PLDI21.

\ref{CDRF} shows that our semantics is not too permissive for $\mRA$-\RMW{}s.
But what about $\mRLX$-\RMW{}s.  The following execution is allowed by \armeight,
and \PS{2.0}, but disallowed by \PS{2.1}.
\begin{gather*}
  \taglabel{RMW-W}
  \begin{gathered}
    \PFADD[\mRLX][\mRLX]{x}{r}{1}\SEMI \PW{y}{1}
    \PAR
    \PR{y}{r}\SEMI \PFADD[\mRLX][\mRLX]{x}{s}{r}
    \\
    \hbox{\begin{tikzinline}[node distance=2em]
        \event{a1}{\DR{x}{1}}{}
        \event{a1b}{\DW{x}{2}}{below=1em of a1}
        \event{a2}{\DW{y}{1}}{right=of a1}
        \event{b1}{\DR{y}{1}}{right=3em of a2}
        \event{b2}{\DR{x}{0}}{right=of b1}
        \event{b2b}{\DW{x}{1}}{below=1em of b2}
        \rmw{a1}{a1b}
        \rmw{b2}{b2b}
        \rf{a2}{b1}
        \po{b1}{b2b}
        \rf[out=-175,in=-20]{b2b}{a1}
      \end{tikzinline}}
  \end{gathered}
\end{gather*}

If this $\ldrfra{z}$?
\begin{gather*}
  \taglabel{Naive-LDRF-RA-Fail}
  \begin{gathered}
    \IF{y}\THEN \PW{x}{z} \ELSE \PW{x}{1} \FI
    \PAR
    \PR{x}{r}\SEMI \PW{z}{1}\SEMI \PW{y}{r}
    \\
    \hbox{\begin{tikzinline}[node distance=2em]
        \event{a1}{\DR{y}{1}}{}
        \event{a2}{\DR{z}{1}}{right=of a1}
        \event{a3}{\DW{x}{1}}{right=of a2}
        \event{b1}{\DR{x}{1}}{right=3em of a3}
        \event{b2}{\DW{z}{1}}{right=of b1}
        \event{b3}{\DW{y}{1}}{right=of b2}
        \po{a2}{a3}
        \po[in=165,out=15]{b1}{b3}
        \rf[out=-165,in=-15]{b2}{a2}
        \rf[out=-165,in=-15]{b3}{a1}
        \rf{a3}{b1}
      \end{tikzinline}}
  \end{gathered}
\intertext{Interpreting $\{z\}$ as $\mRA$:}
    \\
  \begin{gathered}
    \hbox{\begin{tikzinline}[node distance=2em]
        \event{a1}{\DR{y}{1}}{}
        \event{a2}{\DR[\mACQ]{z}{1}}{right=of a1}
        \event{a3}{\DW{x}{1}}{right=of a2}
        \event{b1}{\DR{x}{1}}{right=3em of a3}
        \event{b2}{\DW[\mREL]{z}{1}}{right=of b1}
        \event{b3}{\DW{y}{1}}{right=of b2}
        \po{a2}{a3}
        \po[in=165,out=15]{b1}{b3}
        \rf[out=-165,in=-15]{b2}{a2}
        \rf[out=-165,in=-15]{b3}{a1}
        \rf{a3}{b1}
        \sync{a1}{a2}
        \sync{b2}{b3}
      \end{tikzinline}}
  \end{gathered}
\end{gather*}

Our semantics already disallows \ref{LDRF-Fail-PS}, which is similar to \ref{OOTA4}.
\begin{gather*}  
  \taglabel{LDRF-Fail-PS}
  \begin{gathered}
  \IF{x}\THEN
    \PFADD{w}{}{1}\SEMI
    \PW{y}{1}\SEMI
    \PW{z}{1}
  \FI
  \PAR
  \IF{\BANG z}\THEN
    \PW{x}{1}
  \ELSE
    \IF{\BANG \PFADD{w}{}{1}}\THEN
      \PW{x}{\PR{y}{}}
    \FI
  \FI
    \\
    \hbox{\begin{tikzinline}[node distance=2em]
        \event{a1}{\DR{x}{1}}{}
        \event{a2}{\DR{w}{1}}{right=of a1}
        \event{a3}{\DW{w}{2}}{right=of a2}
        \event{a4}{\DW{y}{1}}{right=of a3}
        \event{a5}{\DW{z}{1}}{right=of a4}
        \event{b1}{\DR{z}{1}}{right=5em of a5}
        \event{b2}{\DR{w}{0}}{right=of b1}
        \event{b3}{\DW{w}{1}}{right=of b2}
        \event{b4}{\DR{y}{1}}{right=of b3}
        \event{b5}{\DW{x}{1}}{right=of b4}
        \rmw{a2}{a3}
        \po[out=15,in=165]{a1}{a3}
        \po[out=15,in=165]{a1}{a4}
        \po[out=15,in=165]{a1}{a5}        
        \rmw{b2}{b3}
        \po{b4}{b5}
        \po[out=15,in=165]{b2}{b5}        
        \po[out=15,in=165]{b1}{b3}
        \rf{a5}{b1}
        \rf[out=15,in=165]{a4}{b4}
        \rf[out=-165,in=-15]{b3}{a2}
        \rf[out=-165,in=-15]{b5}{a1}
      \end{tikzinline}}
  \end{gathered}
\end{gather*}
\begin{gather}
  \taglabel{OOTA4}
  \begin{gathered}
    \PW{y}{x}
    \PAR
    \PR{y}{r} \SEMI \IF{b}\THEN  \PW{x}{r} \SEMI \PW{z}{r} \ELSE \PW{x}{1} \FI
    \PAR
    \PW{b}{1}
    \\[-1ex]
    \hbox{\begin{tikzinline}[node distance=1.5em]
        \event{rx}{\DR{x}{1}}{}
        \event{wy}{\DW{y}{1}}{right=of rx}
        \po{rx}{wy}
        \event{ry}{\DR{y}{1}}{right=3em of wy} 
        \event{wx}{\DW{x}{1}}{right=of ry}
        \event{wz}{\DW{z}{1}}{right=of wx}
        \event{rb}{\DR{b}{1}}{right=of wz}
        \event{wb1}{\DW{b}{1}}{right=3em of rb}
        \po{ry}{wx}
        \rf{wb1}{rb}
        \rf{wy}{ry}
        \rf[out=-170,in=-10]{wx}{rx}
        \po{rb}{wz}
        \po[out=15,in=165]{ry}{wz}
      \end{tikzinline}}
  \end{gathered}  
\end{gather}



\begin{comment}
  \centering  
\begin{verbatim}
a := X                  b := Z                 
if a = 1 then           if b = 0 then          
  _ := FADD(W , 1)        X := 1               
  Y := 1                else                   
  Z := 1                  c := FADD(W, 1) /0   
                          if c = 0 then        
                            d := Y             
                            X := d             
\end{verbatim}
\includegraphics[width=\textwidth]{LDRF-Fail-PS}
\caption{LDRF-Fail-PS}
\end{comment}


If \RMW{}s simply use the same semantics as read and write, then we allow
\ref{LDRF-PF-Fail}, which is used to show failure of $\ldrfsc{}$.
\begin{gather*}  
  \taglabel{LDRF-PF-Fail}
  \begin{gathered}
    \PW{y}{0}\SEMI
    \IF{y}\THEN
      \IF{\BANG\PCAS{x}{}{0}{1}}\THEN
        \IF{z}\THEN
          \PW{x}{2}
    \FI\FI\FI
    \PAR
    \PW{y}{1}\SEMI
    \IF{1{\neq}\PCAS{x}{}{0}{3}}\THEN
      \PW{z}{1}
    \FI
    \\
    \hbox{\begin{tikzinline}[node distance=2em]
        \event{a1}{\DW{y}{0}}{}
        \event{a2}{\DR{y}{1}}{right=of a1}
        \event{a3}{\DR{x}{0}}{right=of a2}
        \event{a4}{\DW{x}{1}}{right=of a3}
        \event{a5}{\DR{z}{1}}{right=of a4}
        \event{a6}{\DW{x}{2}}{right=of a5}
        \event{b1}{\DW{y}{1}}{right=5em of a6}
        \event{b2}{\DR{x}{2}}{right=of b1}
        \event{b3}{\DW{z}{1}}{right=of b2}
        \wk{a1}{a2}
        \rmw{a3}{a4}
        \po[out=15,in=165]{a2}{a6}
        %\po[out=15,in=165]{a3}{a6}
        \po{a5}{a6}
        \wk[out=-20,in=-160]{a4}{a6}
        %\po{b2}{b3}
        \rf[out=15,in=165]{a6}{b2}
        \rf[out=-165,in=-15]{b3}{a5}
        \rf[out=-165,in=-15]{b1}{a2}
      \end{tikzinline}}
  \end{gathered}
\end{gather*}
To disallow this, we need to retain the dependency
\begin{math}
  \DRP{x}{2}\xpo \DWP{z}{1}.
\end{math}
For this, we need to avoid the substitution for $x$.  This is clearer in the
LICS semantics.  You just use L6 rather than L5 for the independent case on
\RMW{}s.

\begin{comment}
  \centering  
\begin{verbatim}
Y := 0                   Y := 1                 
a := Y                   d := CAS(X,0,1) /37?   
if a != 0 then           if d != 42 then        
  b := CAS(X,0,42)         L := 1               
  if b = 0 then
    c := L
    if c = 1 then
      Xsrlx := 37
\end{verbatim}
\includegraphics[width=.8\textwidth]{LDRF-PF-Fail.png}
\caption{LDRF-PF-Fail}
\end{comment}


\section{Example from JAM paper}
From \cite[\textsection 3.3]{DBLP:journals/pacmpl/BenderP19}.  With partial
coherence/weak fulfillment you need to be careful that \RMW{}s are totally
ordered (if that's a property you want).  May not come for free.

 From \cite[\textsection B]{DBLP:journals/pacmpl/BenderP19}:
``Here we demonstrate that it is possible to construct a program that is only
forbidden due to the total coherence order''

\begin{comment}
AArch64 TotalCO
{
0:X1=x; 0:X3=y; 
1:X1=x; 1:X3=y;
2:X1=x; 2:X3=y;
}
 P0            | P1           | P2;
 LDR X2,[X1]   | LDAR X5, [X3]| LDAR X5,[X1];
 MOV X0,#1     | MOV X2,#2    | MOV X0, #1;
 STR X0,[X1]   | STR X2,[X1]  | STR X0, [X3];

exists (0:X2=2 /\ 1:X5=1 /\ 2:X5=1)
\end{comment}


\begin{gather*}
  \PR{x}{r}\SEMI
  \PW{x}{1}
  \PAR
  \PR[\mACQ]{x}{r}\SEMI
  \PW{x}{1}
  \PAR
  \PR[\mACQ]{y}{r}\SEMI
  \PW{x}{2}
  \taglabel{Total-CO}
  \\
  \tag{\xmark\armeight}
  \hbox{\begin{tikzinline}[node distance=1.5em]
      \event{a}{\DR{x}{2}}{}
      \event{b}{\DW{x}{1}}{right=of a}
      \event{c}{\DR[\fACQ]{x}{1}}{right=2.5em of b}
      \event{d}{\DW{y}{1}}{right=of c}
      \event{e}{\DR[\fACQ]{y}{1}}{right=2.5em of d}
      \event{f}{\DW{x}{2}}{right=of e}
      \wki{a}{b}
      \sync{c}{d}
      \sync{e}{f}
      \rf{b}{c}
      \rf{d}{e}
      \rf[out=-165,in=-15]{f}{a}
    \end{tikzinline}}
  % \\
  % \tag{\xmark\armeight}
  % \hbox{\begin{tikzinline}[node distance=1.5em]
  %     \event{a}{\DR{x}{2}}{}
  %     \event{b}{\DW{x}{1}}{right=of a}
  %     \event{c}{\DR[\fACQ]{x}{1}}{right=2.5em of b}
  %     \event{d}{\DW{y}{1}}{right=of c}
  %     \event{e}{\DR[\fACQ]{y}{1}}{right=2.5em of d}
  %     \event{f}{\DW{x}{2}}{right=of e}
  %     \poloc{a}{b}
  %     \co[out=15,in=165]{b}{f}
  %     \rfx[out=-165,in=-15]{f}{a}
  %   \end{tikzinline}}
  \\
  \tag{\xmark\armeight}
  \hbox{\begin{tikzinline}[node distance=1.5em]
      \event{a}{\DR{x}{2}}{}
      \event{b}{\DW{x}{1}}{right=of a}
      \event{c}{\DR[\fACQ]{x}{1}}{right=2.5em of b}
      \event{d}{\DW{y}{1}}{right=of c}
      \event{e}{\DR[\fACQ]{y}{1}}{right=2.5em of d}
      \event{f}{\DW{x}{2}}{right=of e}
      \poloc{a}{b}
      \co[out=15,in=165]{b}{f}
      \rfx{b}{c}
      \fr[out=15,in=165]{c}[below]{f}
      \rfx[out=-165,in=-15]{f}{a}
    \end{tikzinline}}
  \\
  \tag{\xmark\armeight}
  \hbox{\begin{tikzinline}[node distance=1.5em]
      \event{a}{\DR{x}{2}}{}
      \event{b}{\DW{x}{1}}{right=of a}
      \event{c}{\DR[\fACQ]{x}{1}}{right=2.5em of b}
      \event{d}{\DW{y}{1}}{right=of c}
      \event{e}{\DR[\fACQ]{y}{1}}{right=2.5em of d}
      \event{f}{\DW{x}{2}}{right=of e}
      \coe[out=165,in=15]{f}[above]{b}
      %\rfe[out=-165,in=-15]{f}{a}
      \bob{c}{d}
      \bob{e}{f}
      \rfe{b}{c}
      \rfe{d}{e}
    \end{tikzinline}}
\end{gather*}

\section{Two order idea}
The two order idea from OOPSLA talk is:
\begin{itemize}
\item Require: $\bEv\leloc\aEv$ when $\bEv\ledep\aEv$ and they conflict
\end{itemize}
This does not work for the \IMM{} or ARMv7, but it may work for Power, TSO,
ARMv8.  That would be nice.  Let's write $\leloctwo$ for this notion, with
strong fulfillment.

With this there is a cycle in \ref{arm7-weak} (weak/strong fulfillment not relevant here):
\begin{gather*}
  \tag{$\leloctwo$}
  \hbox{\begin{tikzinline}[node distance=1.5em]
      \event{a}{\DR{x}{1}}{}
      \event{b}{d:\DW{x}{1}}{right=of a}
      \wk{a}{b}
      \event{c}{\DR{x}{1}}{right=3em of b}
      \event{d}{\DW{y}{1}}{right=of c}
      %\po{c}{d}
      \event{e}{\DR{y}{1}}{right=3em of d}
      \event{f}{e:\DW{x}{1}}{right=of e}
      % \po{e}{f}
      \po[out=-15,in=-165]{c}{f}
      \rf{b}{c}
      \rf{d}{e}
      \rf[out=172,in=8]{f}{a}
    \end{tikzinline}}    
\end{gather*}
Anton says: \ref{arm7-weak} is forbidden by Power, TSO, ARMv8, but allowed by
ARMv7. Maybe it isn't that important to support it anymore.

There is also a cycle in \ref{pub-rel-rlx-coe}.  Anton says: I checked
Power/ARMv7 models in this regard. They disallow the behavior (as well as
ARMv8 and TSO), so we can in principle strengthen \IMM{} to forbid it as
well.  For that, we may add axiom to \IMM{} forbidding cycles in
\begin{math}
  \rco \cup ([\mathsf{W}]; \rrfe^?; ([\mathsf{R}^{\fACQ}] \cup \rpox;
  [\mathsf{FW}^{\fREL}]); \rar^{*}; [\mathsf{W}]).
\end{math}
This works if we have acquire/release accesses on the path
since they are compiled with fences to Power.

\endinput

\section{OLD Model}

\begin{align*}
  \amode \BNFDEF& \mWK &&\text{{(Weak)}}                      &\ascope \BNFDEF& \sCTA &&\text{(Thread group)} &\hbox{$\;\mkern60mu\;$}&
  \\[-1ex] \BNFSEP& \mRLX &&\text{{(Relaxed)}}                & \BNFSEP&\sGPU   &&\text{(Processor)}                                   
  \\[-1ex] \BNFSEP& \mRA &&\text{{(Release/Acquire)}}         & \BNFSEP&\sSYS  &&\text{(System)}                                         
  \\[-1ex] \BNFSEP& \mSC &&\text{{(Sequentially Consistent)}}    
\end{align*}

Orders/Relations in model
\begin{itemize}
\item $\ledep$ is the old $\le$ (without coherence stuff from \ref{rf4} and \ref{5b}).

  This provides the NO-TAR axiom.
\item $\lesync$ is a the \emph{happens-before} suborder, which only includes $\rrf$ when they are morally strong.

  This serves as a cross-location transitive kernel for the per-location order.
  
\item $\leloc$ is a per-location order that relates morally strong  and $\rpoloc$ accesses

  This includes $\lesync$ for  morally strong accesses.

  This provides the SC-PER-LOC axiom.

  % \item $\rrmw$ is a per-location relation on actions in an \RMW{}
\end{itemize}

Write $\bEv\conflict\aEv$ if they conflict (ie, read/write or write/write, same location).

Write $\bEv\moral\aEv$ if they conflict and are morally strong

\begin{definition}
  A \emph{pomset with preconditions} is a tuple
  $(\Event, \labeling, {\lesync}, {\ledep}, {\leloc})$ where
  \begin{description}
  \item[{\labeltextsc[m1]{(m1)}{m1}}] $\Event$ is a set of \emph{events}
  \item[{\labeltextsc[m2]{(m2)}{m2}}]
    $\labeling: \Event \fun (\Formulae\times\Act)$ is a \emph{labeling} from
    which we derive functions
    \begin{itemize}
    \item $\labelingForm:\Event\fun\Formulae$
      \emph{(formulae)} % include $r{=}v$ $x{=}v$
    \item $\labelingAct:\Event\fun\Act$
      \emph{(actions)} %include $\DW{x}{v}$, $\DR{x}{v}$, and $\DSTOP$
    \end{itemize}
  \item[{\labeltextsc[m3]{(m3)}{m3}}]
    ${\lesync} \subseteq (\Event\times\Event)$,
    ${\ledep} \subseteq (\Event\times\Event)$, and
    ${\leloc} \subseteq (\Event\times\Event)$ are partial orders
  \item[{\labeltextsc[m4]{(m4)}{m-consistency}}] $\bigwedge_{\aEv}\labelingForm(\aEv)$ is satisfiable \emph{(consistency)}
  \item[{\labeltextsc[m5]{(m5)}{m-causal-strengthening}}] if $\bEv\ledep\aEv$ then $\labelingForm(\aEv)$ implies $\labelingForm(\bEv)$ \emph{(causal strengthening)} 
  \item[{\labeltextsc[m6]{(m6)}{m-strong}}] if $\bEv\lesync\aEv$ then $\bEv\ledep\aEv$
  \item[{\labeltextsc[m7]{(m7)}{m-loc}}] if $\bEv\lesync\aEv$ and $\bEv$ conflicts with $\aEv$ then $\bEv\leloc\aEv$
  \end{description}
\end{definition}
% It is important that \ref{m-loc} covers all conflicting access.
% See \ref{pub1sys}.


  % We say $\bEv\ltsync\aEv$ when $\bEv\lesync\aEv$ and $\bEv\neq\aEv$, and similarly
  % for $\ltdep$ and $\ltloc$.

% \begin{definition}
%   Define $\leexists$ %and $\ltexists$
%   as follows:
% \end{definition}


\begin{definition}[Strong fulfillment]
  We say $\labelingAct(\bEv)=(\DW[]{x}{v})$ \emph{fulfills}
  $\labelingAct(\aEv)=(\DR[]{x}{v})$ if
  \begin{description}
  \item[{\labeltextsc[f3a]{(f3a)}{rf3a}}{\labeltextsc[f3]{}{rf3}}]
    $\bEv \ltdep \aEv$
  \item[{\labeltextsc[f3b]{(f3b)}{rf3b}}]
    $\bEv \ltsync \aEv$ if $\bEv$ is morally strong with $\aEv$
  \item[{\labeltextsc[f3c]{(f3c)}{rf3c}}]
    $\bEv \leloc \aEv$ (if $\bEv$ is not morally strong with $\aEv$)
  \item[{\labeltextsc[f4]{(f4)}{rf4}}]
    $\forall\labelingAct(\cEv)=(\DW[]{x}{..})$ either $\cEv \leloc \bEv$ or
    $\aEv \leloc \cEv$,
  \end{description}  
\end{definition}
  
\begin{definition}[Weak fulfillment]
  We say $\labelingAct(\bEv)=(\DW[]{x}{v})$ \emph{fulfills}
  $\labelingAct(\aEv)=(\DR[]{x}{v})$ if
  \begin{description}
  \item[{\labeltextsc[f3a]{(f3a)}{rf3a}}{\labeltextsc[f3]{}{rf3}}]
    $\bEv \ltdep \aEv$
  \item[{\labeltextsc[f3b]{(f3b)}{rf3b}}]
    $\bEv \ltsync \aEv$ if $\bEv$ is morally strong with $\aEv$
  \item[{\labeltextsc[f3c]{(f3c)}{rf3c}}]
    $\aEv \not\leloc \bEv$ (if $\bEv$ is not morally strong with $\aEv$)
  \item[{\labeltextsc[f4]{(f4)}{rf4}}]
    $\forall\labelingAct(\cEv)=(\DW[]{x}{..})$ either $\cEv \leexists \bEv$ or
    $\aEv \leexists \cEv$,
    where
  \begin{align*}
    \bEv\leexists\aEv &\textwhen                      
    \begin{cases}
      \bEv\leloc\aEv &\text{if}\; \bEv \;\text{is morally strong with}\;
      \aEv %\bEv\moral\aEv
      \\
      \aEv\not\ltloc\bEv &\text{otherwise}
    \end{cases}
    % \\
    % \bEv\ltexists\aEv &\textwhen                      
    % \begin{cases}
    %   \bEv\ltloc\aEv &\text{if}\; \bEv \;\text{is morally strong with}\;
    %   \aEv %\bEv\moral\aEv
    %   \\
    %   \aEv\not\leloc\bEv &\text{otherwise}
    % \end{cases}
  \end{align*}    
  \end{description}  
\end{definition}

If all accesses are morally strong with each other, weak fulfillment
degenerates to
\begin{description}
\item[\eqref{rf3}]
  $\bEv \ltsync \aEv$
\item[\eqref{rf4}]
  $\forall\labelingAct(\cEv)=(\DW[]{x}{..})$ either
  $\cEv \leloc \bEv$ or $\aEv \leloc \cEv$
\end{description}

If no accesses are morally strong with each other, weak fulfillment
degenerates to
\begin{description}
\item[\eqref{rf3}]
  $\aEv \not\leloc \bEv$
\item[\eqref{rf4}]
  $\not\mkern-5mu\exists\labelingAct(\cEv)=(\DW[]{x}{..})$ 
  both $\bEv \ltloc \cEv$ and $\cEv \ltloc \aEv$
\end{description}

Note that the difference between strong and weak fulfillment is limited to $\leloc$.
We sometimes write $\lelocstrong$ for strong fulfillment and
$\lelocweak$ for weak fulfillment.

Prefixing is as in OOPSLA, using $\lesync$ for order everywhere except
\ref{5b}, which has $\leloc$.
\begin{definition}
  Let $\aPS'\in(\aForm \mid \aAct) \prefix \aPSS$ when
  $(\exists\aPS\in\aPSS)$ $(\forall\aEv\in\Event)$
  \begin{description}
  \item[{\labeltextsc[P1]{(P1)}{1}}] $\Event' = \Event \cup \{\bEv\}$
  \item[{\labeltextsc[P2]{(P2)}{2}}] ${\lesync'}\supseteq{\lesync}$,
    ${\ledep'}\supseteq{\ledep}$, and ${\leloc'}\supseteq{\leloc}$
  \item[{\labeltextsc[P3]{(P3a)}{3a}\labeltextsc[P3]{}{3}}]%
    $\labelingAct'(\aEv) = \labelingAct(\aEv)$
  \item[{\labeltextsc[P3b]{(P3b)}{3b}}] $\labelingAct'(\bEv) = \aAct$
  \item[{\labeltextsc[P4a]{(P4a)}{4a}\labeltextsc[P4]{}{4}}]%
    $\labelingForm'(\bEv)$ implies
    $\aForm\land(\bEv\not\in\Event\lor\labelingForm(\bEv))$
  \item[{\labeltextsc[P4b]{(P4b)}{4b}}] if $\bEv\neq(\DR[]{..)}{}\mkern79mu$
    then $\aEv=\bEv$ or $\labelingForm'(\aEv)$ implies $\labelingForm(\aEv)$
  \item[{\labeltextsc[P4c]{(P4c)}{4c}}] if
    $\bEv=(\DR[]{\aVal}{\aLoc})\mkern70mu$ then $\aEv=\bEv$ or
    $\labelingForm'(\aEv)$ implies $\labelingForm(\aEv)[\aVal/\aLoc]$
  \item[{\labeltextsc[P5a]{(P5a)}{5a}\labeltextsc[P5]{}{5}}]%
    if $\bEv=(\DR[]{..)}{}$, $\aEv=(\DW[]{..)}{}$ then $\aEv=\bEv$ or
    $\labelingForm'(\aEv)$ implies $\labelingForm(\aEv)$ or $\bEv\lesync'\aEv$
  \item[{\labeltextsc[P5b]{(P5b)}{5b}}] if $\bEv$ conflicts with
    $\aEv$ %$\bEv\conflict\aEv$
    then $\bEv\leloc'\aEv$
  \item[{\labeltextsc[P5c]{(P5c)}{5c}}] if $\bEv$ is an acquire or $\aEv$ is
    a release then $\bEv \lesync' \aEv$
  \item[{\labeltextsc[P5d]{(P5d)}{5d}}] if $\bEv$ is an SC write and $\aEv$
    is an SC read then $\bEv \lesync' \aEv$
  \item[{\labeltextsc[P5e]{(P5e)}{5e}}] if $\bEv$ reads, and $\aEv$ is an
    acquiring fence, then
    $\bEv \lesync' \aEv$
  \item[{\labeltextsc[P5f]{(P5f)}{5f}}] if $\bEv$ is a releasing fence,
    and $\aEv$ writes, then
    $\bEv \lesync' \aEv$
  \end{description}
\end{definition}

% \section{More Model}
% These definitions need to be updated to include the additional orders.
% \begin{definition}
%   A pomset is \emph{$\aLoc$-closed} if
%   \begin{itemize}
%   \item every $\labelingAct(\aEv)=(\DR{\aLoc}{..})$ is fulfilled
%   \item every $\labelingForm(\aEv)$ is independent of $x$:
%     $\bigl(\forall v.\;\labelingForm(\aEv) \vDash
%     \labelingForm(\aEv)[\aVal/\aLoc] \vDash \labelingForm(\aEv)\bigr)$
%   \end{itemize}
% \end{definition}

% \begin{definition}
%   Let $\aPS\phantom{'}\in(\nu\aLoc\!\DOT\!\aPSS) \mkern22mu$ when
%   $\phantom{(\exists}\aPS\in\aPSS$ and $\aPS$ is $\aLoc$-closed
% \end{definition}
% \begin{definition}
%   Let $\aPS\phantom{'}\in(\aForm \guard \aPSS)\mkern16mu$ when
%   $\phantom{(\exists}\aPS\in\aPSS$ and $(\forall\aEv\in\Event)$
%   $\labelingForm(\aEv)$ implies $\aForm$
% \end{definition}

% \begin{definition}
%   Let $\aPS'\in(\aPSS[M/x])\mkern2mu$ when $(\exists\aPS\in\aPSS$)\\\qquad
%   $\Event' = \Event$, ${\lesync'} = {\lesync}$, $\labelingAct' = \labelingAct$, and
%   $(\forall\aEv\in\Event')$ $\labelingForm'(\aEv) = \labelingForm(\aEv)[M/x]$
% \end{definition}
% \begin{definition}
%   Let $\aPS' \in (\aPSS^1 \parallel \aPSS^2)$ when
%   $(\exists\aPS^1 \in \aPSS^1)$ $(\exists\aPS^2 \in \aPSS^2)$
%   \\% $\aPS^1$ is completed exactly when $\aPS^2$ is completed, there is at most one termination in $\Event'$,
%   \qquad $\Event' = \Event^1 \cup \Event^2$,
%   ${\lesync'}\supseteq{\lesync^1}\cup{\lesync^2}$, and $(\forall\aEv\in\Event')$ either
%   \begin{gather*}
%     \begin{aligned}
%       \aEv \not\in \Event^2,\; \labelingAct'(\aEv) &= \labelingAct^1(\aEv)
%       \textand \labelingForm'(\aEv) \textimplies \labelingForm^1(\aEv),
%       \\[-1ex]
%       \aEv \not\in \Event^1,\; \labelingAct'(\aEv) &= \labelingAct^2(\aEv)
%       \textand \labelingForm'(\aEv) \textimplies
%       \labelingForm^2(\aEv),\textor
%       \\[-1ex]
%       \labelingAct'(\aEv) = \labelingAct^1(\aEv) &= \labelingAct^2(\aEv)
%       \textand \labelingForm'(\aEv) \textimplies \labelingForm^1(\aEv) \lor
%       \labelingForm^2(\aEv)
%     \end{aligned}
%   \end{gather*}
% \end{definition}

% Language
% \begin{gather*}
%   % \begin{aligned}
%   %   \aCmd,\,\bCmd
%   %   \BNFDEF& \SKIP
%   %   \BNFSEP \aReg\GETS\aExp\SEMI \aCmd
%   %   \BNFSEP \aReg\GETS\aLoc^{\amode}\SEMI \aCmd 
%   %   \BNFSEP \aLoc^{\amode}\GETS\aExp\SEMI \aCmd
%   %   \\[-.5ex]
%   %   \BNFSEP&\aCmd \PAR[\aThrd][\bThrd] \bCmd
%   %   \BNFSEP \VAR\aLoc\SEMI \aCmd
%   %   \BNFSEP \IF{\aExp} \THEN \aCmd \ELSE \bCmd \FI
%   % \end{aligned}
%   % \\
%   \begin{aligned}
%     \sem[\aThrd]{\SKIP} & \eqdef
%     \{ \DSTOP \}
%     \\
%     \sem[\aThrd]{\aReg\GETS\aExp\SEMI \aCmd} & \eqdef
%     \sem[\aThrd]{\aCmd}[\aExp/\aReg]
%     \\ 
%     \sem{\aReg\GETS\aLoc^{\amode}\SEMI \aCmd} & \eqdef \textstyle\bigcup_\aVal\;
%     (\DR[\amode]\aLoc\aVal)\prefix \sem{\aCmd} [\aLoc/\aReg]
%     \\
%     \sem{\aLoc^{\amode}\GETS\aExp\SEMI \aCmd} & \eqdef
%     \textstyle\bigcup_\aVal\; (\aExp=\aVal \mid \DW[\amode]\aLoc\aVal)\prefix \sem{\aCmd}[\aExp/\aLoc]
%     \\
%     \sem{\FENCE^{\fmode}\SEMI \aCmd} & \eqdef
%     (\DFS{\fmode}) \prefix \sem{\aCmd}
%     \\
%     \sem[\aThrd]{\IF{\aExp} \THEN \aCmd_1 \ELSE \aCmd_2 \FI} & \eqdef
%     \bigl(\aExp \guard \sem[\aThrd]{\aCmd_1}\bigr) \parallel \bigl(\lnot\aExp \guard \sem[\aThrd]{\aCmd_2}\bigr) 
%     \\
%     \sem[\aThrd]{\aCmd_1 \PAR[\bThrd][\bThrd'] \aCmd_2} & \eqdef
%     \sem[\bThrd]{\aCmd_1} \parallel \sem[\bThrd']{\aCmd_2} 
%     \\
%     \sem[\aThrd]{\VAR\aLoc\SEMI \aCmd} & \eqdef
%     \nu \aLoc \DOT \sem[\aThrd]{\aCmd}  
%   \end{aligned}
% \end{gather*}
% \begin{align*}
%   \fmode \BNFDEF&\fREL  &&\text{(Release)} &\hbox{$\;\mkern60mu\;$}&
%   \\ \BNFSEP&\fACQ   &&\text{(Acquire)} 
%   \\      \BNFSEP&\mSC  &&\text{(SC)} 
% \end{align*}





