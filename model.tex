\section{Notes}
GPU stuff:
\begin{itemize}
\item Vulcan/Alloy
\item OpenCL
\item AMD PTX
\item Matthew Sinclair/Sarita Adve stuff ``Chasing Away RAts- Semantics and
  Evaluation for Relaxed Atomics on Heterogeneous Systems'' and his thesis
\end{itemize}

\section{Anton's recent examples relating IMM and PTX}
It looks like we cannot prove compilation correctness from IMM to PTX.
(In this email I assume that all threads are in the same CTA, so any relation is a morally strong one if it is applicable.)
The problem is in the LB-data-rel example:
\begin{comment}
a := [x]  || b := [y]
[y] := a  || [x]_rel := 1
\end{comment}
\begin{gather*}
  \PR{x}{r}\SEMI
  \PW{y}{r}
  \PAR
  \PR{y}{s}\SEMI
  \PW[\mRA]{x}{1}
  \\
  \hbox{\begin{tikzinline}[node distance=1.5em]
      \event{a}{\DR{x}{1}}{}
      \event{b}{\DW{y}{1}}{right=of a}
      \event{c}{\DR{y}{1}}{right=3em of b}
      \event{d}{\DW[\mRA]{x}{1}}{right=of c}
      \data{a}{b}
      \rfe{b}{c}
      \bob{c}{d}
      \rfe[out=-165,in=-15]{d}[below]{a}
    \end{tikzinline}}
\end{gather*}

IMM forbids it, but PTX allows it. The point is that IMM mixes dependencies and release/acquire-induced po-order in its NoOOTA axiom,
whereas PTX doesn't --- release/acquire are only used to have coherence.

The problem is related to the one we have already discussed in the context of the C++ model -- if you don't have acquire reads in the
program, then you can erase release annotations from writes. In this regard, PTX is closer to PL memory models than to hardware ones.

AFAIU for the same reason we won't be able to show compilation correctness from the Pomset model to PTX even directly,
if the Pomset model mixes release/acquire induced order with dependencies in the same causality relation.

Another oddity: PTX includes the $\rbob$ edge below; IMM does not.
\begin{gather*}
  \PW[\mRA]{x}{1}
  \PAR
  \PR{x}{r}\SEMI
  \PW{x}{1}\SEMI
  \PR[\mRA]{x}{1}
  \\
  \hbox{\begin{tikzinline}[node distance=1.5em]
      \event{a}{\DW[\mRA]{x}{1}}{}
      \event{b}{\DR{x}{1}}{right=3em of a}
      \event{c}{\DW{x}{1}}{right=of b}
      \event{d}{\DR[\mRA]{x}{1}}{right=of c}
      \rfe{a}{b}
      \wk{b}{c}
      \rfi{c}{d}
      \bob[in=-165,out=-15]{a}[below]{d}
    \end{tikzinline}}
\end{gather*}

\section{Thin Air}

Need $\lestrong$ to prevent thin air on $\mRLX$:
\begin{gather*}
  \PW{y}{\PR{x}{}}\PAR
  \PW{x}{\PR{y}{}}
  \\
  \tag{$\lestrong$}
  \hbox{\begin{tikzinline}[node distance=1.5em]
      \event{a}{\DR{x}{1}}{}
      \event{b}{\DW{y}{1}}{right=of a}
      \event{c}{\DR{y}{1}}{right=2.5em of b}
      \event{d}{\DW{x}{1}}{right=of c}
      \po{a}{b}
      \rf{b}{c}
      \po{c}{d}
      \rf[out=-165,in=-15]{d}{a}
    \end{tikzinline}}
  \\
  \tag{$\lehb$}
  \hbox{\begin{tikzinline}[node distance=1.5em]
      \event{a}{\DR{x}{1}}{}
      \event{b}{\DW{y}{1}}{right=of a}
      \event{c}{\DR{y}{1}}{right=2.5em of b}
      \event{d}{\DW{x}{1}}{right=of c}
      \po{a}{b}
      \po{c}{d}
    \end{tikzinline}}
  \\
  \tag{$\lelocstrong$}
  \hbox{\begin{tikzinline}[node distance=1.5em]
      \event{a}{\DR{x}{1}}{}
      \event{b}{\DW{y}{1}}{right=of a}
      \event{c}{\DR{y}{1}}{right=2.5em of b}
      \event{d}{\DW{x}{1}}{right=of c}
      \rf{b}{c}
      \rf[out=-165,in=-15]{d}{a}
    \end{tikzinline}}
\end{gather*}

\section{IMM Examples}

Interpreting this definition for the \IMM:
\begin{itemize}
\item No $\mWK$, default is $\mRLX$
\item All threads in same $\sCTA$ (only one scope)
\item Actions are morally strong when both are $\mRA$/$\mSC$, mimicking happens-before
\item Strong fulfillment may do the right thing 
\end{itemize}

Disallowed by \IMM{}:
\begin{gather*}
  \taglabel{pub-rel-acq-coe}
  \PW{x}{2}\SEMI 
  \PW[\mRA]{y}{1} \PAR
  \PR[\mRA]{y}{r}\SEMI
  \PW{x}{1}
  \\
  \tag{\xmark\IMM}
  \hbox{\begin{tikzinline}[node distance=1.5em]
      \event{a}{\DW{x}{2}}{}
      \event{b}{\DW[\mRA]{y}{1}}{right=of a}
      \event{c}{\DR[\mRA]{y}{1}}{right=2.5em of b}
      \event{d}{\DW{x}{1}}{right=of c}
      \bob{a}{b}
      \rfe{b}{c}
      \bob{c}{d}
      \coe[out=-165,in=-15]{d}{a}
    \end{tikzinline}}
  \\
  \tag{${\lestrong}={\lehb}$}
  \hbox{\begin{tikzinline}[node distance=1.5em]
      \event{a}{\DW{x}{2}}{}
      \event{b}{\DW[\mRA]{y}{1}}{right=of a}
      \event{c}{\DR[\mRA]{y}{1}}{right=2.5em of b}
      \event{d}{\DW{x}{1}}{right=of c}
      \sync{a}{b}
      \rf{b}{c}
      \sync{c}{d}
      %\wk[out=-165,in=-15]{d}{a}
    \end{tikzinline}}
  \\
  \tag{$\lelocstrong$}
  \hbox{\begin{tikzinline}[node distance=1.5em]
      \event{a}{\DW{x}{2}}{}
      \event{b}{\DW[\mRA]{y}{1}}{right=of a}
      \event{c}{\DR[\mRA]{y}{1}}{right=2.5em of b}
      \event{d}{\DW{x}{1}}{right=of c}
      %\sync{a}{b}
      \rf{b}{c}
      %\sync{c}{d}
      \po[out=15,in=165]{a}{d}
      %\wk[out=-165,in=-15]{d}{a}
    \end{tikzinline}}
\end{gather*}

Allowed by \IMM, but not by Power/ARMv7/ARMv8/TSO:
\begin{gather*}
  \taglabel{pub-rel-rlx-coe}
  \PW{x}{2}\SEMI 
  \PW[\mRA]{y}{1} \PAR
  \PR{y}{r}\SEMI
  \PW{x}{1}
  \\
  \tag{\cmark\IMM}
  \hbox{\begin{tikzinline}[node distance=1.5em]
      \event{a}{\DW{x}{2}}{}
      \event{b}{\DW[\mRA]{y}{1}}{right=of a}
      \event{c}{\DR{y}{1}}{right=2.5em of b}
      \event{d}{\DW{x}{1}}{right=of c}
      \bob{a}{b}
      \rfe{b}{c}
      \data{c}{d}
      \coe[out=-165,in=-15]{d}{a}
    \end{tikzinline}}
  \\
  \tag{$\lestrong$}
  \hbox{\begin{tikzinline}[node distance=1.5em]
      \event{a}{\DW{x}{2}}{}
      \event{b}{\DW[\mRA]{y}{1}}{right=of a}
      \event{c}{\DR{y}{1}}{right=2.5em of b}
      \event{d}{\DW{x}{1}}{right=of c}
      \sync{a}{b}
      \rf{b}{c}
      \po{c}{d}
      %\wk[out=-165,in=-15]{d}{a}
    \end{tikzinline}}
  \\
  \tag{$\lehb$}
  \hbox{\begin{tikzinline}[node distance=1.5em]
      \event{a}{\DW{x}{2}}{}
      \event{b}{\DW[\mRA]{y}{1}}{right=of a}
      \event{c}{\DR{y}{1}}{right=2.5em of b}
      \event{d}{\DW{x}{1}}{right=of c}
      \sync{a}{b}
      %\rfe{b}{c}
      \po{c}{d}
      %\wk[out=-165,in=-15]{d}{a}
    \end{tikzinline}}
  \\
  \tag{$\lelocstrong$}
  \hbox{\begin{tikzinline}[node distance=1.5em]
      \event{a}{\DW{x}{2}}{}
      \event{b}{\DW[\mRA]{y}{1}}{right=of a}
      \event{c}{\DR{y}{1}}{right=2.5em of b}
      \event{d}{\DW{x}{1}}{right=of c}
      %\sync{a}{b}
      \rf{b}{c}
      %\po{c}{d}
      %\wk[out=-165,in=-15]{d}{a}
    \end{tikzinline}}
\end{gather*}


Example from talk:
\begin{gather*}
  \taglabel{arm7-weak}
  r\GETS x\SEMI x\GETS 1
  \PAR
  y\GETS x 
  \PAR
  x\GETS y 
  \\[-1.2ex]
  \tag{$\lestrong$}
  \hbox{\begin{tikzinline}[node distance=1.5em]
      \event{a}{\DR{x}{1}}{}
      \event{b}{d:\DW{x}{1}}{right=of a}
      %\wk{a}{b}
      \event{c}{\DR{x}{1}}{right=3em of b}
      \event{d}{\DW{y}{1}}{right=of c}
      \po{c}{d}
      \event{e}{\DR{y}{1}}{right=3em of d}
      \event{f}{e:\DW{x}{1}}{right=of e}
      \po{e}{f}
      \rf{b}{c}
      \rf{d}{e}
      \rf[out=172,in=8]{f}{a}
    \end{tikzinline}}
  \\
  \tag{$\lehb$}
  \hbox{\begin{tikzinline}[node distance=1.5em]
      \event{a}{\DR{x}{1}}{}
      \event{b}{d:\DW{x}{1}}{right=of a}
      %\wk{a}{b}
      \event{c}{\DR{x}{1}}{right=3em of b}
      \event{d}{\DW{y}{1}}{right=of c}
      \po{c}{d}
      \event{e}{\DR{y}{1}}{right=3em of d}
      \event{f}{e:\DW{x}{1}}{right=of e}
      \po{e}{f}
      %\rf{b}{c}
      %\rf{d}{e}
      %\rf[out=172,in=8]{f}{a}
    \end{tikzinline}}
  \\
  \tag{$\lelocstrong$}
  \hbox{\begin{tikzinline}[node distance=1.5em]
      \event{a}{\DR{x}{1}}{}
      \event{b}{d:\DW{x}{1}}{right=of a}
      \wk{a}{b}
      \event{c}{\DR{x}{1}}{right=3em of b}
      \event{d}{\DW{y}{1}}{right=of c}
      %\po{c}{d}
      \event{e}{\DR{y}{1}}{right=3em of d}
      \event{f}{e:\DW{x}{1}}{right=of e}
      % \po{e}{f}
      %\po[out=-15,in=-165]{c}{f}
      \rf{b}{c}
      \rf{d}{e}
      \rf[out=172,in=8]{f}{a}
    \end{tikzinline}}
  \\
  \tag{$\lelocweak$}
  \hbox{\begin{tikzinline}[node distance=1.5em]
      \event{a}{\DR{x}{1}}{}
      \event{b}{d:\DW{x}{1}}{right=of a}
      \wk{a}{b}
      \event{c}{\DR{x}{1}}{right=3em of b}
      \event{d}{\DW{y}{1}}{right=of c}
      %\po{c}{d}
      \event{e}{\DR{y}{1}}{right=3em of d}
      \event{f}{e:\DW{x}{1}}{right=of e}
      % \po{e}{f}
      %\po[out=-15,in=-165]{c}{f}
      %\rf{b}{c}
      %\rf{d}{e}
      %\rf[out=172,in=8]{f}{a}
    \end{tikzinline}}
\end{gather*}

\section{Two order idea}
The two order idea from OOPSLA talk is:
\begin{itemize}
\item Require: $\bEv\leloc\aEv$ when $\bEv\lestrong\aEv$ and they conflict
\end{itemize}
This does not work for the \IMM{} or ARMv7, but it may work for Power, TSO,
ARMv8.  That would be nice.  Let's write $\leloctwo$ for this notion, with
strong fulfillment.

With this there is a cycle in \ref{arm7-weak} (weak/strong fulfillment not relevant here):
\begin{gather*}
  \tag{$\leloctwo$}
  \hbox{\begin{tikzinline}[node distance=1.5em]
      \event{a}{\DR{x}{1}}{}
      \event{b}{d:\DW{x}{1}}{right=of a}
      \wk{a}{b}
      \event{c}{\DR{x}{1}}{right=3em of b}
      \event{d}{\DW{y}{1}}{right=of c}
      %\po{c}{d}
      \event{e}{\DR{y}{1}}{right=3em of d}
      \event{f}{e:\DW{x}{1}}{right=of e}
      % \po{e}{f}
      \po[out=-15,in=-165]{c}{f}
      \rf{b}{c}
      \rf{d}{e}
      \rf[out=172,in=8]{f}{a}
    \end{tikzinline}}    
\end{gather*}
Anton says: \ref{arm7-weak} is forbidden by Power, TSO, ARMv8, but allowed by
ARMv7. Maybe it isn't that important to support it anymore.

There is also a cycle in \ref{pub-rel-rlx-coe}.  Anton says: I checked
Power/ARMv7 models in this regard. They disallow the behavior (as well as
ARMv8 and TSO), so we can in principle strengthen \IMM{} to forbid it as
well.  For that, we may add axiom to \IMM{} forbidding cycles in
\begin{math}
  \rco \cup ([\mathsf{W}]; \rrfe^?; ([\mathsf{R}^{\mACQ}] \cup \rpox;
  [\mathsf{FW}^{\mREL}]); \rar^{*}; [\mathsf{W}]).
\end{math}
This works if we have acquire/release accesses on the path
since they are compiled with fences to Power.


\section{PTX Examples}
Based on \cite{DBLP:conf/asplos/LustigSG19,nvidia}.

\PTX{} requires weak fulfillment.

Default scope is $\sCTA$.  In examples, all threads in different $\sCTA$s.

Default mode is $\mWK$.


$(\DR{x}{0})$ must be forbidden.
Before fulfilling the read:
\begin{gather*}
  \taglabel[sys]{pub1}
  \PW{x}{0}\SEMI 
  \PW{x}{1}\SEMI
  \PW[\mRA]{y}[\sSYS]{1} \PAR
  \PR[\mRA]{y}[\sSYS]{r}\SEMI
  \PR{x}{s}
  \\
  \tag{${\lestrong}={\lehb}$}
  \hbox{\begin{tikzinline}[node distance=1.5em]
      \event{wx0}{\DW{x}{0}}{}
      \event{wx1}{\DW{x}{1}}{right=of wx0}
      \event{wy1}{\DW[\mRA]{y}[\sSYS]{1}}{right=of wx1}
      \event{ry1}{\DR[\mRA]{y}[\sSYS]{1}}{right=2.5em of wy1}
      \event{rx}{\DR{x}{}}{right=of ry1}
      \sync[out=-15,in=-165]{wx0}{wy1}
      \sync{wx1}{wy1}
      \sync{ry1}{rx}
      \rf{wy1}{ry1}
    \end{tikzinline}}
  \\
  \tag{$\leloc$}
  \hbox{\begin{tikzinline}[node distance=1.5em]
      \event{wx0}{\DW{x}{0}}{}
      \event{wx1}{\DW{x}{1}}{right=of wx0}
      \event{wy1}{\DW[\mRA]{y}[\sSYS]{1}}{right=of wx1}
      \event{ry1}{\DR[\mRA]{y}[\sSYS]{1}}{right=2.5em of wy1}
      \event{rx}{\DR{x}{}}{right=of ry1}
      \rf{wy1}{ry1}
      \wk{wx0}{wx1}
      \po[out=-15,in=-165]{wx1}{rx}
    \end{tikzinline}}
\end{gather*}
$(\DW{x}{1})\leexists(\DR{x}{})$ is required by \ref{m-loc}, enforcing publication.

$(\DR{x}{0})$ must be allowed:
\begin{gather*}
  \taglabel[cta]{pub1}
  \PW{x}{0}\SEMI 
  \PW{x}{1}\SEMI
  \PW[\mRA]{y}{1} \PAR
  \PR[\mRA]{y}{r}\SEMI
  \PR{x}{s}
  \\
  \tag{${\lestrong}$}
  \hbox{\begin{tikzinline}[node distance=1.5em]
      \event{wx0}{\DW{x}{0}}{}
      \event{wx1}{\DW{x}{1}}{right=of wx0}
      \event{wy1}{\DW[\mRA]{y}{1}}{right=of wx1}
      \event{ry1}{\DR[\mRA]{y}{1}}{right=2.5em of wy1}
      \event{rx}{\DR{x}{}}{right=of ry1}
      \sync[out=-15,in=-165]{wx0}{wy1}
      \sync{wx1}{wy1}
      \sync{ry1}{rx}
      \rf{wy1}{ry1}
    \end{tikzinline}}
  \\
  \tag{${\lehb}$}
  \hbox{\begin{tikzinline}[node distance=1.5em]
      \event{wx0}{\DW{x}{0}}{}
      \event{wx1}{\DW{x}{1}}{right=of wx0}
      \event{wy1}{\DW[\mRA]{y}{1}}{right=of wx1}
      \event{ry1}{\DR[\mRA]{y}{1}}{right=2.5em of wy1}
      \event{rx}{\DR{x}{}}{right=of ry1}
      \sync[out=-15,in=-165]{wx0}{wy1}
      \sync{wx1}{wy1}
      \sync{ry1}{rx}
      % \rf{wy1}{ry1}
    \end{tikzinline}}
  \\
  \tag{$\leloc$}
  \hbox{\begin{tikzinline}[node distance=1.5em]
      \event{wx0}{\DW{x}{0}}{}
      \event{wx1}{\DW{x}{1}}{right=of wx0}
      \event{wy1}{\DW[\mRA]{y}{1}}{right=of wx1}
      \event{ry1}{\DR[\mRA]{y}{1}}{right=2.5em of wy1}
      \event{rx}{\DR{x}{}}{right=of ry1}
      % \rf{wy1}{ry1}
      \wk{wx0}{wx1}
      % \wk[out=-15,in=-165]{wx1}{rx}
    \end{tikzinline}}
\end{gather*}
We do not have $(\DW[\mRA]{y}{1})\lehb (\DR[\mRA]{y}{1})$ since \ref{rf3} only
requires order for things that are morally strong.  

Another example that may be of interest (nothing morally strong).  Can this $(\DR{x}{0})$?
\begin{gather*}
  %\taglabel[cta]{pub1}
  \PW{x}{0} \SEMI
  \PW{x}{1} \PAR 
  \PW{y}{\PR{x}{}} \PAR
  \IF{\PR{y}{}}\THEN \PR{x}{r} \FI
\end{gather*}

\PTX{} allows TC16 for events that are not mutually strong (\ref{tc16wk}),
but disallows it when events are mutually strong (\ref{tc16sys}).  Note that
$\lehb$ imposes no requirements here.  Fulfillment imposes no order.  This
example shows that \ref{rf3c} cannot be strengthened to require that
$\bEv\leloc\aEv$.
\begin{gather*}
  \taglabel[wk]{tc16}
  \PR{x}{r} \SEMI \PW{x}{1}
  \PAR
  \PR{x}{s} \SEMI \PW{x}{2}
  \\
  \tag{$\lestrong$}
  \hbox{\begin{tikzinline}[node distance=1.5em]
      \event{a1}{\DR{x}{2}}{}
      \event{a2}{\DW{x}{1}}{right=of a1}
      \event{b1}{\DR{x}{1}}{right=3em of a2}
      \event{b2}{\DW{x}{2}}{right=of b1}
      \rf{a2}{b1}
      \rf[out=-165,in=-15]{b2}{a1}
    \end{tikzinline}}
  \\
  \tag{$\lehb$}
  \hbox{\begin{tikzinline}[node distance=1.5em]
      \event{a1}{\DR{x}{2}}{}
      \event{a2}{\DW{x}{1}}{right=of a1}
      \event{b1}{\DR{x}{1}}{right=3em of a2}
      \event{b2}{\DW{x}{2}}{right=of b1}
    \end{tikzinline}}
  \\
  \tag{$\leloc$}
  \hbox{\begin{tikzinline}[node distance=1.5em]
      \event{a1}{\DR{x}{2}}{}
      \event{a2}{\DW{x}{1}}{right=of a1}
      \event{b1}{\DR{x}{1}}{right=3em of a2}
      \event{b2}{\DW{x}{2}}{right=of b1}
      \wk{a1}{a2}
      \wk{b1}{b2}
      % \rf{a2}{b1}
      % \rf[out=-165,in=-15]{b2}{a1}
    \end{tikzinline}}
\end{gather*}
\begin{gather*}
  \taglabel[sys]{tc16}
  \PR[\mRLX]{x}[\sSYS]{r} \SEMI \PW[\mRLX]{x}[\sSYS]{1}
  \PAR                                              
  \PR[\mRLX]{x}[\sSYS]{s} \SEMI \PW[\mRLX]{x}[\sSYS]{2}
  \\
  \tag{${\lestrong}={\lehb}$}
  \hbox{\begin{tikzinline}[node distance=1.5em]
      \event{a1}{\DR[\mRLX]{x}[\sSYS]{2}}{}
      \event{a2}{\DW[\mRLX]{x}[\sSYS]{1}}{right=of a1}
      \event{b1}{\DR[\mRLX]{x}[\sSYS]{1}}{right=3em of a2}
      \event{b2}{\DW[\mRLX]{x}[\sSYS]{2}}{right=of b1}
      \rf{a2}{b1}
      \rf[out=-165,in=-15]{b2}{a1}
    \end{tikzinline}}
  \\
  \tag{$\leloc$}
  \hbox{\begin{tikzinline}[node distance=1.5em]
      \event{a1}{\DR[\mRLX]{x}[\sSYS]{2}}{}
      \event{a2}{\DW[\mRLX]{x}[\sSYS]{1}}{right=of a1}
      \event{b1}{\DR[\mRLX]{x}[\sSYS]{1}}{right=3em of a2}
      \event{b2}{\DW[\mRLX]{x}[\sSYS]{2}}{right=of b1}
      \wk{a1}{a2}
      \wk{b1}{b2}
      \rf{a2}{b1}
      \rf[out=-165,in=-15]{b2}{a1}
    \end{tikzinline}}
\end{gather*}

About Release-Acquire semantics.  Anton confirms that the following example
is allowed in C11, but disallowed in the \IMM{}.  It is apparently allowed in
C11 with the intention to allow releasing writes to be downgraded to relaxed
in the case that only fulfill relaxed reads.
\begin{gather*}
  \taglabel{LB-REL}
  \PR[\mRLX]{x}[\sSYS]{r} \SEMI \PW[\mRA]{y}[\sSYS]{1}
  \PAR                                             
  \PR[\mRLX]{y}[\sSYS]{s} \SEMI \PW[\mRA]{x}[\sSYS]{1}
  \\
  \tag{${\lestrong}={\lehb}$}
  \hbox{\begin{tikzinline}[node distance=1.5em]
      \event{a1}{\DR[\mRLX]{x}[\sSYS]{1}}{}
      \event{a2}{\DW[\mRA]{y}[\sSYS]{1}}{right=of a1}
      \event{b1}{\DR[\mRLX]{y}[\sSYS]{1}}{right=3em of a2}
      \event{b2}{\DW[\mRA]{x}[\sSYS]{1}}{right=of b1}
      \rf{a2}{b1}
      \rf[out=-165,in=-15]{b2}{a1}
      \sync{a1}{a2}
      \sync{b1}{b2}
    \end{tikzinline}}
\end{gather*}

Another example from Anton.  This is allowed in PTX because it does not
include synchronization in the no-tar axiom, only in coherence and causality.
\begin{gather*}
  \taglabel{LB-data-rel}
  \PR[\mRLX]{x}[\sSYS]{r} \SEMI \PW[\mRLX]{y}[\sSYS]{r}
  \PAR                                             
  \PR[\mRLX]{y}[\sSYS]{s} \SEMI \PW[\mRA]{x}[\sSYS]{1}
  \\
  \tag{${\lestrong}={\lehb}$}
  \hbox{\begin{tikzinline}[node distance=1.5em]
      \event{a1}{\DR[\mRLX]{x}[\sSYS]{1}}{}
      \event{a2}{\DW[\mRLX]{y}[\sSYS]{1}}{right=of a1}
      \event{b1}{\DR[\mRLX]{y}[\sSYS]{1}}{right=3em of a2}
      \event{b2}{\DW[\mRA]{x}[\sSYS]{1}}{right=of b1}
      \rf{a2}{b1}
      \rf[out=-165,in=-15]{b2}{a1}
      \po{a1}{a2}
      \sync{b1}{b2}
    \end{tikzinline}}
\end{gather*}


\section{RFI Examples}

Anton example 1 (Allowed by ARM) \texttt{[rfi-coe-coe]}
\begin{gather*}
  \taglabel{rfi-coe-coe}
  \PW{x}{2}\SEMI 
  \PR[\mRA]{x}{r}\SEMI
  \PW{y}{1} \PAR
  \PW{y}{2}\SEMI
  \PW[\mRA]{x}{1}
  \\
  \tag{\cmark\armeight}
  \hbox{\begin{tikzinline}[node distance=1.5em]
      \event{a}{\DW{x}{2}}{}
      \event{b}{\DR[\mRA]{x}{2}}{right=of a}
      \event{c}{\DW{y}{1}}{right=of b}
      \event{d}{\DW{y}{2}}{right=2.5em of c}
      \event{e}{\DW[\mRA]{x}{1}}{right=of d}
      \rfi{a}{b}
      \bob{b}{c}
      \coe{c}{d}
      \bob{d}{e}
      \coe[out=-165,in=-15]{e}{a}
    \end{tikzinline}}
\end{gather*}
Internal reads survive acquires \texttt{[rfi-acq-coe-coe]} (where SC read =
\texttt{LDAR})
\begin{gather*}
  \taglabel{rfi-acq-coe-coe}
  \PW{x}{2}\SEMI 
  \PR[\mSC]{z}{s}\SEMI
  \PR[\mSC]{x}{r}\SEMI
  \PW{y}{1} \PAR
  \PW{y}{2}\SEMI
  \PW[\mRA]{x}{1}
  \\
  \tag{\cmark\armeight}
  \hbox{\begin{tikzinline}[node distance=1.5em]
      \event{a}{\DW{x}{2}}{}
      \event{b0}{\DR[\mSC]{z}{0}}{right=of a}
      \event{b}{\DR[\mSC]{x}{2}}{right=of b0}
      \event{c}{\DW{y}{1}}{right=of b}
      \event{d}{\DW{y}{2}}{right=2.5em of c}
      \event{e}{\DW[\mRA]{x}{1}}{right=of d}
      \rfi[out=20,in=160]{a}{b}
      %\bob[out=20,in=160]{b0}{c}
      \bob{b0}{b}
      \bob{b}{c}
      \coe{c}{d}
      \bob{d}{e}
      \coe[out=-165,in=-15]{e}{a}
    \end{tikzinline}}
\end{gather*}
And release-acquire pairs \texttt{[rfi-ra-coe-coe]} (where acquiring read
= \texttt{LDAPR})
\begin{gather*}
  \taglabel{rfi-ra-coe-coe2}
  \PW{x}{2}\SEMI 
  \PW[\mRA]{w}{1}\SEMI
  \PR[\mRA]{z}{s}\SEMI
  \PR[\mRA]{x}{r}\SEMI
  \PW{y}{1}
  \\[-1ex]\PAR
  \PW{y}{2}\SEMI
  \PW[\mRA]{x}{1}
  \PAR
  \PR{r}{w}\SEMI
  \PW{r}{1}\SEMI
  \\
  \tag{\cmark\armeight}
  \hbox{\begin{tikzinline}[node distance=1.5em]
      \event{a}{\DW{x}{2}}{}
      \event{b1}{\DW[\mRA]{w}{1}}{right=of a}
      \event{b0}{\DR[\mRA]{z}{1}}{right=of b1}
      \event{b}{\DR[\mRA]{x}{2}}{right=of b0}
      \event{c}{\DW{y}{1}}{right=of b}
      \event{d}{\DW{y}{2}}{right=2.5em of c}
      \event{e}{\DW[\mRA]{x}{1}}{right=of d}
      \rfi[out=20,in=160]{a}{b}
      %\bob[out=20,in=160]{b0}{c}
      \bob{a}{b1}
      %\bob{b1}{b0}
      \bob{b0}{b}
      \bob{b}{c}
      \coe{c}{d}
      \bob{d}{e}
      \coe[out=-165,in=-15]{e}{a}
      \event{f1}{\DR{w}{1}}{below=of b1}
      \event{f0}{\DW{z}{1}}{below=of b0}
      %\data{f1}{f0}
      \rfe{b1}{f1}
      \rfe{f0}{b0}
    \end{tikzinline}}
\end{gather*}
% \begin{gather*}
%   \taglabel{rfi-ra-coe-coe}
%   \PW{x}{2}\SEMI 
%   \PW[\mRA]{w}{1}\SEMI
%   \PR[\mRA]{z}{s}\SEMI
%   \PR[\mRA]{x}{r}\SEMI
%   \PW{y}{1} \PAR
%   \PW{y}{2}\SEMI
%   \PW[\mRA]{x}{1}
%   \\
%   \tag{\cmark\armeight}
%   \hbox{\begin{tikzinline}[node distance=1.5em]
%       \event{a}{\DW{x}{2}}{}
%       \event{b1}{\DW[\mRA]{w}{1}}{right=of a}
%       \event{b0}{\DR[\mRA]{z}{0}}{right=of b1}
%       \event{b}{\DR[\mRA]{x}{2}}{right=of b0}
%       \event{c}{\DW{y}{1}}{right=of b}
%       \event{d}{\DW{y}{2}}{right=2.5em of c}
%       \event{e}{\DW[\mRA]{x}{1}}{right=of d}
%       \rfi[out=20,in=160]{a}{b}
%       %\bob[out=20,in=160]{b0}{c}
%       \bob{a}{b1}
%       %\bob{b1}{b0}
%       \bob{b0}{b}
%       \bob{b}{c}
%       \coe{c}{d}
%       \bob{d}{e}
%       \coe[out=-165,in=-15]{e}{a}
%     \end{tikzinline}}
% \end{gather*}
But not if either acquire is strengthened to SC (where SC read =
\texttt{LDAR}).  The execution is also disallowed if an external thread
places order between the $\mRA$ accesses:
\begin{gather*}
  \taglabel{rfi-ra-data-coe-coe}
  \PW{x}{2}\SEMI 
  \PW[\mRA]{w}{1}\SEMI
  \PR[\mRA]{z}{s}\SEMI
  \PR[\mRA]{x}{r}\SEMI
  \PW{y}{1}
  \\[-1ex]\PAR
  \PW{y}{2}\SEMI
  \PW[\mRA]{x}{1}
  \PAR
  \PR{r}{w}\SEMI
  \PW{r}{z}\SEMI
  \\
  \tag{\xmark\armeight}
  \hbox{\begin{tikzinline}[node distance=1.5em]
      \event{a}{\DW{x}{2}}{}
      \event{b1}{\DW[\mRA]{w}{1}}{right=of a}
      \event{b0}{\DR[\mRA]{z}{1}}{right=of b1}
      \event{b}{\DR[\mRA]{x}{2}}{right=of b0}
      \event{c}{\DW{y}{1}}{right=of b}
      \event{d}{\DW{y}{2}}{right=2.5em of c}
      \event{e}{\DW[\mRA]{x}{1}}{right=of d}
      \rfi[out=20,in=160]{a}{b}
      %\bob[out=20,in=160]{b0}{c}
      \bob{a}{b1}
      %\bob{b1}{b0}
      \bob{b0}{b}
      \bob{b}{c}
      \coe{c}{d}
      \bob{d}{e}
      \coe[out=-165,in=-15]{e}{a}
      \event{f1}{\DR{w}{1}}{below=of b1}
      \event{f0}{\DW{z}{1}}{below=of b0}
      \data{f1}{f0}
      \rfe{b1}{f1}
      \rfe{f0}{b0}
    \end{tikzinline}}
\end{gather*}

To allow this, weaken $\mRA$ to $\mRLX$ when read fulfilled by relaxed write
of same thread (don't need to allow this when the write is part of an \RMW{}).
\begin{gather*}
  \PW{x}{2}\SEMI 
  \PR[\mRA]{x}{r}\SEMI
  \PW{y}{1} \PAR
  \PW{y}{2}\SEMI
  \PW[\mRA]{x}{1}
  \\
  \hbox{\begin{tikzinline}[node distance=1.5em]
      \event{a}{\DW{x}{2}}{}
      \event{b}{\DR{x}{2}}{right=of a}
      \event{c}{\DW{y}{1}}{right=of b}
      \event{d}{\DW{y}{2}}{right=2.5em of c}
      \event{e}{\DW[\mRA]{x}{1}}{right=of d}
      \rf{a}{b}
      %\sync{b}{c}
      \wk{c}{d}
      \sync{d}{e}
      \wk[out=-165,in=-15]{e}{a}
    \end{tikzinline}}
\end{gather*}

RF variant \texttt{[rfi-rfe-coe]}:
\begin{gather*}
  \taglabel{rfi-rfe-coe}
  \PW{x}{2}\SEMI 
  \PR[\mRA]{x}{r}\SEMI
  \PW{y}{1} \PAR
  \PR{y}{s}\SEMI
  \PW[\mRA]{x}{1}
  \\
  \tag{\cmark\armeight}
  \hbox{\begin{tikzinline}[node distance=1.5em]
      \event{a}{\DW{x}{2}}{}
      \event{b}{\DR[\mRA]{x}{2}}{right=of a}
      \event{c}{\DW{y}{1}}{right=of b}
      \event{d}{\DR{y}{1}}{right=2.5em of c}
      \event{e}{\DW[\mRA]{x}{1}}{right=of d}
      \rfi{a}{b}
      \bob{b}{c}
      \rfe{c}{d}
      \bob{d}{e}
      \coe[out=-165,in=-15]{e}{a}
    \end{tikzinline}}
\end{gather*}

\tso{} variant \texttt{[rfi-fre-coe]}:
\begin{gather*}
  \taglabel{rfi-coe-coe}
  \PW{x}{2}\SEMI 
  \PR[\mRA]{x}{r}\SEMI
  \PR{y}{s} \PAR
  \PW{y}{2}\SEMI
  \PW[\mRA]{x}{1}
  \\
  \tag{\cmark\armeight}
  \hbox{\begin{tikzinline}[node distance=1.5em]
      \event{a}{\DW{x}{2}}{}
      \event{b}{\DR[\mRA]{x}{2}}{right=of a}
      \event{c}{\DR{y}{0}}{right=of b}
      \event{d}{\DW{y}{2}}{right=2.5em of c}
      \event{e}{\DW[\mRA]{x}{1}}{right=of d}
      \rfi{a}{b}
      \bob{b}{c}
      \fre{c}{d}
      \bob{d}{e}
      \coe[out=-165,in=-15]{e}{a}
    \end{tikzinline}}
  \\
  \tag{\cmark\tso}
  \hbox{\begin{tikzinline}[node distance=1.5em]
      \event{a}{\DW{x}{2}}{}
      \event{b}{\DR{x}{2}}{right=of a}
      \event{c}{\DR{y}{0}}{right=of b}
      \event{d}{\DW{y}{2}}{right=2.5em of c}
      \event{e}{\DW{x}{1}}{right=of d}
      %\rfi{a}{b}
      \lob{b}{c}
      \fre{c}{d}
      \lob{d}{e}
      \coe[out=-165,in=-15]{e}{a}
    \end{tikzinline}}
\end{gather*}
Note that \tso{} does not order W to R in local order, even in poloc.
Nonetheless, \tso{} disallows the following because of local visibility in first thread.
\begin{gather*}
  \PW{x}{2}\SEMI 
  \PR{x}{r} \PAR
  \PW{x}{1}\SEMI
  \PR{x}{s}
  \\
  \tag{\xmark\tso}
  \hbox{\begin{tikzinline}[node distance=1.5em]
      \event{a}{\DW{x}{2}}{}
      \event{b}{\DR{x}{1}}{right=of a}
      \event{c}{\DW{x}{1}}{right=2.5em of b}
      \event{d}{\DR{x}{2}}{right=of c}
      \coe[out=-165,in=-15]{c}{a}
      \rfe{c}[above]{b}
      \rfe[out=15,in=165]{a}{d}
      \fr{b}[above]{a}
    \end{tikzinline}}
\end{gather*}
\cite{DBLP:conf/hipc/HighamK00} describe \tso{} as a linearization of partial
order including:
\begin{itemize}
\item ${\rpoloc}$
\item lws = ${\rpox};[\mathsf{W}]$
\item $\bEv\xpox\aEv$ when $\cEv\xrfe\bEv\xpox\aEv$
\end{itemize}
\cite{armed-cats} describe \tso{} as linearization of partial order
satisfying internal visibility and including
\begin{itemize}
\item $[\mathsf{W}];\rpox;[\mathsf{W}]$
\item $\bEv\xpox\aEv$ when $\cEv\xrfe\bEv\xpox\aEv$, from \verb|(range(rfe) * _)|
\item $[\mathsf{R}];\rpox;[\mathsf{W}]$, from \verb|(rfi^-1; lob)|
\end{itemize}
Ignoring fences and \RMW{}s:
\begin{verbatim}
let rec lob = po \ ([W]; po; [R])
let IM0 = loc & ((IW * (M\IW)) | ((W\FW) * FW))
let gc-req = (W * _) | ((R * _) & ((range(rfe) * _) | (rfi^-1; lob))
let preorder-gcb = IM0 | lob & gc-req
\end{verbatim}
% \begin{verbatim}
% let rec lob = po \ ([W]; po; [R])
%         | [W]; po; [MFENCE]; po; [R]
%         | [W]; po; [R & X]
%         | [W & X]; po; [R]
%         | lob; lob
% let IM0 = loc & ((IW * (M\IW)) | ((W\FW) * FW))
% let gc-req = (W * _) | ((R * _) & ((range(rfe) * _) | (rfi^-1; lob))
% let preorder-gcb = IM0 | lob & gc-req
% \end{verbatim}


Double FRE variant \texttt{[rfi-fre-fre]}:
\begin{gather*}
  \taglabel{rfi-fre-fre}
  \PW{x}{2}\SEMI 
  \PR[\mRA]{x}{r}\SEMI
  \PR{y}{s} \PAR
  \PW{y}{2}\SEMI
  \PF{}\SEMI
  \PR{x}{r}
  \\
  \tag{\cmark\armeight}
  \hbox{\begin{tikzinline}[node distance=1.5em]
      \event{a}{\DW{x}{2}}{}
      \event{b}{\DR[\mRA]{x}{2}}{right=of a}
      \event{c}{\DR{y}{0}}{right=of b}
      \event{d}{\DW{y}{2}}{right=2.5em of c}
      \event{e}{\DF{}}{right=of d}
      \event{f}{\DR{x}{0}}{right=of e}
      \rfi{a}{b}
      \bob{b}{c}
      \fre{c}{d}
      \bob{d}{e}
      \bob{e}{f}
      \fre[out=-165,in=-15]{f}{a}
    \end{tikzinline}}
\end{gather*}

It does not seem possible to do this only with $\rrfe$.
ARM disallows this \texttt{[data-rfi-rfe-rfe]}:
\begin{gather*}
  \taglabel{data-rfi-rfe-rfe}
  \PW{x}{\PR{z}{}} \SEMI
  \PR[\mRA]{x}{r}\SEMI
  \PW{y}{1} \PAR
  \PW{z}{\PR{y}{}}
  \\
  \tag{\xmark\armeight}
  \hbox{\begin{tikzinline}[node distance=1.5em]
      \event{a}{\DR{z}{1}}{}
      \event{b}{\DW{x}{1}}{right=of a}
      \event{c}{\DR[\mRA]{x}{1}}{right=of b}
      \event{d}{\DW{y}{1}}{right=of c}
      \event{e}{\DW{y}{1}}{right=2.5em of d}
      \event{f}{\DW{z}{1}}{right=of e}
      \data{a}{b}
      \rfi{b}{c}
      \bob{c}{d}
      \data{e}{f}
      \rfe[out=-165,in=-15]{f}{a}
      \rfe{d}{e}
    \end{tikzinline}}
\end{gather*}

It also disallows \texttt{[ctrl-rfi-rfe-rfe]}:
\begin{gather*}
  \taglabel{ctrl-rfi-rfe-rfe}
  \IF{\PR{z}{}}\THEN\FI \SEMI
  \PW{x}{1} \SEMI
  \PR[\mRA]{x}{r}\SEMI
  \PW{y}{1}
  \PAR
  \PW{z}{\PR{y}{}}
  \\
  \tag{\xmark\armeight}
  \hbox{\begin{tikzinline}[node distance=1.5em]
      \event{a}{\DR{z}{1}}{}
      \event{b}{\DW{x}{1}}{right=of a}
      \event{c}{\DR[\mRA]{x}{1}}{right=of b}
      \event{d}{\DW{y}{1}}{right=of c}
      \event{e}{\DW{y}{1}}{right=2.5em of d}
      \event{f}{\DW{z}{1}}{right=of e}
      \ctrl[out=15,in=165]{a}{d}
      \rfi{b}{c}
      \bob{c}[below]{d}
      \data{e}{f}
      \rfe[out=-165,in=-15]{f}{a}
      \rfe{d}{e}
    \end{tikzinline}}
\end{gather*}

ARM allows some counterintuitive results for SC access \texttt{[ctrl-rfi-fre-rfe]}:
\begin{gather*}
  \taglabel{ctrl-rfi-fre-rfe}
  \IF{\PR{x}{}}\THEN\FI\SEMI
  \PW{x}{2} \SEMI
  \PR[\mSC]{x}{r}\SEMI
  \PR[\mSC]{y}{s} \PAR
  \PW[\mSC]{y}{2}\SEMI
  \PW[\mSC]{x}{1}
  \\
  \tag{\cmark\armeight}
  \hbox{\begin{tikzinline}[node distance=1.5em]
      \event{a}{\DR{x}{1}}{}
      \event{b}{\DW{x}{2}}{right=of a}
      \event{c}{\DR[\mSC]{x}{2}}{right=of b}
      \event{d}{\DR[\mSC]{y}{0}}{right=of c}
      \event{e}{\DW[\mSC]{y}{2}}{right=2.5em of d}
      \event{f}{\DW[\mSC]{x}{1}}{right=of e}
      \ctrl{a}{b}
      \rfi{b}{c}
      \bob{c}{d}
      \bob{e}{f}
      \fre{d}{e}
      \rfe[out=-165,in=-15]{f}{a}
    \end{tikzinline}}
\end{gather*}
Not possible with $\rcoe$ \texttt{[ctrl-rfi-coe-rfe]}:
\begin{gather*}
  \taglabel{ctrl-rfi-coe-rfe}
  \IF{\PR{x}{}}\THEN\FI\SEMI
  \PW{x}{2} \SEMI
  \PR[\mSC]{x}{r}\SEMI
  \PW[\mSC]{y}{1} \PAR
  \PW[\mSC]{y}{2}\SEMI
  \PW[\mSC]{x}{1}
  \\
  \tag{\xmark\armeight}
  \hbox{\begin{tikzinline}[node distance=1.5em]
      \event{a}{\DR{x}{1}}{}
      \event{b}{\DW{x}{2}}{right=of a}
      \event{c}{\DR[\mSC]{x}{2}}{right=of b}
      \event{d}{\DW[\mSC]{y}{1}}{right=of c}
      \event{e}{\DW[\mSC]{y}{2}}{right=2.5em of d}
      \event{f}{\DW[\mSC]{x}{1}}{right=of e}
      \ctrl[out=15,in=165]{a}{d}
      \rfi{b}{c}
      \bob{c}{d}
      \bob{e}{f}
      \coe{d}{e}
      \rfe[out=-165,in=-15]{f}{a}
    \end{tikzinline}}
\end{gather*}

This is not allowed with a data dependency instead of a control dependency \texttt{[data-rfi-fre-rfe]}:
\begin{gather*}
  \taglabel{data-rfi-fre-rfe}
  \PW{x}{\PR{x}{}{+}1} \SEMI
  \PR[\mSC]{x}{r}\SEMI
  \PR[\mSC]{y}{s} \PAR
  \PW[\mSC]{y}{1}\SEMI
  \PW[\mSC]{x}{1}
  \\
  \tag{\xmark\armeight}
  \hbox{\begin{tikzinline}[node distance=1.5em]
      \event{a}{\DR{x}{1}}{}
      \event{b}{\DW{x}{2}}{right=of a}
      \event{c}{\DR[\mSC]{x}{2}}{right=of b}
      \event{d}{\DR[\mSC]{y}{0}}{right=of c}
      \event{e}{\DW[\mSC]{y}{1}}{right=2.5em of d}
      \event{f}{\DW[\mSC]{x}{1}}{right=of e}
      \data{a}{b}
      \rfi{b}{c}
      \bob{c}{d}
      \bob{e}{f}
      \fre{d}{e}
      \rfe[out=-165,in=-15]{f}{a}
    \end{tikzinline}}
\end{gather*}

\section{SC Examples}

\ref{IRIW-aqc-sc} is allowed by trailing-sync compilation to power
\cite[\textsection 1]{DBLP:conf/pldi/LahavVKHD17}.
\begin{gather*}
  \PW[\mSC]{x}{1}
  \PAR
  \PW[\mSC]{y}{1}
  \PAR
  \PR[\mRA]{x}{r}\SEMI \PR[\mSC]{y}{s}
  \PAR
  \PR[\mRA]{y}{r}\SEMI \PR[\mSC]{x}{s}
  \taglabel{IRIW-aqc-sc}
  \\
  \tag{\cmark\ppc,\rcXI}
  \hbox{\begin{tikzinline}[node distance=1.5em]
      % \event{wx0}{\DW{x}{0}}{}
      % \event{wx1}{\DW{x}{1}}{right=of wx0}
      % \event{wy0}{\DW{y}{0}}{below=4ex of wx0}
      % \event{wy1}{\DW{y}{1}}{right=of wy0}
      \event{wx1}{\DW[\mSC]{x}{1}}{}
      \event{wy1}{\DW[\mSC]{y}{1}}{below=4ex of wx1}
      \event{ry1}{\DR[\mRA]{y}{1}}{right=2.5em of wy1}
      \event{rx0}{\DR[\mSC]{x}{0}}{right=of ry1}
      \event{rx1}{\DR[\mRA]{x}{1}}{right=2.5 em of wx1}
      \event{ry0}{\DR[\mSC]{y}{0}}{right=of rx1}
      % \wk{wx0}{wx1}
      % \wk{wy0}{wy1}
      % \rf[bend left]{wy0}{ry0}
      % \rf[bend right]{wx0}{rx0}
      \sync{rx1}{ry0}
      \sync{ry1}{rx0}
      \rf{wx1}{rx1}
      \rf{wy1}{ry1}
      \wk{rx0}{wx1}
      \wk{ry0}{wy1}
    \end{tikzinline}}
\end{gather*}
Leading sync is also unsound in \cXI{} with \RMW{}
\cite[\textsection 2.1]{DBLP:conf/pldi/LahavVKHD17}.
\begin{gather*}
  \PW[\mSC]{x}{1} \SEMI \PW[\mRA]{y}{1}
  \PAR
  \FADD^{\mSC,\mSC}(y,1) \SEMI \PR{y}{s}
  \PAR
  \PW[\mSC]{y}{3} \SEMI \PR[\mSC]{x}{s}
  \taglabel{Z6.U}
  \\
  \tag{\cmark\ppc,\rcXI}
  \hbox{\begin{tikzinline}[node distance=1.5em]
      \event{a}{\DW[\mSC]{x}{1}}{}
      \event{b}{\DW[\mRA]{y}{1}}{right=of a}
      \event{c1}{\DR[\mSC]{y}{1}}{right=2.5em of b}
      \event{c2}{\DW[\mSC]{y}{2}}{right=of c1}
      \event{d}{\DR{y}{3}}{right=of c2}
      \event{e}{\DW[\mSC]{y}{3}}{right=2.5 em of d}
      \event{f}{\DR[\mSC]{x}{0}}{right=of e}
      \sync{a}{b}
      \rf{b}{c1}
      \rf{e}{d}
      \rmw{c1}{c2}
      %\wk{c2}{d}
      \wk[out=-15,in=-165]{c2}{e}
      % \sync[out=-15,in=-165]{c1}{d}
      %\wk{c2}{d}
      \sync{e}{f}
      \wk[out=-165,in=-15]{f}{a}
    \end{tikzinline}}
\end{gather*}
Leading sync is also unsound in \cXI{} with SC fences
\cite[\textsection A.1]{DBLP:conf/pldi/LahavVKHD17}.
\begin{gather*}
  \PW{x}{2} \SEMI \PF{\mSC} \SEMI \PR{y}{r}
  \PAR
  \PW[\mSC]{y}{1}
  \PAR
  \PR[\mRA]{y}{r} \SEMI \PW[\mRA]{x}{1}  \SEMI \PR{x}{s}
  \PAR
  \PR[\mSC]{x}{r}
   \taglabel{rsync+rsc}
  \\
  \tag{\cmark\rcXI}
  \hbox{\begin{tikzinline}[node distance=1.5em]
      \event{a}{\DW{x}{2}}{}
      \event{b}{\DF{\mSC}}{right=of a}
      \event{c}{\DR{y}{0}}{right=of b}
      \event{d}{\DW[\mSC]{y}{1}}{right=2.5em of c}
      \event{e}{\DR[\mRA]{y}{1}}{right=2.5em of d}
      \event{f}{\DW[\mRA]{x}{1}}{right=of e}
      \event{g}{\DR{x}{2}}{right=of f}
      \event{h}{\DR[\mSC]{x}{1}}{right=2.5em of g}
      \sync{a}{b}
      \sync{b}{c}
      \rf{d}{e}
      \rf[out=-15,in=-165]{f}{h}
      \wk[in=-15,out=-165]{f}{a}
      %\rf[out=-15,in=-165]{a}{g}
      \wk{c}{d}
      \wk{f}{g}
      \sync{e}{f}
      %\sync[out=15,in=165]{e}{g}
    \end{tikzinline}}
\end{gather*}
Fulfillment of $(\DR{x}{2})$ requires that either
\begin{math}
  (\DW[\mRA]{x}{1})
  \xwk
  (\DW{x}{2})
\end{math}
or 
\begin{math}
  (\DR{x}{2})
  \xwk
  (\DW[\mRA]{x}{1}).
\end{math}
It's interesting that in the pomset, $(\DR[\mSC]{x}{1})$ is not needed to get
a cycle.

There is a long discussion of this in \cite[\textsection 5.2,
Fig.~17]{DBLP:journals/pacmpl/BenderP19}, where they also discuss this example:
\begin{gather*}
  \PW[\mSC]{x}{1}\SEMI \PW{x}{2}
  \PAR
  \PW[\mSC]{y}{1}\SEMI \PW{y}{2}
  \PAR
  \PR[\mRA]{x}{r}\SEMI \PR[\mSC]{y}{s}
  \PAR
  \PR[\mRA]{y}{r}\SEMI \PR[\mSC]{x}{s}
  \taglabel{IRIW-sc-rlx-acq}
  \\
  \tag{\cmark\rcXI}
  \hbox{\begin{tikzinline}[node distance=1.5em]
      \event{wx1}{\DW[\mSC]{x}{1}}{}
      \event{wx2}{\DW{x}{2}}{right=of wx1}
      \event{wy1}{\DW[\mSC]{y}{1}}{below=4ex of wx1}
      \event{wy2}{\DW{y}{2}}{right=of wy1}
      \event{ry1}{\DR[\mRA]{y}{2}}{right=2.5em of wy2}
      \event{rx0}{\DR[\mSC]{x}{0}}{right=of ry1}
      \event{rx1}{\DR[\mRA]{x}{2}}{right=2.5 em of wx2}
      \event{ry0}{\DR[\mSC]{y}{0}}{right=of rx1}
      \sync{rx1}{ry0}
      \sync{ry1}{rx0}
      \rf{wx2}{rx1}
      \rf{wy2}{ry1}
      \wk{rx0}{wx1}
      \wk{ry0}{wy1}
      \wk{wx1}{wx2}
      \wk{wy1}{wy2}
    \end{tikzinline}}
\end{gather*}


\cite[\textsection A.2]{DBLP:conf/pldi/LahavVKHD17} claims that \armeight{}
allows this \texttt{[RWC+acq+sc]}, but \href{http://diy.inria.fr/www/?record=aarch64}{herd7} rejects it.
%\verbatiminput{litmus/RWC+acq+sc.litmus}
% More legibly:
% \begin{verbatim}
% STLR#1,[x]     | LDR a, [x] /1    | STLR #1, [y] 
%                | DMB LD           | LDAR c, [x] /0
%                | LDAR b, [y] /0
% \end{verbatim}
Reason: they are citing the flowing/pop model
\cite{DBLP:conf/popl/FlurGPSSMDS16} rather than
\cite{DBLP:journals/pacmpl/PulteFDFSS18}.
\begin{gather*}
  \taglabel{rwc+acq+sc}
  \PW[\mSC]{x}{1} \PAR
  \PR{x}{r}\SEMI
  \PF{\mACQ}\SEMI
  \PR[\mSC]{y}{s} \PAR
  \PW[\mSC]{y}{1}\SEMI
  \PR[\mSC]{x}{r}
  \\
  \tag{\xmark\armeight}
  \hbox{\begin{tikzinline}[node distance=1.5em]
      \event{a}{\DW[\mSC]{x}{1}}{}
      \event{b}{\DR{x}{1}}{right=2.5em of a}
      \event{c}{\DF{\mACQ}}{right=of b}
      \event{d}{\DR[\mSC]{y}{0}}{right=of c}
      \event{e}{\DW[\mSC]{y}{1}}{right=2.5em of d}
      \event{f}{\DR[\mSC]{x}{0}}{right=of e}
      \rfe{a}{b}
      \sync{b}{c}
      \sync{c}{d}
      \fre[out=-165,in=-15]{f}{a}
      \fre{d}{e}
      \sync{e}{f}
    \end{tikzinline}}
\end{gather*}

\section{RMWs}
From \cite[\textsection 3.3]{DBLP:journals/pacmpl/BenderP19}.  With partial
coherence/weak fulfillment you need to be careful that \RMW{}s are totally
ordered (if that's a property you want).  May not come for free.



\section{Example from JAM paper}
From \cite[\textsection B]{DBLP:journals/pacmpl/BenderP19}:
``Here we demonstrate that it is possible to construct a program that is only
forbidden due to the total coherence order''

\begin{comment}
AArch64 TotalCO
{
0:X1=x; 0:X3=y; 
1:X1=x; 1:X3=y;
2:X1=x; 2:X3=y;
}
 P0            | P1           | P2;
 LDR X2,[X1]   | LDAR X5, [X3]| LDAR X5,[X1];
 MOV X0,#1     | MOV X2,#2    | MOV X0, #1;
 STR X0,[X1]   | STR X2,[X1]  | STR X0, [X3];

exists (0:X2=2 /\ 1:X5=1 /\ 2:X5=1)
\end{comment}


\begin{gather*}
  \PR{x}{r}\SEMI
  \PW{x}{1}
  \PAR
  \PR[\mRA]{x}{r}\SEMI
  \PW{x}{1}
  \PAR
  \PR[\mRA]{y}{r}\SEMI
  \PW{x}{2}
  \taglabel{Total-CO}
  \\
  \tag{\xmark\armeight}
  \hbox{\begin{tikzinline}[node distance=1.5em]
      \event{a}{\DR{x}{2}}{}
      \event{b}{\DW{x}{1}}{right=of a}
      \event{c}{\DR[\mACQ]{x}{1}}{right=2.5em of b}
      \event{d}{\DW{y}{1}}{right=of c}
      \event{e}{\DR[\mACQ]{y}{1}}{right=2.5em of d}
      \event{f}{\DW{x}{2}}{right=of e}
      \wk{a}{b}
      \sync{c}{d}
      \sync{e}{f}
      \rf{b}{c}
      \rf{d}{e}
      \rf[out=-165,in=-15]{f}{a}
    \end{tikzinline}}
  % \\
  % \tag{\xmark\armeight}
  % \hbox{\begin{tikzinline}[node distance=1.5em]
  %     \event{a}{\DR{x}{2}}{}
  %     \event{b}{\DW{x}{1}}{right=of a}
  %     \event{c}{\DR[\mACQ]{x}{1}}{right=2.5em of b}
  %     \event{d}{\DW{y}{1}}{right=of c}
  %     \event{e}{\DR[\mACQ]{y}{1}}{right=2.5em of d}
  %     \event{f}{\DW{x}{2}}{right=of e}
  %     \poloc{a}{b}
  %     \co[out=15,in=165]{b}{f}
  %     \rfx[out=-165,in=-15]{f}{a}
  %   \end{tikzinline}}
  \\
  \tag{\xmark\armeight}
  \hbox{\begin{tikzinline}[node distance=1.5em]
      \event{a}{\DR{x}{2}}{}
      \event{b}{\DW{x}{1}}{right=of a}
      \event{c}{\DR[\mACQ]{x}{1}}{right=2.5em of b}
      \event{d}{\DW{y}{1}}{right=of c}
      \event{e}{\DR[\mACQ]{y}{1}}{right=2.5em of d}
      \event{f}{\DW{x}{2}}{right=of e}
      \poloc{a}{b}
      \co[out=15,in=165]{b}{f}
      \rfx{b}{c}
      \fr[out=15,in=165]{c}[below]{f}
      \rfx[out=-165,in=-15]{f}{a}
    \end{tikzinline}}
  \\
  \tag{\xmark\armeight}
  \hbox{\begin{tikzinline}[node distance=1.5em]
      \event{a}{\DR{x}{2}}{}
      \event{b}{\DW{x}{1}}{right=of a}
      \event{c}{\DR[\mACQ]{x}{1}}{right=2.5em of b}
      \event{d}{\DW{y}{1}}{right=of c}
      \event{e}{\DR[\mACQ]{y}{1}}{right=2.5em of d}
      \event{f}{\DW{x}{2}}{right=of e}
      \coe[out=165,in=15]{f}[above]{b}
      %\rfe[out=-165,in=-15]{f}{a}
      \bob{c}{d}
      \bob{e}{f}
      \rfe{b}{c}
      \rfe{d}{e}
    \end{tikzinline}}
\end{gather*}


\section{OLD Model}

\begin{align*}
  \amode \BNFDEF& \mWK &&\text{{(Weak)}}                      &\ascope \BNFDEF& \sCTA &&\text{(Thread group)} &\hbox{$\;\mkern60mu\;$}&
  \\[-1ex] \BNFSEP& \mRLX &&\text{{(Relaxed)}}                & \BNFSEP&\sGPU   &&\text{(Processor)}                                   
  \\[-1ex] \BNFSEP& \mRA &&\text{{(Release/Acquire)}}         & \BNFSEP&\sSYS  &&\text{(System)}                                         
  \\[-1ex] \BNFSEP& \mSC &&\text{{(Sequentially Consistent)}}    
\end{align*}

Orders/Relations in model
\begin{itemize}
\item $\lestrong$ is the old $\le$ (without coherence stuff from \ref{rf4} and \ref{5b}).

  This provides the NO-TAR axiom.
\item $\lehb$ is a the \emph{happens-before} suborder, which only includes $\rrf$ when they are morally strong.

  This serves as a cross-location transitive kernel for the per-location order.
  
\item $\leloc$ is a per-location order that relates morally strong  and $\rpoloc$ accesses

  This includes $\lehb$ for  morally strong accesses.

  This provides the SC-PER-LOC axiom.

  % \item $\rrmw$ is a per-location relation on actions in an \RMW{}
\end{itemize}

Write $\bEv\conflict\aEv$ if they conflict (ie, read/write or write/write, same location).

Write $\bEv\moral\aEv$ if they conflict and are morally strong

\begin{definition}
  A \emph{pomset with preconditions} is a tuple
  $(\Event, \labeling, {\lehb}, {\lestrong}, {\leloc})$ where
  \begin{description}
  \item[{\labeltextsc[m1]{(m1)}{m1}}] $\Event$ is a set of \emph{events}
  \item[{\labeltextsc[m2]{(m2)}{m2}}]
    $\labeling: \Event \fun (\Formulae\times\Act)$ is a \emph{labeling} from
    which we derive functions
    \begin{itemize}
    \item $\labelingForm:\Event\fun\Formulae$
      \emph{(formulae)} % include $r{=}v$ $x{=}v$
    \item $\labelingAct:\Event\fun\Act$
      \emph{(actions)} %include $\DW{x}{v}$, $\DR{x}{v}$, and $\DSTOP$
    \end{itemize}
  \item[{\labeltextsc[m3]{(m3)}{m3}}]
    ${\lehb} \subseteq (\Event\times\Event)$,
    ${\lestrong} \subseteq (\Event\times\Event)$, and
    ${\leloc} \subseteq (\Event\times\Event)$ are partial orders
  \item[{\labeltextsc[m4]{(m4)}{m-consistency}}] $\bigwedge_{\aEv}\labelingForm(\aEv)$ is satisfiable \emph{(consistency)}
  \item[{\labeltextsc[m5]{(m5)}{m-causal-strengthening}}] if $\bEv\lestrong\aEv$ then $\labelingForm(\aEv)$ implies $\labelingForm(\bEv)$ \emph{(causal strengthening)} 
  \item[{\labeltextsc[m6]{(m6)}{m-strong}}] if $\bEv\lehb\aEv$ then $\bEv\lestrong\aEv$
  \item[{\labeltextsc[m7]{(m7)}{m-loc}}] if $\bEv\lehb\aEv$ and $\bEv$ conflicts with $\aEv$ then $\bEv\leloc\aEv$
  \end{description}
\end{definition}
% It is important that \ref{m-loc} covers all conflicting access.
% See \ref{pub1sys}.


  We say $\bEv\lthb\aEv$ when $\bEv\lehb\aEv$ and $\bEv\neq\aEv$, and similarly
  for $\ltstrong$ and $\ltloc$.

% \begin{definition}
%   Define $\leexists$ %and $\ltexists$
%   as follows:
% \end{definition}


\begin{definition}[Strong fulfillment]
  We say $\labelingAct(\bEv)=(\DW[]{x}{v})$ \emph{fulfills}
  $\labelingAct(\aEv)=(\DR[]{x}{v})$ if
  \begin{description}
  \item[{\labeltextsc[f3a]{(f3a)}{rf3a}}{\labeltextsc[f3]{}{rf3}}]
    $\bEv \ltstrong \aEv$
  \item[{\labeltextsc[f3b]{(f3b)}{rf3b}}]
    $\bEv \lthb \aEv$ if $\bEv$ is morally strong with $\aEv$
  \item[{\labeltextsc[f3c]{(f3c)}{rf3c}}]
    $\bEv \leloc \aEv$ (if $\bEv$ is not morally strong with $\aEv$)
  \item[{\labeltextsc[f4]{(f4)}{rf4}}]
    $\forall\labelingAct(\cEv)=(\DW[]{x}{..})$ either $\cEv \leloc \bEv$ or
    $\aEv \leloc \cEv$,
  \end{description}  
\end{definition}
  
\begin{definition}[Weak fulfillment]
  We say $\labelingAct(\bEv)=(\DW[]{x}{v})$ \emph{fulfills}
  $\labelingAct(\aEv)=(\DR[]{x}{v})$ if
  \begin{description}
  \item[{\labeltextsc[f3a]{(f3a)}{rf3a}}{\labeltextsc[f3]{}{rf3}}]
    $\bEv \ltstrong \aEv$
  \item[{\labeltextsc[f3b]{(f3b)}{rf3b}}]
    $\bEv \lthb \aEv$ if $\bEv$ is morally strong with $\aEv$
  \item[{\labeltextsc[f3c]{(f3c)}{rf3c}}]
    $\aEv \not\leloc \bEv$ (if $\bEv$ is not morally strong with $\aEv$)
  \item[{\labeltextsc[f4]{(f4)}{rf4}}]
    $\forall\labelingAct(\cEv)=(\DW[]{x}{..})$ either $\cEv \leexists \bEv$ or
    $\aEv \leexists \cEv$,
    where
  \begin{align*}
    \bEv\leexists\aEv &\textwhen                      
    \begin{cases}
      \bEv\leloc\aEv &\text{if}\; \bEv \;\text{is morally strong with}\;
      \aEv %\bEv\moral\aEv
      \\
      \aEv\not\ltloc\bEv &\text{otherwise}
    \end{cases}
    % \\
    % \bEv\ltexists\aEv &\textwhen                      
    % \begin{cases}
    %   \bEv\ltloc\aEv &\text{if}\; \bEv \;\text{is morally strong with}\;
    %   \aEv %\bEv\moral\aEv
    %   \\
    %   \aEv\not\leloc\bEv &\text{otherwise}
    % \end{cases}
  \end{align*}    
  \end{description}  
\end{definition}

If all accesses are morally strong with each other, weak fulfillment
degenerates to
\begin{description}
\item[\eqref{rf3}]
  $\bEv \lthb \aEv$
\item[\eqref{rf4}]
  $\forall\labelingAct(\cEv)=(\DW[]{x}{..})$ either
  $\cEv \leloc \bEv$ or $\aEv \leloc \cEv$
\end{description}

If no accesses are morally strong with each other, weak fulfillment
degenerates to
\begin{description}
\item[\eqref{rf3}]
  $\aEv \not\leloc \bEv$
\item[\eqref{rf4}]
  $\not\mkern-5mu\exists\labelingAct(\cEv)=(\DW[]{x}{..})$ 
  both $\bEv \ltloc \cEv$ and $\cEv \ltloc \aEv$
\end{description}

Note that the difference between strong and weak fulfillment is limited to $\leloc$.
We sometimes write $\lelocstrong$ for strong fulfillment and
$\lelocweak$ for weak fulfillment.

Prefixing is as in OOPSLA, using $\lehb$ for order everywhere except
\ref{5b}, which has $\leloc$.
\begin{definition}
  Let $\aPS'\in(\aForm \mid \aAct) \prefix \aPSS$ when
  $(\exists\aPS\in\aPSS)$ $(\forall\aEv\in\Event)$
  \begin{description}
  \item[{\labeltextsc[P1]{(P1)}{1}}] $\Event' = \Event \cup \{\bEv\}$
  \item[{\labeltextsc[P2]{(P2)}{2}}] ${\lehb'}\supseteq{\lehb}$,
    ${\lestrong'}\supseteq{\lestrong}$, and ${\leloc'}\supseteq{\leloc}$
  \item[{\labeltextsc[P3]{(P3a)}{3a}\labeltextsc[P3]{}{3}}]%
    $\labelingAct'(\aEv) = \labelingAct(\aEv)$
  \item[{\labeltextsc[P3b]{(P3b)}{3b}}] $\labelingAct'(\bEv) = \aAct$
  \item[{\labeltextsc[P4a]{(P4a)}{4a}\labeltextsc[P4]{}{4}}]%
    $\labelingForm'(\bEv)$ implies
    $\aForm\land(\bEv\not\in\Event\lor\labelingForm(\bEv))$
  \item[{\labeltextsc[P4b]{(P4b)}{4b}}] if $\bEv\neq(\DR[]{..)}{}\mkern79mu$
    then $\aEv=\bEv$ or $\labelingForm'(\aEv)$ implies $\labelingForm(\aEv)$
  \item[{\labeltextsc[P4c]{(P4c)}{4c}}] if
    $\bEv=(\DR[]{\aVal}{\aLoc})\mkern70mu$ then $\aEv=\bEv$ or
    $\labelingForm'(\aEv)$ implies $\labelingForm(\aEv)[\aVal/\aLoc]$
  \item[{\labeltextsc[P5a]{(P5a)}{5a}\labeltextsc[P5]{}{5}}]%
    if $\bEv=(\DR[]{..)}{}$, $\aEv=(\DW[]{..)}{}$ then $\aEv=\bEv$ or
    $\labelingForm'(\aEv)$ implies $\labelingForm(\aEv)$ or $\bEv\lehb'\aEv$
  \item[{\labeltextsc[P5b]{(P5b)}{5b}}] if $\bEv$ conflicts with
    $\aEv$ %$\bEv\conflict\aEv$
    then $\bEv\leloc'\aEv$
  \item[{\labeltextsc[P5c]{(P5c)}{5c}}] if $\bEv$ is an acquire or $\aEv$ is
    a release then $\bEv \lehb' \aEv$
  \item[{\labeltextsc[P5d]{(P5d)}{5d}}] if $\bEv$ is an SC write and $\aEv$
    is an SC read then $\bEv \lehb' \aEv$
  \item[{\labeltextsc[P5e]{(P5e)}{5e}}] if $\bEv$ reads, and $\aEv$ is an
    acquiring fence, then
    $\bEv \lehb' \aEv$
  \item[{\labeltextsc[P5f]{(P5f)}{5f}}] if $\bEv$ is a releasing fence,
    and $\aEv$ writes, then
    $\bEv \lehb' \aEv$
  \end{description}
\end{definition}

% \section{More Model}
% These definitions need to be updated to include the additional orders.
% \begin{definition}
%   A pomset is \emph{$\aLoc$-closed} if
%   \begin{itemize}
%   \item every $\labelingAct(\aEv)=(\DR{\aLoc}{..})$ is fulfilled
%   \item every $\labelingForm(\aEv)$ is independent of $x$:
%     $\bigl(\forall v.\;\labelingForm(\aEv) \vDash
%     \labelingForm(\aEv)[\aVal/\aLoc] \vDash \labelingForm(\aEv)\bigr)$
%   \end{itemize}
% \end{definition}

% \begin{definition}
%   Let $\aPS\phantom{'}\in(\nu\aLoc\!\DOT\!\aPSS) \mkern22mu$ when
%   $\phantom{(\exists}\aPS\in\aPSS$ and $\aPS$ is $\aLoc$-closed
% \end{definition}
% \begin{definition}
%   Let $\aPS\phantom{'}\in(\aForm \guard \aPSS)\mkern16mu$ when
%   $\phantom{(\exists}\aPS\in\aPSS$ and $(\forall\aEv\in\Event)$
%   $\labelingForm(\aEv)$ implies $\aForm$
% \end{definition}

% \begin{definition}
%   Let $\aPS'\in(\aPSS[M/x])\mkern2mu$ when $(\exists\aPS\in\aPSS$)\\\qquad
%   $\Event' = \Event$, ${\lehb'} = {\lehb}$, $\labelingAct' = \labelingAct$, and
%   $(\forall\aEv\in\Event')$ $\labelingForm'(\aEv) = \labelingForm(\aEv)[M/x]$
% \end{definition}
% \begin{definition}
%   Let $\aPS' \in (\aPSS^1 \parallel \aPSS^2)$ when
%   $(\exists\aPS^1 \in \aPSS^1)$ $(\exists\aPS^2 \in \aPSS^2)$
%   \\% $\aPS^1$ is completed exactly when $\aPS^2$ is completed, there is at most one termination in $\Event'$,
%   \qquad $\Event' = \Event^1 \cup \Event^2$,
%   ${\lehb'}\supseteq{\lehb^1}\cup{\lehb^2}$, and $(\forall\aEv\in\Event')$ either
%   \begin{gather*}
%     \begin{aligned}
%       \aEv \not\in \Event^2,\; \labelingAct'(\aEv) &= \labelingAct^1(\aEv)
%       \textand \labelingForm'(\aEv) \textimplies \labelingForm^1(\aEv),
%       \\[-1ex]
%       \aEv \not\in \Event^1,\; \labelingAct'(\aEv) &= \labelingAct^2(\aEv)
%       \textand \labelingForm'(\aEv) \textimplies
%       \labelingForm^2(\aEv),\textor
%       \\[-1ex]
%       \labelingAct'(\aEv) = \labelingAct^1(\aEv) &= \labelingAct^2(\aEv)
%       \textand \labelingForm'(\aEv) \textimplies \labelingForm^1(\aEv) \lor
%       \labelingForm^2(\aEv)
%     \end{aligned}
%   \end{gather*}
% \end{definition}

% Language
% \begin{gather*}
%   % \begin{aligned}
%   %   \aCmd,\,\bCmd
%   %   \BNFDEF& \SKIP
%   %   \BNFSEP \aReg\GETS\aExp\SEMI \aCmd
%   %   \BNFSEP \aReg\GETS\aLoc^{\amode}\SEMI \aCmd 
%   %   \BNFSEP \aLoc^{\amode}\GETS\aExp\SEMI \aCmd
%   %   \\[-.5ex]
%   %   \BNFSEP&\aCmd \PAR[\aThrd][\bThrd] \bCmd
%   %   \BNFSEP \VAR\aLoc\SEMI \aCmd
%   %   \BNFSEP \IF{\aExp} \THEN \aCmd \ELSE \bCmd \FI
%   % \end{aligned}
%   % \\
%   \begin{aligned}
%     \sem[\aThrd]{\SKIP} & \eqdef
%     \{ \DSTOP \}
%     \\
%     \sem[\aThrd]{\aReg\GETS\aExp\SEMI \aCmd} & \eqdef
%     \sem[\aThrd]{\aCmd}[\aExp/\aReg]
%     \\ 
%     \sem{\aReg\GETS\aLoc^{\amode}\SEMI \aCmd} & \eqdef \textstyle\bigcup_\aVal\;
%     (\DR[\amode]\aLoc\aVal)\prefix \sem{\aCmd} [\aLoc/\aReg]
%     \\
%     \sem{\aLoc^{\amode}\GETS\aExp\SEMI \aCmd} & \eqdef
%     \textstyle\bigcup_\aVal\; (\aExp=\aVal \mid \DW[\amode]\aLoc\aVal)\prefix \sem{\aCmd}[\aExp/\aLoc]
%     \\
%     \sem{\FENCE^{\fmode}\SEMI \aCmd} & \eqdef
%     (\DFS{\fmode}) \prefix \sem{\aCmd}
%     \\
%     \sem[\aThrd]{\IF{\aExp} \THEN \aCmd_1 \ELSE \aCmd_2 \FI} & \eqdef
%     \bigl(\aExp \guard \sem[\aThrd]{\aCmd_1}\bigr) \parallel \bigl(\lnot\aExp \guard \sem[\aThrd]{\aCmd_2}\bigr) 
%     \\
%     \sem[\aThrd]{\aCmd_1 \PAR[\bThrd][\bThrd'] \aCmd_2} & \eqdef
%     \sem[\bThrd]{\aCmd_1} \parallel \sem[\bThrd']{\aCmd_2} 
%     \\
%     \sem[\aThrd]{\VAR\aLoc\SEMI \aCmd} & \eqdef
%     \nu \aLoc \DOT \sem[\aThrd]{\aCmd}  
%   \end{aligned}
% \end{gather*}
% \begin{align*}
%   \fmode \BNFDEF&\mREL  &&\text{(Release)} &\hbox{$\;\mkern60mu\;$}&
%   \\ \BNFSEP&\mACQ   &&\text{(Acquire)} 
%   \\      \BNFSEP&\mSC  &&\text{(SC)} 
% \end{align*}





