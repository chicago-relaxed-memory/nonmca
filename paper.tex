\documentclass[anonymous,letterpaper,USenglish,cleveref, autoref, thm-restate]{lipics-v2019}
%This is a template for producing LIPIcs articles. 
%See lipics-manual.pdf for further information.
%for A4 paper format use option "a4paper", for US-letter use option "letterpaper"
%for british hyphenation rules use option "UKenglish", for american hyphenation rules use option "USenglish"
%for section-numbered lemmas etc., use "numberwithinsect"
%for enabling cleveref support, use "cleveref"
%for enabling autoref support, use "autoref"
%for anonymousing the authors (e.g. for double-blind review), add "anonymous"
%for enabling thm-restate support, use "thm-restate"

%\graphicspath{{./graphics/}}%helpful if your graphic files are in another directory

\bibliographystyle{plainurl}% the mandatory bibstyle

\title{Dummy title} %TODO Please add

\titlerunning{Dummy short title} %TODO optional, please use if title is longer than one line

\author{John Q. Public}{Dummy University Computing Laboratory, [optional: Address], Country \and My second affiliation, Country \and \url{http://www.myhomepage.edu} }{johnqpublic@dummyuni.org}{https://orcid.org/0000-0002-1825-0097}{(Optional) author-specific funding acknowledgements}%TODO mandatory, please use full name; only 1 author per \author macro; first two parameters are mandatory, other parameters can be empty. Please provide at least the name of the affiliation and the country. The full address is optional

\author{Joan R. Public\footnote{Optional footnote, e.g. to mark corresponding author}}{Department of Informatics, Dummy College, [optional: Address], Country}{joanrpublic@dummycollege.org}{[orcid]}{[funding]}

\authorrunning{J.\,Q. Public and J.\,R. Public} %TODO mandatory. First: Use abbreviated first/middle names. Second (only in severe cases): Use first author plus 'et al.'

\Copyright{John Q. Public and Joan R. Public} %TODO mandatory, please use full first names. LIPIcs license is "CC-BY";  http://creativecommons.org/licenses/by/3.0/

\ccsdesc[100]{\textcolor{red}{Replace ccsdesc macro with valid one}} %TODO mandatory: Please choose ACM 2012 classifications from https://dl.acm.org/ccs/ccs_flat.cfm 

\keywords{Dummy keyword} %TODO mandatory; please add comma-separated list of keywords

\category{} %optional, e.g. invited paper

\relatedversion{} %optional, e.g. full version hosted on arXiv, HAL, or other respository/website
%\relatedversion{A full version of the paper is available at \url{...}.}

\supplement{}%optional, e.g. related research data, source code, ... hosted on a repository like zenodo, figshare, GitHub, ...

%\funding{(Optional) general funding statement \dots}%optional, to capture a funding statement, which applies to all authors. Please enter author specific funding statements as fifth argument of the \author macro.

\acknowledgements{I want to thank \dots}%optional

%\nolinenumbers %uncomment to disable line numbering

%\hideLIPIcs  %uncomment to remove references to LIPIcs series (logo, DOI, ...), e.g. when preparing a pre-final version to be uploaded to arXiv or another public repository

%Editor-only macros:: begin (do not touch as author)%%%%%%%%%%%%%%%%%%%%%%%%%%%%%%%%%%
\EventEditors{John Q. Open and Joan R. Access}
\EventNoEds{2}
\EventLongTitle{42nd Conference on Very Important Topics (CVIT 2016)}
\EventShortTitle{CVIT 2016}
\EventAcronym{CVIT}
\EventYear{2016}
\EventDate{December 24--27, 2016}
\EventLocation{Little Whinging, United Kingdom}
\EventLogo{}
\SeriesVolume{42}
\ArticleNo{23}
%%%%%%%%%%%%%%%%%%%%%%%%%%%%%%%%%%%%%%%%%%%%%%%%%%%%%%

\usepackage{macros}
\begin{document}

% \maketitle
% \begin{abstract}
% Extend pomsets semantics to PTX.
% \end{abstract}
\section{Example from JAM paper}
From \cite[\textsection 3.3]{DBLP:journals/pacmpl/BenderP19}.  With partial
coherence/weak fulfillment you need to be careful that \RMW{}s are totally
ordered (if that's a property you want).  May not come for free.

 From \cite[\textsection B]{DBLP:journals/pacmpl/BenderP19}:
``Here we demonstrate that it is possible to construct a program that is only
forbidden due to the total coherence order''

\begin{comment}
AArch64 TotalCO
{
0:X1=x; 0:X3=y; 
1:X1=x; 1:X3=y;
2:X1=x; 2:X3=y;
}
 P0            | P1           | P2;
 LDR X2,[X1]   | LDAR X5, [X3]| LDAR X5,[X1];
 MOV X0,#1     | MOV X2,#2    | MOV X0, #1;
 STR X0,[X1]   | STR X2,[X1]  | STR X0, [X3];

exists (0:X2=2 /\ 1:X5=1 /\ 2:X5=1)
\end{comment}


\begin{gather*}
  \PR{x}{r}\SEMI
  \PW{x}{1}
  \PAR
  \PR[\mACQ]{x}{r}\SEMI
  \PW{y}{1}
  \PAR
  \PR[\mACQ]{y}{r}\SEMI
  \PW{x}{2}
  \taglabel{Total-CO}
  \\
  \tag{\xmark\armeight}
  \hbox{\begin{tikzinline}[node distance=1.5em]
      \event{a}{\DR{x}{2}}{}
      \event{b}{\DW{x}{1}}{right=of a}
      \event{c}{\DR[\fACQ]{x}{1}}{right=2.5em of b}
      \event{d}{\DW{y}{1}}{right=of c}
      \event{e}{\DR[\fACQ]{y}{1}}{right=2.5em of d}
      \event{f}{\DW{x}{2}}{right=of e}
      \wki{a}{b}
      \sync{c}{d}
      \sync{e}{f}
      \rf{b}{c}
      \rf{d}{e}
      \rf[out=-165,in=-15]{f}{a}
    \end{tikzinline}}
  % \\
  % \tag{\xmark\armeight}
  % \hbox{\begin{tikzinline}[node distance=1.5em]
  %     \event{a}{\DR{x}{2}}{}
  %     \event{b}{\DW{x}{1}}{right=of a}
  %     \event{c}{\DR[\fACQ]{x}{1}}{right=2.5em of b}
  %     \event{d}{\DW{y}{1}}{right=of c}
  %     \event{e}{\DR[\fACQ]{y}{1}}{right=2.5em of d}
  %     \event{f}{\DW{x}{2}}{right=of e}
  %     \poloc{a}{b}
  %     \co[out=15,in=165]{b}{f}
  %     \rfx[out=-165,in=-15]{f}{a}
  %   \end{tikzinline}}
  \\
  \tag{\xmark\armeight}
  \hbox{\begin{tikzinline}[node distance=1.5em]
      \event{a}{\DR{x}{2}}{}
      \event{b}{\DW{x}{1}}{right=of a}
      \event{c}{\DR[\fACQ]{x}{1}}{right=2.5em of b}
      \event{d}{\DW{y}{1}}{right=of c}
      \event{e}{\DR[\fACQ]{y}{1}}{right=2.5em of d}
      \event{f}{\DW{x}{2}}{right=of e}
      \poloc{a}{b}
      \co[out=15,in=165]{b}{f}
      \rfx{b}{c}
      \fr[out=15,in=165]{c}[below]{f}
      \rfx[out=-165,in=-15]{f}{a}
    \end{tikzinline}}
  \\
  \tag{\xmark\armeight}
  \hbox{\begin{tikzinline}[node distance=1.5em]
      \event{a}{\DR{x}{2}}{}
      \event{b}{\DW{x}{1}}{right=of a}
      \event{c}{\DR[\fACQ]{x}{1}}{right=2.5em of b}
      \event{d}{\DW{y}{1}}{right=of c}
      \event{e}{\DR[\fACQ]{y}{1}}{right=2.5em of d}
      \event{f}{\DW{x}{2}}{right=of e}
      \coe[out=165,in=15]{f}[above]{b}
      %\rfe[out=-165,in=-15]{f}{a}
      \bob{c}{d}
      \bob{e}{f}
      \rfe{b}{c}
      \rfe{d}{e}
    \end{tikzinline}}
\end{gather*}

\section{IRIW}
Status of IRIW is unclear in our model, since we allow everything allowed by
power...
\begin{gather*}
  \begin{gathered}
    % \PW{x}{0}\SEMI
    \PW{x}{1}
    \PAR
    \PR[\mRA]{x}{r}\SEMI \PR{y}{s}
    \PAR
    % \PW{y}{0}\SEMI
    \PW{y}{1}
    \PAR
    \PR[\mRA]{y}{s} \SEMI \PR{x}{r}
    \\
    %\smash[b]{
      \hbox{\begin{tikzinline}[node distance=1.5em]
          % \event{wx0}{\DW{x}{0}}{}
          % \event{wx1}{\DW{x}{1}}{right=of wx0}
          % \event{wy0}{\DW{y}{0}}{below=4ex of wx0}
          % \event{wy1}{\DW{y}{1}}{right=of wy0}
          \event{wx1}{\DW{x}{1}}{}
          \event{rx1}{\DR[\mRA]{x}{1}}{right=3em of wx1}
          \event{ry0}{\DR{y}{0}}{right=of rx1}
          \event{wy1}{\DW{y}{1}}{right=3em of ry0}
          \event{ry1}{\DR[\mRA]{y}{1}}{right=3em of wy1}
          \event{rx0}{\DR{x}{0}}{right=of ry1}
          % \wk{wx0}{wx1}
          % \wk{wy0}{wy1}
          % \rf[bend left]{wy0}{ry0}
          % \rf[bend right]{wx0}{rx0}
          \sync{rx1}{ry0}
          \sync{ry1}{rx0}
          \rf{wx1}{rx1}
          \rf{wy1}{ry1}
          \wk[out=170,in=10]{rx0}{wx1}
          \wk{ry0}{wy1}
        \end{tikzinline}}
    %}
  \end{gathered}
\end{gather*}


\section{Sync examples}

The first of these is seen in hardware.  All are allowed by \PTX.
Showing $\rrfx$ that is not included in the order using a dotted arrow.
\begin{gather*}
  {
    \PW{x}{1}
    \SEMI
    \PW[\mREL]{y}{1}
  }\PAR{
    \PR[\mACQ]{y}{r}
    \SEMI
    \PW{z}[\sSYS]{r}
  }\LPAR[\bScp]{
    \PR[\mACQ]{z}[\sSYS]{r}
    \SEMI
    \PR{x}{s}
  }
  \\
  \tag{$\lesync$}
  \hbox{\begin{tikzinline}[node distance=1.5em]
      \event{a1}{\DW{x}{1}[\aScp]}{}
      \event{a2}{\DW[\mREL]{y}{1}[\aScp]}{right=of a1}
      \event{b1}{\DR[\mACQ]{y}{1}[\aScp]}{right=3em of a2}
      \event{b2}{\DW{z}[\sSYS]{1}[\aScp]}{right=of b1}
      \event{c1}{\DR[\mACQ]{z}[\sSYS]{1}[\bScp]}{right=3em of b2}
      \event{c2}{\DR{x}{0}[\bScp]}{right=of c1}
      \sync{a1}{a2}
      \sync{b1}{b2}
      \sync{c1}{c2}
      \rf{a2}{b1}
      \rfint{b2}{c1}
      %\wk[out=-165,in=-15]{c2}{a1}
    \end{tikzinline}}
\end{gather*}

\begin{gather*}
  {
    \PW{x}{1}
    \SEMI
    \PW[\mREL]{y}{1}
  }\PAR{
    \PR[\mACQ]{y}{r}
    \SEMI
    \PW{z}{r}
  }\PAR{
    \PR[\mACQ]{z}{r}
    \SEMI
    \PR{x}{s}
  }
  % \\
  % \tag{$\ledep$}
  % \hbox{\begin{tikzinline}[node distance=1.5em]
  %     \event{a1}{\DW{x}{1}}{}
  %     \event{a2}{\DW[\mREL]{y}{1}}{right=of a1}
  %     \event{b1}{\DR[\mACQ]{y}{1}}{right=3em of a2}
  %     \event{b2}{\DW{z}{1}}{right=of b1}
  %     \event{c1}{\DR[\mACQ]{z}{1}}{right=3em of b2}
  %     \event{c2}{\DR{x}{0}}{right=of c1}
  %     %\sync{a1}{a2}
  %     \po{b1}{b2}
  %     %\sync{c1}{c2}
  %     \rf{a2}{b1}
  %     \rf{b2}{c1}
  %     %\wk[out=-165,in=-15]{c2}{a1}
  %   \end{tikzinline}}
  \\
  \tag{$\lesync$}
  \hbox{\begin{tikzinline}[node distance=1.5em]
      \event{a1}{\DW{x}{1}}{}
      \event{a2}{\DW[\mREL]{y}{1}}{right=of a1}
      \event{b1}{\DR[\mACQ]{y}{1}}{right=3em of a2}
      \event{b2}{\DW{z}{1}}{right=of b1}
      \event{c1}{\DR[\mACQ]{z}{1}}{right=3em of b2}
      \event{c2}{\DR{x}{0}}{right=of c1}
      \sync{a1}{a2}
      \sync{b1}{b2}
      \sync{c1}{c2}
      \rf{a2}{b1}
      \rfint{b2}{c1}
      %\wk[out=-165,in=-15]{c2}{a1}
    \end{tikzinline}}
  % \\
  % \tag{$\leloc$}
  % \hbox{\begin{tikzinline}[node distance=1.5em]
  %     \event{a1}{\DW{x}{1}}{}
  %     \event{a2}{\DW[\mREL]{y}{1}}{right=of a1}
  %     \event{b1}{\DR[\mACQ]{y}{1}}{right=3em of a2}
  %     \event{b2}{\DW{z}{1}}{right=of b1}
  %     \event{c1}{\DR[\mACQ]{z}{1}}{right=3em of b2}
  %     \event{c2}{\DR{x}{0}}{right=of c1}
  %     %\sync{a1}{a2}
  %     %\sync{b1}{b2}
  %     %\sync{c1}{c2}
  %     \rf{a2}{b1}
  %     \rf{b2}{c1}
  %     \wk[out=-165,in=-15]{c2}{a1}
  %   \end{tikzinline}}
\end{gather*}

\begin{gather*}
  {
    \PW{x}{1}
    \SEMI
    \PW[\mREL]{y}{1}
  }\PAR{
    \PR{y}{r}
    \SEMI
    \PW[\mREL]{z}{r}
  }\PAR{
    \PR[\mACQ]{z}{r}
    \SEMI
    \PR{x}{s}
  }
  \\
  \tag{$\lesync$}
  \hbox{\begin{tikzinline}[node distance=1.5em]
      \event{a1}{\DW{x}{1}}{}
      \event{a2}{\DW[\mREL]{y}{1}}{right=of a1}
      \event{b1}{\DR{y}{1}}{right=3em of a2}
      \event{b2}{\DW[\mREL]{z}{1}}{right=of b1}
      \event{c1}{\DR[\mACQ]{z}{1}}{right=3em of b2}
      \event{c2}{\DR{x}{0}}{right=of c1}
      \sync{a1}{a2}
      \sync{b1}{b2}
      \sync{c1}{c2}
      \rfint{a2}{b1}
      \rf{b2}{c1}
      %\wk[out=-165,in=-15]{c2}{a1}
    \end{tikzinline}}
\end{gather*}

To get publication using fences we need an additional closure property for
$\rrfx$ on sync order:
\begin{gather*}
  {
    \PW{x}{1}
    \SEMI
    \PF{\fREL}
    \SEMI
    \PW{y}{1}
  }\PAR{
    \PR{y}{r}
    \SEMI
    \PF{\fACQ}
    \SEMI
    \PR{x}{s}
  }
  \\
  \tag{$\lesync$}
  \hbox{\begin{tikzinline}[node distance=1.5em]
      \event{a1}{\DW{x}{1}}{}
      \event{a2}{\DF{\fREL}}{right=of a1}
      \event{a3}{\DW{y}{1}}{right=of a2}
      \event{b1}{\DR{y}{1}}{right=3em of a3}
      \event{b2}{\DF{\fACQ}}{right=of b1}
      \event{b3}{\DR{x}{0}}{right=of b2}
      \sync{a1}{a2}
      \sync{a2}{a3}
      \sync{b1}{b2}
      \sync{b2}{b3}
      \rfint{a3}{b1}
      %\wk[out=-165,in=-15]{b3}{a1}
    \end{tikzinline}}
\end{gather*}
Previous def of candidate requires:
\begin{itemize}
\item[(\ref{cand-lesync-rf})]
  if $\bEv\xrfx\aEv$ and $\labeling(\bEv) \rsmatches \labeling(\aEv)$ then $\bEv \lesync \aEv$.
\end{itemize}
This is not good enough for fences.
A possible fix is the following closure condition:
\begin{itemize}
\item[(\ref{cand-lesync-rf}$'$)]
  if $\bEv'\lesync\bEv\xrfx\aEv\lesync\aEv'$ and $\labeling(\bEv') \rsmatches \labeling(\aEv')$ then $\bEv' \lesync \aEv'$.
\end{itemize}
With that we have the following, using $\xliftrf$ for edges induced by closure
when $\bEv'\neq\bEv$ or $\aEv'\neq\aEv$:
\begin{gather*}
  {
    \PW{x}{1}
    \SEMI
    \PF{\fREL}
    \SEMI
    \PW{y}{1}
  }\PAR{
    \PR{y}{r}
    \SEMI
    \PF{\fACQ}
    \SEMI
    \PR{x}{s}
  }
  \\
  \tag{$\lesync$}
  \hbox{\begin{tikzinline}[node distance=1.5em]
      \event{a1}{\DW{x}{1}}{}
      \event{a2}{\DF{\fREL}}{right=of a1}
      \event{a3}{\DW{y}{1}}{right=of a2}
      \event{b1}{\DR{y}{1}}{right=3em of a3}
      \event{b2}{\DF{\fACQ}}{right=of b1}
      \event{b3}{\DR{x}{0}}{right=of b2}
      \sync{a1}{a2}
      \sync{a2}{a3}
      \sync{b1}{b2}
      \sync{b2}{b3}
      \rfint{a3}{b1}
      \liftrf[out=-15,in=-165]{a2}{b2}
      %\wk[out=-165,in=-15]{b3}{a1}
    \end{tikzinline}}
\end{gather*}
This seems to work for the above examples, but it could be too strong in general.
\begin{itemize}
\item One possibility is to restrict to preceding and following things in the
  same thread:
  \begin{itemize}
  \item[(\ref{cand-lesync-rf}$''$)]
    if $\bEv'\lesyncpo\bEv\xrfx\aEv\lesyncpo\aEv'$ and $\labeling(\bEv') \rsmatches \labeling(\aEv')$ then $\bEv' \lesync \aEv'$.
  \end{itemize}
  where $\lesyncpo$ is the obvious restriction of $\lesync$ to actions on the
  same thread.
\item With either (\ref{cand-lesync-rf}$'$) or (\ref{cand-lesync-rf}$''$) is it too strong to require $\lesync$ that be
  transitive?   In particular:
  \begin{itemize}
  \item if we restrict to $\lesyncpo$, the closure condition
    (\ref{cand-lesync-rf}$''$) could add order between actions on the same thread
    via cross-thread reads.
  \item How does transitivity interact with scopes?
  \end{itemize}
\end{itemize}
Anton proposes:
\begin{itemize}
\item[(\ref{pom-rmw-lesync}$'$)]
  if $\bEv\xrmw\aEv$ then %$\bEv \lesync \aEv$ and
  $\bEv \leloc \aEv$,    
\item[(\ref{cand-lesync-rf}$'''$)]
  if $\bEv'\lesync\bEv\mathrel{(\xrfx;(\xrmw;\xrfx)^{*})}\aEv\lesync\aEv'$ and $\labeling(\bEv') \rsmatches \labeling(\aEv')$ then $\bEv' \lesync \aEv'$.
\end{itemize}

The following behavior is allowed by Arm, IMM, and C11, but forbidden by \PTX.
\PTX{} forbids it since acquire reads work as fences for po-previous reads from
the same location (symmetrically to release writes for po-latter writes to
the same location in \IMM, \cXI, and \PTX).
\begin{gather*}
  {
    \PW{x}{1}
    \SEMI
    \PW[\mREL]{y}{1}
  }\PAR{
    \PR{y}{r}
    \SEMI
    \PW{y}{2}
    \SEMI
    \PR[\mACQ]{y}{s}
     \SEMI
    \PR{x}{t}
  }
  \\
  \tag{$\lesync$}
  \hbox{\begin{tikzinline}[node distance=1.5em]
      \event{a}{\DW{x}{1}}{}
      \raevent{b}{\DW[\mREL]{y}{1}}{right=of a}
      \event{c}{\DR{y}{1}}{right=3em of b}
      \event{d}{\DW{y}{2}}{right=of c}
      \raevent{e}{\DR[\mACQ]{y}{2}}{right=of d}
      \event{f}{\DR{x}{0}}{right=of e}
      \sync{a}{b}
      \rfint{b}{c}
      \sync[out=15,in=165]{c}{e}
      %\wk{c}{d}
      \rfint{d}{e}
      \sync{e}{f}
      %\wk[out=-165,in=-15]{f}{a}
      \liftrf[out=-15,in=-165]{b}{e}
    \end{tikzinline}}
\end{gather*}
To allow this on for IMM, we need to drop
\begin{math}
  (\DR{\aLoc}{}, \DR[\gemode\mACQ]{\aLoc}{})
\end{math}
from $\rsyncdelays$.

The following is allowed by \cXI{}, but not \IMM{} or \PTX.
The goal here is to construct a cycle
$a\xrfx b \xhb c \xrfx d \xhb a$
where $\rrfx$  will be included in synch-relation.
In relational notation, the cycle has the following form:
\begin{displaymath}
  \PBRbig{{\rrmw} ; ({\rrfe} ; {\rrmw})^2 ; {\rppo} ; \Wclass[\mREL]; {\rrfe} ; \Rclass[\mACQ] ; {\rppo}}^2
\end{displaymath}
\begin{gather*}
  {
    \PR[\mACQ]{x}{r}
    \SEMI
    \PINC{y}{}
  }\PAR{
    \PINC{y}{}
  }\PAR{
    \PINC{y}{}
    \SEMI
    \PW[\mREL]{z}{1}
  }\PAR{
    \PR[\mACQ]{z}{s}
    \SEMI
    \PINC{w}{}
  }\PAR{
    \PINC{w}{}
  }\PAR{
    \PINC{w}{}
    \SEMI
    \PW[\mREL]{x}{1}
  }
  \\
  \hbox{\begin{tikzinline}[node distance=1.5em]
      \event{a1}{\DR[\mACQ]{x}{1}}{}
      \event{a2}{\DR{y}{0}}{below=of a1}
      \event{a3}{\DW{y}{1}}{below=of a2}
      \sync{a1}{a2}
      \rmw{a2}{a3}
      \event{b1}{\DR{y}{1}}{right=of a1}
      \event{b2}{\DW{y}{2}}{below=of b1}
      \rmw{b1}{b2}
      \event{c1}{\DR{y}{2}}{right=of b1}
      \event{c2}{\DW{y}{3}}{below=of c1}
      \event{c3}{\DW[\mREL]{z}{1}}{below=of c2}
      \rmw{c1}{c2}
      \sync{c2}{c3}
      \event{d1}{\DR[\mACQ]{z}{1}}{right=of c1}
      \event{d2}{\DR{w}{0}}{below=of d1}
      \event{d3}{\DW{w}{1}}{below=of d2}
      \sync{d1}{d2}
      \rmw{d2}{d3}
      \event{e1}{\DR{w}{1}}{right=of d1}
      \event{e2}{\DW{w}{2}}{below=of e1}
      \rmw{e1}{e2}
      \event{f1}{\DR{w}{2}}{right=of e1}
      \event{f2}{\DR{w}{3}}{below=of f1}
      \event{f3}{\DW[\mREL]{x}{1}}{below=of f2}
      \rmw{f1}{f2}
      \sync{f2}{f3}
      \rf{a3}{b1}
      \rf{b2}{c1}
      \rf{c3}{d1}
      \rf{d3}{e1}
      \rf{e2}{f1}
      \rf{f3}{a1}
    \end{tikzinline}}
\end{gather*}

\section{Relating IMM and PTX}
It looks like we cannot prove compilation correctness from IMM to PTX.
(In this email I assume that all threads are in the same CTA, so any relation is a morally strong one if it is applicable.)
The problem is in the LB-data-rel example:
\begin{comment}
a := [x]  || b := [y]
[y] := a  || [x]_rel := 1
\end{comment}
\begin{gather*}
  \PR{x}{r}\SEMI
  \PW{y}{r}
  \PAR
  \PR{y}{s}\SEMI
  \PW[\mREL]{x}{1}
  \\
  \hbox{\begin{tikzinline}[node distance=1.5em]
      \event{a}{\DR{x}{1}}{}
      \event{b}{\DW{y}{1}}{right=of a}
      \event{c}{\DR{y}{1}}{right=3em of b}
      \raevent{d}{\DW[\mREL]{x}{1}}{right=of c}
      \data{a}{b}
      \rfe{b}{c}
      \bob{c}{d}
      \rfe[out=-165,in=-15]{d}[above]{a}
    \end{tikzinline}}
  \\
  \tag{$\ledep$}  
  \hbox{\begin{tikzinline}[node distance=1.5em]
      \event{a}{\DR{x}{1}}{}
      \event{b}{\DW{y}{1}}{right=of a}
      \event{c}{\DR{y}{1}}{right=3em of b}
      \raevent{d}{\DW[\mREL]{x}{1}}{right=of c}
      \po{a}{b}
      \rf{b}{c}
      %\sync{c}{d}
      \rf[out=-165,in=-15]{d}{a}
    \end{tikzinline}}
  \\
  \tag{$\lesync$}  
  \hbox{\begin{tikzinline}[node distance=1.5em]
      \event{a}{\DR{x}{1}}{}
      \event{b}{\DW{y}{1}}{right=of a}
      \event{c}{\DR{y}{1}}{right=3em of b}
      \raevent{d}{\DW[\mREL]{x}{1}}{right=of c}
      %\po{a}{b}
      %\rf{b}{c}
      \sync{c}{d}
      %\rf[out=-165,in=-15]{d}{a}
    \end{tikzinline}}
  \\
  \tag{$\leloc$}  
  \hbox{\begin{tikzinline}[node distance=1.5em]
      \event{a}{\DR{x}{1}}{}
      \event{b}{\DW{y}{1}}{right=of a}
      \event{c}{\DR{y}{1}}{right=3em of b}
      \raevent{d}{\DW[\mREL]{x}{1}}{right=of c}
      %\po{a}{b}
      %\rf{b}{c}
      %\sync{c}{d}
      %\rf[out=-165,in=-15]{d}{a}
    \end{tikzinline}}
\end{gather*}

IMM forbids it, but PTX allows it. The point is that IMM mixes dependencies and release/acquire-induced po-order in its NoOOTA axiom,
whereas PTX doesn't --- release/acquire are only used to have coherence.

The problem is related to the one we have already discussed in the context of the C++ model -- if you don't have acquire reads in the
program, then you can erase release annotations from writes. In this regard, PTX is closer to PL memory models than to hardware ones.

AFAIU for the same reason we won't be able to show compilation correctness from the Pomset model to PTX even directly,
if the Pomset model mixes release/acquire induced order with dependencies in the same causality relation.

The previous example in the section shows that IMM's acquires are stronger
than PTX for this pattern.  The next example shows that acquiring reads in
PTX are stronger than in IMM for a different pattern.  Thus the acquires in
PTX and IMM are incomparable.

The following behavior is allowed by IMM and C11, but forbidden by PTX.  PTX
forbids it since acquire reads work as fences for po-previous reads from the
same location (symmetrically to release writes for po-latter writes to the
same location in IMM, C11, and PTX).
\begin{gather*}
  \PW{x}{1}\SEMI
  \PW[\mREL]{y}{1}
  \PAR
  \PR{y}{r}\SEMI
  \PW{y}{2}\SEMI
  \PR[\mACQ]{y}{s}\SEMI
  \PR{x}{t}
  \\
  \hbox{\begin{tikzinline}[node distance=1.5em]
      \event{a}{\DW{x}{1}}{}
      \raevent{b}{\DW[\mREL]{y}{1}}{right=of a}
      \event{c}{\DR{y}{1}}{right=3em of b}
      \event{d}{\DW{y}{2}}{right=of c}
      \raevent{e}{\DR[\mACQ]{y}{2}}{right=of d}
      \event{f}{\DR{x}{0}}{right=of e}
      \sync{a}{b}
      \rf{b}{c}
      \wk{c}{d}
      \rf{d}{e}
      \sync{e}{f}
      \wk[out=-165,in=-15]{f}{a}
    \end{tikzinline}}
\end{gather*}


\section{Thin Air}

Need $\ledep$ to prevent thin air on $\mRLX$:
\begin{gather*}
  \PW{y}{\PR{x}{}}\PAR
  \PW{x}{\PR{y}{}}
  \\
  \tag{$\ledep$}
  \hbox{\begin{tikzinline}[node distance=1.5em]
      \event{a}{\DR{x}{1}}{}
      \event{b}{\DW{y}{1}}{right=of a}
      \event{c}{\DR{y}{1}}{right=2.5em of b}
      \event{d}{\DW{x}{1}}{right=of c}
      \po{a}{b}
      \rf{b}{c}
      \po{c}{d}
      \rf[out=-165,in=-15]{d}{a}
    \end{tikzinline}}
  \\
  \tag{$\lesync$}
  \hbox{\begin{tikzinline}[node distance=1.5em]
      \event{a}{\DR{x}{1}}{}
      \event{b}{\DW{y}{1}}{right=of a}
      \event{c}{\DR{y}{1}}{right=2.5em of b}
      \event{d}{\DW{x}{1}}{right=of c}
      \po{a}{b}
      \po{c}{d}
    \end{tikzinline}}
  \\
  \tag{$\lelocstrong$}
  \hbox{\begin{tikzinline}[node distance=1.5em]
      \event{a}{\DR{x}{1}}{}
      \event{b}{\DW{y}{1}}{right=of a}
      \event{c}{\DR{y}{1}}{right=2.5em of b}
      \event{d}{\DW{x}{1}}{right=of c}
      \rf{b}{c}
      \rf[out=-165,in=-15]{d}{a}
    \end{tikzinline}}
\end{gather*}

\section{IMM Examples}

Disallowed by \IMM{}:
\begin{gather*}
  \taglabel{pub-rel-acq-coe}
  \PW{x}{2}\SEMI 
  \PW[\mREL]{y}{1} \PAR
  \PR[\mACQ]{y}{r}\SEMI
  \PW{x}{1}
  \\
  \tag{\xmark\IMM}
  \hbox{\begin{tikzinline}[node distance=1.5em]
      \event{a}{\DW{x}{2}}{}
      \raevent{b}{\DW[\mREL]{y}{1}}{right=of a}
      \raevent{c}{\DR[\mACQ]{y}{1}}{right=2.5em of b}
      \event{d}{\DW{x}{1}}{right=of c}
      \bob{a}{b}
      \rfe{b}{c}
      \bob{c}{d}
      \coe[out=-165,in=-15]{d}{a}
    \end{tikzinline}}
  \\
  \tag{${\ledep}={\lesync}$}
  \hbox{\begin{tikzinline}[node distance=1.5em]
      \event{a}{\DW{x}{2}}{}
      \raevent{b}{\DW[\mREL]{y}{1}}{right=of a}
      \raevent{c}{\DR[\mACQ]{y}{1}}{right=2.5em of b}
      \event{d}{\DW{x}{1}}{right=of c}
      \sync{a}{b}
      \rf{b}{c}
      \sync{c}{d}
      %\wk[out=-165,in=-15]{d}{a}
    \end{tikzinline}}
  \\
  \tag{$\lelocstrong$}
  \hbox{\begin{tikzinline}[node distance=1.5em]
      \event{a}{\DW{x}{2}}{}
      \raevent{b}{\DW[\mREL]{y}{1}}{right=of a}
      \raevent{c}{\DR[\mACQ]{y}{1}}{right=2.5em of b}
      \event{d}{\DW{x}{1}}{right=of c}
      %\sync{a}{b}
      \rf{b}{c}
      %\sync{c}{d}
      \po[out=15,in=165]{a}{d}
      %\wk[out=-165,in=-15]{d}{a}
    \end{tikzinline}}
\end{gather*}

Allowed by \IMM, but not by Power/ARMv7/ARMv8/TSO:
\begin{gather*}
  \taglabel{pub-rel-rlx-coe}
  \PW{x}{2}\SEMI 
  \PW[\mREL]{y}{1} \PAR
  \PR{y}{r}\SEMI
  \PW{x}{1}
  \\
  \tag{\cmark\IMM}
  \hbox{\begin{tikzinline}[node distance=1.5em]
      \event{a}{\DW{x}{2}}{}
      \raevent{b}{\DW[\mREL]{y}{1}}{right=of a}
      \event{c}{\DR{y}{1}}{right=2.5em of b}
      \event{d}{\DW{x}{1}}{right=of c}
      \bob{a}{b}
      \rfe{b}{c}
      \data{c}{d}
      \coe[out=-165,in=-15]{d}{a}
    \end{tikzinline}}
  \\
  \tag{$\ledep$}
  \hbox{\begin{tikzinline}[node distance=1.5em]
      \event{a}{\DW{x}{2}}{}
      \raevent{b}{\DW[\mREL]{y}{1}}{right=of a}
      \event{c}{\DR{y}{1}}{right=2.5em of b}
      \event{d}{\DW{x}{1}}{right=of c}
      \sync{a}{b}
      \rf{b}{c}
      \po{c}{d}
      %\wk[out=-165,in=-15]{d}{a}
    \end{tikzinline}}
  \\
  \tag{$\lesync$}
  \hbox{\begin{tikzinline}[node distance=1.5em]
      \event{a}{\DW{x}{2}}{}
      \raevent{b}{\DW[\mREL]{y}{1}}{right=of a}
      \event{c}{\DR{y}{1}}{right=2.5em of b}
      \event{d}{\DW{x}{1}}{right=of c}
      \sync{a}{b}
      %\rfe{b}{c}
      \po{c}{d}
      %\wk[out=-165,in=-15]{d}{a}
    \end{tikzinline}}
  \\
  \tag{$\lelocstrong$}
  \hbox{\begin{tikzinline}[node distance=1.5em]
      \event{a}{\DW{x}{2}}{}
      \raevent{b}{\DW[\mREL]{y}{1}}{right=of a}
      \event{c}{\DR{y}{1}}{right=2.5em of b}
      \event{d}{\DW{x}{1}}{right=of c}
      %\sync{a}{b}
      \rf{b}{c}
      %\po{c}{d}
      %\wk[out=-165,in=-15]{d}{a}
    \end{tikzinline}}
\end{gather*}


Example from talk:
\begin{gather*}
  \taglabel{arm7-weak}
  \PR{x}{r}\SEMI \PW{x}{1}
  \PAR
  \PW{y}{x} 
  \PAR
  \PW{x}{y} 
  \\[-1.2ex]
  \tag{$\ledep$}
  \hbox{\begin{tikzinline}[node distance=1.5em]
      \event{a}{\DR{x}{1}}{}
      \event{b}{d:\DW{x}{1}}{right=of a}
      %\wk{a}{b}
      \event{c}{\DR{x}{1}}{right=3em of b}
      \event{d}{\DW{y}{1}}{right=of c}
      \po{c}{d}
      \event{e}{\DR{y}{1}}{right=3em of d}
      \event{f}{e:\DW{x}{1}}{right=of e}
      \po{e}{f}
      \rf{b}{c}
      \rf{d}{e}
      \rf[out=172,in=8]{f}{a}
    \end{tikzinline}}
  \\
  \tag{$\lesync$}
  \hbox{\begin{tikzinline}[node distance=1.5em]
      \event{a}{\DR{x}{1}}{}
      \event{b}{d:\DW{x}{1}}{right=of a}
      %\wk{a}{b}
      \event{c}{\DR{x}{1}}{right=3em of b}
      \event{d}{\DW{y}{1}}{right=of c}
      \po{c}{d}
      \event{e}{\DR{y}{1}}{right=3em of d}
      \event{f}{e:\DW{x}{1}}{right=of e}
      \po{e}{f}
      %\rf{b}{c}
      %\rf{d}{e}
      %\rf[out=172,in=8]{f}{a}
    \end{tikzinline}}
  \\
  \tag{$\lelocstrong$}
  \hbox{\begin{tikzinline}[node distance=1.5em]
      \event{a}{\DR{x}{1}}{}
      \event{b}{d:\DW{x}{1}}{right=of a}
      \wk{a}{b}
      \event{c}{\DR{x}{1}}{right=3em of b}
      \event{d}{\DW{y}{1}}{right=of c}
      %\po{c}{d}
      \event{e}{\DR{y}{1}}{right=3em of d}
      \event{f}{e:\DW{x}{1}}{right=of e}
      % \po{e}{f}
      %\po[out=-15,in=-165]{c}{f}
      \rf{b}{c}
      \rf{d}{e}
      \rf[out=172,in=8]{f}{a}
    \end{tikzinline}}
  \\
  \tag{$\lelocweak$}
  \hbox{\begin{tikzinline}[node distance=1.5em]
      \event{a}{\DR{x}{1}}{}
      \event{b}{d:\DW{x}{1}}{right=of a}
      \wk{a}{b}
      \event{c}{\DR{x}{1}}{right=3em of b}
      \event{d}{\DW{y}{1}}{right=of c}
      %\po{c}{d}
      \event{e}{\DR{y}{1}}{right=3em of d}
      \event{f}{e:\DW{x}{1}}{right=of e}
      % \po{e}{f}
      %\po[out=-15,in=-165]{c}{f}
      %\rf{b}{c}
      %\rf{d}{e}
      %\rf[out=172,in=8]{f}{a}
    \end{tikzinline}}
\end{gather*}

\section{PTX Examples}
Based on \cite{DBLP:conf/asplos/LustigSG19,nvidia-model}.

In examples, all threads in different $\sCTA$s.

$(\DR{x}{0})$ must be forbidden.
Before fulfilling the read:
\begin{gather*}
  \taglabel[sys]{pub1}
  \PW[\mWK]{x}[\sCTA]{0}\SEMI 
  \PW[\mWK]{x}[\sCTA]{1}\SEMI
  \PW[\mREL]{y}{1} \PAR
  \PR[\mACQ]{y}{r}\SEMI
  \PR[\mWK]{x}[\sCTA]{s}
  \\
  \tag{${\ledep}={\lesync}$}
  \hbox{\begin{tikzinline}[node distance=1.5em]
      \event{wx0}{\DW[\mWK]{x}[\sCTA]{0}}{}
      \event{wx1}{\DW[\mWK]{x}[\sCTA]{1}}{right=of wx0}
      \raevent{wy1}{\DW[\mREL]{y}{1}}{right=of wx1}
      \raevent{ry1}{\DR[\mACQ]{y}{1}}{right=2.5em of wy1}
      \event{rx}{\DR[\mWK]{x}[\sCTA]{}}{right=of ry1}
      \sync[out=-15,in=-165]{wx0}{wy1}
      \sync{wx1}{wy1}
      \sync{ry1}{rx}
      \rf{wy1}{ry1}
    \end{tikzinline}}
  \\
  \tag{$\leloc$}
  \hbox{\begin{tikzinline}[node distance=1.5em]
      \event{wx0}{\DW[\mWK]{x}[\sCTA]{0}}{}
      \event{wx1}{\DW[\mWK]{x}[\sCTA]{1}}{right=of wx0}
      \raevent{wy1}{\DW[\mREL]{y}[\sSYS]{1}}{right=of wx1}
      \raevent{ry1}{\DR[\mACQ]{y}[\sSYS]{1}}{right=2.5em of wy1}
      \event{rx}{\DR[\mWK]{x}[\sCTA]{}}{right=of ry1}
      \rf{wy1}{ry1}
      \wki{wx0}{wx1}
      \rf[out=-15,in=-165]{wx1}{rx}
    \end{tikzinline}}
\end{gather*}
$(\DW{x}{1})\leexists(\DR{x}{})$ is required by \ref{cand-leloc-block}, enforcing publication.

$(\DR{x}{0})$ must be allowed:
\begin{gather*}
  \taglabel[cta]{pub1}
  \PW[\mWK]{x}[\sCTA]{0}\SEMI 
  \PW[\mWK]{x}[\sCTA]{1}\SEMI
  \PW[\mREL]{y}[\sCTA]{1} \PAR
  \PR[\mACQ]{y}[\sCTA]{r}\SEMI
  \PR[\mWK]{x}[\sCTA]{s}
  \\
  \tag{${\ledep}$}
  \hbox{\begin{tikzinline}[node distance=1.5em]
      \event{wx0}{\DW[\mWK]{x}[\sCTA]{0}}{}
      \event{wx1}{\DW[\mWK]{x}[\sCTA]{1}}{right=of wx0}
      \raevent{wy1}{\DW[\mREL]{y}[\sCTA]{1}}{right=of wx1}
      \raevent{ry1}{\DR[\mACQ]{y}[\sCTA]{1}}{right=2.5em of wy1}
      \event{rx}{\DR[\mWK]{x}[\sCTA]{}}{right=of ry1}
      \sync[out=-15,in=-165]{wx0}{wy1}
      \sync{wx1}{wy1}
      \sync{ry1}{rx}
      \rf{wy1}{ry1}
    \end{tikzinline}}
  \\
  \tag{${\lesync}$}
  \hbox{\begin{tikzinline}[node distance=1.5em]
      \event{wx0}{\DW[\mWK]{x}[\sCTA]{0}}{}
      \event{wx1}{\DW[\mWK]{x}[\sCTA]{1}}{right=of wx0}
      \raevent{wy1}{\DW[\mREL]{y}[\sCTA]{1}}{right=of wx1}
      \raevent{ry1}{\DR[\mACQ]{y}[\sCTA]{1}}{right=2.5em of wy1}
      \event{rx}{\DR[\mWK]{x}[\sCTA]{}}{right=of ry1}
      \sync[out=-15,in=-165]{wx0}{wy1}
      \sync{wx1}{wy1}
      \sync{ry1}{rx}
      % \rf{wy1}{ry1}
    \end{tikzinline}}
  \\
  \tag{$\leloc$}
  \hbox{\begin{tikzinline}[node distance=1.5em]
      \event{wx0}{\DW[\mWK]{x}[\sCTA]{0}}{}
      \event{wx1}{\DW[\mWK]{x}[\sCTA]{1}}{right=of wx0}
      \raevent{wy1}{\DW[\mREL]{y}[\sCTA]{1}}{right=of wx1}
      \raevent{ry1}{\DR[\mACQ]{y}[\sCTA]{1}}{right=2.5em of wy1}
      \event{rx}{\DR[\mWK]{x}[\sCTA]{}}{right=of ry1}
      % \rf{wy1}{ry1}
      \wki{wx0}{wx1}
      % \wk[out=-15,in=-165]{wx1}{rx}
    \end{tikzinline}}
\end{gather*}
We do not have $(\DW[\mREL]{y}{1})\lesync (\DR[\mACQ]{y}{1})$ since \ref{cand-lesync-rf} only
requires order for things that are morally strong.  

Another example that may be of interest (nothing morally strong).  Can this $(\DR{x}{0})$?
\begin{gather*}
  %\taglabel[cta]{pub1}
  \PW{x}{0} \SEMI
  \PW{x}{1} \PAR 
  \PW{y}{\PR{x}{}} \PAR
  \IF{\PR{y}{}}\THEN \PR{x}{r} \FI
\end{gather*}

\PTX{} allows TC16 for events that are not mutually strong (\ref{tc16wk}),
but disallows it when events are mutually strong (\ref{tc16sys}).  Note that
$\lesync$ imposes no requirements here.  Fulfillment imposes no order.  This
example shows that \ref{cand-leloc-block} cannot be strengthened to replace
$\leexists$ with $\leloc$.
\begin{gather*}
  \taglabel[wk]{tc16}
  \PR[\mWK]{x}[\sCTA]{r} \SEMI
  \PW[\mWK]{x}[\sCTA]{1}
  \PAR
  \PR[\mWK]{x}[\sCTA]{s} \SEMI
  \PW[\mWK]{x}[\sCTA]{2}
  \\
  \tag{$\ledep$}
  \hbox{\begin{tikzinline}[node distance=1.5em]
      \event{a1}{\DR[\mWK]{x}[\sCTA]{2}}{}
      \event{a2}{\DW[\mWK]{x}[\sCTA]{1}}{right=of a1}
      \event{b1}{\DR[\mWK]{x}[\sCTA]{1}}{right=3em of a2}
      \event{b2}{\DW[\mWK]{x}[\sCTA]{2}}{right=of b1}
      \rf{a2}{b1}
      \rf[out=-165,in=-15]{b2}{a1}
    \end{tikzinline}}
  \\
  \tag{$\lesync$}
  \hbox{\begin{tikzinline}[node distance=1.5em]
      \event{a1}{\DR[\mWK]{x}[\sCTA]{2}}{}
      \event{a2}{\DW[\mWK]{x}[\sCTA]{1}}{right=of a1}
      \event{b1}{\DR[\mWK]{x}[\sCTA]{1}}{right=3em of a2}
      \event{b2}{\DW[\mWK]{x}[\sCTA]{2}}{right=of b1}
    \end{tikzinline}}
  \\
  \tag{$\leloc$}
  \hbox{\begin{tikzinline}[node distance=1.5em]
      \event{a1}{\DR[\mWK]{x}[\sCTA]{2}}{}
      \event{a2}{\DW[\mWK]{x}[\sCTA]{1}}{right=of a1}
      \event{b1}{\DR[\mWK]{x}[\sCTA]{1}}{right=3em of a2}
      \event{b2}{\DW[\mWK]{x}[\sCTA]{2}}{right=of b1}
      \wki{a1}{a2}
      \wki{b1}{b2}
      % \rf{a2}{b1}
      % \rf[out=-165,in=-15]{b2}{a1}
    \end{tikzinline}}
\end{gather*}
\begin{gather*}
  \taglabel[sys]{tc16}
  \PR{x}{r} \SEMI \PW{x}{1}
  \PAR                                              
  \PR{x}{s} \SEMI \PW{x}{2}
  \\
  \tag{${\ledep}={\lesync}$}
  \hbox{\begin{tikzinline}[node distance=1.5em]
      \event{a1}{\DR{x}{2}}{}
      \event{a2}{\DW{x}{1}}{right=of a1}
      \event{b1}{\DR{x}{1}}{right=3em of a2}
      \event{b2}{\DW{x}{2}}{right=of b1}
      \rf{a2}{b1}
      \rf[out=-165,in=-15]{b2}{a1}
    \end{tikzinline}}
  \\
  \tag{$\leloc$}
  \hbox{\begin{tikzinline}[node distance=1.5em]
      \event{a1}{\DR{x}{2}}{}
      \event{a2}{\DW{x}{1}}{right=of a1}
      \event{b1}{\DR{x}{1}}{right=3em of a2}
      \event{b2}{\DW{x}{2}}{right=of b1}
      \wk{a1}{a2}
      \wk{b1}{b2}
      \rf{a2}{b1}
      \rf[out=-165,in=-15]{b2}{a1}
    \end{tikzinline}}
\end{gather*}

About Release-Acquire semantics.  Anton confirms that the following example
is allowed in C11, but disallowed in the \IMM{}.  It is apparently allowed in
C11 with the intention to allow releasing writes to be downgraded to relaxed
in the case that only fulfill relaxed reads.
\begin{gather*}
  \taglabel{LB-REL}
  \PR{x}{r} \SEMI \PW[\mREL]{y}{1}
  \PAR                                             
  \PR{y}{s} \SEMI \PW[\mREL]{x}{1}
  \\
  \tag{${\ledep}={\lesync}$}
  \hbox{\begin{tikzinline}[node distance=1.5em]
      \event{a1}{\DR{x}{1}}{}
      \raevent{a2}{\DW[\mREL]{y}{1}}{right=of a1}
      \event{b1}{\DR{y}{1}}{right=3em of a2}
      \raevent{b2}{\DW[\mREL]{x}{1}}{right=of b1}
      \rf{a2}{b1}
      \rf[out=-165,in=-15]{b2}{a1}
      \sync{a1}{a2}
      \sync{b1}{b2}
    \end{tikzinline}}
\end{gather*}

Another example from Anton.  This is allowed in PTX because it does not
include synchronization in the no-tar axiom, only in coherence and causality.
\begin{gather*}
  \taglabel{LB-data-rel}
  \PR{x}{r} \SEMI \PW{y}{r}
  \PAR                                             
  \PR{y}{s} \SEMI \PW[\mREL]{x}{1}
  \\
  \tag{${\ledep}={\lesync}$}
  \hbox{\begin{tikzinline}[node distance=1.5em]
      \event{a1}{\DR{x}{1}}{}
      \event{a2}{\DW{y}{1}}{right=of a1}
      \event{b1}{\DR{y}{1}}{right=3em of a2}
      \raevent{b2}{\DW[\mREL]{x}{1}}{right=of b1}
      \rf{a2}{b1}
      \rf[out=-165,in=-15]{b2}{a1}
      \po{a1}{a2}
      \sync{b1}{b2}
    \end{tikzinline}}
\end{gather*}


\section{RFI Examples}

Bad example:
\begin{gather*}
  \PEXCHG{x}{r}{2}\SEMI 
  \PR{x}{s}\SEMI
  \PW{y}{s{-}1} \PAR
  \PR{y}{r}\SEMI
  \PW{x}{r}
  \\
  \tag{\cmark\armeight}
  \hbox{\begin{tikzinline}[node distance=1.5em]
      \event{a}{\DR{x}{1}}{}
      \event{b}{\DW{x}{2}}{right=of a}
      \event{c}{\DR{x}{2}}{right=of b}
      \event{d}{\DW{y}{1}}{right=of c}
      \event{e}{\DR{y}{1}}{right=3em of d}
      \event{f}{\DW{x}{1}}{right=of e}
      \rmw{a}{b}
      \rfi{b}{c}
      \dob{c}{d}
      \rfe{d}{e}
      \dob{e}{f}
      \rfe[out=-165,in=-15]{f}{a}
    \end{tikzinline}}
  \\
  \tag{$\ledep$}
  \hbox{\begin{tikzinline}[node distance=1.5em]
      \event{a}{\DR{x}{1}}{}
      \event{b}{\DW{x}{2}}{right=of a}
      \event{c}{\DR{x}{2}}{right=of b}
      \event{d}{\DW{y}{1}}{right=of c}
      \event{e}{\DR{y}{1}}{right=3em of d}
      \event{f}{\DW{x}{1}}{right=of e}
      %\rmw{a}{b}
      \rf{b}{c}
      \po{c}{d}
      \rf{d}{e}
      \po{e}{f}
      \rf[out=-165,in=-15]{f}{a}
    \end{tikzinline}}
  \\
  \tag{${\lesync}$}
  \hbox{\begin{tikzinline}[node distance=1.5em]
      \event{a}{\DR{x}{1}}{}
      \event{b}{\DW{x}{2}}{right=of a}
      \event{c}{\DR{x}{2}}{right=of b}
      \event{d}{\DW{y}{1}}{right=of c}
      \event{e}{\DR{y}{1}}{right=3em of d}
      \event{f}{\DW{x}{1}}{right=of e}
      \rmw{a}{b}
      % \rf{b}{c}
      % \po{c}{d}
      % \rf{d}{e}
      % \po{e}{f}
      % \rf[out=-165,in=-15]{f}{a}
    \end{tikzinline}}
  \\
  \tag{$\leloc$}
  \hbox{\begin{tikzinline}[node distance=1.5em]
      \event{a}{\DR{x}{1}}{}
      \event{b}{\DW{x}{2}}{right=of a}
      \event{c}{\DR{x}{2}}{right=of b}
      \event{d}{\DW{y}{1}}{right=of c}
      \event{e}{\DR{y}{1}}{right=3em of d}
      \event{f}{\DW{x}{1}}{right=of e}
      \wki{a}{b}
      \rf{b}{c}
      % \po{c}{d}
      \rf{d}{e}
      % \po{e}{f}
      \rf[out=-165,in=-15]{f}{a}
      % \wk[out=15,in=165]{c}{f}
    \end{tikzinline}}
\end{gather*}
\begin{gather*}
  \PR{x}{r}\SEMI 
  \PW{x}{2}\SEMI
  \PR{x}{s}\SEMI
  \PW{y}{s{-}1} \PAR
  \PR{y}{r}\SEMI
  \PW{x}{r}
  \\
  \tag{${\ledep}$}
  \hbox{\begin{tikzinline}[node distance=1.5em]
      \event{a}{\DR{x}{1}}{}
      \event{b}{\DW{x}{2}}{right=of a}
      \event{c}{\DR{x}{2}}{right=of b}
      \event{d}{\DW{y}{1}}{right=of c}
      \event{e}{\DR{y}{1}}{right=3em of d}
      \event{f}{\DW{x}{1}}{right=of e}
      % \wk{a}{b}
      \rf{b}{c}
      \po{c}{d}
      \rf{d}{e}
      \po{e}{f}
      \rf[out=-165,in=-15]{f}{a}
    \end{tikzinline}}
  \\
  \tag{${\lesync}$}
  \hbox{\begin{tikzinline}[node distance=1.5em]
      \event{a}{\DR{x}{1}}{}
      \event{b}{\DW{x}{2}}{right=of a}
      \event{c}{\DR{x}{2}}{right=of b}
      \event{d}{\DW{y}{1}}{right=of c}
      \event{e}{\DR{y}{1}}{right=3em of d}
      \event{f}{\DW{x}{1}}{right=of e}
      % \wk{a}{b}
      % \rf{b}{c}
      % \po{c}{d}
      % \rf{d}{e}
      % \po{e}{f}
      % \rf[out=-165,in=-15]{f}{a}
    \end{tikzinline}}
  \\
  \tag{$\leloc$}
  \hbox{\begin{tikzinline}[node distance=1.5em]
      \event{a}{\DR{x}{1}}{}
      \event{b}{\DW{x}{2}}{right=of a}
      \event{c}{\DR{x}{2}}{right=of b}
      \event{d}{\DW{y}{1}}{right=of c}
      \event{e}{\DR{y}{1}}{right=3em of d}
      \event{f}{\DW{x}{1}}{right=of e}
      \wki{a}{b}
      \rf{b}{c}
      % \po{c}{d}
      \rf{d}{e}
      % \po{e}{f}
      \rf[out=-165,in=-15]{f}{a}
      % \wk[out=15,in=165]{c}{f}
    \end{tikzinline}}
\end{gather*}


Anton example 1 (Allowed by ARM) \texttt{[rfi-coe-coe]}
\begin{gather*}
  \taglabel{rfi-coe-coe}
  \PW{x}{2}\SEMI 
  \PR[\mACQ]{x}{r}\SEMI
  \PW{y}{1} \PAR
  \PW{y}{2}\SEMI
  \PW[\mREL]{x}{1}
  \\
  \tag{\cmark\armeight}
  \hbox{\begin{tikzinline}[node distance=1.5em]
      \event{a}{\DW{x}{2}}{}
      \raevent{b}{\DR[\mACQ]{x}{2}}{right=of a}
      \event{c}{\DW{y}{1}}{right=of b}
      \event{d}{\DW{y}{2}}{right=2.5em of c}
      \raevent{e}{\DW[\mREL]{x}{1}}{right=of d}
      \rfi{a}{b}
      \bob{b}{c}
      \coe{c}{d}
      \bob{d}{e}
      \coe[out=-165,in=-15]{e}{a}
    \end{tikzinline}}
  \\
  \tag{$\ledep$}
  \hbox{\begin{tikzinline}[node distance=1.5em]
      \event{a}{\DW{x}{2}}{}
      \raevent{b}{\DR[\mACQ]{x}{2}}{right=of a}
      \event{c}{\DW{y}{1}}{right=of b}
      \event{d}{\DW{y}{2}}{right=2.5em of c}
      \raevent{e}{\DW[\mREL]{x}{1}}{right=of d}
      \rf{a}{b}
      %\sync{b}{c}
      %\wk{c}{d}
      %\sync{d}{e}
      %\wk[out=-165,in=-15]{e}{a}
    \end{tikzinline}}
  \\
  \tag{$\lesync$}
  \hbox{\begin{tikzinline}[node distance=1.5em]
      \event{a}{\DW{x}{2}}{}
      \raevent{b}{\DR[\mACQ]{x}{2}}{right=of a}
      \event{c}{\DW{y}{1}}{right=of b}
      \event{d}{\DW{y}{2}}{right=2.5em of c}
      \raevent{e}{\DW[\mREL]{x}{1}}{right=of d}
      %\rf{a}{b}
      \sync{b}{c}
      %\wk{c}{d}
      \sync{d}{e}
      %\wk[out=-165,in=-15]{e}{a}
    \end{tikzinline}}
  \\
  \tag{$\leloc$}
  \hbox{\begin{tikzinline}[node distance=1.5em]
      \event{a}{\DW{x}{2}}{}
      \raevent{b}{\DR[\mACQ]{x}{2}}{right=of a}
      \event{c}{\DW{y}{1}}{right=of b}
      \event{d}{\DW{y}{2}}{right=2.5em of c}
      \raevent{e}{\DW[\mREL]{x}{1}}{right=of d}
      \rf{a}{b}
      %\sync{b}{c}
      \wk{c}{d}
      %\sync{d}{e}
      \wk[out=-165,in=-15]{e}{a}
    \end{tikzinline}}
\end{gather*}
Internal reads survive acquires \texttt{[rfi-acq-coe-coe]} (where SC read =
\texttt{LDAR})
\begin{gather*}
  \taglabel{rfi-acq-coe-coe}
  \PW{x}{2}\SEMI 
  \PR[\mSC]{z}{s}\SEMI
  \PR[\mSC]{x}{r}\SEMI
  \PW{y}{1} \PAR
  \PW{y}{2}\SEMI
  \PW[\mREL]{x}{1}
  \\
  \tag{\cmark\armeight}
  \hbox{\begin{tikzinline}[node distance=1.5em]
      \event{a}{\DW{x}{2}}{}
      \scevent{b0}{\DR[\mSC]{z}{0}}{right=of a}
      \scevent{b}{\DR[\mSC]{x}{2}}{right=of b0}
      \event{c}{\DW{y}{1}}{right=of b}
      \event{d}{\DW{y}{2}}{right=2.5em of c}
      \raevent{e}{\DW[\mREL]{x}{1}}{right=of d}
      \rfi[out=20,in=160]{a}{b}
      %\bob[out=20,in=160]{b0}{c}
      \bob{b0}{b}
      \bob{b}{c}
      \coe{c}{d}
      \bob{d}{e}
      \coe[out=-165,in=-15]{e}{a}
    \end{tikzinline}}
\end{gather*}
And release-acquire pairs \texttt{[rfi-ra-coe-coe]} (where acquiring read
= \texttt{LDAPR})
\begin{gather*}
  \taglabel{rfi-ra-coe-coe2}
  \PW{x}{2}\SEMI 
  \PW[\mREL]{w}{1}\SEMI
  \PR[\mACQ]{z}{s}\SEMI
  \PR[\mACQ]{x}{r}\SEMI
  \PW{y}{1}
  \\[-1ex]\PAR
  \PW{y}{2}\SEMI
  \PW[\mREL]{x}{1}
  \PAR
  \PR{w}{r}\SEMI
  \PW{z}{1}\SEMI
  \\
  \tag{\cmark\armeight}
  \hbox{\begin{tikzinline}[node distance=1.5em]
      \event{a}{\DW{x}{2}}{}
      \raevent{b1}{\DW[\mREL]{w}{1}}{right=of a}
      \raevent{b0}{\DR[\mACQ]{z}{1}}{right=of b1}
      \raevent{b}{\DR[\mACQ]{x}{2}}{right=of b0}
      \event{c}{\DW{y}{1}}{right=of b}
      \event{d}{\DW{y}{2}}{right=2.5em of c}
      \raevent{e}{\DW[\mREL]{x}{1}}{right=of d}
      \rfi[out=20,in=160]{a}{b}
      %\bob[out=20,in=160]{b0}{c}
      \bob{a}{b1}
      %\bob{b1}{b0}
      \bob{b0}{b}
      \bob{b}{c}
      \coe{c}{d}
      \bob{d}{e}
      \coe[out=-165,in=-15]{e}{a}
      \event{f1}{\DR{w}{1}}{below=of b1}
      \event{f0}{\DW{z}{1}}{below=of b0}
      %\data{f1}{f0}
      \rfe{b1}{f1}
      \rfe{f0}{b0}
    \end{tikzinline}}
\end{gather*}
% \begin{gather*}
%   \taglabel{rfi-ra-coe-coe}
%   \PW{x}{2}\SEMI 
%   \PW[\mREL]{w}{1}\SEMI
%   \PR[\mACQ]{z}{s}\SEMI
%   \PR[\mACQ]{x}{r}\SEMI
%   \PW{y}{1} \PAR
%   \PW{y}{2}\SEMI
%   \PW[\mREL]{x}{1}
%   \\
%   \tag{\cmark\armeight}
%   \hbox{\begin{tikzinline}[node distance=1.5em]
%       \event{a}{\DW{x}{2}}{}
%       \raevent{b1}{\DW[\mREL]{w}{1}}{right=of a}
%       \raevent{b0}{\DR[\mACQ]{z}{0}}{right=of b1}
%       \raevent{b}{\DR[\mACQ]{x}{2}}{right=of b0}
%       \event{c}{\DW{y}{1}}{right=of b}
%       \event{d}{\DW{y}{2}}{right=2.5em of c}
%       \raevent{e}{\DW[\mREL]{x}{1}}{right=of d}
%       \rfi[out=20,in=160]{a}{b}
%       %\bob[out=20,in=160]{b0}{c}
%       \bob{a}{b1}
%       %\bob{b1}{b0}
%       \bob{b0}{b}
%       \bob{b}{c}
%       \coe{c}{d}
%       \bob{d}{e}
%       \coe[out=-165,in=-15]{e}{a}
%     \end{tikzinline}}
% \end{gather*}
But not if either acquire is strengthened to SC (where SC read =
\texttt{LDAR}).  The execution is also disallowed if an external thread
places order between the $\mRA$ accesses:
\begin{gather*}
  \taglabel{rfi-ra-data-coe-coe}
  \PW{x}{2}\SEMI 
  \PW[\mREL]{w}{1}\SEMI
  \PR[\mACQ]{z}{s}\SEMI
  \PR[\mACQ]{x}{r}\SEMI
  \PW{y}{1}
  \\[-1ex]\PAR
  \PW{y}{2}\SEMI
  \PW[\mREL]{x}{1}
  \PAR
  \PR{w}{r}\SEMI
  \PW{z}{r}\SEMI
  \\
  \tag{\xmark\armeight}
  \hbox{\begin{tikzinline}[node distance=1.5em]
      \event{a}{\DW{x}{2}}{}
      \raevent{b1}{\DW[\mREL]{w}{1}}{right=of a}
      \raevent{b0}{\DR[\mACQ]{z}{1}}{right=of b1}
      \raevent{b}{\DR[\mACQ]{x}{2}}{right=of b0}
      \event{c}{\DW{y}{1}}{right=of b}
      \event{d}{\DW{y}{2}}{right=2.5em of c}
      \raevent{e}{\DW[\mREL]{x}{1}}{right=of d}
      \rfi[out=20,in=160]{a}{b}
      %\bob[out=20,in=160]{b0}{c}
      \bob{a}{b1}
      %\bob{b1}{b0}
      \bob{b0}{b}
      \bob{b}{c}
      \coe{c}{d}
      \bob{d}{e}
      \coe[out=-165,in=-15]{e}{a}
      \event{f1}{\DR{w}{1}}{below=of b1}
      \event{f0}{\DW{z}{1}}{below=of b0}
      \data{f1}{f0}
      \rfe{b1}{f1}
      \rfe{f0}{b0}
    \end{tikzinline}}
\end{gather*}

To allow this, weaken $\mRA$ to $\mRLX$ when read fulfilled by relaxed write
of same thread (don't need to allow this when the write is part of an \RMW{}).
\begin{gather*}
  \PW{x}{2}\SEMI 
  \PR[\mACQ]{x}{r}\SEMI
  \PW{y}{1} \PAR
  \PW{y}{2}\SEMI
  \PW[\mREL]{x}{1}
  \\
  \hbox{\begin{tikzinline}[node distance=1.5em]
      \event{a}{\DW{x}{2}}{}
      \event{b}{\DR{x}{2}}{right=of a}
      \event{c}{\DW{y}{1}}{right=of b}
      \event{d}{\DW{y}{2}}{right=2.5em of c}
      \raevent{e}{\DW[\mREL]{x}{1}}{right=of d}
      \rf{a}{b}
      %\sync{b}{c}
      \wk{c}{d}
      \sync{d}{e}
      \wk[out=-165,in=-15]{e}{a}
    \end{tikzinline}}
\end{gather*}

RF variant \texttt{[rfi-rfe-coe]}:
\begin{gather*}
  \taglabel{rfi-rfe-coe}
  \PW{x}{2}\SEMI 
  \PR[\mACQ]{x}{r}\SEMI
  \PW{y}{1} \PAR
  \PR{y}{s}\SEMI
  \PW[\mREL]{x}{1}
  \\
  \tag{\cmark\armeight}
  \hbox{\begin{tikzinline}[node distance=1.5em]
      \event{a}{\DW{x}{2}}{}
      \raevent{b}{\DR[\mACQ]{x}{2}}{right=of a}
      \event{c}{\DW{y}{1}}{right=of b}
      \event{d}{\DR{y}{1}}{right=2.5em of c}
      \raevent{e}{\DW[\mREL]{x}{1}}{right=of d}
      \rfi{a}{b}
      \bob{b}{c}
      \rfe{c}{d}
      \bob{d}{e}
      \coe[out=-165,in=-15]{e}{a}
    \end{tikzinline}}
\end{gather*}

\tso{} variant \texttt{[rfi-fre-coe2]}:
\begin{gather*}
  \taglabel{rfi-coe-coe2}
  \PW{x}{2}\SEMI 
  \PR[\mACQ]{x}{r}\SEMI
  \PR{y}{s} \PAR
  \PW{y}{2}\SEMI
  \PW[\mREL]{x}{1}
  \\
  \tag{\cmark\armeight}
  \hbox{\begin{tikzinline}[node distance=1.5em]
      \event{a}{\DW{x}{2}}{}
      \raevent{b}{\DR[\mACQ]{x}{2}}{right=of a}
      \event{c}{\DR{y}{0}}{right=of b}
      \event{d}{\DW{y}{2}}{right=2.5em of c}
      \raevent{e}{\DW[\mREL]{x}{1}}{right=of d}
      \rfi{a}{b}
      \bob{b}{c}
      \fre{c}{d}
      \bob{d}{e}
      \coe[out=-165,in=-15]{e}{a}
    \end{tikzinline}}
  \\
  \tag{\cmark\tso}
  \hbox{\begin{tikzinline}[node distance=1.5em]
      \event{a}{\DW{x}{2}}{}
      \event{b}{\DR{x}{2}}{right=of a}
      \event{c}{\DR{y}{0}}{right=of b}
      \event{d}{\DW{y}{2}}{right=2.5em of c}
      \event{e}{\DW{x}{1}}{right=of d}
      %\rfi{a}{b}
      \lob{b}{c}
      \fre{c}{d}
      \lob{d}{e}
      \coe[out=-165,in=-15]{e}{a}
    \end{tikzinline}}
\end{gather*}
Note that \tso{} does not order W to R in local order, even in poloc.
Nonetheless, \tso{} disallows the following because of local visibility in first thread.
\begin{gather*}
  \PW{x}{2}\SEMI 
  \PR{x}{r} \PAR
  \PW{x}{1}\SEMI
  \PR{x}{s}
  \\
  \tag{\xmark\tso}
  \hbox{\begin{tikzinline}[node distance=1.5em]
      \event{a}{\DW{x}{2}}{}
      \event{b}{\DR{x}{1}}{right=of a}
      \event{c}{\DW{x}{1}}{right=2.5em of b}
      \event{d}{\DR{x}{2}}{right=of c}
      \coe[out=-165,in=-15]{c}{a}
      \rfe{c}[above]{b}
      \rfe[out=15,in=165]{a}{d}
      \fr{b}[above]{a}
    \end{tikzinline}}
\end{gather*}
\cite{DBLP:conf/hipc/HighamK00} describe \tso{} as a linearization of partial
order including:
\begin{itemize}
\item ${\rpoloc}$
\item lws = ${\rpox};[\mathsf{W}]$
\item $\bEv\xpox\aEv$ when $\cEv\xrfe\bEv\xpox\aEv$
\end{itemize}
\cite{armed-cats} describe \tso{} as linearization of partial order
satisfying internal visibility and including
\begin{itemize}
\item $[\mathsf{W}];\rpox;[\mathsf{W}]$
\item $\bEv\xpox\aEv$ when $\cEv\xrfe\bEv\xpox\aEv$, from \verb|(range(rfe) * _)|
\item $[\mathsf{R}];\rpox;[\mathsf{W}]$, from \verb|(rfi^-1; lob)|
\end{itemize}
Ignoring fences and \RMW{}s:
\begin{verbatim}
let rec lob = po \ ([W]; po; [R])
let IM0 = loc & ((IW * (M\IW)) | ((W\FW) * FW))
let gc-req = (W * _) | ((R * _) & ((range(rfe) * _) | (rfi^-1; lob))
let preorder-gcb = IM0 | lob & gc-req
\end{verbatim}
% \begin{verbatim}
% let rec lob = po \ ([W]; po; [R])
%         | [W]; po; [MFENCE]; po; [R]
%         | [W]; po; [R & X]
%         | [W & X]; po; [R]
%         | lob; lob
% let IM0 = loc & ((IW * (M\IW)) | ((W\FW) * FW))
% let gc-req = (W * _) | ((R * _) & ((range(rfe) * _) | (rfi^-1; lob))
% let preorder-gcb = IM0 | lob & gc-req
% \end{verbatim}


Double FRE variant \texttt{[rfi-fre-fre]}:
\begin{gather*}
  \taglabel{rfi-fre-fre}
  \PW{x}{2}\SEMI 
  \PR[\mACQ]{x}{r}\SEMI
  \PR{y}{s} \PAR
  \PW{y}{2}\SEMI
  \PF{}\SEMI
  \PR{x}{r}
  \\
  \tag{\cmark\armeight}
  \hbox{\begin{tikzinline}[node distance=1.5em]
      \event{a}{\DW{x}{2}}{}
      \raevent{b}{\DR[\mACQ]{x}{2}}{right=of a}
      \event{c}{\DR{y}{0}}{right=of b}
      \event{d}{\DW{y}{2}}{right=2.5em of c}
      \event{e}{\DF{}}{right=of d}
      \event{f}{\DR{x}{0}}{right=of e}
      \rfi{a}{b}
      \bob{b}{c}
      \fre{c}{d}
      \bob{d}{e}
      \bob{e}{f}
      \fre[out=-165,in=-15]{f}{a}
    \end{tikzinline}}
\end{gather*}

It does not seem possible to do this only with $\rrfe$.
ARM disallows this \texttt{[data-rfi-rfe-rfe]}:
\begin{gather*}
  \taglabel{data-rfi-rfe-rfe}
  \PW{x}{\PR{z}{}} \SEMI
  \PR[\mACQ]{x}{r}\SEMI
  \PW{y}{1} \PAR
  \PW{z}{\PR{y}{}}
  \\
  \tag{\xmark\armeight}
  \hbox{\begin{tikzinline}[node distance=1.5em]
      \event{a}{\DR{z}{1}}{}
      \event{b}{\DW{x}{1}}{right=of a}
      \raevent{c}{\DR[\mACQ]{x}{1}}{right=of b}
      \event{d}{\DW{y}{1}}{right=of c}
      \event{e}{\DW{y}{1}}{right=2.5em of d}
      \event{f}{\DW{z}{1}}{right=of e}
      \data{a}{b}
      \rfi{b}{c}
      \bob{c}{d}
      \data{e}{f}
      \rfe[out=-165,in=-15]{f}{a}
      \rfe{d}{e}
    \end{tikzinline}}
\end{gather*}

It also disallows \texttt{[ctrl-rfi-rfe-rfe]}:
\begin{gather*}
  \taglabel{ctrl-rfi-rfe-rfe}
  \IF{\PR{z}{}}\THEN\FI \SEMI
  \PW{x}{1} \SEMI
  \PR[\mACQ]{x}{r}\SEMI
  \PW{y}{1}
  \PAR
  \PW{z}{\PR{y}{}}
  \\
  \tag{\xmark\armeight}
  \hbox{\begin{tikzinline}[node distance=1.5em]
      \event{a}{\DR{z}{1}}{}
      \event{b}{\DW{x}{1}}{right=of a}
      \raevent{c}{\DR[\mACQ]{x}{1}}{right=of b}
      \event{d}{\DW{y}{1}}{right=of c}
      \event{e}{\DW{y}{1}}{right=2.5em of d}
      \event{f}{\DW{z}{1}}{right=of e}
      \ctrl[out=15,in=165]{a}{d}
      \rfi{b}{c}
      \bob{c}[below]{d}
      \data{e}{f}
      \rfe[out=-165,in=-15]{f}{a}
      \rfe{d}{e}
    \end{tikzinline}}
\end{gather*}

ARM allows some counterintuitive results for SC access \texttt{[ctrl-rfi-fre-rfe]}:
\begin{gather*}
  \taglabel{ctrl-rfi-fre-rfe}
  \IF{\PR{x}{}}\THEN\FI\SEMI
  \PW{x}{2} \SEMI
  \PR[\mSC]{x}{r}\SEMI
  \PR[\mSC]{y}{s} \PAR
  \PW[\mSC]{y}{2}\SEMI
  \PW[\mSC]{x}{1}
  \\
  \tag{\cmark\armeight}
  \hbox{\begin{tikzinline}[node distance=1.5em]
      \event{a}{\DR{x}{1}}{}
      \event{b}{\DW{x}{2}}{right=of a}
      \scevent{c}{\DR[\mSC]{x}{2}}{right=of b}
      \scevent{d}{\DR[\mSC]{y}{0}}{right=of c}
      \scevent{e}{\DW[\mSC]{y}{2}}{right=2.5em of d}
      \scevent{f}{\DW[\mSC]{x}{1}}{right=of e}
      \ctrl{a}{b}
      \rfi{b}{c}
      \bob{c}{d}
      \bob{e}{f}
      \fre{d}{e}
      \rfe[out=-165,in=-15]{f}{a}
    \end{tikzinline}}
\end{gather*}
Not possible with $\rcoe$ \texttt{[ctrl-rfi-coe-rfe]}:
\begin{gather*}
  \taglabel{ctrl-rfi-coe-rfe}
  \IF{\PR{x}{}}\THEN\FI\SEMI
  \PW{x}{2} \SEMI
  \PR[\mSC]{x}{r}\SEMI
  \PW[\mSC]{y}{1} \PAR
  \PW[\mSC]{y}{2}\SEMI
  \PW[\mSC]{x}{1}
  \\
  \tag{\xmark\armeight}
  \hbox{\begin{tikzinline}[node distance=1.5em]
      \event{a}{\DR{x}{1}}{}
      \event{b}{\DW{x}{2}}{right=of a}
      \scevent{c}{\DR[\mSC]{x}{2}}{right=of b}
      \scevent{d}{\DW[\mSC]{y}{1}}{right=of c}
      \scevent{e}{\DW[\mSC]{y}{2}}{right=2.5em of d}
      \scevent{f}{\DW[\mSC]{x}{1}}{right=of e}
      \ctrl[out=15,in=165]{a}{d}
      \rfi{b}{c}
      \bob{c}{d}
      \bob{e}{f}
      \coe{d}{e}
      \rfe[out=-165,in=-15]{f}{a}
    \end{tikzinline}}
\end{gather*}

This is not allowed with a data dependency instead of a control dependency \texttt{[data-rfi-fre-rfe]}:
\begin{gather*}
  \taglabel{data-rfi-fre-rfe}
  \PW{x}{\PR{x}{}{+}1} \SEMI
  \PR[\mSC]{x}{r}\SEMI
  \PR[\mSC]{y}{s} \PAR
  \PW[\mSC]{y}{1}\SEMI
  \PW[\mSC]{x}{1}
  \\
  \tag{\xmark\armeight}
  \hbox{\begin{tikzinline}[node distance=1.5em]
      \event{a}{\DR{x}{1}}{}
      \event{b}{\DW{x}{2}}{right=of a}
      \scevent{c}{\DR[\mSC]{x}{2}}{right=of b}
      \scevent{d}{\DR[\mSC]{y}{0}}{right=of c}
      \scevent{e}{\DW[\mSC]{y}{1}}{right=2.5em of d}
      \scevent{f}{\DW[\mSC]{x}{1}}{right=of e}
      \data{a}{b}
      \rfi{b}{c}
      \bob{c}{d}
      \bob{e}{f}
      \fre{d}{e}
      \rfe[out=-165,in=-15]{f}{a}
    \end{tikzinline}}
\end{gather*}

\section{SC Examples}

\begin{example}
  Consider \iriw{} with all $\mRA$ access:
  \begin{gather*}
    \PW[\mREL]{x}{1}
    \PAR
    \PR[\mACQ]{x}{r}\SEMI \PR[\mACQ]{y}{s}
    \PAR
    \PW[\mREL]{y}{1}
    \PAR
    \PR[\mACQ]{y}{r}\SEMI \PR[\mACQ]{x}{s}
    \taglabel{IRIW-acq-acq}
    \\
    \tag{\cmark\ppc,\cXI}
    \hbox{\begin{tikzinline}[node distance=1.5em]
        \raevent{wx1}{\DW[\mREL]{x}{1}}{}
        \raevent{rx1}{\DR[\mACQ]{x}{1}}{right=2.5em of wx1}
        \raevent{ry0}{\DR[\mACQ]{y}{0}}{right=of rx1}
        \raevent{wy1}{\DW[\mREL]{y}{1}}{right=2.5em of ry0}
        \raevent{ry1}{\DR[\mACQ]{y}{1}}{right=2.5em of wy1}
        \raevent{rx0}{\DR[\mACQ]{x}{0}}{right=of ry1}
        \sync{rx1}{ry0}
        \sync{ry1}{rx0}
        \rf{wx1}{rx1}
        \rf{wy1}{ry1}
        \wk[out=-165,in=-15]{rx0}{wx1}
        \wk{ry0}{wy1}
      \end{tikzinline}}
  \end{gather*}
  We allow this execution:
  \begin{gather*}
    \tag{$\ledep$}
    \hbox{\begin{tikzinline}[node distance=1.5em]
        \raevent{wx1}{\DW[\mREL]{x}{1}}{}
        \raevent{rx1}{\DR[\mACQ]{x}{1}}{right=2.5em of wx1}
        \raevent{ry0}{\DR[\mACQ]{y}{0}}{right=of rx1}
        \raevent{wy1}{\DW[\mREL]{y}{1}}{right=2.5em of ry0}
        \raevent{ry1}{\DR[\mACQ]{y}{1}}{right=2.5em of wy1}
        \raevent{rx0}{\DR[\mACQ]{x}{0}}{right=of ry1}
        \rf{wx1}{rx1}
        \rf{wy1}{ry1}
      \end{tikzinline}}
    \\
    \tag{$\lesync$}
    \hbox{\begin{tikzinline}[node distance=1.5em]
        \raevent{wx1}{\DW[\mREL]{x}{1}}{}
        \raevent{rx1}{\DR[\mACQ]{x}{1}}{right=2.5em of wx1}
        \raevent{ry0}{\DR[\mACQ]{y}{0}}{right=of rx1}
        \raevent{wy1}{\DW[\mREL]{y}{1}}{right=2.5em of ry0}
        \raevent{ry1}{\DR[\mACQ]{y}{1}}{right=2.5em of wy1}
        \raevent{rx0}{\DR[\mACQ]{x}{0}}{right=of ry1}
        \sync{rx1}{ry0}
        \sync{ry1}{rx0}
        \rf{wx1}{rx1}
        \rf{wy1}{ry1}
      \end{tikzinline}}
    \\
    \tag{$\leloc$}
    \hbox{\begin{tikzinline}[node distance=1.5em]
        \raevent{wx1}{\DW[\mREL]{x}{1}}{}
        \raevent{rx1}{\DR[\mACQ]{x}{1}}{right=2.5em of wx1}
        \raevent{ry0}{\DR[\mACQ]{y}{0}}{right=of rx1}
        \raevent{wy1}{\DW[\mREL]{y}{1}}{right=2.5em of ry0}
        \raevent{ry1}{\DR[\mACQ]{y}{1}}{right=2.5em of wy1}
        \raevent{rx0}{\DR[\mACQ]{x}{0}}{right=of ry1}
        \rf{wx1}{rx1}
        \rf{wy1}{ry1}
        \wk[out=-165,in=-15]{rx0}{wx1}
        \wk{ry0}{wy1}
      \end{tikzinline}}
  \end{gather*}
  \ref{IRIW-acq-sc1}, is allowed by trailing-sync compilation to power
  \cite[\textsection 1]{DBLP:conf/pldi/LahavVKHD17}.
  \begin{gather*}
    \PW[\mSC]{x}{1}
    \PAR
    \PR[\mACQ]{x}{r}\SEMI \PR[\mSC]{y}{s}
    \PAR
    \PW[\mSC]{y}{1}
    \PAR
    \PR[\mACQ]{y}{r}\SEMI \PR[\mSC]{x}{s}
    \taglabel{IRIW-acq-sc1}
    \\
    \tag{\cmark\ppc,\xmark\cXI}
    \hbox{\begin{tikzinline}[node distance=1.5em]
        \scevent{wx1}{\DW[\mSC]{x}{1}}{}
        \raevent{rx1}{\DR[\mACQ]{x}{1}}{right=2.5em of wx1}
        \scevent{ry0}{\DR[\mSC]{y}{0}}{right=of rx1}
        \scevent{wy1}{\DW[\mSC]{y}{1}}{right=2.5em of ry0}
        \raevent{ry1}{\DR[\mACQ]{y}{1}}{right=2.5em of wy1}
        \scevent{rx0}{\DR[\mSC]{x}{0}}{right=of ry1}
        \sync{rx1}{ry0}
        \sync{ry1}{rx0}
        \rf{wx1}{rx1}
        \rf{wy1}{ry1}
        \wk[out=-165,in=-15]{rx0}{wx1}
        \wk{ry0}{wy1}
      \end{tikzinline}}
  \end{gather*}
  To model this it is convenient that synchronization is not included in
  dependency order:
  \begin{itemize}
  \item add $\mSC$ bullet to def of $\leexists$ in \ref{cand-leloc-block},
  \item add SC access to $\rsyncdelaysdef$.
  \end{itemize}
  \begin{gather*}
    \tag{$\ledep$}
    \hbox{\begin{tikzinline}[node distance=1.5em]
        \scevent{wx1}{\DW[\mSC]{x}{1}}{}
        \raevent{rx1}{\DR[\mACQ]{x}{1}}{right=2.5em of wx1}
        \scevent{ry0}{\DR[\mSC]{y}{0}}{right=of rx1}
        \scevent{wy1}{\DW[\mSC]{y}{1}}{right=2.5em of ry0}
        \raevent{ry1}{\DR[\mACQ]{y}{1}}{right=2.5em of wy1}
        \scevent{rx0}{\DR[\mSC]{x}{0}}{right=of ry1}
        \rf{wx1}{rx1}
        \rf{wy1}{ry1}
        \wk[out=-165,in=-15]{rx0}{wx1}
        \wk{ry0}{wy1}
      \end{tikzinline}}    
    \\
    \tag{$\lesync$}
    \hbox{\begin{tikzinline}[node distance=1.5em]
        \scevent{wx1}{\DW[\mSC]{x}{1}}{}
        \raevent{rx1}{\DR[\mACQ]{x}{1}}{right=2.5em of wx1}
        \scevent{ry0}{\DR[\mSC]{y}{0}}{right=of rx1}
        \scevent{wy1}{\DW[\mSC]{y}{1}}{right=2.5em of ry0}
        \raevent{ry1}{\DR[\mACQ]{y}{1}}{right=2.5em of wy1}
        \scevent{rx0}{\DR[\mSC]{x}{0}}{right=of ry1}
        \sync{rx1}{ry0}
        \sync{ry1}{rx0}
        \rf{wx1}{rx1}
        \rf{wy1}{ry1}
      \end{tikzinline}}    
    \\
    \tag{$\leloc$}
    \hbox{\begin{tikzinline}[node distance=1.5em]
        \scevent{wx1}{\DW[\mSC]{x}{1}}{}
        \raevent{rx1}{\DR[\mACQ]{x}{1}}{right=2.5em of wx1}
        \scevent{ry0}{\DR[\mSC]{y}{0}}{right=of rx1}
        \scevent{wy1}{\DW[\mSC]{y}{1}}{right=2.5em of ry0}
        \raevent{ry1}{\DR[\mACQ]{y}{1}}{right=2.5em of wy1}
        \scevent{rx0}{\DR[\mSC]{x}{0}}{right=of ry1}
        \rf{wx1}{rx1}
        \rf{wy1}{ry1}
        \wk[out=-165,in=-15]{rx0}{wx1}
        \wk{ry0}{wy1}
      \end{tikzinline}}    
  \end{gather*}
  This correctly forbids the all $\mSC$ version:
  \begin{gather*}
    \PW[\mSC]{x}{1}
    \PAR
    \PR[\mSC]{x}{r}\SEMI \PR[\mSC]{y}{s}
    \PAR
    \PW[\mSC]{y}{1}
    \PAR
    \PR[\mSC]{y}{r}\SEMI \PR[\mSC]{x}{s}
    \taglabel{IRIW-sc-sc}
    \\
    \tag{$\ledep$}
    \hbox{\begin{tikzinline}[node distance=1.5em]
        \scevent{wx1}{\DW[\mSC]{x}{1}}{}
        \scevent{rx1}{\DR[\mSC]{x}{1}}{right=2.5em of wx1}
        \scevent{ry0}{\DR[\mSC]{y}{0}}{right=of rx1}
        \scevent{wy1}{\DW[\mSC]{y}{1}}{right=2.5em of ry0}
        \scevent{ry1}{\DR[\mSC]{y}{1}}{right=2.5em of wy1}
        \scevent{rx0}{\DR[\mSC]{x}{0}}{right=of ry1}
        \sync{rx1}{ry0}
        \sync{ry1}{rx0}
        \rf{wx1}{rx1}
        \rf{wy1}{ry1}
        \wk[out=-165,in=-15]{rx0}{wx1}
        \wk{ry0}{wy1}
      \end{tikzinline}}
  \end{gather*}
  
\end{example}  

\begin{example}
  Thin air with an SC antidependency:
  \begin{gather*}
    \PW[\mSC]{y}{\PR{x}{}}
    \PAR \PW[\mSC]{y}{2}
    \PAR \PW{x}{\PR{y}{}{-}1}
    \\
    \tag{$\ledep$}
    \hbox{\begin{tikzinline}[node distance=1.5em]
        \event{a}{\DR{x}{1}}{}
        \scevent{b}{\DW[\mSC]{y}{1}}{right=of a}
        \scevent{c}{\DW[\mSC]{y}{2}}{right=2.5em of b}
        \event{d}{\DR{y}{2}}{right=2.5em of c}
        \event{e}{\DW{x}{1}}{right=of d}
        \po{a}{b}
        \wk{b}{c}
        \rf{c}{d}
        \po{d}{e}
        \rf[out=-165,in=-15]{e}{a}
      \end{tikzinline}}
  \end{gather*}
\end{example}


\ref{IRIW-acq-sc2} is allowed by trailing-sync compilation to power
\cite[\textsection 1]{DBLP:conf/pldi/LahavVKHD17}.
\begin{gather*}
  \PW[\mSC]{x}{1}
  \PAR
  \PR[\mACQ]{x}{r}\SEMI \PR[\mSC]{y}{s}
  \PAR
  \PW[\mSC]{y}{1}
  \PAR
  \PR[\mACQ]{y}{r}\SEMI \PR[\mSC]{x}{s}
  \taglabel{IRIW-acq-sc2}
  \\
  \tag{\cmark\ppc,\rcXI}
  \hbox{\begin{tikzinline}[node distance=1.5em]
      \scevent{wx1}{\DW[\mSC]{x}{1}}{}
      \raevent{rx1}{\DR[\mACQ]{x}{1}}{right=2.5em of wx1}
      \scevent{ry0}{\DR[\mSC]{y}{0}}{right=of rx1}
      \scevent{wy1}{\DW[\mSC]{y}{1}}{right=2.5em of ry0}
      \raevent{ry1}{\DR[\mACQ]{y}{1}}{right=2.5em of wy1}
      \scevent{rx0}{\DR[\mSC]{x}{0}}{right=of ry1}
      \sync{rx1}{ry0}
      \sync{ry1}{rx0}
      \rf{wx1}{rx1}
      \rf{wy1}{ry1}
      \wk[out=-165,in=-15]{rx0}{wx1}
      \wk{ry0}{wy1}
    \end{tikzinline}}
\end{gather*}
This example is hard to get right for power because it must be allowed with
$\mRA$ reads, but disallowed with $\mSC$ reads.  This seems unsolvable: To
allow the version with $\mRA$, we would need to weaken the order between the
reads in each thread for the $\mRA$ case, and that would break publication.



Leading sync is also unsound in \cXI{} with \RMW{}
\cite[\textsection 2.1]{DBLP:conf/pldi/LahavVKHD17}.
\begin{gather*}
  \PW[\mSC]{x}{1} \SEMI \PW[\mREL]{y}{1}
  \PAR
  \PFADD[\mSC][\mSC]{y}{}{1} \SEMI \PR{y}{s}
  \PAR
  \PW[\mSC]{y}{3} \SEMI \PR[\mSC]{x}{s}
  \taglabel{Z6.U}
  \\
  \tag{\cmark\ppc,\rcXI}
  \hbox{\begin{tikzinline}[node distance=1.5em]
      \scevent{a}{\DW[\mSC]{x}{1}}{}
      \raevent{b}{\DW[\mREL]{y}{1}}{right=of a}
      \scevent{c1}{\DR[\mSC]{y}{1}}{right=2.5em of b}
      \scevent{c2}{\DW[\mSC]{y}{2}}{right=of c1}
      \event{d}{\DR{y}{3}}{right=of c2}
      \scevent{e}{\DW[\mSC]{y}{3}}{right=2.5 em of d}
      \scevent{f}{\DR[\mSC]{x}{0}}{right=of e}
      \sync{a}{b}
      \rf{b}{c1}
      \rf{e}{d}
      \rmw{c1}{c2}
      %\wk{c2}{d}
      \wk[out=-15,in=-165]{c2}{e}
      % \sync[out=-15,in=-165]{c1}{d}
      %\wk{c2}{d}
      \sync{e}{f}
      \wk[out=-165,in=-15]{f}{a}
    \end{tikzinline}}
\end{gather*}
Leading sync is also unsound in \cXI{} with SC fences
\cite[\textsection A.1]{DBLP:conf/pldi/LahavVKHD17}.
\begin{gather*}
  \PW{x}{2} \SEMI \PF{\mSC} \SEMI \PR{y}{r}
  \PAR
  \PW[\mSC]{y}{1}
  \PAR
  \PR[\mACQ]{y}{r} \SEMI \PW[\mREL]{x}{1}  \SEMI \PR{x}{s}
  \PAR
  \PR[\mSC]{x}{r}
   \taglabel{rsync+rsc}
  \\
  \tag{\cmark\rcXI}
  \hbox{\begin{tikzinline}[node distance=1.5em]
      \event{a}{\DW{x}{2}}{}
      \event{b}{\DF{\mSC}}{right=of a}
      \event{c}{\DR{y}{0}}{right=of b}
      \scevent{d}{\DW[\mSC]{y}{1}}{right=2.5em of c}
      \raevent{e}{\DR[\mACQ]{y}{1}}{right=2.5em of d}
      \raevent{f}{\DW[\mREL]{x}{1}}{right=of e}
      \event{g}{\DR{x}{2}}{right=of f}
      \scevent{h}{\DR[\mSC]{x}{1}}{right=2.5em of g}
      \sync{a}{b}
      \sync{b}{c}
      \rf{d}{e}
      \rf[out=-15,in=-165]{f}{h}
      \wk[in=-15,out=-165]{f}{a}
      %\rf[out=-15,in=-165]{a}{g}
      \wk{c}{d}
      \wki{f}{g}
      \sync{e}{f}
      %\sync[out=15,in=165]{e}{g}
    \end{tikzinline}}
\end{gather*}
Fulfillment of $(\DR{x}{2})$ requires that either
\begin{math}
  (\DW[\mREL]{x}{1})
  \xwk
  (\DW{x}{2})
\end{math}
or 
\begin{math}
  (\DR{x}{2})
  \xwk
  (\DW[\mREL]{x}{1}).
\end{math}
It's interesting that in the pomset, $(\DR[\mSC]{x}{1})$ is not needed to get
a cycle.

There is a long discussion of this in \cite[\textsection 5.2,
Fig.~17]{DBLP:journals/pacmpl/BenderP19}, where they also discuss this example:
\begin{gather*}
  \PW[\mSC]{x}{1}\SEMI \PW{x}{2}
  \PAR
  \PW[\mSC]{y}{1}\SEMI \PW{y}{2}
  \PAR
  \PR[\mACQ]{x}{r}\SEMI \PR[\mSC]{y}{s}
  \PAR
  \PR[\mACQ]{y}{r}\SEMI \PR[\mSC]{x}{s}
  \taglabel{IRIW-sc-rlx-acq}
  \\
  \tag{\cmark\rcXI}
  \hbox{\begin{tikzinline}[node distance=1.5em]
      \scevent{wx1}{\DW[\mSC]{x}{1}}{}
      \event{wx2}{\DW{x}{2}}{right=of wx1}
      \scevent{wy1}{\DW[\mSC]{y}{1}}{below=4ex of wx1}
      \event{wy2}{\DW{y}{2}}{right=of wy1}
      \raevent{ry1}{\DR[\mACQ]{y}{2}}{right=2.5em of wy2}
      \scevent{rx0}{\DR[\mSC]{x}{0}}{right=of ry1}
      \raevent{rx1}{\DR[\mACQ]{x}{2}}{right=2.5 em of wx2}
      \scevent{ry0}{\DR[\mSC]{y}{0}}{right=of rx1}
      \sync{rx1}{ry0}
      \sync{ry1}{rx0}
      \rf{wx2}{rx1}
      \rf{wy2}{ry1}
      \wk{rx0}{wx1}
      \wk{ry0}{wy1}
      \wk{wx1}{wx2}
      \wk{wy1}{wy2}
    \end{tikzinline}}
\end{gather*}


\cite[\textsection A.2]{DBLP:conf/pldi/LahavVKHD17} claims that \armeight{}
allows this \texttt{[RWC+acq+sc]}, but \href{http://diy.inria.fr/www/?record=aarch64}{herd7} rejects it.
%\verbatiminput{litmus/RWC+acq+sc.litmus}
% More legibly:
% \begin{verbatim}
% STLR#1,[x]     | LDR a, [x] /1    | STLR #1, [y] 
%                | DMB LD           | LDAR c, [x] /0
%                | LDAR b, [y] /0
% \end{verbatim}
Reason: they are citing the flowing/pop model
\cite{DBLP:conf/popl/FlurGPSSMDS16} rather than
\cite{DBLP:journals/pacmpl/PulteFDFSS18}.
\begin{gather*}
  \taglabel{rwc+acq+sc}
  \PW[\mSC]{x}{1} \PAR
  \PR{x}{r}\SEMI
  \PF{\fACQ}\SEMI
  \PR[\mSC]{y}{s} \PAR
  \PW[\mSC]{y}{1}\SEMI
  \PR[\mSC]{x}{r}
  \\
  \tag{\xmark\armeight}
  \hbox{\begin{tikzinline}[node distance=1.5em]
      \scevent{a}{\DW[\mSC]{x}{1}}{}
      \event{b}{\DR{x}{1}}{right=2.5em of a}
      \event{c}{\DF{\fACQ}}{right=of b}
      \scevent{d}{\DR[\mSC]{y}{0}}{right=of c}
      \scevent{e}{\DW[\mSC]{y}{1}}{right=2.5em of d}
      \scevent{f}{\DR[\mSC]{x}{0}}{right=of e}
      \rfe{a}{b}
      \sync{b}{c}
      \sync{c}{d}
      \fre[out=-165,in=-15]{f}{a}
      \fre{d}{e}
      \sync{e}{f}
    \end{tikzinline}}
\end{gather*}

\section{Two order idea}
The two order idea from OOPSLA talk is:
\begin{itemize}
\item Require: $\bEv\leloc\aEv$ when $\bEv\ledep\aEv$ and they conflict
\end{itemize}
This does not work for the \IMM{} or ARMv7, but it may work for Power, TSO,
ARMv8.  That would be nice.  Let's write $\leloctwo$ for this notion, with
strong fulfillment.

With this there is a cycle in \ref{arm7-weak} (weak/strong fulfillment not relevant here):
\begin{gather*}
  \tag{$\leloctwo$}
  \hbox{\begin{tikzinline}[node distance=1.5em]
      \event{a}{\DR{x}{1}}{}
      \event{b}{d:\DW{x}{1}}{right=of a}
      \wk{a}{b}
      \event{c}{\DR{x}{1}}{right=3em of b}
      \event{d}{\DW{y}{1}}{right=of c}
      %\po{c}{d}
      \event{e}{\DR{y}{1}}{right=3em of d}
      \event{f}{e:\DW{x}{1}}{right=of e}
      % \po{e}{f}
      \po[out=-15,in=-165]{c}{f}
      \rf{b}{c}
      \rf{d}{e}
      \rf[out=172,in=8]{f}{a}
    \end{tikzinline}}    
\end{gather*}
Anton says: \ref{arm7-weak} is forbidden by Power, TSO, ARMv8, but allowed by
ARMv7. Maybe it isn't that important to support it anymore.

There is also a cycle in \ref{pub-rel-rlx-coe}.  Anton says: I checked
Power/ARMv7 models in this regard. They disallow the behavior (as well as
ARMv8 and TSO), so we can in principle strengthen \IMM{} to forbid it as
well.  For that, we may add axiom to \IMM{} forbidding cycles in
\begin{math}
  \rco \cup ([\mathsf{W}]; \rrfe^?; ([\mathsf{R}^{\fACQ}] \cup \rpox;
  [\mathsf{FW}^{\fREL}]); \rar^{*}; [\mathsf{W}]).
\end{math}
This works if we have acquire/release accesses on the path
since they are compiled with fences to Power.

\endinput

\section{OLD Model}

\begin{align*}
  \amode \BNFDEF& \mWK &&\text{{(Weak)}}                      &\ascope \BNFDEF& \sCTA &&\text{(Thread group)} &\hbox{$\;\mkern60mu\;$}&
  \\[-1ex] \BNFSEP& \mRLX &&\text{{(Relaxed)}}                & \BNFSEP&\sGPU   &&\text{(Processor)}                                   
  \\[-1ex] \BNFSEP& \mRA &&\text{{(Release/Acquire)}}         & \BNFSEP&\sSYS  &&\text{(System)}                                         
  \\[-1ex] \BNFSEP& \mSC &&\text{{(Sequentially Consistent)}}    
\end{align*}

Orders/Relations in model
\begin{itemize}
\item $\ledep$ is the old $\le$ (without coherence stuff from \ref{rf4} and \ref{5b}).

  This provides the NO-TAR axiom.
\item $\lesync$ is a the \emph{happens-before} suborder, which only includes $\rrf$ when they are morally strong.

  This serves as a cross-location transitive kernel for the per-location order.
  
\item $\leloc$ is a per-location order that relates morally strong  and $\rpoloc$ accesses

  This includes $\lesync$ for  morally strong accesses.

  This provides the SC-PER-LOC axiom.

  % \item $\rrmw$ is a per-location relation on actions in an \RMW{}
\end{itemize}

Write $\bEv\conflict\aEv$ if they conflict (ie, read/write or write/write, same location).

Write $\bEv\moral\aEv$ if they conflict and are morally strong

\begin{definition}
  A \emph{pomset with preconditions} is a tuple
  $(\aEvs, \labeling, {\lesync}, {\ledep}, {\leloc})$ where
  \begin{description}
  \item[{\labeltextsc[m1]{(m1)}{m1}}] $\aEvs$ is a set of \emph{events}
  \item[{\labeltextsc[m2]{(m2)}{m2}}]
    $\labeling: \aEvs \fun (\Formulae\times\Act)$ is a \emph{labeling} from
    which we derive functions
    \begin{itemize}
    \item $\labelingForm:\aEvs\fun\Formulae$
      \emph{(formulae)} % include $r{=}v$ $x{=}v$
    \item $\labelingAct:\aEvs\fun\Act$
      \emph{(actions)} %include $\DW{x}{v}$, $\DR{x}{v}$, and $\DSTOP$
    \end{itemize}
  \item[{\labeltextsc[m3]{(m3)}{m3}}]
    ${\lesync} \subseteq (\aEvs\times\aEvs)$,
    ${\ledep} \subseteq (\aEvs\times\aEvs)$, and
    ${\leloc} \subseteq (\aEvs\times\aEvs)$ are partial orders
  \item[{\labeltextsc[m4]{(m4)}{m-consistency}}] $\bigwedge_{\aEv}\labelingForm(\aEv)$ is satisfiable \emph{(consistency)}
  \item[{\labeltextsc[m5]{(m5)}{m-causal-strengthening}}] if $\bEv\ledep\aEv$ then $\labelingForm(\aEv)$ implies $\labelingForm(\bEv)$ \emph{(causal strengthening)} 
  \item[{\labeltextsc[m6]{(m6)}{m-strong}}] if $\bEv\lesync\aEv$ then $\bEv\ledep\aEv$
  \item[{\labeltextsc[m7]{(m7)}{m-loc}}] if $\bEv\lesync\aEv$ and $\bEv$ conflicts with $\aEv$ then $\bEv\leloc\aEv$
  \end{description}
\end{definition}
% It is important that \ref{m-loc} covers all conflicting access.
% See \ref{pub1sys}.


  % We say $\bEv\ltsync\aEv$ when $\bEv\lesync\aEv$ and $\bEv\neq\aEv$, and similarly
  % for $\ltdep$ and $\ltloc$.

% \begin{definition}
%   Define $\leexists$ %and $\ltexists$
%   as follows:
% \end{definition}


\begin{definition}[Strong fulfillment]
  We say $\labelingAct(\bEv)=(\DW[]{x}{v})$ \emph{fulfills}
  $\labelingAct(\aEv)=(\DR[]{x}{v})$ if
  \begin{description}
  \item[{\labeltextsc[f3a]{(f3a)}{rf3a}}{\labeltextsc[f3]{}{rf3}}]
    $\bEv \ltdep \aEv$
  \item[{\labeltextsc[f3b]{(f3b)}{rf3b}}]
    $\bEv \ltsync \aEv$ if $\bEv$ is morally strong with $\aEv$
  \item[{\labeltextsc[f3c]{(f3c)}{rf3c}}]
    $\bEv \leloc \aEv$ (if $\bEv$ is not morally strong with $\aEv$)
  \item[{\labeltextsc[f4]{(f4)}{rf4}}]
    $\forall\labelingAct(\cEv)=(\DW[]{x}{..})$ either $\cEv \leloc \bEv$ or
    $\aEv \leloc \cEv$,
  \end{description}  
\end{definition}
  
\begin{definition}[Weak fulfillment]
  We say $\labelingAct(\bEv)=(\DW[]{x}{v})$ \emph{fulfills}
  $\labelingAct(\aEv)=(\DR[]{x}{v})$ if
  \begin{description}
  \item[{\labeltextsc[f3a]{(f3a)}{rf3a}}{\labeltextsc[f3]{}{rf3}}]
    $\bEv \ltdep \aEv$
  \item[{\labeltextsc[f3b]{(f3b)}{rf3b}}]
    $\bEv \ltsync \aEv$ if $\bEv$ is morally strong with $\aEv$
  \item[{\labeltextsc[f3c]{(f3c)}{rf3c}}]
    $\aEv \not\leloc \bEv$ (if $\bEv$ is not morally strong with $\aEv$)
  \item[{\labeltextsc[f4]{(f4)}{rf4}}]
    $\forall\labelingAct(\cEv)=(\DW[]{x}{..})$ either $\cEv \leexists \bEv$ or
    $\aEv \leexists \cEv$,
    where
  \begin{align*}
    \bEv\leexists\aEv &\textwhen                      
    \begin{cases}
      \bEv\leloc\aEv &\text{if}\; \bEv \;\text{is morally strong with}\;
      \aEv %\bEv\moral\aEv
      \\
      \aEv\not\ltloc\bEv &\text{otherwise}
    \end{cases}
    % \\
    % \bEv\ltexists\aEv &\textwhen                      
    % \begin{cases}
    %   \bEv\ltloc\aEv &\text{if}\; \bEv \;\text{is morally strong with}\;
    %   \aEv %\bEv\moral\aEv
    %   \\
    %   \aEv\not\leloc\bEv &\text{otherwise}
    % \end{cases}
  \end{align*}    
  \end{description}  
\end{definition}

If all accesses are morally strong with each other, weak fulfillment
degenerates to
\begin{description}
\item[\eqref{rf3}]
  $\bEv \ltsync \aEv$
\item[\eqref{rf4}]
  $\forall\labelingAct(\cEv)=(\DW[]{x}{..})$ either
  $\cEv \leloc \bEv$ or $\aEv \leloc \cEv$
\end{description}

If no accesses are morally strong with each other, weak fulfillment
degenerates to
\begin{description}
\item[\eqref{rf3}]
  $\aEv \not\leloc \bEv$
\item[\eqref{rf4}]
  $\not\mkern-5mu\exists\labelingAct(\cEv)=(\DW[]{x}{..})$ 
  both $\bEv \ltloc \cEv$ and $\cEv \ltloc \aEv$
\end{description}

Note that the difference between strong and weak fulfillment is limited to $\leloc$.
We sometimes write $\lelocstrong$ for strong fulfillment and
$\lelocweak$ for weak fulfillment.

Prefixing is as in OOPSLA, using $\lesync$ for order everywhere except
\ref{5b}, which has $\leloc$.
\begin{definition}
  Let $\aPS'\in(\aForm \mid \aAct) \prefix \aPSS$ when
  $(\exists\aPS\in\aPSS)$ $(\forall\aEv\in\aEvs)$
  \begin{description}
  \item[{\labeltextsc[P1]{(P1)}{1}}] $\aEvs' = \aEvs \cup \{\bEv\}$
  \item[{\labeltextsc[P2]{(P2)}{2}}] ${\lesync'}\supseteq{\lesync}$,
    ${\ledep'}\supseteq{\ledep}$, and ${\leloc'}\supseteq{\leloc}$
  \item[{\labeltextsc[P3]{(P3a)}{3a}\labeltextsc[P3]{}{3}}]%
    $\labelingAct'(\aEv) = \labelingAct(\aEv)$
  \item[{\labeltextsc[P3b]{(P3b)}{3b}}] $\labelingAct'(\bEv) = \aAct$
  \item[{\labeltextsc[P4a]{(P4a)}{4a}\labeltextsc[P4]{}{4}}]%
    $\labelingForm'(\bEv)$ implies
    $\aForm\land(\bEv\not\in\aEvs\lor\labelingForm(\bEv))$
  \item[{\labeltextsc[P4b]{(P4b)}{4b}}] if $\bEv\neq(\DR[]{..)}{}\mkern79mu$
    then $\aEv=\bEv$ or $\labelingForm'(\aEv)$ implies $\labelingForm(\aEv)$
  \item[{\labeltextsc[P4c]{(P4c)}{4c}}] if
    $\bEv=(\DR[]{\aVal}{\aLoc})\mkern70mu$ then $\aEv=\bEv$ or
    $\labelingForm'(\aEv)$ implies $\labelingForm(\aEv)[\aVal/\aLoc]$
  \item[{\labeltextsc[P5a]{(P5a)}{5a}\labeltextsc[P5]{}{5}}]%
    if $\bEv=(\DR[]{..)}{}$, $\aEv=(\DW[]{..)}{}$ then $\aEv=\bEv$ or
    $\labelingForm'(\aEv)$ implies $\labelingForm(\aEv)$ or $\bEv\lesync'\aEv$
  \item[{\labeltextsc[P5b]{(P5b)}{5b}}] if $\bEv$ conflicts with
    $\aEv$ %$\bEv\conflict\aEv$
    then $\bEv\leloc'\aEv$
  \item[{\labeltextsc[P5c]{(P5c)}{5c}}] if $\bEv$ is an acquire or $\aEv$ is
    a release then $\bEv \lesync' \aEv$
  \item[{\labeltextsc[P5d]{(P5d)}{5d}}] if $\bEv$ is an SC write and $\aEv$
    is an SC read then $\bEv \lesync' \aEv$
  \item[{\labeltextsc[P5e]{(P5e)}{5e}}] if $\bEv$ reads, and $\aEv$ is an
    acquiring fence, then
    $\bEv \lesync' \aEv$
  \item[{\labeltextsc[P5f]{(P5f)}{5f}}] if $\bEv$ is a releasing fence,
    and $\aEv$ writes, then
    $\bEv \lesync' \aEv$
  \end{description}
\end{definition}

% \section{More Model}
% These definitions need to be updated to include the additional orders.
% \begin{definition}
%   A pomset is \emph{$\aLoc$-closed} if
%   \begin{itemize}
%   \item every $\labelingAct(\aEv)=(\DR{\aLoc}{..})$ is fulfilled
%   \item every $\labelingForm(\aEv)$ is independent of $x$:
%     $\bigl(\forall v.\;\labelingForm(\aEv) \vDash
%     \labelingForm(\aEv)[\aVal/\aLoc] \vDash \labelingForm(\aEv)\bigr)$
%   \end{itemize}
% \end{definition}

% \begin{definition}
%   Let $\aPS\phantom{'}\in(\nu\aLoc\!\DOT\!\aPSS) \mkern22mu$ when
%   $\phantom{(\exists}\aPS\in\aPSS$ and $\aPS$ is $\aLoc$-closed
% \end{definition}
% \begin{definition}
%   Let $\aPS\phantom{'}\in(\aForm \guard \aPSS)\mkern16mu$ when
%   $\phantom{(\exists}\aPS\in\aPSS$ and $(\forall\aEv\in\aEvs)$
%   $\labelingForm(\aEv)$ implies $\aForm$
% \end{definition}

% \begin{definition}
%   Let $\aPS'\in(\aPSS[M/x])\mkern2mu$ when $(\exists\aPS\in\aPSS$)\\\qquad
%   $\aEvs' = \aEvs$, ${\lesync'} = {\lesync}$, $\labelingAct' = \labelingAct$, and
%   $(\forall\aEv\in\aEvs')$ $\labelingForm'(\aEv) = \labelingForm(\aEv)[M/x]$
% \end{definition}
% \begin{definition}
%   Let $\aPS' \in (\aPSS^1 \parallel \aPSS^2)$ when
%   $(\exists\aPS^1 \in \aPSS^1)$ $(\exists\aPS^2 \in \aPSS^2)$
%   \\% $\aPS^1$ is completed exactly when $\aPS^2$ is completed, there is at most one termination in $\aEvs'$,
%   \qquad $\aEvs' = \aEvs^1 \cup \aEvs^2$,
%   ${\lesync'}\supseteq{\lesync^1}\cup{\lesync^2}$, and $(\forall\aEv\in\aEvs')$ either
%   \begin{gather*}
%     \begin{aligned}
%       \aEv \not\in \aEvs^2,\; \labelingAct'(\aEv) &= \labelingAct^1(\aEv)
%       \textand \labelingForm'(\aEv) \textimplies \labelingForm^1(\aEv),
%       \\[-1ex]
%       \aEv \not\in \aEvs^1,\; \labelingAct'(\aEv) &= \labelingAct^2(\aEv)
%       \textand \labelingForm'(\aEv) \textimplies
%       \labelingForm^2(\aEv),\textor
%       \\[-1ex]
%       \labelingAct'(\aEv) = \labelingAct^1(\aEv) &= \labelingAct^2(\aEv)
%       \textand \labelingForm'(\aEv) \textimplies \labelingForm^1(\aEv) \lor
%       \labelingForm^2(\aEv)
%     \end{aligned}
%   \end{gather*}
% \end{definition}

% Language
% \begin{gather*}
%   % \begin{aligned}
%   %   \aCmd,\,\bCmd
%   %   \BNFDEF& \SKIP
%   %   \BNFSEP \aReg\GETS\aExp\SEMI \aCmd
%   %   \BNFSEP \aReg\GETS\aLoc^{\amode}\SEMI \aCmd 
%   %   \BNFSEP \aLoc^{\amode}\GETS\aExp\SEMI \aCmd
%   %   \\[-.5ex]
%   %   \BNFSEP&\aCmd \PAR[\aThrd][\bThrd] \bCmd
%   %   \BNFSEP \VAR\aLoc\SEMI \aCmd
%   %   \BNFSEP \IF{\aExp} \THEN \aCmd \ELSE \bCmd \FI
%   % \end{aligned}
%   % \\
%   \begin{aligned}
%     \sem[\aThrd]{\SKIP} & \eqdef
%     \{ \DSTOP \}
%     \\
%     \sem[\aThrd]{\aReg\GETS\aExp\SEMI \aCmd} & \eqdef
%     \sem[\aThrd]{\aCmd}[\aExp/\aReg]
%     \\ 
%     \sem{\aReg\GETS\aLoc^{\amode}\SEMI \aCmd} & \eqdef \textstyle\bigcup_\aVal\;
%     (\DR[\amode]\aLoc\aVal)\prefix \sem{\aCmd} [\aLoc/\aReg]
%     \\
%     \sem{\aLoc^{\amode}\GETS\aExp\SEMI \aCmd} & \eqdef
%     \textstyle\bigcup_\aVal\; (\aExp=\aVal \mid \DW[\amode]\aLoc\aVal)\prefix \sem{\aCmd}[\aExp/\aLoc]
%     \\
%     \sem{\FENCE^{\fmode}\SEMI \aCmd} & \eqdef
%     (\DFS{\fmode}) \prefix \sem{\aCmd}
%     \\
%     \sem[\aThrd]{\IF{\aExp} \THEN \aCmd_1 \ELSE \aCmd_2 \FI} & \eqdef
%     \bigl(\aExp \guard \sem[\aThrd]{\aCmd_1}\bigr) \parallel \bigl(\lnot\aExp \guard \sem[\aThrd]{\aCmd_2}\bigr) 
%     \\
%     \sem[\aThrd]{\aCmd_1 \PAR[\bThrd][\bThrd'] \aCmd_2} & \eqdef
%     \sem[\bThrd]{\aCmd_1} \parallel \sem[\bThrd']{\aCmd_2} 
%     \\
%     \sem[\aThrd]{\VAR\aLoc\SEMI \aCmd} & \eqdef
%     \nu \aLoc \DOT \sem[\aThrd]{\aCmd}  
%   \end{aligned}
% \end{gather*}
% \begin{align*}
%   \fmode \BNFDEF&\fREL  &&\text{(Release)} &\hbox{$\;\mkern60mu\;$}&
%   \\ \BNFSEP&\fACQ   &&\text{(Acquire)} 
%   \\      \BNFSEP&\mSC  &&\text{(SC)} 
% \end{align*}






\end{document}





\section{Typesetting instructions -- Summary}
\label{sec:typesetting-summary}

LIPIcs is a series of open access high-quality conference proceedings across all fields in informatics established in cooperation with Schloss Dagstuhl. 
In order to do justice to the high scientific quality of the conferences that publish their proceedings in the LIPIcs series, which is ensured by the thorough review process of the respective events, we believe that LIPIcs proceedings must have an attractive and consistent layout matching the standard of the series.
Moreover, the quality of the metadata, the typesetting and the layout must also meet the requirements of other external parties such as indexing service, DOI registry, funding agencies, among others. The guidelines contained in this document serve as the baseline for the authors, editors, and the publisher to create documents that meet as many different requirements as possible. 

Please comply with the following instructions when preparing your article for a LIPIcs proceedings volume. 
\paragraph*{Minimum requirements}

\begin{itemize}
\item Use pdflatex and an up-to-date \LaTeX{} system.
\item Use further \LaTeX{} packages and custom made macros carefully and only if required.
\item Use the provided sectioning macros: \verb+\section+, \verb+\subsection+, \verb+\subsubsection+, \linebreak \verb+\paragraph+, \verb+\paragraph*+, and \verb+\subparagraph*+.
\item Provide suitable graphics of at least 300dpi (preferably in PDF format).
\item Use BibTeX and keep the standard style (\verb+plainurl+) for the bibliography.
\item Please try to keep the warnings log as small as possible. Avoid overfull \verb+\hboxes+ and any kind of warnings/errors with the referenced BibTeX entries.
\item Use a spellchecker to correct typos.
\end{itemize}

\paragraph*{Mandatory metadata macros}
Please set the values of the metadata macros carefully since the information parsed from these macros will be passed to publication servers, catalogues and search engines.
Avoid placing macros inside the metadata macros. The following metadata macros/environments are mandatory:
\begin{itemize}
\item \verb+\title+ and, in case of long titles, \verb+\titlerunning+.
\item \verb+\author+, one for each author, even if two or more authors have the same affiliation.
\item \verb+\authorrunning+ and \verb+\Copyright+ (concatenated author names)\\
The \verb+\author+ macros and the \verb+\Copyright+ macro should contain full author names (especially with regard to the first name), while \verb+\authorrunning+ should contain abbreviated first names.
\item \verb+\ccsdesc+ (ACM classification, see \url{https://www.acm.org/publications/class-2012}).
\item \verb+\keywords+ (a comma-separated list of keywords).
\item \verb+\relatedversion+ (if there is a related version, typically the ``full version''); please make sure to provide a persistent URL, e.\,g., at arXiv.
\item \verb+\begin{abstract}...\end{abstract}+ .
\end{itemize}

\paragraph*{Please do not \ldots} %Do not override the \texttt{\seriesstyle}-defaults}
Generally speaking, please do not override the \texttt{lipics-v2019}-style defaults. To be more specific, a short checklist also used by Dagstuhl Publishing during the final typesetting is given below.
In case of \textbf{non-compliance} with these rules Dagstuhl Publishing will remove the corresponding parts of \LaTeX{} code and \textbf{replace it with the \texttt{lipics-v2019} defaults}. In serious cases, we may reject the LaTeX-source and expect the corresponding author to revise the relevant parts.
\begin{itemize}
\item Do not use a different main font. (For example, the \texttt{times} package is forbidden.)
\item Do not alter the spacing of the \texttt{lipics-v2019.cls} style file.
\item Do not use \verb+enumitem+ and \verb+paralist+. (The \texttt{enumerate} package is preloaded, so you can use
 \verb+\begin{enumerate}[(a)]+ or the like.)
\item Do not use ``self-made'' sectioning commands (e.\,g., \verb+\noindent{\bf My+ \verb+Paragraph}+).
\item Do not hide large text blocks using comments or \verb+\iffalse+ $\ldots$ \verb+\fi+ constructions. 
\item Do not use conditional structures to include/exclude content. Instead, please provide only the content that should be published -- in one file -- and nothing else.
\item Do not wrap figures and tables with text. In particular, the package \texttt{wrapfig} is not supported.
\item Do not change the bibliography style. In particular, do not use author-year citations. (The
\texttt{natbib} package is not supported.)
\end{itemize}

\enlargethispage{\baselineskip}

This is only a summary containing the most relevant details. Please read the complete document ``LIPIcs: Instructions for Authors and the \texttt{lipics-v2019} Class'' for all details and don't hesitate to contact Dagstuhl Publishing (\url{mailto:publishing@dagstuhl.de}) in case of questions or comments:
\href{http://drops.dagstuhl.de/styles/lipics-v2019/lipics-v2019-authors/lipics-v2019-authors-guidelines.pdf}{\texttt{http://drops.dagstuhl.de/styles/lipics-v2019/\newline lipics-v2019-authors/lipics-v2019-authors-guidelines.pdf}}

\section{Lorem ipsum dolor sit amet}

Lorem ipsum dolor sit amet, consectetur adipiscing elit \cite{DBLP:journals/cacm/Knuth74}. Praesent convallis orci arcu, eu mollis dolor. Aliquam eleifend suscipit lacinia. Maecenas quam mi, porta ut lacinia sed, convallis ac dui. Lorem ipsum dolor sit amet, consectetur adipiscing elit. Suspendisse potenti. Donec eget odio et magna ullamcorper vehicula ut vitae libero. Maecenas lectus nulla, auctor nec varius ac, ultricies et turpis. Pellentesque id ante erat. In hac habitasse platea dictumst. Curabitur a scelerisque odio. Pellentesque elit risus, posuere quis elementum at, pellentesque ut diam. Quisque aliquam libero id mi imperdiet quis convallis turpis eleifend. 

\begin{lemma}[Lorem ipsum]
\label{lemma:lorem}
Vestibulum sodales dolor et dui cursus iaculis. Nullam ullamcorper purus vel turpis lobortis eu tempus lorem semper. Proin facilisis gravida rutrum. Etiam sed sollicitudin lorem. Proin pellentesque risus at elit hendrerit pharetra. Integer at turpis varius libero rhoncus fermentum vitae vitae metus.
\end{lemma}

\begin{proof}
Cras purus lorem, pulvinar et fermentum sagittis, suscipit quis magna.

\begin{claim}
content...
\end{claim}
\begin{claimproof}
content...
\end{claimproof}

\end{proof}

\begin{corollary}[Curabitur pulvinar, \cite{DBLP:books/mk/GrayR93}]
\label{lemma:curabitur}
Nam liber tempor cum soluta nobis eleifend option congue nihil imperdiet doming id quod mazim placerat facer possim assum. Lorem ipsum dolor sit amet, consectetuer adipiscing elit, sed diam nonummy nibh euismod tincidunt ut laoreet dolore magna aliquam erat volutpat.
\end{corollary}

\begin{proposition}\label{prop1}
This is a proposition
\end{proposition}

\autoref{prop1} and \cref{prop1} \ldots

\subsection{Curabitur dictum felis id sapien}

Curabitur dictum \cref{lemma:curabitur} felis id sapien \autoref{lemma:curabitur} mollis ut venenatis tortor feugiat. Curabitur sed velit diam. Integer aliquam, nunc ac egestas lacinia, nibh est vehicula nibh, ac auctor velit tellus non arcu. Vestibulum lacinia ipsum vitae nisi ultrices eget gravida turpis laoreet. Duis rutrum dapibus ornare. Nulla vehicula vulputate iaculis. Proin a consequat neque. Donec ut rutrum urna. Morbi scelerisque turpis sed elit sagittis eu scelerisque quam condimentum. Pellentesque habitant morbi tristique senectus et netus et malesuada fames ac turpis egestas. Aenean nec faucibus leo. Cras ut nisl odio, non tincidunt lorem. Integer purus ligula, venenatis et convallis lacinia, scelerisque at erat. Fusce risus libero, convallis at fermentum in, dignissim sed sem. Ut dapibus orci vitae nisl viverra nec adipiscing tortor condimentum \cite{DBLP:journals/cacm/Dijkstra68a}. Donec non suscipit lorem. Nam sit amet enim vitae nisl accumsan pretium. 

\begin{lstlisting}[caption={Useless code},label=list:8-6,captionpos=t,float,abovecaptionskip=-\medskipamount]
for i:=maxint to 0 do 
begin 
    j:=square(root(i));
end;
\end{lstlisting}

\subsection{Proin ac fermentum augue}

Proin ac fermentum augue. Nullam bibendum enim sollicitudin tellus egestas lacinia euismod orci mollis. Nulla facilisi. Vivamus volutpat venenatis sapien, vitae feugiat arcu fringilla ac. Mauris sapien tortor, sagittis eget auctor at, vulputate pharetra magna. Sed congue, dui nec vulputate convallis, sem nunc adipiscing dui, vel venenatis mauris sem in dui. Praesent a pretium quam. Mauris non mauris sit amet eros rutrum aliquam id ut sapien. Nulla aliquet fringilla sagittis. Pellentesque eu metus posuere nunc tincidunt dignissim in tempor dolor. Nulla cursus aliquet enim. Cras sapien risus, accumsan eu cursus ut, commodo vel velit. Praesent aliquet consectetur ligula, vitae iaculis ligula interdum vel. Integer faucibus faucibus felis. 

\begin{itemize}
\item Ut vitae diam augue. 
\item Integer lacus ante, pellentesque sed sollicitudin et, pulvinar adipiscing sem. 
\item Maecenas facilisis, leo quis tincidunt egestas, magna ipsum condimentum orci, vitae facilisis nibh turpis et elit. 
\end{itemize}

\begin{remark}
content...
\end{remark}

\section{Pellentesque quis tortor}

Nec urna malesuada sollicitudin. Nulla facilisi. Vivamus aliquam tempus ligula eget ornare. Praesent eget magna ut turpis mattis cursus. Aliquam vel condimentum orci. Nunc congue, libero in gravida convallis \cite{DBLP:conf/focs/HopcroftPV75}, orci nibh sodales quam, id egestas felis mi nec nisi. Suspendisse tincidunt, est ac vestibulum posuere, justo odio bibendum urna, rutrum bibendum dolor sem nec tellus. 

\begin{lemma} [Quisque blandit tempus nunc]
Sed interdum nisl pretium non. Mauris sodales consequat risus vel consectetur. Aliquam erat volutpat. Nunc sed sapien ligula. Proin faucibus sapien luctus nisl feugiat convallis faucibus elit cursus. Nunc vestibulum nunc ac massa pretium pharetra. Nulla facilisis turpis id augue venenatis blandit. Cum sociis natoque penatibus et magnis dis parturient montes, nascetur ridiculus mus.
\end{lemma}

Fusce eu leo nisi. Cras eget orci neque, eleifend dapibus felis. Duis et leo dui. Nam vulputate, velit et laoreet porttitor, quam arcu facilisis dui, sed malesuada risus massa sit amet neque.

\section{Morbi eros magna}

Morbi eros magna, vestibulum non posuere non, porta eu quam. Maecenas vitae orci risus, eget imperdiet mauris. Donec massa mauris, pellentesque vel lobortis eu, molestie ac turpis. Sed condimentum convallis dolor, a dignissim est ultrices eu. Donec consectetur volutpat eros, et ornare dui ultricies id. Vivamus eu augue eget dolor euismod ultrices et sit amet nisi. Vivamus malesuada leo ac leo ullamcorper tempor. Donec justo mi, tempor vitae aliquet non, faucibus eu lacus. Donec dictum gravida neque, non porta turpis imperdiet eget. Curabitur quis euismod ligula. 


%%
%% Bibliography
%%

%% Please use bibtex, 

\bibliography{lipics-v2019-sample-article}

\appendix

\section{Styles of lists, enumerations, and descriptions}\label{sec:itemStyles}

List of different predefined enumeration styles:

\begin{itemize}
\item \verb|\begin{itemize}...\end{itemize}|
\item \dots
\item \dots
%\item \dots
\end{itemize}

\begin{enumerate}
\item \verb|\begin{enumerate}...\end{enumerate}|
\item \dots
\item \dots
%\item \dots
\end{enumerate}

\begin{alphaenumerate}
\item \verb|\begin{alphaenumerate}...\end{alphaenumerate}|
\item \dots
\item \dots
%\item \dots
\end{alphaenumerate}

\begin{romanenumerate}
\item \verb|\begin{romanenumerate}...\end{romanenumerate}|
\item \dots
\item \dots
%\item \dots
\end{romanenumerate}

\begin{bracketenumerate}
\item \verb|\begin{bracketenumerate}...\end{bracketenumerate}|
\item \dots
\item \dots
%\item \dots
\end{bracketenumerate}

\begin{description}
\item[Description 1] \verb|\begin{description} \item[Description 1]  ...\end{description}|
\item[Description 2] Fusce eu leo nisi. Cras eget orci neque, eleifend dapibus felis. Duis et leo dui. Nam vulputate, velit et laoreet porttitor, quam arcu facilisis dui, sed malesuada risus massa sit amet neque.
\item[Description 3]  \dots
%\item \dots
\end{description}

\cref{testenv-proposition} and \autoref{testenv-proposition} ...

\section{Theorem-like environments}\label{sec:theorem-environments}

List of different predefined enumeration styles:

\begin{theorem}\label{testenv-theorem}
Fusce eu leo nisi. Cras eget orci neque, eleifend dapibus felis. Duis et leo dui. Nam vulputate, velit et laoreet porttitor, quam arcu facilisis dui, sed malesuada risus massa sit amet neque.
\end{theorem}

\begin{lemma}\label{testenv-lemma}
Fusce eu leo nisi. Cras eget orci neque, eleifend dapibus felis. Duis et leo dui. Nam vulputate, velit et laoreet porttitor, quam arcu facilisis dui, sed malesuada risus massa sit amet neque.
\end{lemma}

\begin{corollary}\label{testenv-corollary}
Fusce eu leo nisi. Cras eget orci neque, eleifend dapibus felis. Duis et leo dui. Nam vulputate, velit et laoreet porttitor, quam arcu facilisis dui, sed malesuada risus massa sit amet neque.
\end{corollary}

\begin{proposition}\label{testenv-proposition}
Fusce eu leo nisi. Cras eget orci neque, eleifend dapibus felis. Duis et leo dui. Nam vulputate, velit et laoreet porttitor, quam arcu facilisis dui, sed malesuada risus massa sit amet neque.
\end{proposition}

\begin{exercise}\label{testenv-exercise}
Fusce eu leo nisi. Cras eget orci neque, eleifend dapibus felis. Duis et leo dui. Nam vulputate, velit et laoreet porttitor, quam arcu facilisis dui, sed malesuada risus massa sit amet neque.
\end{exercise}

\begin{definition}\label{testenv-definition}
Fusce eu leo nisi. Cras eget orci neque, eleifend dapibus felis. Duis et leo dui. Nam vulputate, velit et laoreet porttitor, quam arcu facilisis dui, sed malesuada risus massa sit amet neque.
\end{definition}

\begin{example}\label{testenv-example}
Fusce eu leo nisi. Cras eget orci neque, eleifend dapibus felis. Duis et leo dui. Nam vulputate, velit et laoreet porttitor, quam arcu facilisis dui, sed malesuada risus massa sit amet neque.
\end{example}

\begin{note}\label{testenv-note}
Fusce eu leo nisi. Cras eget orci neque, eleifend dapibus felis. Duis et leo dui. Nam vulputate, velit et laoreet porttitor, quam arcu facilisis dui, sed malesuada risus massa sit amet neque.
\end{note}

\begin{note*}
Fusce eu leo nisi. Cras eget orci neque, eleifend dapibus felis. Duis et leo dui. Nam vulputate, velit et laoreet porttitor, quam arcu facilisis dui, sed malesuada risus massa sit amet neque.
\end{note*}

\begin{remark}\label{testenv-remark}
Fusce eu leo nisi. Cras eget orci neque, eleifend dapibus felis. Duis et leo dui. Nam vulputate, velit et laoreet porttitor, quam arcu facilisis dui, sed malesuada risus massa sit amet neque.
\end{remark}

\begin{remark*}
Fusce eu leo nisi. Cras eget orci neque, eleifend dapibus felis. Duis et leo dui. Nam vulputate, velit et laoreet porttitor, quam arcu facilisis dui, sed malesuada risus massa sit amet neque.
\end{remark*}

\begin{claim}\label{testenv-claim}
Fusce eu leo nisi. Cras eget orci neque, eleifend dapibus felis. Duis et leo dui. Nam vulputate, velit et laoreet porttitor, quam arcu facilisis dui, sed malesuada risus massa sit amet neque.
\end{claim}

\begin{claim*}\label{testenv-claim2}
Fusce eu leo nisi. Cras eget orci neque, eleifend dapibus felis. Duis et leo dui. Nam vulputate, velit et laoreet porttitor, quam arcu facilisis dui, sed malesuada risus massa sit amet neque.
\end{claim*}

\begin{proof}
Fusce eu leo nisi. Cras eget orci neque, eleifend dapibus felis. Duis et leo dui. Nam vulputate, velit et laoreet porttitor, quam arcu facilisis dui, sed malesuada risus massa sit amet neque.
\end{proof}

\begin{claimproof}
Fusce eu leo nisi. Cras eget orci neque, eleifend dapibus felis. Duis et leo dui. Nam vulputate, velit et laoreet porttitor, quam arcu facilisis dui, sed malesuada risus massa sit amet neque.
\end{claimproof}

\end{document}
